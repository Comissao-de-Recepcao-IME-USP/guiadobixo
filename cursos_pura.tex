\begin{subsecao}{Pura}
{\em Paula Corradi, Marina Trindade e Mauricio ``=o)'' Camilo,
Andre “Shinji” Rodrigues, David, Vinicius Rodrigues}

\quadrinhos{19}

Ufa, você chegou à Pura! Seja bem vindo! Nesse curso você será apresentado a muitas das diversas áreas da matemática pura. Muitas pessoas entram no curso sem saber direito do que ele se trata e descobrem que afinal era bem diferente daquilo que esperavam que fosse. Mas estamos aqui para ajudar vocês!
%\begin{enumerate}[label=\roman{*})]

i) Como é o curso da Pura?

Agora você deve estar pensando como ele é...
\begin{itemize}


\item  Existem matérias extremamente úteis e práticas para o dia-a-dia e é
exatamente por isso que você vai acabar as odiando (Ex. Estatística, Física,
Computação...até português, se for forte). Mas felizmente elas acabam antes do fim do segundo ano!
\item  Logo depois disso começam as matérias específicas do curso (aquelas que a Poli nem sonha!). Topologia, Análise Funcional, Álgebra III, Teoria dos Conjuntos... Aí é que a coisa fica interessante!
\item  Você vai ter que estudar muito, mas nem por isso desista.

\end{itemize}
ii) O que fazer depois de se formar??

Agora você deve estar pensando: "O que eu faço depois de formado?”... (se você
não estava pensando, aposto que agora está).

O objetivo principal do Bacharelado em Matemática é formar bons pesquisadores em... Matemática!. Para quem não
sabe a matemática não está completa (nem nunca vai estar!), isto é, sempre tem alguma coisa nova para
descobrir. Caso seja isso que você queira, Mestrado e Doutorado te aguardam depois desse curso! Se você pensa que quem se forma nesse curso só pode ser
professor/pesquisador, você está muito enganado! O curso forma pessoas que
sabem analisar e resolver problemas metodicamente (você vai ver que estará
pensando com mais clareza em breve). Com seu potente raciocínio lógico, um
bacharel em Matemática pode fazer Pós-Graduação em Engenharia (argh!),
Computação, Estatística (argh$^2$!), Física (argh$^3$!), Economia (argh$^5$!).
Ele pode trabalhar em vários locais: universidades, colégios, bancos,
empresas... Enfim, a vida se torna muito mais fácil se você é matemático.

iii) Como lidar com o curso da Pura?

O essencial é gostar de matemática, ter gosto pela descoberta e pelo raciocínio em matemática! Um meio para lhe ajudar é trocar ideias com seus amigos: Além de conversar sobre matemática (você vai fazer isso bastante por aqui), vocês podem formar um grupo unido que esteja
disposto a enfrentar as matérias, línguas estrangeiras (de eventuais
professores), EPs (sim, você também faz EPs, se é que você sabe o que é isso),
provas, etc, além de estudarem juntos e, claro, aprenderem juntos. O curso da pura
é até bem divertido depois que você se enturma. Claro que
formar esse grupo não é a coisa mais fácil, já que, quando você começar a
prestar atenção nos seus colegas de turma, vai achar eles bem estranhos, mas,
depois de um certo tempo, você percebe que eles são bem parecidos com você. Não se esqueça também que há muitos veteranos que gostam de ajudar e o farão se você pedir!

Talvez já tenham te contado, mas esse curso pode ser fácil de entrar, mas
costumam formar-se uns 6 de nós por ano (e olhe lá!). E foi em 2011 que a Pura bateu o seu recorde de formandos ao mesmo tempo,
foram 16! Isso não acontecia desde pelo menos a época do Jacy Monteiro (que
você ainda vai saber quem é)! E ainda tiveram uns três que formaram em
três anos. Nunca desanime com as pessoas que querem abaixar a sua moral dizendo que você nunca vai se formar. É mentira! Todos nós amamos matemática, mas sabemos que matemática é trabalho duro e persistência (mas isso é ruim?). Então nunca desanime com os comentários dos outros e não desista!

v) O quê mais preciso saber sobre a Pura???

Primeiramente, apesar de toda a dificuldade, a Pura tem uma carga horária
relativamente menor do que a maioria dos outros cursos... Teoricamente, é
possível se formar em 3 anos e meio, ou até menos. E existem pessoas que o
fazem (ou tentam pelo menos). Mas tome cuidado: Além de difícil, você
corre o risco de não aprender nada e tirar notas bem mais baixas. Normalmente,
o tempo que você pode vir a ter a menos de aula precisará ser gasto estudando
por conta própria. Por isso, tome cuidado para não se sobrecarregar. Mas não se esqueça que há muitas atividades em que podem ser abandonadas sem se prejudicar  caso o curso fique pesado (e isso não é feio, viu?).

Tente tirar proveito da relativa flexibilidade da grade de horários: enquanto
que o primeiro ano você tem todas as aulas certinhas todo dia, com o passar do
curso você terá menos aulas (as quais tenderão a ficar mais difíceis), e sua
grade poderá ficar cheia de buracos. Não tenha medo do trancamento parcial,
quando você tiver medo de bombar alguma matéria, ou quando não se der bem com
um professor: em boa parte dos cursos pode valer a pena deixar determinada
matéria para depois do que fazer com algum professor com quem você não se dê
bem. Mas antes de tomar alguma atitude tão extrema, você tem sempre seus colegas de turma e seus colegas veteranos para te ajudar. No IME tem sim muita gente disposta a te ajudar, com o tempo você vai descobrir!

Acredite: a Pura só começa realmente no segundo ano. No primeiro, você terá
todas as matérias junto com outros cursos, como a estat, a aplicada e o BCC.
Aproveite para fazer contatos com as pessoas dos outros cursos, pois depois
disso a tendência é se distanciar delas. Só que por esse mesmo motivo, você
verá bastante coisa que provavelmente não usará no resto da Pura, além de que
no primeiro ano você não vai ter ainda uma boa noção do que será a pura. Você
terá uma ideia melhor do que é Matemática de verdade a partir de cursos
como Álgebra I e Análise Real.


Recentemente, houve alterações no currículo da Pura. Parabéns bixos! Vocês não
precisam mais fazer matérias chatas e que não tem nada a ver com o curso, como
Laboratório de Física e Português. Em compensação, vocês terão que fazer mais
duas matérias que não eram obrigatórias: Geometria Diferencial II e Análise Funcional.

Finalizando, deixamos para vocês os seguintes conselhos:
\begin{enumerate}


\item	Persista e lute;
\item	Lembre-se sempre que você gosta de Matemática;

\item	Informe-se sobre atividades extracurriculares como o programa de
Iniciação Científica (que é muito bom para formação, talvez até essencial) e
uma série de palestras com professores que, muito possivelmente, realizar-se-ão
durante o ano. Também há programas voltados para o primeiro ano com o propósito único de te auxiliar. A oportunidade é única, então se infome e aproveite!
\item	Tome consciência de que você, na grande maioria das vezes, vai ter que
estudar muito;
\item	Não desanime com as pessoas que dizem que você não vai conseguir. Se você gostar da coisa, você vai! É normal dar escorregadas e ir mal em algumas provas durante o seu curso, e pode ser até que você reprove em alguma coisa, mas isso não quer dizer que você não serve para a coisa! Se você gosta do curso, você vai conseguir chegar tão longe quanto você quiser. Mas é claro que nada acontece sem esforço!
\item Se você está com dúvidas, pergunte. Não importa se você vai perguntar pro professor, pro colega, pro monitor, pro cachorro, pro defunto, pro exú, pro Chuck Norris... Mas dúvidas pequenas hoje geralmente se tornam problemas enormes no fim do semestre, e esse tipo de coisa tem o potencial de te reprovar em alguma disciplina, além de, é claro, prejudicar seu aprendizado.
\item	Existem muitos veteranos que gostam de ajudar e que vão o fazer se você pedir. Com o tempo você vai descobrir quem são. Não tenha medo deles!
\item	Se você entrou sem saber ao certo como é o curso, seus colegas podem te ajudar a gostar! Esperamos fortemente que você goste e estamos dispostos a te ajudar com isso!
\item	E para você entrou sabendo, não seja arrogante... você pode ajudar os seus colegas a gostar! E claro, esperamos que o curso seja mesmo aquilo que você espera e que você também seja feliz com ele.

\end{enumerate}
Para terminar, faça amigos no IME, só eles vão te entender. Qualquer dúvida,
você pode nos procurar. Estaremos sempre dispostos a ajudá-lo para, assim,
preservarmos a nossa espécie !!!


\end{subsecao}
