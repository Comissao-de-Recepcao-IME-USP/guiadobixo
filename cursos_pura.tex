\begin{subsecao}{Pura}
{\em Paula Corradi, Marina Trindade e Mauricio ``=o)'' Camilo,
Andre “Shinji” Rodrigues, David}

Ufa, você chegou à Pura! Seja bem vindo! Mas, um aviso: se você entrou nesse
curso porque se dava bem com a matemática no colégio você vai descobrir que
aqui não é bem daquele jeito. Nem por isso desista (vamos até separar nosso
texto em itens para ficar mais fácil para você).
%\begin{enumerate}[label=\roman{*})]

i) Como é o curso da Pura?

Agora você deve estar pensando como ele é...
\begin{itemize}

\item Ele é Super-Duper-Mega-Mor-Ever-DeTodos DIFÍCIL... e nem por isso desista.
\item  Você vai ter que estudar muito$^5$, mas nem por isso desista.
\item  Existem matérias extremamente úteis e práticas para o dia-a-dia e é
exatamente por isso que você vai acabar odiando elas (Ex. Estatística, Física,
Computação...até português, se for forte), mas nem por isso desista.
\item  Você vai ter que fazer umas duas optativas fora do IME, por isso
aproveite para relaxar e abrir sua cabeça. Já pensou em aprender alguma outra
língua? Logo você vai perceber que já sabe o alfabeto grego inteiro (maiúsculas
e minúsculas), nada mais justo que saber associá-las. Há quem faça mímica na
ECA, microeconomia na FEA, métodos anticoncepcionais na enfermagem e até a
lenda sobre o ex-aluno que fez "Fauna e Flora” na Biologia!!!

\end{itemize}
ii) O que fazer depois de se formar??

Agora você deve estar pensando: "O que eu faço depois de formado?”... (se você
não estava pensando aposto que agora está)

Sim, as pessoas se formam nesse curso, acredite. O objetivo principal do
Bacharelado em Matemática é formar (!?) bons (?!) pesquisadores. Para quem não
sabe a matemática não está completa, isto é, sempre tem alguma coisa nova para
descobrir. Se você pensa que quem se forma nesse curso só pode ser
professor/pesquisador, você está muito enganado! O curso forma pessoas que
sabem analisar e resolver problemas metodicamente (você vai ver que está
pensando com mais clareza em breve). Com seu potente raciocínio lógico, um
bacharel em Matemática pode fazer Pós-Graduação em Engenharia (argh!),
Computação, Estatística (argh$^2$!), Física (argh$^3$!), Economia (argh$^5$!).
Ele pode trabalhar em vários locais: universidades, colégios, bancos,
empresas... Enfim, a vida se torna muito mais fácil se você é matemático. (E se
nem tudo der certo você pode vender pipoca na frente de algum teatro de São
Paulo.)

iii) Como sobreviver ao curso da Pura???

Bom, como nós ainda estamos cursando, não podemos dizer se vamos sobreviver ou
não, mas, de qualquer jeito, podemos dar umas dicas. Um meio para ser bem
sucedido é se apoiar em seus amigos: formando um grupo unido que esteja
disposto a enfrentar as matérias, línguas estrangeiras (de eventuais
professores), EPs (sim, você também faz EPs, se é que você sabe o que é isso),
provas, subs, recs, as mesmas matérias de novo todos juntos, o curso da pura
nem chega a ser tão doloroso e, na verdade, é até bem divertido. Claro que
formar esse grupo não é a coisa mais fácil, já que, quando você começar a
prestar atenção nos seus colegas de turma, vai achar eles bem estranhos, mas,
depois de um certo tempo, você percebe que eles são bem parecidos com você.

Talvez ja tenham te contado, mas esse curso pode ser fácil de entrar, mas
costumam formar-se uns 3 de nós por ano (e olhe la!). E foi no ano
passado, 2011, que a Pura bateu o seu recorde de formandos ao mesmo tempo,
foram 16! Isso não acontecia desde pelo menos a época do Jacy Monteiro (que
você ainda vai saber quem é)! E ainda tiveram uns três malucos que formaram em
três anos, mas isso, bixo, isso você não vai contar pra ninguem, nem pros seus
pais, que quando você tiver na metade do seu sétimo ano vão te perguntar pela
n$^16$-ésima vez por que você não se formou no mesmo tempo daqueles seus amigos.

v) O que precisa saber sobre a Pura???

Primeiramente, apesar de toda a dificuldade, a Pura tem uma carga horária
relativamente menor do que a maioria dos outros cursos... Teoricamente, é
possível se formar em 3 anos e meio, ou até menos. E existem pessoas que o
fazem (ou tentam pelo menos). Mas tome cuidado: Além de extremamente difícil (o
curso já é normalmente difícil, não queira torná-lo mais difícil ainda), você
corre o risco de não aprender nada e tirar notas bem mais baixas. Normalmente,
o tempo que você pode vir a ter a menos de aula precisará ser gasto estudando
por conta própria. Por isso, tome cuidado para não se sobrecarregar.

Tente tirar proveito da relativa flexibilidade da grade de horários: enquanto
que o primeiro ano você tem todas as aulas certinhas todo dia, com o passar do
curso você terá menos aulas (as quais tenderão a ficar mais difíceis), e sua
grade poderá ficar cheia de buracos. Não tenha medo do trancamento parcial,
quando você tiver medo de bombar alguma matéria, ou quando não se der bem com
um professor: em boa parte dos cursos vale mais a pena deixar determinada
matérias para depois do que fazer com algum professor com quem você não se dê
bem.

Acredite: a Pura só começa realmente no segundo ano. No primeiro, você terá
todas as matérias junto com outros cursos, como a estat, a aplicada e o BCC.
Aproveite para fazer contatos com as pessoas dos outros cursos, pois depois
disso a tendência é se distanciar deles. Só que por esse mesmo motivo, você
verá bastante coisa que provavelmente não usará no resto da Pura, além de que
no primeiro ano você não vai ter ainda uma boa noção do que será a pura. Você
terá uma idéia melhor do que é Matemática de verdade a partir de cursos
como Álgebra I e Análise Real.

Muito cuidado com o $5^{o}$ semestre, e o trio parada dura: Álgebra III,
topologia e Funções Analíticas.

Recentemente, houve alterações no currículo da Pura. Parabéns bixos! Vocês não
precisam mais fazer matérias chatas e que não tem nada a ver com a Pura, como
Laboratório de Física e Português. Em compensação, vocês terão que fazer mais
duas matérias que não eram obrigatórias: Geometria Diferencial II e Análise
Matemática II\footnote{Que mudou de nome para ''Análise Funcional'' em 2011,
mas os seus VETERANOS ainda insitirão em chamá-la de Análise Matemática II por
um bom tempo...}, além do que terão que fazer mais créditos de optativas livres
fora do IME.

Ah, além disso temos os sacrossantos conselhos que são passados há várias
gerações:
\begin{enumerate}
\item	Lembre-se sempre que você gosta de Matemática;
\item	Não tome um curso ruim como parâmetro de como é um determinado assunto;
\item	Lembre-se sempre que você gosta de Matemática;
\item	Persista e lute;
\item	Lembre-se sempre que você gosta de Matemática;
\item	Tome consciência de que você, na grande maioria das vezes, vai ter que
estudar muito;
\item	Lembre-se sempre que você gosta de Matemática;
\item	Informe-se sobre atividades extracurriculares como o programa de
Iniciação Científica (que é muito bom para formação, talvez até essencial) e
uma série de palestras com professores que, muito possivelmente, realizar-se-ão
durante o ano;
\item	Lembre-se sempre que você gosta de Matemática;
\item	Não desanime;
\item	Lembre-se sempre que você gosta de Matemática.

\end{enumerate}
Para terminar, faça amigos na Pura, só eles vão te entender. Qualquer dúvida,
você pode nos procurar. Estaremos sempre dispostos a ajudá-lo para, assim,
preservarmos a nossa espécie !!!

\end{subsecao}
