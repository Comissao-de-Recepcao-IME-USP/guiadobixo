\begin{secao}{Dicas}

Como todo bixo é perdido, aqui vão algumas dicas pra você não ficar perguntando
o tempo todo:

{\bf Biblioteca:} Para fazer a inscrição na Biblioteca, você precisa
apresentar 2 fotos 3x4, comprovante de residência e o cartão de matrícula. Não
deixe de assistir à palestra da Biblioteca na primeira semana de aula! Consulte
a programação.

{\bf Correios:} tem uma agência na Antiga Reitoria (que agora virou nova),
perto do Banco do Brasil.

{\bf Bancos e Caixas Eletrônicos:} Agências do Santander Banespa, Bradesco,
Nossa Caixa, Banco do Brasil, HSBC, Itau e Banco Real na Av. Prof. Luciano
Gualberto; Caixas eletrônicos perto do bandejão da Física, no bandejão da
Química, em frente à reitoria, no CEPE, enfim, em lugares aleatórios.

{\bf Farmácia:} Ao lado do bandejão Central, a farmácia funciona de segunda a
sexta, das 9h às 17h e no HU.

{\bf Posto de Gasolina:} na Avenida da Universidade, em frente à Faculdade de
Educação.

\begin{subsecao}{Museus pertencentes a USP}

{\bf Museu de Arqueologia e Etnologia (MAE):} ao lado da Prefeitura do Campus.

{\bf Museu de Arte Contemporânea (MAC):} próximo ao CRUSP.

{\bf Paço das Artes:} em frente à Faculdade de Educação.

{\bf Museu do Brinquedo:} fica na Faculdade de Educação, Bloco B.

{\bf Museu do Crime:} na Academia de Polícia perto da P1.

{\bf Museu do Instituto Oceanográfico:} adivinha?

{\bf Museu da Geociências:} Lá mesmo.

{\bf Instituto Butantan:} Próximo à História.

{\bf Museu da Faculdade de Veterinária} Perto do P3. Ponto final do circular.

{\bf Estação Ciência:} fica fora da USP. R. Guaicurus, 1274 - Lapa

{\bf Museu Paulista, vulgo Ipiranga: }também fora da USP, no Parque da
Independência - S/N  - Ipiranga.

{\bf Museu de Zoologia: }também fora da USP, na Av. Nazaré, 481  -
Ipiranga (atrás do Ipiranga).

\end{subsecao}

\begin{subsecao}{Onde beber?}

Se você é um bixo que curte entornar os canecos de vez em quando, então agora
deve estar pensando, até que enfim vamos falar de algo que presta! Lembre-se de
levar um Veterano para pagar a ele algumas doses, pois graças a eles que você
está recebendo essas dicas, então sem mais delongas, aqui vão alguns lugares
firmeza para se fazer isso à vontade:

{\bf Vet:} Na verdade não fica na veterinária que todos conhecem, é o C.A. da
Vet, um lugar paradisíaco, com breja razoavelmente barata, que ficou mais
famoso por volta de 2009, em vista das restrições alcoólicas que houveram na
USP. Seu local não é revelado por escrito, só no boca-a-boca, por isso vá com
um VETERANO experiente, principalmente para pagar cerveja para ele... É, essa
parte do guia foi preparado para isso, afinal não passamos meses das nossas
vidas fazendo esse guia para não receber nada de volta né???

{\bf Física:} Embora seja a Física, lá é um lugar gostoso para tomar vários
tipos de cerveja, que só é vendida após as 18:00 horas, e comer alguns salgados
baratos. No próprio C.A. deles se vende cerveja.

{\bf ECA:} Famosa Quinta i Breja, adivinha que dia da semana isso acontece?
Acertou bixo, toda Quinta-feira, parabéns, como prêmio você pode pagar uma
breja para o seu veterano favorito!

{\bf Rei das batidas:} Muito famoso não só por quem estuda na USP, o Rei,
como é carinhosamente chamado, fica fora da USP, saindo pela P1. Vende diversas
batidas e, é claro, cerveja...

{\bf Bar do frango:} Não se sabe qual é o verdadeiro nome desse bar, mas ele é
uma alternativa ao Rei, quando este se encontra muito lotado. Apesar do péssimo
atendimento e do aspecto horrível do lugar é bom para beber sem tumulto. É
frequentado principalmente no começo do ano. Se encontra atrás do Rei.

{\bf Morro da coruja:} Um morro perto da física, onde geralmente as pessoas se
reúnem para tomar as mais diversas bebidas e misturas alcoólicas possíveis.
Um ótimo lugar quando não está chovendo.

É claro que também há as festas, que ocorrem em qualquer lugar da USP, e nelas
há ainda outras misturas alcoólicas impossíveis.

É nosso dever informar também que o álcool é uma substância altamente viciante
e, quando bebida em excesso, pode trazer graves consequências à sua saúde, às
vezes à saúde de outra pessoa, à sua família e principalmente ao seu bolso.
Portanto, não se esqueça de abastecer os seus VETERANOS.

\quadrinhos{9} %FIXME passar essa imagem para preto e branco

\end{subsecao}
\end{secao}
