\begin{secao}{Editorial}

Bixo (opa, bixo é com letra minúscula), foi difícil chegar até aqui. Você está
meio ou completamente perdido. Temos apenas uma sugestão: aproveite esta etapa.
Faça da sua estadia na USP o melhor tempo da sua vida. Você verá que a USP tem
muitas e muitas coisas a oferecer. Não se preocupe apenas em estudar e passar de
ano, como você fez durante sua vida inteira; aproveite TUDO (você ainda vai
descobrir a definição de TUDO). Você pode não acreditar nisto agora, mas saiba
que viverá momentos inesquecíveis aqui no IME, alguns fantásticos e outros deploráveis. 

Este guia foi feito para você, bixo que já sabe ler (se não souber, apenas olhe
as figuras), possa aprender um pouquinho do que é a USP, o IME e a vida de
universitário que se inicia agora. Ele foi escrito numa forma descontraída e
fácil para que você consiga entender; mesmo assim, se pintar alguma dúvida,
você pode se dirigir a qualquer VETERANO, e sua dúvida será sanada (e, quem sabe,
talvez você também comece uma nova e forte amizade). Outra coisa: LEIA E DECORE
COMPLETAMENTE ESTE GUIA PARA NÃO PAGAR MICO. Pensando bem, você vai pagar mico
de qualquer jeito; ainda assim, seja um mínimo precavido e leia.

Lembre-se: este é seu último ano como bixo. Aproveite!

\vspace{\stretch{1}}
\rule{\textwidth}{0.5ex}\rule{2ex}{0.5ex}

\begin{small}
\begin{tabular}{|p{\textwidth}|}
\hline
\\[0.2pt]
{\large\bf Guia do bixo 2013} \\
Uma publicação da Comissão de Trote \\
\\
\makebox[4cm][l]{{\bf Editores}} Luiz, David e Godinez.\\
%
%\makebox[4cm][l]{{\bf Capa}} David e Wil-Kazuo.\\ %FIXME Arrumar após fazer a capa
%
%FIXME Colocar toda a lista de autores aqui e tirar das seções individualmente
%FIXME Arrumar a formatação do box
\makebox[4cm][l]{{\bf Textos}} André Verri (Deco), Antonieta, Fábio da Yumi, Gizela Fonseca,\\
\makebox[4cm][l]{{\bf       }} Lucas Cavalcanti, Marina Trindade, Mauricio Camilo,\\
\makebox[4cm][l]{{\bf       }} Paula Corradi, Pedrosa, Pedrão, Renata Aguemi,\\ 
\makebox[4cm][l]{{\bf       }} Ricardo Yasuda,  Yumi, David, Wil-Kazuo e autores dos textos\\ 
\makebox[4cm][l]{{\bf       }} dos guias anteriores não supracitados.\\
%
\makebox[4cm][l]{{\bf Layout}} btco (Guia 2007)                          \\
\makebox[4cm][l]{{\bf Revisão geral}} Dado a falta de tempo, bixo, a revisão é com você!\\
\makebox[4cm][l]{{\bf Agradecimentos:}} \\
%FIXME Colocar um tab aqui no início
Ao Donald Knuth (por inventar o \TeX\makebox{} e salvar-nos do Word na preparação
desse guia), ao btco pela iniciativa de começar este guia em \LaTeX\makebox{},
ao gimp, por nos ajudar a editar as figuras, aos bixos, por lerem o guia todo
e decorarem as músicas e os dez mandamentos. Ao museu de filatelia de São Jorge, %FIXME O que é esse museu aí?
por nos auxiliar na capa do guia e aos organizadores dos guias anteriores,
afinal nada se cria,
tudo se copia. Por último, à gráfica e às pessoas da Comissão que
perderam as férias para que este guia ficasse pronto. \\
\makebox[4cm][l]{{\bf       }}                                            \\
\hline
\end{tabular}
\end{small}

\pagebreak

\begin{subsecao}{Os Dez Mandamentos}
  \begin{enumerate}
  \item O VETERANO tem sempre razão;
  \item Na improvável hipótese de o bixo ter razão, entra imediatamente
        em vigor o primeiro mandamento;
  \item Em qualquer evento social, as despesas correm sempre por conta
        do bixo;
  \item O bixo tem o direito de permanecer calado (exceto quando interpelado
        por um VETERANO). Tudo o que disser pode e será usado contra ele;
  \item O bixo deve se apresentar imediatamente em caso de convocação por
        um VETERANO. Os desertores serão severamente punidos;
  \item Não são válidos no IME os direitos constitucionais do bixo à vida,
        liberdade e igualdade;
  \item O bixo deve estar pronto para assumir as seguintes funções para um
        VETERANO: cadeira, cinzeiro, moleque de recados, etc, quando as
        circunstâncias assim o exigirem; e também quando não o exigirem.
  \item O bixo deve amar respeitar e os seus VETERANOS acima de qualquer
        coisa;
  \item Para os casos não abrangidos por estas regras, a decisão final
        correrá por conta dos VETERANOS.
  \item Todo bixo é BURRO.
  \end{enumerate}

  
Como bixo, você tem todo o direito de reclamar sobre os mandamentos! Qualquer
reclamação deverá ser protocolada em três vias datadas, assinadas e autenticadas,
com firma reconhecida em cartório, e assim encaminhada à Comissão de Trote 2012
via mala direta. As reclamações serão incineradas e os reclamantes severamente
punidos. Obs.: alguns VETERANOS sugeriram que incinerássemos os reclamantes também.
A medida está em estudo, devido ao custo da operação e ao lixo tóxico produzido.


\end{subsecao}
\end{secao}
