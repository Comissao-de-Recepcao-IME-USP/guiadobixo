\begin{secao}{Olimpíadas de Conhecimento}

\begin{itemize}

\item{\bf Informática: }
{\em Renato (aka Totoro), Yan (aka Pikachu), Giovana Delfino} % refererencia, depois tirar

``Informática? Vocês mexem com Word e PowerPoint então?''

Responder essa pergunta já virou rotina para competidores da
Olimpíada Brasileira de Informática (OBI). Não, Informática 
não é Word. Oras, então o que é a OBI?

A OBI é uma competição de lógica e matemática. As provas são pequenos
problemas que você deve resolver com programas de computador.

Apesar de ser uma olimpíada que requeira um conhecimento mínimo de 
linguagem de computação, as provas em si geralmente não contém nada 
de teoria computacional mais avançada. 

Isso por que a OBI é voltada para alunos do Ensino Médio e recém
ingressantes. Quer dizer que você, bixo recém formado do Médio,
é nossa única esperança de ganhar medalhas e trazer glória ao IME!
Isso também quer dizer que essa é sua única chance de fazer a OBI,
uma competição relativamente tranquila comparada à Maratona de 
Programação.

Para participar, basta falar com o Professor Carlinhos (\url{http://www.ime.usp.br/~cef/}), 
ou com ex-competidores, amáveis veteranos e lindas criaturas:

-- Giovana Delfino

-- Renato ``Tororo" Geh

-- Yan ``Pikachu'' Couto

Para mais informações, acesse \url{http://olimpiada.ic.unicamp.br/}

\item{\bf Matemática: }
{\em Lucas Colucci, Giovana Delfino} % refererencia, depois tirar

Bom pessoal, se você entrou no IME, muito provavelmente já participou
de alguma Olimpíada de Matemática no Ensino Fundamental e/ou Médio. A 
boa notícia é que vocês vão poder continuar participando se quiserem,
e quem nunca participou tem a oportunidade de começar agora.

Mas por que participar? As Olimpíadas Universitárias de Matemática são uma
oportunidade de se divertir resolvendo problemas difíceis de Matemática e agregar
valor ao currículo ao mesmo tempo. Elas são parecidas com as Olimpíadas de
Ensino Médio, mas com conteúdo de Matemática da graduação (essencialmente 
Cálculo, Análise, Álgebra Linear, Álgebra, Combinatória e Teoria dos Números), 
mas com enfoque em problemas que exigem criatividade e técnicas menos standards,
muitas delas que você provavelmente não verá durante toda sua graduação.

De quais olimpíadas posso participar? Como aluno de graduação, você pode
participar da Olimpíada Iberoamericana de Matemática Universitária (OIMU),
Olimpíada Brasileira de Matemática (OBM) e Olimpíada Internacional de 
Matemática (IMC).

Como faço para me preparar? Os sites institucionais dessas olimpíadas
tem todo o material necessário para você que quer estudar e estar preparado
para elas.

Como faço para participar? Inscreva-se pelo site ou entre em contato com
o professor Yoshiharu. Para o IMC aconselha-se ter ganhado medalha na OBM,
já que é necessário apoio financeiro do IME por ser uma olimpíada internacional.

Mas eu tenho chance? Como foi o desempenho de IMEanos nelas? Nós obtivemos
sucesso nestas olimpíadas. Ganhamos medalhas em todas as três competições e
o resultado mais recente foi uma medalha de bronze na IMC e ouro na OBM.

Links institucionais:

\url{http://www.cimat.mx/Eventos/oimu/}

\url{http://www.imc-math.org/}

\url{http://www.obm.org.br/opencms/}

Se tiverem alguma dúvida, não hesite em contatar o famoso olimpíada Lucas Colucci Souza

\end{itemize}

\end{secao}
