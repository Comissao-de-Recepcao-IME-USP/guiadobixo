\begin{secao}{A Atlética}

  \begin{subsecao}{O que é a AAAMat?}

    É a Associação Atlética Acadêmica da Matemática. É formada por alunos do IME e tem como objetivo organizar e divulgar atividades esportivas e eventos (festas) para a comunidade imeana, visando o seu desenvolvimento físico e mental, e também a integração [$\int$] entre os alunos de diferentes cursos, de diversos anos e de outras faculdades. 

Dentre essas atividades há a IMEteria, a bateria dos alunos do nosso instituto, que acompanha as equipes imeanas em jogos, eventos, etc. 

  
  \end{subsecao}  
  
  Algumas das atividades da Atlética:

\pagebreak

  \begin{subsecao}{Atividades esportivas internas:}

A Atlética promove anualmente campeonatos internos com o objetivo de integrar ($\int$) os alunos do IME. Já foram promovidos campeonatos de xadrez, futsal, pebolim, sinuca e seletivas de natação e de truco.

Novos desafios estão vindo por aí! Idéias e sugestões são sempre bem vindas!


    \end{subsecao}
   
  A Atlética também representa o IME em diversos campeonatos universitários. São eles:

  \begin{subsecao}{Bixusp}


Como o próprio nome já diz, o Bixusp é um campeonato disputado entre a maioria das faculdades da USP em que apenas os bixos participam. O campeonato é disputado nos finais de semana do mês de março e conta com grande participação das torcidas incentivando seus times à vitória e já mostrando a você, bixo, a rivalidade histórica entre algumas faculdades.

O IME tem como tradição sempre revelar grandes talentos e grandes equipes: em 2005 o basquete masculino ganhou a primeira medalha de ouro em uma modalidade coletiva na história da participação imeana. Neste mesmo ano faturamos o Tênis de Mesa Feminino e ficamos em segundo lugar no Masculino. Em 2007, o Xadrez imeano, o Basquete Masculino, o Futsal Masculino e o Futebol de Campo conquistaram o terceiro lugar e fomos campões no Atletismo Masculino, garantindo, assim, o segundo lugar no Atletismo geral e quarto lugar na classificação geral. Em 2008, ficamos em segundo lugar no Futsal Masculino. Em 2009, o Atletismo foi campeão, o Tênis de Mesa Masculino foi vice. Em 2010, o Tênis de Mesa Masculino e o Atletismo Feminino foram vice.

O importante no Bixusp não é saber jogar, mas ter vontade de participar. Portanto, se você acha que não joga muito bem, não tem problema. Se por acaso você tem alergia a esporte, compareça e torça pelos nossos times. O Bixusp é uma ótima maneira de começar a conhecer melhor os VETERANOS e bixos (e bixetes) do IME e de outras faculdades. Fique atento aos treinos especiais para o Bixusp. Os dias e horários dos treinos e jogos serão sempre informados através do site e do mural da Atlética, localizado na entrada do bloco B.

  \end{subsecao}

  \begin{subsecao}{Copa USP e Jogos da Liga}

São os campeonatos internos da USP. A maioria das faculdades da USP participa destas competições. A Copa USP ocorre no primeiro semestre e os Jogos da Liga, no segundo semestre. Os jogos ocorrem sempre aos finais de semana.

A Copa USP é um dos mais tradicionais campeonatos da USP, mantendo uma rivalidade muito grande entre algumas faculdades. Os times são separados em duas divisões de acordo com as colocações nos anos anteriores. O IME, embora tenha poucos atletas, teve a maioria de seus times disputando a Série Azul, divisão mais forte, e muitos deles alcançando resultados expressivos.

O Handebol Masculino foi vice-campeão dos Jogos da Liga em 2006 e campeão da Copa USP em 2009. O Basquete Masculino se destacou em 2004 e em 2005, sendo campeão tanto da Copa USP como dos Jogos da Liga nos dois anos seguidos e ficou em terceiro lugar em 2006, assim como as meninas do Basquete. O Futebol de Campo e o Xadrez ficaram em primeiro lugar na Copa USP em 2006. O Vôlei Masculino foi quarto colocado da Copa USP Série Azul em 2007 e campeão da Copa USP em 2010, além de vice nos Jogos da Liga em 2010. O Futebol Masculino chegou a terceiro nos Jogos da Liga em 2010.

Em 2008, tivemos excelentes resultados nos Jogos da Liga. Ficamos entre os primeiros lugares em praticamente todas as modalidades, nos colocando entre as maiores atléticas da USP.

Como dissemos, o IME é famoso por sua inflamada torcida, que muitas vezes já ajudou nossas equipes nos momentos mais difíceis. Portanto, se você pratica esporte terá muitas opções e caso você não jogue nada, junte-se à nossa torcida!

  \end{subsecao}

%Tirado na edição do Guia 2009 pra 2012---------------------------------------
%  \begin{subsecao}{Intercomp}

%O Intercomp fez parte do nosso passado glorioso. Em 1997, 1998, 2001, 2002 e 2003 fomos campeões e em 1999, 2000, 2004 e 2005 fomos vice. Em 2006 o IME não participou por problemas técnicos. Em 2007 tivemos uma participação um tanto quanto frustrante, cultivando muitos vices-campeonatos como o Futebol de Campo, Futsal Masculino, Basquete Masculino, Vôlei Masculino, Natação Masculina, Handebol Feminino e no Xadrez.

%Ao lado da Unicamp, fomos uma das faculdades que mais venceu o Intercomp, faturando 8 torneios no total de 18 disputados. Em 2008, ficamos com o caneco.

%\end{subsecao}

\begin{subsecao}{BIFE}

O BIFE é uma competição contendo apenas faculdades da USP. Essas competições esportivas acontecem em cidades do interior ou litoral, onde ficamos 4 dias alojados pra cochilos de poucas horas, com jogos durante o dia e festas durante a noite e a madrugada.

Em 2011 as faculdades presentes foram: BIO, IME, FAU, ECA, FFLCH, Veterinária, Geologia, Física, Psicologia, Química, FOFITO e RI. As letras são as iniciais dos quatro fundadores: BIO, IME, FAU e ECA.

É uma competição que terá sua décima terceira edição em 2012 e que vem crescendo a cada ano, com um grande número de alunos comparecendo.

Temos um histórico muito bom no BIFE. Em 2010, ganhamos o Atletismo, ficamos em terceiro lugar em 2005, com o vice-campeonato em 2002 e 2004 e fomos campeões em 2000, 2001, 2003, 2006, 2007, 2008, 2009, 2010 e 2011! Sim, somos os atuais campeões do Bife e você, bixo, terá o compromisso de ajudar o IME a vencer em 2012 também! 


\end{subsecao}

Outras atividades:

\begin{subsecao}{Vendas}

Além desse monte de atividades citadas acima, a Atlética também faz e vende adesivos, canecas, chaveiros, camisetas e agasalhos do IME. Não deixe de visitar a Atlética e conhecer os nossos produtos.

ATENÇÃO: fique esperto e compre logo o seu agasalho para não ficar sem!

  \end{subsecao}

\begin{subsecao}{Festas}

A Atlética e o CAMAT já promoveram muitas festas e Happy Hours. O sucesso dessas festas depende em grande parte da participação dos imeanos. Convide seus amigos e participe das festas, uma outra maneira de conhecer e se integrar ($\int$) com seus colegas bixos e VETERANOS.

Uma das nossas festas é a Melhores do Ano, onde os melhores atletas do ano anterior são premiados. E o melhor: é open bar! A festa de 2011 foi muito boa, e com certeza esse ano será melhor ainda! Contamos com sua presença para prestigiar os VETERANOS e se divertir na festa.

\end{subsecao}

\begin{subsecao}{Como falar com a Atlética?}


Sempre que tiver alguma dúvida, reclamação ou sugestão você pode falar pessoalmente com qualquer membro da Atlética ou ir até a sala da Atlética, sala B-18, dentro da Vivência. Pergunte por aí, veja no mural, ligue no telefone: 3091-6378 ou mande um e-mail para atletica@ime.usp.br.

Visite também nosso site na Internet: www.ime.usp.br/~atletica. 

A Atlética inicia 2012 esperando idéias e sugestões que serão muito bem recebidas. Não deixe de colaborar e participar dos eventos por ela promovidos. A Atlética agradece!!!

\end{subsecao}
\end{secao}
