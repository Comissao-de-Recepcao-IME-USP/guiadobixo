\begin{secao}{A Atlética}

\begin{subsecao}{O que é a AAAMat?}

É a Associação Atlética Acadêmica da Matemática. É formada por alunos do IME e
tem como objetivo organizar e divulgar atividades esportivas e eventos (festas)
para a comunidade imeana, visando o seu desenvolvimento físico e mental,
e também a integração [$\int$] entre os alunos de diferentes cursos, de diversos
anos e de outras faculdades.

Dentre essas atividades há a IMEteria, a bateria dos alunos do nosso instituto,
que acompanha as equipes imeanas em jogos, eventos, etc.

\end{subsecao}

Algumas das atividades da Atlética:

\begin{subsecao}{Atividades esportivas internas:}

A AAAMat possui responsáveis para organizar treinos e campeonatos das seguintes 
modalidades: futebol de campo, futsal, basquete, vôlei, handebol, atletismo, 
natação, tênis de mesa, tênis de campo, xadrez, judô, beisebol e softbol. Para 
que os treinos continuem ocorrendo precisamos sempre da presença de novos 
atletas. Então, se você, bixo, está interessado em aprender qualquer um desses 
esportes ou se já está familiarizado com algum, avise-nos. Venha treinar com os 
nossos times e competir com eles.

Os dias e horários dos treinos e jogos serão sempre informados através do site
e do mural da Atlética, localizado na entrada do bloco B.

Além disso, a Atlética promove anualmente campeonatos internos com o objetivo de
integrar ($\int$) os alunos do IME. Já foram promovidos campeonatos de xadrez,
futsal, truco, Winning Eleven, sinuca e seletivas de natação, tênis de mesa e 
campo.

Idéias e sugestões sobre novas modalidades, campeonatos, inters, etc. são sempre
 bem-vindos! Converse com a gente!

\end{subsecao}

A AAAMat também representa o IME em diversos campeonatos universitários. São
eles:

\begin{subsecao}{Bixusp}

Como o próprio nome já diz, o Bixusp é um campeonato disputado entre a maioria
das faculdades da USP em que apenas os bixos participam. O campeonato é
disputado nos finais de semana do mês de março e conta com grande participação
das torcidas incentivando seus times à vitória e já mostrando a você, bixo, a
rivalidade histórica entre algumas faculdades.

O IME tem como tradição sempre revelar grandes talentos e grandes equipes: em
2005 o basquete masculino ganhou a primeira medalha de ouro em uma modalidade
coletiva na história da participação imeana. Neste mesmo ano faturamos o Tênis
de Mesa Feminino e ficamos em segundo lugar no Masculino. Em 2007, o Xadrez, o 
Basquete Masculino, o Futsal Masculino e o Futebol de Campo conquistaram o 
terceiro lugar e fomos campões no Atletismo Masculino, garantindo, assim, o 
segundo lugar no Atletismo geral e quarto lugar na classificação geral. Em 2008,
ficamos em segundo lugar no Futsal Masculino. Em 2009, o Atletismo foi campeão, 
o Tênis de Mesa Masculino foi vice. Em 2010, o Tênis de Mesa Masculino e o 
Atletismo Feminino foram vice. Em 2011, o Tênis de Campo Feminino foi campeão. 
Ano passado, o Basquete Feminino foi campeão.

O importante no Bixusp não é saber jogar, mas ter vontade de participar.
Portanto, se você acha que não joga muito bem, não tem problema. Se por acaso
você tem alergia a esporte, compareça e torça pelos nossos times. O Bixusp é uma
ótima maneira de começar a conhecer melhor os VETERANOS e bixos (e bixetes) do
IME e de outras faculdades. Fique atento aos treinos especiais para o Bixusp, 
visitando o site e observando o mural da Atlética.

\end{subsecao}
\begin{subsecao}{Copa USP e Jogos da Liga}

São os campeonatos internos da USP. A maioria das faculdades da USP participa
destas competições. A Copa USP ocorre no primeiro semestre e os Jogos da Liga,
no segundo. Os jogos ocorrem sempre aos finais de semana.

A Copa USP é um dos mais tradicionais campeonatos da USP, mantendo uma
rivalidade muito grande entre algumas faculdades. Os times são separados em
duas divisões de acordo com as colocações nos anos anteriores. O IME, embora
tenha poucos atletas, teve a maioria de seus times disputando a Série Azul,
divisão mais forte, e muitos deles alcançando resultados expressivos.

O Handebol Masculino foi vice-campeão dos Jogos da Liga em 2006 e campeão da
Copa USP em 2009. O Basquete Masculino se destacou em 2004 e em 2005, sendo
campeão tanto da Copa USP como dos Jogos da Liga nos dois anos seguidos e ficou
em terceiro lugar em 2006, assim como as meninas do Basquete. O Futebol de
Campo e o Xadrez ficaram em primeiro lugar na Copa USP em 2006. O Vôlei
Masculino foi quarto colocado da Copa USP Série Azul em 2007 e campeão da Copa
USP em 2010, além de vice nos Jogos da Liga em 2010. O Futebol Masculino chegou
a terceiro nos Jogos da Liga em 2010.

Em 2008, tivemos excelentes resultados nos Jogos da Liga. Ficamos entre os
primeiros lugares em praticamente todas as modalidades, nos colocando entre as
maiores atléticas da USP.

Como dissemos, o IME é famoso por sua inflamada torcida, que muitas vezes já
ajudou nossas equipes nos momentos mais difíceis. Portanto, se você pratica
esporte terá muitas opções e caso você não jogue nada, junte-se à nossa torcida!

Neste ano, além das modalidades tradicionais que o IME treina, haverá na Copa 
USP: pólo aquático, rugby, judô, karatê, badminton, jiu-jitsu e tae-kwon-do. Se 
você tem noção de qualquer um desses esportes avise a Atlética, participe e 
represente o IME.

\end{subsecao}
\begin{subsecao}{BIFE}

O BIFE é a competição mais tradicional do IME, onde você verá mais de 170 
imeanos confraternizando, jogando, torcendo e festejando juntos.

Trata-se de um campeonato contendo apenas faculdades da USP. Essas competições
esportivas acontecem em cidades do interior ou litoral, onde ficamos 4 dias
alojados pra cochilos de poucas horas, com diversos jogos durante o dia e ótimas
festas durante a noite e a madrugada.

Em 2012 as faculdades presentes foram: BIO, IME, FAU, ECA, FFLCH, Veterinária,
Geologia, Física, Psicologia, Química e FOFITO. As letras são as iniciais
dos quatro fundadores: BIO, IME, FAU e ECA.

É uma competição que terá sua décima quarta edição em 2013 e que vem
crescendo a cada ano, com um grande número de alunos comparecendo.

Temos um histórico muito bom no BIFE:
%FIXME dar um jeito de deixar essa tabela na mesma página que a linha de cima
\begin{center}
	\begin{tabular}{c|c|c}
	  Ano & Cidade & Campeão\\
	  \hline
	  1999 & Jacareí & IME\\
	  2000 & Não Houve & - \\
	  2001 & Serra Negra & IME\\
	  2002 & Socorro & ECA\\
	  2003 & São Sebastião & IME\\
	  2004 & Cruzeiro & FFLCH\\
	  2005 & Jacareí & FFLCH\\
	  2006 & Lorena & IME\\
	  2007 & Piedade & IME\\
	  2008 & Itapeva & IME\\
	  2009 & Cruzeiro & IME\\
	  2010 & Barra Bonita & IME\\
	  2011 & Casa Branca & IME\\
	  2012 & Barra Bonita & IME\\
	  2013 & ????????? & ???
	\end{tabular}
\end{center}

Sim, somos decacampeões do BIFE. Sete vezes consecutivas nos últimos anos.

E você, bixo, terá o compromisso de ajudar o IME a conquistar o décimo primeiro 
título em 2013 e poder gritar junto com a maioria imeana:

\textbf{``EU NUNCA PERDI UM BIFE!!!''}.

\end{subsecao}
\begin{subsecao}{IntegraMIX}

Ano passado, o IME participou dessa grande competição entre as faculdades: EACH,
 ESPM, FEA-PUC, FAAP, Med Paulista, Med ABC e UNESP Bauru.

Foram quatro dias sensacionais com direito a baladas e tendas open-bar, muitos 
jogos e diversão.

Converse com um VETERANO que participou e saiba tudo sobre o melhor inter que 
ele já participou na vida.

\end{subsecao}

Outras atividades:

\begin{subsecao}{Vendas}

Além desse monte de atividades citadas acima, a Atlética também vende
adesivos, calculadoras, canecas, chaveiros, bonés, camisetas e agasalhos do IME,
 entre outros. Não deixe de visitá-la e conhecer nossos produtos.

ATENÇÃO: compre logo sua camiseta 2013 e reserve seu agasalho, para não ficar 
sem!

\end{subsecao}
\begin{subsecao}{Festas}

A Atlética e o CAMat já promoveram muitas festas e Happy Hours. O sucesso
deles depende em grande parte da participação dos imeanos. Convide seus
amigos e participe das festas, uma outra maneira de conhecer e se
integrar ($\int$) com seus colegas bixos e VETERANOS.

Promovemos as festas da IMEteria, ForrIME, Rompendo o hIMEn, Melhores do Ano, 
entre outras e auxiliamos as festas pré-BIFE (Desmame, Engorda e Abate). Todas 
imperdíveis.

Contamos com sua presença para prestigiar os VETERANOS e se divertir muito.

\end{subsecao}
\begin{subsecao}{Como falar com a Atlética?}

Sempre que tiver alguma dúvida, reclamação ou sugestão você pode falar
pessoalmente com qualquer membro da Atlética ou ir até a sala da Atlética, sala
B-18, dentro da Vivência. Pergunte por aí, veja no mural, ligue para o
telefone: 3091-6378 ou mande um e-mail para {\tt atletica@ime.usp.br}.
Saiba todas as novidades nos seguintes endereços eletrônicos:

Site da Atlética: {\tt https://www.ime.usp.br/$\sim$atletica.}

Curta nossa página no Facebook: {\tt https://www.facebook.com/aaamat.ime}

Siga-nos no Twitter: {\tt @aaamat}

A Atlética inicia 2013 esperando idéias e sugestões que serão muito bem
recebidas. Não deixe de colaborar e participar dos eventos por ela promovidos.

Você é SEMPRE bem-vindo em nossa sala, atividades e eventos!

\end{subsecao}
\end{secao}
