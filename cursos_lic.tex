\begin{subsecao}{Licenciatura}
{\em Pedrosa, Cartola e outros}

Olá, bixo! Se você chegou até aqui então parabéns!

Não só porque passou na FUVEST mas também porque entrou em Licenciatura em
Matemática, mesmo sendo chamado pelos seus colegas de doido, louco entre
outros simpáticos adjetivos.

Se você ainda não sabe exatamente o que você fará com o seu curso, tentaremos
te explicar, mas esperamos mesmo que você tenha em mente uma coisa: Você será
Professor (a), aquele que tem o dom de sanar as dúvidas dos outros, então
aprenda o suficiente para isso. E como fazer isso? Temos algumas sugestões:

Primeiramente, não caia na conversa de seus VETERANOS e colegas bacharelandos
que insistem em dizer que o curso de licenciatura é mais fácil que o deles. São
cursos diferentes:

Um bacharel é um pesquisador. Portanto, usa a Matemática explorando seus
problemas em aberto na esperança de solucionar algum deles e, consequentemente,
criar outros mais.

Já um licenciado é um professor. Apto a lecionar na Escola Básica e com
competências para fazer o aluno compreender esse universo tão mágico que é a
Matemática. Se você chegou até aqui com a vontade de ser um professor (a) então
provavelmente teve bons professores de matemática. Inspire-se neles, supere-os.
Aqui você tem a condição ideal para tanto. Somente através de você o mundo
poderá ver que matemática também é legal. Ainda mais aquela aprendida na
escola, pois a parte difícil fica para ser aprofundada na faculdade, e é o que
você estará fazendo nesses n anos que se seguirão.

Você terá uma base de vários ramos da matemática: Geometrias,
Cálculos (importante: não bombe neles ou seu curso vai demorar mais para ser
concluído!), Estatísticas, Álgebra, Computação entre outros. Com o decorrer do
curso, você descobrirá qual área acadêmica você prefere fazer as disciplinas de
aprofundamento, onde você deverá escolher que matérias você quer se
especializar. Tanto pode ser na área de física (para você se tornar um
professor de física também!), quanto educação, estatística, álgebra,
computação, matemática aplicada em saúde animal e o que mais a sua
imaginação (e o Júpiter) permitir. Como pode ver, esse curso é um “coringa” se
comparado aos outros.

Além disso, sua formação também abrangerá questões como: o contexto social do
aluno, preparação para sala de aula, psicologia da educação e diversas
metodologias de ensino. Para isso, você fará disciplinas na Faculdade de
Educação a qual lhe preparará melhor nesse contexto (ou pelo menos deveria. É,
vá se acostumando...).

Com a nova reforma do MEC para as licenciaturas, implantada na USP em 2006,
você também fará mais atividades acadêmicas científicas e culturais, que são:
projetos de iniciação científica, oficinas e cursos de aperfeiçoamento,
participação em eventos e outras ações que enriqueçam a sua formação
profissional e pessoal. Fique esperto: você terá que correr atrás de tudo isso
sozinho. Esteja atento com os prazos de entrega dos relatórios de cada
semestre. São 200 horas para cumprir! Mas veja pelo lado bom, várias dessas
atividades são prazerosas!

Como pode ver, o curso lhe dá um leque bem amplo de escolhas que podem
transformá-lo em um excelente professor, basta você querer. Portanto, bixo, aja!

\begin{subsubsecao}{Dicas da cartola!}

Agora algumas dicas tiradas da cartola:

Você pode fazer diversas coisas acadêmicas e muitas outras não acadêmicas e
consequentemente mais divertidas, porém tudo tem um preço.
\begin{enumerate}
\item	Podemos passar o ano todo só participando de festas e levar o curso nas
coxas, o que será bem divertido e estenderá o tempo que você ficará na
faculdade, mas cuidado, tudo tem um limite, e jubilar, apesar dessa palavra vir
de júbilo, nesse caso não é uma boa coisa!
\item	Podemos passar o ano todo na Biblioteca estudando até rachar, ser o nerd
da turma (ei, vê se passa cola viu!) e com isso diminuindo o tempo de
faculdade. Você será um bom candidato a RD, já pensou nisso? Isso gera coisas
boas com relação a bolsas e empregos, então também vale a pena, mas não vá se
esquecer de fazer amizades, pois é a única coisa realmente importante.
\item	O tão difícil meio-termo. É um ideal difícil de ser conquistado, afinal
quem já viu um nerd em todas as baladas, ou o baladeiro de plantão que só
tira 10? Aliás, vá se acostumando, pois o 10 aqui no IME é virtual... você vai
entender isso mais cedo ou mais tarde! Bom, se tudo der certo você vai tirar
boas notas (leia-se algo entre 5 até 7), ser mais conhecido/chegado dos
professores por se formar de um a três anos a mais que o normal e ainda vai
participar das melhores baladas!! Se isso não é bom então vou voltar a fazer
minhas listas de Cálculo...
\item	Passe em Cálculo, se tenho algo que presta para te dizer é isso, passe em
cálculo, bombar aqui vai te atrapalhar muito! Claro que tem outras matérias
muito importantes para passar também, mas essa é pré-esquisito para muitas
coisas. Faça uma lista das coisas que tem pré-esquisito para cursar e dê
prioridade a elas.
\item	Faça amigos, são eles que vão te ajudar a prosseguir. Muitas vezes
pensamos em desistir, e os amigos são aqueles que em último caso nos arrastam,
literalmente, para o caminho certo!

\end{enumerate}

\end{subsubsecao}
\quadrinhos{3}

\end{subsecao}
