\begin{secao}{Glossário}

Esta parte é, sem dúvida, uma das mais importantes do guia. Aqui você
encontrará todas as explicações para as maiores dúvidas do universo. Com
certeza, após ler este trecho sua vida vai mudar: você saberá, por exemplo,
porque o céu é azul e com quantos paus se faz uma canoa.

\begin{subsecao}{Sobre o IME}
{\bf Armários:} Ficam na sala 18 do bloco B (sala da Vivência). Além de
servirem para aliviar imeanos do excesso de peso dos livros e cadernos, ajudam
o CAMAT a sobreviver financeiramente.

{\bf Banheiro:} Temos vários e de três tipos: para Homens, para Mulheres e os
matinhos para os bixos. Recentemente foram reformados os três tipos, então você
bixo, poderá usufruir de um matinho totalmente novo!

{\bf Bichinho esquisito:} Aquele negócio que está nas camisetas, agasalhos e
bonés do IME. Alguns dizem que é o mascote da Atlética; alguns o chamam de
Fluffy; ninguém sabe dizer o que ele realmente é, mas apareceu depois que
surgiram aquelas bolinhas, estilo porco-espinho, feitas com elásticos
coloridos (você provavelmente deve lembrar; foi mais ou menos na época do seu
primário). Primo do Cariboo.

{\bf CEC:} É a sigla do Centro de Ensino de Computação. É um dos lugares onde o
bixo pode fazer (ou pelo menos tentar) seus EP's e onde são ministrados cursos
de ``Computação Instrumental'' para alunos de toda USP (e até de fora). Ideal
para os que não sabem nem dar login na Linux.

{\bf GRECIME:} É o grêmio dos funcionários. Portanto quando ouvir falar em
grêmio não pense CAMat. Não tem nada a ver. Fica em frente às máquinas de
salgadinhos (se elas voltarem; se não, fica ao lado da Gráfica). De vez em
quando, até 16h ou 20h (varia numa função quadraticamente proporcional ao
bom-humor do vendedor e ao próprio vendedor, que não é único), tem bolos
maravilhosos de dar água na boca. Outros dias são biscoitos excelentes para se
mastigar enquanto se estuda, ou salgados para almoçar quando você tem EP para
entregar. Claro que se você que é do noturno e trabalha de dia pode pular esse
tópico.

{\bf IMEJr:} Empresa administrada pelos alunos do IME.

{\bf RD:} Representante Discente. É aquele aluno que representará você nas
comissões do IME, Comissões de Curso, Comissão de Graduação, entre outras.
Nestas comissões serão tomadas decisões que irão influenciar a sua vida
acadêmica. Então a presença de um aluno nelas é imprescindível, pois nada
melhor que um aluno para saber das necessidades dos alunos.

{\bf Sala das ET's:} É uma sala no bloco A onde estão as Work
Stations (Estações de Trabalho), usadas pelos professores, alunos de
pós-graduação e de iniciação científica.

{\bf Sala de Vivência:} É a sala 18 do bloco B, a sala mais importante de todas
do IME e onde os alunos aproveitam para jogar cartas, sinuca (Formiga está
sempre disposto a perder uma lá), pebolim (Cartola, Jerônimo, Adalberto - vocês
ainda vão perder para eles), xadrez (Pedrosa), fliperama, Pokémon, Magic, ouvir
música, assistir TV, dormir no sofá, e o que mais der na telha de quem quiser
relaxar. Você encontrará maus elementos que irão lhe levar ao submundo das
drogas, tentando lhe ensinar a jogar king ou bridge... seja careta, caia fora
logo e assuma que você gosta de TRUCO.

{\bf Seção de Alunos:} Aquele lugar onde você faz matrícula (aliás, você deve ir
confirmar a matrícula logo, bixo!), declaração, requerimentos, etc. (assuntos
burocráticos). Vá até o saguão à esquerda do Bloco B (lá onde fica o CEC) entre
na fila. O horário de funcionamento é meio restrito, então não deixe as coisas
pra última hora.

{\bf Xerox:} Não ficou pronta até a edição final desse guia. Provavelmente a
xerox será entre a Vivência e o Grecime, entrado no Bloco B à
direita (aproveite essa informação, bixo, é de primeira mão).

\end{subsecao}

\begin{subsecao}{Sobre as matérias}

{\bf Teorema:} Um teorema é uma afirmação que pode ser provada. Provar
teoremas é a principal atividade dos matemáticos. Deles surgem Lemas,
Corolários, Proposições, e tantas outras coisas que você só vai entender
completamente o que significam quando precisar escrever sobre eles, o que vai
acontecer logo logo!

{\bf Iniciação Científica:} grupo de alunos, coordenado por um professor, que
estuda um determinado assunto, paralelamente ao curso. No caso de alunos que
queiram bolsas de estudo, é adotado um plano de estudos mais rigoroso.

{\bf SUB:} prova que você faz quando vai mal em alguma outra avaliação, ou
quando você simplesmente não vai. Sua aplicação e utilidade depende do
professor ministrante

{\bf REC:} prova que você faz quando vai mal na SUB.

{\bf DP:} matéria que você faz quando vai mal na REC.
\end{subsecao}

\begin{subsecao}{Sobre programas}

{\bf Computador:} objeto com vontade própria, sensível, que requer muito
carinho e atenção. Normalmente comparado às mulheres, com duas pequenas
diferenças: ele faz direito o que você pede e neles podemos fazer upgrades
quando quisermos.

{\bf EP:} Exercício-Programa. Algo que você vai ter que fazer muitas vezes, e
vai dar muito trabalho.

{\bf GCC:} compilador mais recomendado para seus EP's, por suas inúmeras
qualidades. Atenção: ele ainda fará você se sentir burro.

{\bf Hello World:} Um clássico da programação universal.

{\bf Segmentation Fault:} Efeito computacional aleatório causado pela ``véspera
de entrega de EP''. Desenvolvido por Murphy.

{\bf Stack Overflow:} mensagem que aparece na tela do computador (Windows)
quando ele se recusa a funcionar. Isso ocorre quando ele está magoado, cansado,
ou simplesmente está ``naqueles dias''.

{\bf Teorema Fundamental do EP:} ``O EP só funcionará no dia da entrega.'' Não
confunda com o Corolário 42 da Lei de Murphy: ``O EP só {\bf não} funcionará no
dia da entrega!''

{\bf Linux:} Sistema operacional criado totalmente em linguagem C, graças a um
esforço mundial de milhares de programadores e experts em informática, composto
por aproximadamente 7 mil arquivos e 5 milhões de linhas, e com o qual você não
tem capacidade para trabalhar.

{\bf Windows:} Vírus. Porém tão bem mascarado que parece até a coisa correta a
se usar.
\end{subsecao}

\begin{subsecao}{Sobre a fauna da USP}

{\bf politécnico:} Podemos defini-los através de seu próprio hino: ``Mulher,
mulher pra quê / eu quero a HP; Mulher, mulher que nada / eu quero a derivada;
Mulher, mulher faz mal / eu quero a integral...''

{\bf VETERANO:} O ser supremo. O senhor de sua vida imeana. Nunca responda,
desrespeite, agrida, afronte ou encare um VETERANO. Se um VETERANO lhe dirigir
a palavra agradeça ao seu Deus, pois você foi abençoado. Com todas as letras
maiúsculas e sempre em fontes TrueType.

{\bf Monitor:} VETERANO que é mal pago pelo instituto para tirar dúvidas e
corrigir listas de exercícios e/ou EP's de uma determinada matéria.

{\bf Formado:} VETERANO + diploma

{\bf Mestre:} VETERANO + diploma + dissertação

{\bf Doutor:} VETERANO + diploma + dissertação + experiência + tese

{\bf Professor:} Aquele que sabe muito, só não conta pra você. A maior parte
não fala português; alguns apenas emitem ruídos estranhos.

(N.do E.): Como vocês podem perceber, o VETERANO está em todas. Portanto, você
deve se esforçar ao máximo para se tornar um de nós (esperamos realmente que
você consiga, pois não vai ser fácil)

{\bf bixo:} O ser mais inferior da face da Terra. Para encontrar um basta olhar
no espelho. Sempre em minúsculas.
\end{subsecao}

\begin{subsecao}{Sobre a USP}

{\bf Bandejão:} Local de torturas diárias. Ali você irá receber a sua dose de
substâncias estranhas. Se você criar o hábito de comer lá todos os dias, quando
houver guerra biológica ou radioativa você estará imune.

{\bf CEPE:} Centro de Práticas Esportivas - lugar onde você poderá praticar
todos os esportes que quiser.

{\bf Circular:} ônibus interno da USP. Pegue um (se passar) e conheça a
universidade toda. Dizemos TODA, porque esse ônibus dá voltas incríveis. Por
outro lado, é incrivelmente mais barato que os ônibus da prefeitura, diriamos
até que é o preço que vale andar com tal opção (ou falta dela).

{\bf COSEAS:} ao lado da praça do relógio. É onde os alunos fazem a carteirinha
de passes de ônibus, EMTU e metrô, além de solicitar os auxílios eteceteras que
estão explicados na seção respectiva.

{\bf CRUSP:} Conjunto residencial da USP. Se você se inscrever, torça para
pegar um apartamento num dos blocos já reformados, ou então torça para não
conseguir nenhum.

{\bf Colméia:} conjunto de favos.

{\bf Favos:} um monte de prédios hexagonais encravados no meio do CRUSP.

{\bf CINUSP:} Cinema da USP localizado no favo 4 da Colméia. Toda semana ele
passa um filme de qualidade. Informe-se sobre a programação em {\tt www.usp.br/cinusp}

{\bf Pelletron:} prédio da Física que na realidade é um acelerador de
partículas. Nada de ter idéias mirabolantes, Joselito...

{\bf PUTUSP:} Se você, bixo, quiser fazer Iniciação (não científica), vá até a
Avenida Valdemar Ferreira (saída principal da USP). Lá estarão os(as)
instrutores(as) dispostos a iniciá-lo 24 horas por dia.

{\bf Vet (Veterinária):} Para onde são mandados os bixos que se machucaram no
trote.

{\bf H.U. (Hospital Universitário):} para onde são mandados os bixos que não
foram aceitos na Vet. Lá são realizados os tratamentos e experiências com
exposição a radiação, exposição a aspirantes a médicos, teste de
paciência/resistência a dor assim que pega a senha, alto nível de gesso no
estômago, queda espontânea (ou não) de cabelo etc.

{\bf Psico:} Para onde mandamos os bixos que pensam que são gente.

\end{subsecao}
\end{secao}
