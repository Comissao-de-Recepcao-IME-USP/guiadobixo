\begin{secao}{(in)Utilidades}

\begin{subsecao}{Na WEB}

{\tt www.usp.br} - Página da USP. Aqui você encontrará notícias e eventos da
universidade, bem como informações gerais.

{\tt www.ime.usp.br} - Página do IME.
Nesse link, você poderá ver detalhes sobre os cursos e obter informações sobre
a faculdade.

{\tt www.linux.ime.usp.br} - Página da Rede Linux do IME. A partir dessa página,
você poderá aprender a explorar alguns dos recursos existentes na rede Linux,
através do FAQ, além de poder acessar seus e-mails da rede Linux pelo Webmail.

{\tt camat.ime.usp.br} - Página do CAMAT. O site foi remodelado no começo de
2012, não deixem de dar uma olhada. Lá você ficará por dentro não só de eventos,
debates e festas, mas também terá acesso a um banco de provas de anos anteriores,
que os seus VETERANOS gentil e coletivamente construíram.

{\tt www.ime.usp.br/$\sim$atletica} - Página da Atlética. Quer acompanhar os
jogos do IME nos campeonatos? Então visite o nosso site e fique ligado em tudo
o que está rolando (literalmente), além de se informar sobre as modalidades que
existem no IME.

{\tt uspdigital.usp.br/jupiterweb} - Sistema JúpiterWeb. Aqui você vai
encontrar a sua grade horária e, mais tarde, você poderá fazer as matrículas nas
matérias que irá cursar no semestre, além de ter acesso às suas notas.

{\tt paca.ime.usp.br} - É bixo... acha que vai ser essa moleza pra sempre? Se
você acha, está muito enganado! Daqui a pouco você vai receber uma senha para
poder enviar seus EPs (vide glossário) nesse endereço... (e não adianta fazer
chantagem emocional que o Paca só vai aceitar até 23h55... não entendeu? você
vai entender..).

{\tt www.parperfeito.com.br} - Tá precisando, hein, bixo!?

{\tt www.google.com.br} - Tudo o que você precisa tá no google. Se não estiver,
então não existe.

{\tt www.wikipedia.org} - Uma enciclopédia livre, onde qualquer pessoa pode
copiar coisas, e acrescentar também.

{\tt www.xkcd.com} - Webcomic sobre matemática e computação. Origem de muitas
piadas que você ouvirá por aí.

\end{subsecao}

\begin{subsecao}{Telefones}
Colocamos essa seção aqui apenas para que os bixos não nos encham com perguntas
sobre passes e plantão:

{\bf COSEAS:} Seção de Passe Escolar: {\tt 3091-3581}

{\bf Seção de Alunos:} {\tt 3091-6149}

{\bf Telefonista da USP:} {\tt 3091-4313}

{\bf Plantão de Cálculo e Álgebra Linear (serviço gratuito):} {\tt 3037-1773}
%Número do orelhão do IME - ou seja, existe uma grande propabilidade de ser atendido ^^

\end{subsecao}
\end{secao}
