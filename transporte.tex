\begin{secao}{Tudo Que Vai Volta (até bixo)}

\begin{subsecao}{Ônibus}

Se você é um bixo que não tem como ir nem como voltar, temos algumas dicas:

\begin{enumerate}
  \item Trabalhe muito para comprar um carro,
  trabalhe mais para pagar a gasolina e
  venha para a USP de carro e, obrigatoriamente, dê carona a um VETERANO;

  \item Peça a uma pessoa amiga para trazê-lo e buscá-lo durante seus
  longos anos de IME;

  \item Conheça alguém que, por sorte, mora perto da sua casa, estuda na USP,
  tenha o mesmo horário que você, seja legal e tenha carro. Traduzindo, s-o-n-h-e;

  \item Estique o dedão e espere, espere, espere, espere... a boa vontade
  de alguém para te dar carona;

  \item Mude-se para uma casa perto da USP;

  \item Desista do curso e diga: ``Eu não queria mesmo!!'';

  \item Se nenhuma das alternativas anteriores foi satisfeita, então
  transforme-se numa pessoa normal e pegue ônibus. Abaixo vão as
  principais linhas que passam pela USP.
\end{enumerate}

Eis as principais linhas que entram na USP. Se você não achou um local perto procure outras linhas em algum guia municipal. As fontes usadas foram: o site da USP, o da SPTrans e "o boca-a-boca”, já que como algumas coisas não constavam nos sites ou estavam desatualizadas tivemos que perguntar mesmo. 

{\bf Linhas municipais:}

Essas linhas tem como seu ponto final (com algumas exceções) a Portaria 2,
também conhecida como portaria da POLI (blerg..). Os itinerários não estão completos,
apenas colocamos as ruas/avenidas mais importantes. Para mais detalhes entre no site
da sptrans {\tt www.sptrans.com.br}.
Ei-las:


\begin{itemize}
  \item {\bf 107T - Cidade Universiária / Metrô Tucuruvi}\\
    Cor: Azul-escuro.\\
    Onde pegar para sair da USP: ponto da FAU.\\
    Itinerário: Av Tucuruvi, Av Nova Cantareira, Pr. Orlando Silva, Av Cruzeiro do Sul,
	Av Tiradentes, R. Augusta, Av Cidade Jardim, Av Valdemar Ferreira, Rua dos Bancos.
  
  \item {\bf 701U - Butantã-USP / Jaçanã}\\
    Cor: Azul-escuro.\\
    Onde pegar para sair da USP: ponto da FEA.\\
    Itinerário: Av Tucuruvi, Av Nova Cantareira, Av Cruzeiro do Sul, R. Voluntários da
    Pátria, Av Tiradentes, Av Ipiranga, R. Consolação, Av Dr. Arnaldo, R.
    Cardeal Arcoverde*, Av Eusébio Matoso, Av Valdemar Ferreira**, Rua do HU.

  \item {\bf 177H  - Butantã-USP / Casa Verde (Cuidado!!! Não confundir com a próxima)}\\
    Cor: Azul-escuro.\\
    Onde pegar para sair da USP: ponto da FAU.\\
    Itinerário: Rua Jaguarete, Av Brás Leme, Viad Pacaembu, Av Angélica, Av Dr. Arnaldo,
	R. Cardeal Arcoverde*, Av Valdemar Ferreira**, Rua dos Bancos.

  \item {\bf 177P - Butantã-USP / Casa Verde}\\
    Cor: Azul-escuro.\\
    Onde pegar para sair da USP: ponto da FAU.\\
    Itinerário: Rua Jaguaret, Av Brás Leme, Av Pacaembu, R. Cardoso de Almeida, Av
    Dr. Arnaldo, R. Cardeal Arcoverde*, Av Valdemar Ferreira**, Rua dos Bancos
    
  \item {\bf 637G - Butantã / Grajaú (até a Av. Afrânio Peixoto)}\\
    Cor: Azul-claro.\\
    Onde pegar para sair da USP: na Av. Afrânio Peixoto.\\
    Itinerário: Av Sen. Teotônio Vilela, Av Interlagos, Av Robert Kennedy, Lgo. Socorro, Av
    Adolfo Pinheiro, Av Sto. Amaro, R. Joaquim Floriano, Av Faria Lima, R.
    Sumidouro, Av Valdemar Ferreira, Pr. Vicente Rodrigues, Av Afrânio Peixoto

  \item {\bf 702P - Butantã (até a Portaria 1) / Term. Pq. D. Pedro II (não confundir com
      o próximo)}\\
    Cor: Amarela.\\
    Onde pegar para sair da USP: no ponto da portaria P1.\\
    Itinerário: R. Augusta, R. Colômbia, Av Europa, Av Cidade Jardim, Av Lineu de Paula
    Machado Av Valdemar Ferreira, Pr. Vicente Rodrigues (P1).

  \item {\bf 702U - Butantã-USP / Term. Pq. D.Pedro II}\\
    Cor: Laranja.\\
    Onde pegar para sair da USP: ponto da FEA.\\
    Itinerário: Viad Vinte e cinco de Março, Av Ipiranga, Pr. República, R. Consolação, Av
    Rebouças, Av Eusébio Matoso, Av Valdemar Ferreira**, Rua do HU.

  \item {\bf 7181 - Cidade Universitária / Term. PRINC. Isabel}\\
    Cor: Laranja.\\
    Onde pegar para sair da USP: ponto da FAU.\\
    Itinerário: Av Rio Branco, R. Aurora, Pça República, Av S. Luís, R. Augusta, Av Cidade
    Jardim, Av Lineu de Paula Machado, Av Valdemar Ferreira, Rua dos Bancos

  \item {\bf 7411 - Cidade Universitária / Pça. Da Sé}\\
    Cor: Laranja.\\
    Onde pegar para sair da USP: ponto da FAU.\\
    Itinerário: Pr. João Mendes, Lgo. S. Bento (volta Lgo. S. Francisco) , R. Líbero Badaró,
    Viad do chá, Pr. Ramos de Azevedo, Lgo. Paissandu, Pr. República, Av
    Ipiranga, R. Consolação, Av Rebouças, Av Vital Brasil, Rua dos Bancos.

  \item {\bf 7725 - Metrô V. Madalena / Rio Pequeno (não confundir com o próximo)}\\
    Cor: Laranja.\\
    Onde pegar para sair da USP: Se for sentido R. Pequeno, então ponto da FAU,
    senão ponto da FEA.\\
    Itinerário: Av Gustavo Berthier, Av Rio Pequeno, Av Corifeu de Azevedo Marques, Av
    Escola Politécnica, R. Mello Moraes, Av Lúcio Martins Rodrigues, Rua dos Bancos,
    Rua do HU, Pte. Cidade Universitária, Pr. Arcipreste
    Anselmo de Oliveira, Av Manoel José Chaves, Pr. Panamericana, R. Heitor
    Penteado

  \item {\bf 7725 - Metrô V. Madalena / Expresso-USP (não confundir com o próximo)}\\
    Cor: Laranja.\\
    Onde pegar para sair da USP: Ponto da FEA.\\
    Itinerário:  Rua dos Bancos, Rua do HU, Pte. Cidade Universitária, Pr. Arcipreste Anselmo de Oliveira, Av Manoel José Chaves, Pr. Panamericana, R. Heitor Penteado

  \item {\bf  724A - Aclimação / Cidade Universitária}\\
    Cor: Laranja\\
    Onde pegar pra sair da USP: Ponto da FEA.\\
    Itinerário:  Lgo. Nsa. Sra. da Conceição, R. Cons. Furtado, Av. da Aclimação, 
    R. Topázio, R. Dr. Nicolau de Souza Queiroz, Av. Bernardino de Campos,
    Av. Paulista, R. Card. Arcoverde*, R. Sumidouro, R. Eugênio de Medeiros,
    Av. Vital Brasil, Av. Afrânio Peixoto, P1, Rua dos Bancos.

  \item {\bf 7702 - Terminal Lapa / USP}
    Cor: Laranja\\
    Onde pegar pra sair da USP: Rua do Matão\\
    Itinerário:  R. do Matão, R. dos HU, Av. Afrânio Peixoto, Av Valentim Gentil,	 
    R. Magalhães de Castro, Pte. da Cidade Universitária,Av. Prof. Manuel José Chaves,
    Pr. Panamericana, Av. S. Gualter, R. Bairi, R. Pio XI, R. Tito, R. Francisco Alves,
	R. Jeroaquara, R. Scipiao, Term. Lapa,	 

\end{itemize}
* volta R. Teodoro de Sampaio\\
** volta Av Vital Brasil
 

Dica: Aqui na USP a maioria dos ônibus podem ser dividos em dois tipos: os que
passam antes pela Rua do HU (Av. Prof Lineu Prestes) e os
que passam antes na Rua dos Bancos (Av. Prof. Luciano Gualberto).


{\bf Linhas intermunicipais:}

Agora, se você mora mais longe ainda (outra cidade, outro estado, outro país...) e não quer ou não pode se mudar para São Paulo, existem algumas linhas de ônibus fretados para cidades mais próximas (ou não). Se por acaso a sua cidade não está aí, procure se informar a respeito, pois não significa necessariamente que não haja ônibus da USP para lá. Aí estão elas:

\begin{itemize}
  \item {\bf Empresa Urubupungá.}\\
    Tel: 3658-7777
    Site: {\tt www.urubupunga.com.br}\\
    280BI1- São Bernardo do Campo (Centro)\\
    Cor: Cinza\\
    Onde pegar para sair da USP: ponto da FAU\\
    Av Magalhães de Castro, Av Marginal Pinheiros (Shopping Eldorado), Av Dos
    Bandeirantes, Av Eng. Luiz Carlos Berrini, Av Roque Petroni Jr. (Shopping
    Morumbi), Av Prof Vicente Rão, Av Cupecê (Diadema), Av Fábio Eduardo Ramos
    Esquivel (Diadema), Av Piraporinha (Diadema), Av Lucas Nogueira Garcez
    (Diadema), Av Urubupungá.

  \item {\bf Fretados Jundiaí - USP}\\
    Viação MIMO\\
    Tel: 4522-7788\\
    {\tt www.viacaomimo.com.br}\\
    Principais horários:\\
    Ida: 6h20; 7h20; 12h50; 18h00 (na Rodoviaria de Jundiaí)\\
    Volta: 11h50; 17h20; 23h00 (No ponto da FEA)

  \item {\bf São José dos Campos}\\
    Redenção\\
    tel: (12) 3931-3047\\
    {\tt www.redencaoturismo.com.br}\\
    Ida: 06h15 (na gruta em S. José)\\
    Volta:17h40 (no ponto da p2)

  \item {\bf Bragança}\\
    N.S. Fátima\\
    {\tt www.saexbra.com.br}\\
    Tel: 4032-4723 e 7344-2007\\
    Ida: 05h50 (Lgo do Tabão/Habbibs)\\
    Volta:17h00 (Av prof almeida Prado)

  \item {\bf Campinas}\\
    Sta. Cruz\\
    Tel:3868-5995\\
    {\tt www.gruposantacruz.com.br}

  \item {\bf Santos}\\
    Náutica Turismo\\
    Tel: (13) 9112-8860

  \item {\bf Santos / S. Vicente}\\
    Transul\\
    Tel: 6954-4466

  \item {\bf ABC}\\
    Dinâmica ABC-USP\\
    4352-0565 / 4109-0172

  \item {\bf Sorocaba}\\
    Fretado Diurno\\
    (15) 9715-1676 (Márcio)

  \item {\bf Van (diurno e noturno)}\\
    RR transportes\\
    6919-3345 / 7144-3934\\
    4474-1222 \\

\end{itemize}

\end{subsecao}

\begin{subsecao}{Circular}

Também conhecido como “circulenda” ou “secular” (aos sábados, “milenar”, e aos domingos, “anos-luz”), é o meio de transporte mais barato dentro da USP. Foi criado para os USPianos se locomoverem dentro do Campus, mas em muitas vezes é melhor andar do que ficar esperando. Existem 2 itinerários distintos, com trajetos aproximadamente reversos. Fique atento para não dar uma de bixo burro (duh!) e se perder, hein? 

Obs.: há controvérsias incontáveis, mas há circulares aos fins-de-semana, bixo dedicado. Surge uma terceira linha chamada “Museus”, e fica a cargo do leitor adivinhar por onde ela passa. De qualquer forma, é um itinerário alterado com notável frequência... No último modelo conhecido, os pontos mais próximos para vir para o IME, seriam: para quem vem do P3, o Acesso Vila Indiana, e então se deve descer TODA a Rua do Estupro por vezes chamada de Rua do Matão, e para quem vem do P1, o ponto da FEA. Vale comentar que, aos sábados, esses ônibus passam a cada uma hora e meia, e, aos domingos, existe uma média de 2 circulares passando num ponto, com margem de erro igual a 3. 


\begin{itemize}
  \item {\bf Circular 1}\\
    P3/ Acesso pedestre Vila Indiana/ Biociências/ Filosofia/
    FAU(próximo ao IME)/ IAG(Bandejão da Física)/ Clube dos Funcionários/
    Poli-Civil/ Poli-Metalurgia (Term. De Ônibus e P2)/ Poli-Mecânica/ Pça do
    Relógio/ CRUSP/ acesso de pedestre FEPASA(para pegar o trêm)/ Educação
    Física (P1)/ Educação/ Reitoria (Bandejão Central e Coseas)/ Pça dos
    Reitores/ Antiga Reitoria (Bancos)/ FEA(próximo ao IME)/ Poli-Biênio/
    Prefeitura(Bandejão)/ MAE(acesso de pedestres Rio Pequeno)/ HU/ Biomédicas
    III/ Odontologia/ P3.

  \item {\bf Circular 2}\\
    P3/ Ipen/ Copesp(HU)/ acesso pedestres Rio Pequeno/
    Prefeitura(Bandejão)/ Física(Bandejão)/ Oceanográfico/ Biociências/ CEPAM
    (Bandejão da Química)/ Butantã/ Cultura Japonesa/ Paço das artes/
    P1/ Educação Física/ Acesso pedestres FEPASA(para pegar o trêm)/ Raia
    Olímpica/ Poratria 2/ Terminal de ônibus (mais próximo da P2)/ IPT/
    Eletrotécnica/ FAU (próximo ao IME)/ Geociências(Bancos)/ Letras(mais
    próximo do Bandejão Central)/ História e Geografia/ Fármácia e Química
    (Bandejão da Química)/ Rua do Lago/ Biomédicas/ P3.
\end{itemize}
\end{subsecao}

\begin{subsecao}{Veículos no Campus}
Saiba por onde entrar na USP. Lembre-se de ter sempre a sua carteirinha USP ou
seu comprovante de matrícula com RG em mãos. 
\begin{itemize}
  \item {\bf Portaria 1 (P1):} R. Afrânio Peixoto. Funciona 24h por dia todos os
    dias, mas a entrada é controlada de segunda à sexta das 20h às 05h, aos sábados
    após as 14h e domingo o dia inteiro. É por onde entram os ônibus municipais. 
    
  \item {\bf Portaria 2 (P2):} Av. Escola Politécnica. Funciona das 5h30 às 20h
    de segunda a sexta. Entrada controlada de seg a sex das 20h às 24h. Fechada
    aos sábados, domingos e feriados. Única entrada para caminhões. 
    
  \item {\bf Portaria 3 (P3):} Av. Corifeu de Azevedo Marques. Tem o mesmo horário
    de funcionamento da P2.

  \item {\bf Portaria 1/2:} R. Eng. Teixeira Soares. Funciona de segunda a sexta das 5h30 às 20h. 
    
  \item {\bf Portarias de pedestre (Mercadinho, São Remo, HU, FEPASA e
      Vila Indiana):} Funcionam de 2ª a 6ª, das 05 às 23 hs.

\end{itemize}

Aos fins de semana, ficam abertas apenas as portarias 1 e 3 (esta, fecha mais cedo). É preciso se identificar.

Os ônibus entram no sábado até as 14h e não entram no domingo. Se você vem de carro, saiba que a universidade dispões de bolsões de estacionamento
gratuito em torno das Unidades.

\end{subsecao}

\begin{subsecao}{Pontos de taxi}
Existem alguns pontos de taxi espalhados pela Cidade Universitária. Eis suas
localidades:

\begin{itemize}
\item Ponto da FEA / ECA (atrás do Banespa)\\
Fone: 3091-4488

\item Ponto da Reitoria\\
Fone: 3091-3556

\item Ponto do Hospital Universitário\\
Fone: 3091-3536
\end{itemize}
\end{subsecao}


\end{secao}
