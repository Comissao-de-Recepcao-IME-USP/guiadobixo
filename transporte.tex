\begin{secao}{Tudo Que Vai Volta (até bixo)}

\begin{subsecao}{Ônibus}

Se você é um bixo que não tem como ir nem como voltar, temos algumas dicas:

\begin{enumerate}
  \item Trabalhe muito para comprar um carro,
  trabalhe mais para pagar a gasolina e
  venha para a USP de carro e, obrigatoriamente, dê carona a um VETERANO;

  \item Peça a uma pessoa amiga para trazê-lo e buscá-lo durante seus
  longos anos de IME;

  \item Conheça alguém que, por sorte, mora perto da sua casa, estuda na USP,
  tenha o mesmo horário que você, seja legal e tenha carro. Traduzindo, s-o-n-h-e;

  \item Estique o dedão e espere, espere, espere, espere... a boa vontade
  de alguém para te dar carona;

  \item Mude-se para uma casa perto da USP;

  \item Desista do curso e diga: ``Eu não queria mesmo!!'';

  \item Se nenhuma das alternativas anteriores foi satisfeita, então
  transforme-se numa pessoa normal e pegue ônibus. Abaixo vão as
  principais linhas que passam pela USP.
\end{enumerate}

Eis as principais linhas que entram na USP. Se você não achou um local perto procure outras linhas em algum guia municipal. As fontes usadas foram: o site da USP, o da SPTrans e "o boca-a-boca”, já que como algumas coisas não constavam nos sites ou estavam desatualizadas tivemos que perguntar mesmo. 

{\bf Linhas municipais:}

Os itinerários não estão completos, apenas colocamos as ruas/avenidas mais
importantes. Para mais detalhes entre no site da sptrans 
{\tt www.sptrans.com.br}.

Ei-las:

\begin{itemize}

	\item {\bf 7181/10 – Term. Princesa Isabel / Cid. Universitária}\\
		Cor: Laranja\\
    	Onde pegar para sair da USP: ponto da FAU\\
    	Itinerário: Term. USP, Av. Prof. Almeida Prado, Av. Prof. Luciano
    	Gualberto, Av. Prof. Lineu Prestes, Av. Da Universidade, Av. Afrânio
    	Peixoto, Av. Valdemar Ferreira, Av. Lineu de Paula Machado, Av. Dos 
    	Tajurás, Av. Cidade Jardim, Av. Brigadeiro Faria Lima, Av. Europa, R. 
    	Augusta, R. Frei Caneca, R. da Consolação, Lgo. Do Arouche, Av. Duque
    	de Caxias, Term. Princesa Isabel.

	\item {\bf 701U/10 – Metrô Santana / Butantã – USP}\\
    	Cor: Azul-escuro\\
		Onde pegar para sair da USP: ponto da FEA\\
		Itinerário: Term. USP, Av. Prof. Melo Morais, Av. Prof. Lúcio Martins
		Rodrigues, Av. Prof. Luciano Gualberto, Av. Prof. Almeida Prado, Av. 
		Prof. Lineu Prestes, Av. Da Universidade, Av. Dr. Vital Brasil, R. 
		Butantã, R. Teodoro Sampaio, Av. Dr. Arnaldo, R. da Consolação, Pça.
		Ramos de Azevedo, Lgo. Paissandu, Av. Prestes Maia, Av. Tiradentes, Av.
		Cruzeiro do Sul, R. Ezequiel Freire.

	\item {\bf 177H/10 – Metro Santana/Butantã – USP}\\
	Cor: Azul-escuro\\
	Onde pegar para sair da USP: ponto da FAU\\
	Itinerário: Term. USP, Av. Prof. Almeida Prado, Av. Prof. Luciano 
	Gualberto,Av. Prof. Lineu Prestes, Av. da Universidade, Av. Dr. Vital
	Brasil, R. Butantã, R. Teodoro Sampaio, Av. Dr. Arnaldo, R. Bela Cintra,
	Av. Angélica, Av. S. João, Av. Pacaembu, Av. Brás Leme, Av. Casa Verde,
	Av. Cruzeiro do Sul, R. Ezequiel Freire.

	\item {\bf 702U/10 – Term. Pq. Dom Pedro II/Butantã – USP}\\
	Cor: Laranja\\
	Onde pegar para sair da USP: ponto da FEA\\
	Itinerário: Term. USP, Av. Prof. Melo Morais, Av. Prof. Lúcio Martins
	Rodrigues, Av. Prof. Luciano Gualberto, Av. Prof. Almeida Prado, Av. Prof.
	Lineu Prestes, Av. Da Universidade, Av.	Dr. Vital Brasil, Av. Rebouças, R.
	da Consolação, Viad. Do Chá, Lgo. S. Francisco, Pça da Sé, Av. Rangel 
	Pestana, Term. Pq. Dom Pedro II.

	\item {\bf 7411/10 – Pça. Da Sé/Cidade Universitária}\\
	Cor: Laranja\\
	Onde pegar para sair da USP: ponto da FAU\\
	Itinerário: Term. USP, Av. Prof. Almeida Prado, Av. Prof. Luciano 
	Gualberto, Av. Prof. Lineu Prestes, Av. Da Universidade, Av. Dr. Vital 
	Brasil, Av. Rebouças, R. Da Consolação, Viad. Do Chá, R. Líbero Badaró,
	Lgo. S. Francisco, Pça da Sé.

	\item {\bf 7725/10 – Rio Pequeno/Metro V. Madalena}\\
	Cor: Laranja\\
	Onde pegar para sair da USP: ponto da FAU (sentido Rio Pequeno), ponto da
	FEA (sentido metro V. Madalena)\\
	Itinerário: R. Rui Amaral Lemos, R. Jorge Ward, Av. Do Rio Pequeno, Av.
	Corifeu de Azevedo Marques, Av. Escola Politécnica, Av. Prof. Melo Moraes,
	Av. Prof. Lúcio Martins Rodrigues, Av. Prof. Luciano Gualberto, Av. Prof.
	Almeida Prado, Av. Prof. Lineu Prestes, Av. Da Universidade, Av. Afrânio
	Peixoto, Av. Valentim Gentil, Pte. Da Cidade Universitária, Pça. 
	Panamericana, R. Alvilandia, R. Pereira Leite, R. Heitor Penteado, Term.
	Metro V. Madalena.\\

	Obs.: 7725/10 – Rio Pequeno/ Metro V. Madalena – Expresso USP: mesmo
	itinerário, a diferença é que este só para nos pontos dentro da cidade
	universitária.

	\item {\bf 7702/10 – Term. Lapa/USP}\\
	Cor: Laranja\\
	Onde pegar para sair da USP: ponto na Av. Prof. Almeida Prado\\
	Itinerário: Term. USP, Av. Prof. Almeida Prado, Av. Prof. Lineu Prestes,
	Av. Da Universidade, Av. Afrânio Peixoto, Av. Valentim Gentil, Pte. Da
	Cidade Universitária, Pça. Panamericana, Av. S. Gualter, R. Bairi, R. Pio
	XI, R. Tito, R. Aurélia, Term. Lapa.

	\item {\bf 908T/10 – Term. Pq. Dom Pedro II/Butantã}\\
	Cor: Amarelo\\
	Onde pegar para sair da USP: ponto do P1 a partir das 20 horas.\\
	Itinerário: Av. Afrânio Peixoto, Av. Valdemar Ferreira, Av. Lineu de Paula
	Machado, Av. dos Tajurás, Av. Cidade Jardim, Av. Brig. Faria Lima, Av. 
	Europa, R. Augusta, Viad. Nove de Julho, Av. Rangel Pestana, R. Gen. 
	Carneiro, Term. Pq. Dom Pedro II.
  
\end{itemize}

Dica: Aqui na USP os ônibus podem ser divididos em dois tipos: os que passam
pela Rua dos Bancos e os que passam pela Rua do HU.

Para tentar facilitar:
\begin{itemize}
	\item Av. Prof. Luciano Gualberto = Rua dos Bancos;
	\item Av. Prof. Lineu Prestes = Rua do HU;
	\item Av. Prof. Mello Moraes = Rua da Raia.
\end{itemize}

{\bf Linhas intermunicipais:}

Agora, se você mora mais longe ainda (outra cidade, outro estado, outro país...) e não quer ou não pode se mudar para São Paulo, existem algumas linhas de ônibus fretados para cidades mais próximas (ou não). Se por acaso a sua cidade não está aí, procure se informar a respeito, pois não significa necessariamente que não haja ônibus da USP para lá. Aí estão elas:

\begin{itemize}
  \item {\bf Empresa Urubupungá.}\\
    Tel: 3658-7777
    Site: {\tt www.urubupunga.com.br}\\
    280BI1- São Bernardo do Campo (Centro)\\
    Cor: Cinza\\
    Onde pegar para sair da USP: ponto da FAU\\
    Av Magalhães de Castro, Av Marginal Pinheiros (Shopping Eldorado), Av Dos
    Bandeirantes, Av Eng. Luiz Carlos Berrini, Av Roque Petroni Jr. (Shopping
    Morumbi), Av Prof Vicente Rão, Av Cupecê (Diadema), Av Fábio Eduardo Ramos
    Esquivel (Diadema), Av Piraporinha (Diadema), Av Lucas Nogueira Garcez
    (Diadema), Av Urubupungá.

  \item {\bf Fretados Jundiaí - USP}\\
    Viação MIMO\\
    Tel: 4522-7788\\
    {\tt www.viacaomimo.com.br}\\
    Principais horários:\\
    Ida: 6h20; 7h20; 12h50; 18h00 (na Rodoviaria de Jundiaí)\\
    Volta: 11h50; 17h20; 23h00 (No ponto da FEA)

  \item {\bf São José dos Campos}\\
    Redenção\\
    tel: (12) 3931-3047\\
    {\tt www.redencaoturismo.com.br}\\
    Ida: 06h15 (na gruta em S. José)\\
    Volta:17h40 (no ponto da p2)

  \item {\bf Bragança}\\
    N.S. Fátima\\
    {\tt www.saexbra.com.br}\\
    Tel: 4032-4723 e 7344-2007\\
    Ida: 05h50 (Lgo do Tabão/Habbibs)\\
    Volta:17h00 (Av prof almeida Prado)

  \item {\bf Campinas}\\
    Sta. Cruz\\
    Tel:3868-5995\\
    {\tt www.gruposantacruz.com.br}

  \item {\bf Santos}\\
    Náutica Turismo\\
    Tel: (13) 9112-8860

  \item {\bf Santos / S. Vicente}\\
    Transul\\
    Tel: 6954-4466

  \item {\bf ABC}\\
    Dinâmica ABC-USP\\
    4352-0565 / 4109-0172

  \item {\bf Sorocaba}\\
    Fretado Diurno\\
    (15) 9715-1676 (Márcio)

  \item {\bf Van (diurno e noturno)}\\
    RR transportes\\
    6919-3345 / 7144-3934\\
    4474-1222 \\

\end{itemize}

\end{subsecao}

\begin{subsecao}{Circular}

Também conhecido como ``circulenda'' ou ``secular'' (aos sábados, ``milenar'', e aos domingos, ``anos-luz''), é o meio de transporte mais barato dentro da USP. Foi criado para os USPianos se locomoverem dentro do Campus, mas em muitas vezes é melhor andar do que ficar esperando. Existem 2 itinerários distintos, com trajetos aproximadamente reversos. Fique atento para não dar uma de bixo burro (duh!) e se perder, hein? 

Há controvérsias incontáveis, mas há circulares aos fins-de-semana, bixo dedicado. Surge uma terceira linha chamada “Museus”, e fica a cargo do leitor adivinhar por onde ela passa. De qualquer forma, é um itinerário alterado com notável frequência... No último modelo conhecido, os pontos mais próximos para vir para o IME, seriam: para quem vem do P3, o Acesso Vila Indiana, e então se deve descer TODA a Rua do Estupro por vezes chamada de Rua do Matão, e para quem vem do P1, o ponto da FEA. Vale comentar que, aos sábados, esses ônibus passam a cada uma hora e meia, e, aos domingos, existe uma média de 2 circulares passando num ponto, com margem de erro igual a 3. 

No ano passado foram implantadas as linhas 8012/10 e 8022/10 - Metro Butantã /Cidade Universitária, que funcionam como circulares USP, e nós alunos não pagamos, pois elas aceitam o bilhete USP (BUSP).

\begin{itemize}
  \item {\bf 8012-10}\\
    Metrô Butantã / P1 / Educação/ Reitoria (Bandejão Central e
    Coseas)/ Pça dos Reitores/ Antiga Reitoria (Bancos)/ FEA (próximo ao IME)/
    Poli-Biênio/ Prefeitura (Bandejão)/ MAE (acesso de pedestres Rio Pequeno)/
    HU/ Biomédicas III/ Odontologia/ P3/ Acesso pedestre Vila Indiana/ 	  
    Biociências/ Filosofia/
    FAU (próximo ao IME)/ IAG (Bandejão da Física)/ Clube dos Funcionários/
    Poli-Civil/ Poli-Metalurgia (Term. De Ônibus e P2)/ Poli-Mecânica/ ECA / 
    Pça do Relógio/ CRUSP/ acesso de pedestre FEPASA (para pegar o trem)/ 
    Educação Física / P1 / Metrô Butantã.

  \item {\bf 8022-10}\\
    Metrô Butantã / P1 / Educação Física/ Acesso pedestres FEPASA (para
    pegar o trêm)/ Raia Olímpica/ Poratria 2/ Terminal de ônibus (mais próximo
    da P2)/ IPT/ Eletrotécnica/ FAU (próximo ao IME)/ Geociências (Bancos)/
    Letras (mais próximo do Bandejão Central)/ História e Geografia/ Farmácia e
    Química (Bandejão da Química)/ Rua do Lago/ Biomédicas/ P3/ Ipen/ Copesp
    (HU)/ acesso pedestres Rio Pequeno/
    Prefeitura (Bandejão)/ Física (Bandejão)/ Oceanográfico/ Biociências/ CEPAM
    (Bandejão da Química)/ Butantã/ Cultura Japonesa/ Paço das artes/
    P1 / Metrô Butantã.
    
\end{itemize}
\end{subsecao}

\begin{subsecao}{Veículos no Campus}
Saiba por onde entrar na USP. Lembre-se de ter sempre a sua carteirinha USP ou
seu comprovante de matrícula com RG em mãos. 
\begin{itemize}
  \item {\bf Portaria 1 (P1):} R. Afrânio Peixoto. Funciona 24h por dia todos os
    dias, mas a entrada é controlada de segunda à sexta das 20h às 05h, aos sábados
    após as 14h e domingo o dia inteiro. É por onde entram os ônibus municipais. 
    
  \item {\bf Portaria 2 (P2):} Av. Escola Politécnica. Funciona das 5h30 às 20h
    de segunda a sexta. Entrada controlada de seg a sex das 20h às 24h. Fechada
    aos sábados, domingos e feriados. Única entrada para caminhões. 
    
  \item {\bf Portaria 3 (P3):} Av. Corifeu de Azevedo Marques. Tem o mesmo horário
    de funcionamento da P2.

  \item {\bf Portaria 1/2:} R. Eng. Teixeira Soares. Funciona de segunda a sexta das 5h30 às 20h. 
    
  \item {\bf Portarias de pedestre (Mercadinho, São Remo, HU, FEPASA e
      Vila Indiana):} Funcionam de 2ª a 6ª, das 05 às 23 hs.

\end{itemize}

Aos fins de semana, ficam abertas apenas as portarias 1 e 3 (esta, fecha mais cedo). É preciso se identificar.

Os ônibus entram no sábado até as 14h e não entram no domingo. Se você vem de carro, saiba que a universidade dispões de bolsões de estacionamento
gratuito em torno das Unidades.

\end{subsecao}

\begin{subsecao}{Pontos de táxi}
Existem alguns pontos de táxi espalhados pela Cidade Universitária. Eis suas
localidades:

\begin{itemize}
\item Ponto da FEA / ECA (atrás do Banespa)\\
Fone: 3091-4488

\item Ponto da Reitoria\\
Fone: 3091-3556

\item Ponto do Hospital Universitário\\
Fone: 3091-3536
\end{itemize}
\end{subsecao}


\end{secao}
