\begin{secao}{Contas na Rede GNU/Linux}
\\
\\
\begin{subsecao}{Introdução}

A rede GNU/Linux é uma rede de computadores, administrada por alunos do IME e
que fornece diversos serviços para os VETERANOS e até mesmo para vocês, bixos.
Ela disponibiliza:

\begin{itemize}
\item 4 salas de computadores (no bloco A) com todo\footnote{se um programa
estiver faltando, mande um email pra admin@linux.ime.usp.br pedindo-o} tipo de
programa necessário para suas atividades acadêmicas (com pelo menos uma que fica
aberta 24 horas por dia, 7 dias por semana\footnote{mas
talvez você não consiga entrar no bloco A depois da meia noite, que é quando a
portaria fecha});
\item Uma página na internet para cada aluno;
\item Um e-mail para cada aluno;
\item Espaço para você guardar seus arquivos;
\item Acesso remoto via ssh (shell.linux.ime.usp.br);
\item Repositórios svn e git;
\item Impressoras;
\item Listas de discussão (em parte pra você e seus colegas bixos discutirem
coisas das matérias e sobreviverem ao IME)\footnote{
users-<curso>-ano@linux.ime.usp.br, onde curso é bcc, bma, bm, be, bmap, lic ou
licn. Para mais informações sobre as listas, acesse listas.linux.ime.usp.br};
\item Admins dispostos e capazes, para o caso de algum usuário ter alguma boa
idéia para adicionar a essa lista;
\end{itemize}
\end{subsecao}

\begin{subsecao}{O GNU/linux }

A rede utiliza em todos os seus computadores um sistema operacional chamado
GNU/Linux. Esse é um sistema desenvolvido de forma colaborativa pelos usuários
e empresas interessados nele. (se quiser saber mais a respeito, pesquise
por "software livre” ou passe na admin e pergunte!).

O GNU/Linux não é um sistema mais difícil de usar que o Windows. Ele é apenas
diferente em alguns aspectos. Além de tudo, existem cursos de GNU/Linux que são
organizados pelos alunos do IME. Os admins costumam promover esses cursos, então
fique atento!

Então, não se deixe intimidar pelo sistema. Se você se der ao trabalho de
aprender a utilizá-lo bem, verá que ele é bastante flexível, e até mesmo
interessante (tanto quanto um sistema operacional pode ser =P).

\end{subsecao}

\begin{subsecao}{Os admins}

Os admins são alunos do bacharelado em ciência da computação (vulgo BCC) que
são responsáveis por administrar a rede. Entre outras coisas, isso quer dizer
manter os micros funcionando, ajudar os alunos a usar a rede (com
cursos \footnote{ veja na página da rede (www.linux.ime.usp.br) para saber
quando. Talvez os admins passem na sua sala avisando também} e resolvendo
dúvidas nos horários de plantão \footnote{na página da rede, estão os horários
de todos os admins (especificamente, em {\tt http://www.linux.ime.usp.br/admin/atendimento.html})}
) e também implementar coisas novas na rede (aceitamos sugestões !)

Os admins são escolhidos por um treinamento que acontece de dois em dois anos,
em todo ano par. Mais informações serão divulgadas quando este estiver próximo
a ocorrer.

\end{subsecao}
\begin{subsecao}{Como criar uma conta?}

Basta passar na admin, na sala 125 do bloco A (como você é bixo: bloco A é o da
biblioteca, bloco B aquele que tem muitas salas de aula e que você vai passar boa 
parte da sua vida). Contatos:
\begin{description}

\item [Telefone:] 3091-6482
\item [e-mail:] admin@linux.ime.usp.br
\item [Página:] www.linux.ime.usp.br
\item [Sala:] 125, bloco A

\end{description}
\end{subsecao}
\end{secao}
