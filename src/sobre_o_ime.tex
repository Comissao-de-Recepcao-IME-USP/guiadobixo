\begin{secao}{Um Pouco Sobre o IME}

O IME, Instituto de Matemática e Estatística (não, bixos, não está errado. 
Computação não faz parte da sigla mesmo! Aliás, faz sim, mas só os inteligentes 
podem ver!), tem vários blocos: o bloco A, bloco B, bloco C, o bloco D e o 
novo bloco, até então um mistério para todos. Leia atentamente, porque com essa 
confusão toda, até alguns VETERANOS insistem em dizer que só existem os blocos A, B e C.

\begin{subsecao}{Bloco A}
No bloco A, as coisas mais importantes são: as salas dos professores, a parte
administrativa (diretoria, secretarias de departamento), algumas salas de aula
(para a pós-graduação), as máquinas de café, a IMEjr, a biblioteca, as ET's e
a Rede Linux.

\begin{itemize}

\item {\bf IMEjr:} é a empresa júnior do IME, que é administrada pelos próprios
alunos da graduação. Fica na sala 258A.

\item {\bf Biblioteca:} fica logo na entrada, do lado direito. Há algumas mesas
individuais (que são muito confortáveis para tirar um cochilo) e algumas salas
com lousa para estudo em grupo. Vocês podem pegar livros emprestados para consulta
na hora, mas para levarem para casa vão precisar fazer um cadastro, mostrando a
carteirinha provisória.

O empréstimo é unificado, sendo possível retirar livros em qualquer uma das bibliotecas da USP.

Horário de atendimento:
\begin{itemize}
\item[-] Período letivo: de segunda a sexta, das 08:00 às 21:30; sábado das 08:00 às 13:00.
\item[-] Férias: de segunda a sexta, 08:00 às 19:00.
\end{itemize}

\item {\bf ET's (Estações de Trabalho):} são salas com alguns terminais que têm
acesso à Internet e muito mais. Não, bixos, não se animem, pois ela não é para
vocês! Essa sala só pode ser usada pelos alunos da pós-graduação e por alguns
alunos que fazem iniciação científica.

\item {\bf Rede Linux:} podem ser usadas por alunos da graduação (ieiii!).
Utilizam o Linux, que, ao contrário do Windows, é um Sistema Operacional.

\end{itemize}
\end{subsecao}

\begin{subsecao}{Bloco B}

No bloco B, vocês devem conhecer as salas de aula (da graduação), a Vivência, as
mesas azuis, o CEC, a seção de alunos, o GRECIME, o CAEM e o Xerox para vocês
copiarem \sout{o caderno do colega} o capítulo do livro para estudar para a prova.

\begin{itemize}
\item {\bf Vivência:} sala 18, arrasadora de graduações, é onde as pessoas podem 
dormir, assistir TV, jogar pebolim, sinuca, cartas, xadrez, vídeo-game e até 
mesmo estudar. Guardem bem esse nome, vocês vão passar a maior parte do tempo lá. 
Lá também estão os armários e as salas do CAMat e da Atlética.

Os armários são renovados semestralmente. Após o período de renovação os que 
estiverem sobrando ficam livres para serem alugados. Aí cabe a vocês, bixos, ficarem 
atentos às datas divulgadas pelo CAMat e demonstrarem interesse pelos armários para 
conseguirem entrar no sorteio (que ocorre porque infelizmente não há armários 
suficiente para todos os alunos).

\item {\bf Lanchonete:} Até 2011 havia uma no IME, mas nos parece que o inquilino
teve suas síndromes de Seu Madruga, atrasou o aluguel por anos \footnote{A verdade 
é que o cara não somente atrasou o aluguel, mas anunciou com cada palavra que não 
o pagaria! E, apesar desta clareza cartesiana, só foi feito o despejo em 2011, 
após anos sem aluguel pago.}, e teve a lanchonete fechada. Como nada nesse
mundo não é tão ruim que não possa ficar pior, hoje sentimos saudades do Sandrão
enquanto pegamos filas quilométricas para pegar um salgado requentado no GRECIME.
Há um projeto sobre o que será feito no local, mas isso ainda leva um tempo e, 
por enquanto, tudo que temos é o \sout{péssimo} serviço da tia do GRECIME... 

\item {\bf GRECIME:} Grêmio dos funcionários. Fica na sala 21, e além de vender
vários produtos desde perfumes a vestidos, agora é a única alternativa à
lanchonete dentro do IME.

\item {\bf CEC (Centro de Ensino de Computação):} é um centro munido de dezenas
de computadores rodando Windows, e alguns rodando Linux. Também é possível fazer 
(ou pelo menos tentar) EPs em situações de desespero (vocês vão descobrir o que é 
isso logo, logo). Algumas aulas de computação são ministradas lá (como Desafios 
de Programação do BCC).
 
\item {\bf Seção de Alunos:} é aqui que vocês vão resolver quase todos os pepinos 
de vocês durante a Graduação. Desde descobrir por quê o Júpiter não quer matricular vocês
em uma disciplina obrigatória até CONFIRMAR A MATRÍCULA nos dias 23/02 e 24/02. %REFTIME

A Seção de Alunos fica na sala 12, ao lado do CEC e das mesas azuis.
\begin{itemize}
\item[-] Horário de atendimento: 10:00 às 13:00 e 18:00 às 20:00 de segunda a sexta.
\item[-] E-mail: \tt{saol@ime.usp}
\item[-] Telefone: \tt{11 3091-6149} e \tt{11 3091-6279}
\end{itemize}

\item {\bf CAEM:} sigla para o Centro de Aperfeiçoamento do Ensino da Matemática.
É um órgão de extensão, que oferece cursos, oficinas, palestras e presta serviços
de assessoria para professores de Matemática. Vocês que fazem Licenciatura (futuros 
professores) também podem usufruir dos serviços do CAEM, e até mesmo estagiarem lá. 
Fica no 1º andar, em frente às escadas.

\item{\bf Xerox:} fica no térreo, seguindo o corredor do Grecime, na próxima
                  porta à esquerda. Para fazer cópias, é preciso primeiro
                  comprar fichas na Atlética.
\end{itemize}
\end{subsecao}

\begin{subsecao}{Bloco C}

Apesar de atualmente abrigar os professores e a secretaria do departamento de
Computação, o misterioso Bloco C é um local a que nós, estudantes, infelizmente não
temos livre acesso. Para garantir tranquilidade e boas condições de trabalho aos
docentes, os alunos devem ser anunciados ou apresentarem uma boa desculpa para 
adentrar o local. (Na verdade, os professores do MAC acham que os alunos são monstros
verdes e gosmentos, com $e^{10}$ braços, $\pi$ olhos e que comem criancinhas;
por isso, não querem esse tipo de criatura perambulando pelos seus corredores.)

\end{subsecao}

\begin{subsecao}{Bloco D (vulgo C')}

O bloco D não passa de uma extensão do bloco C. Na verdade, eles são o mesmo bloco,
só que têm entradas independentes. Pertence ao NUMEC, que ninguém sabe ao certo
o que significa. Alguns dizem que ele não existe, e que é apenas fruto da sua
imaginação.

\end{subsecao}

\begin{subsecao}{Nova Construção - CCSL}

O anexo do Bloco C, abriga o CCSL --- Centro de Competência em \textit{Software} 
Livre. O que tem lá? Algumas salas de docentes (afinal, ele é uma extensão do 
Bloco C), alguns laboratórios (de sistemas, de inteligência artificial, 
de visão computacional, de computação musical...), um pequeno auditório 
onde rolam palestras e seminários e, finalmente, o laboratório de extensão, 
que é usado para cursos rápidos e palestras com parte prática onde o público 
precise de um computador.\\
Se vocês se interessam por \textit{Software} Livre, não deixem de ler a seção 
do CCSL desse Guia!

\end{subsecao}
\end{secao}
