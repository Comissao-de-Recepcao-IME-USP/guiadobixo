\begin{secao}{Um Pouco Sobre o IME}

O IME, Instituto de Matemática e Estatística (não, bixes, não está errado.
Computação não faz parte da sigla mesmo! Aliás, faz sim, mas só os inteligentes
podem ver!), tem vários blocos: bloco A, bloco B, bloco C, bloco D e o
novo bloco, até então um mistério para todos. Leia atentamente, porque com essa
confusão toda, até alguns veteranos insistem em dizer que só existem os blocos
A, B e C.


\begin{subsecao}{Bloco A}
No bloco A, as coisas mais importantes são: as salas dos professores, a parte
administrativa (diretoria, secretarias de departamento), algumas salas de aula
(para a pós-graduação), a IMEjr, a biblioteca, as ET's e a Rede Linux.

\begin{itemize}

\item {\bf IMEjr:} é a empresa júnior do IME, que é administrada pelos próprios
alunos da graduação. Fica na sala 258A.

\item {\bf Biblioteca:} fica logo na entrada, do lado direito. Há algumas mesas
individuais e sofás (ambos muito confortáveis para tirar um cochilo) e algumas salas
com lousa para estudo em grupo. Tem também uma sala de estudos enorme, 
mas lá não pode fazer barulho (cuidado até mesmo ao abrir e fechar a porta!).
Veja mais informações sobre funcionamento e empréstimos adiante.

\item {\bf ET's (Estações de Trabalho):} são salas com alguns terminais que têm
acesso à Internet e muito mais. Não, bixes, não se animem, pois ela não é para
vocês! Essa sala só pode ser usada pelos alunos da pós-graduação e por alguns
alunos que fazem iniciação científica.

\item {\bf Rede Linux:} podem ser usadas por alunos da graduação (ieiii!).
Utilizam o Linux, que, ao contrário do Windows, é um Sistema Operacional.

\end{itemize}

\end{subsecao}

\begin{subsecao}{Bloco B}


No bloco B, vocês devem conhecer as salas de aula (da graduação), a Vivência, as
mesas azuis (que não são todas azuis), o CEC, a seção de alunos, a lanchonete, 
o CAEM e o Xerox para vocês copiarem \sout{o caderno do colega} o capítulo do 
livro para estudar para a prova.

\begin{itemize}
\item {\bf Vivência:} sala 18, arrasadora de graduações, é onde as pessoas podem
dormir, assistir TV, jogar sinuca, cartas, xadrez, vídeo-game e até
mesmo estudar. Guardem bem esse nome, vocês vão passar a maior parte do tempo lá.
Lá também estão os armários e as salas do CAMat e da Atlética.

Os armários são renovados semestralmente. Após o período de renovação os que
estiverem sobrando ficam livres para serem alugados. Aí cabe a vocês, bixes, ficarem
atentos às datas divulgadas pelo CAMat e demonstrarem interesse pelos armários para
conseguirem entrar no sorteio.

\item {\bf Lanchonete:} uma alternativa para os dias chuvosos que você está sem
  guarda-chuva para ir até o bandejão é a lanchonete do IME. Tem um vasto
  cardápio de salgados e congelados esquentados no microondas, além de sucos,
  refrigerantes, chocolates, bolachas e similares.

%FIXME gambz para os mapas ficarem no lugar certo
\end{itemize}

\mapaime{bloco_A_terreo}

\mapaime{bloco_A_piso1}

\mapaime{bloco_A_piso2}

\begin{itemize}

\item {\bf CEC (Centro de Ensino de Computação):} é um centro munido de dezenas
de computadores rodando Windows, e alguns rodando Linux. Também é possível fazer
(ou pelo menos tentar) EPs em situações de desespero (vocês vão descobrir o que é
isso logo, logo). Algumas aulas de computação são ministradas lá (como Desafios
de Programação do BCC).

\item {\bf Seção de Alunos:} é aqui que vocês vão resolver quase todos os
pepinos de vocês durante a Graduação. Desde descobrir por quê o Júpiterweb não
quer matricular vocês em uma disciplina obrigatória até fazer a
MATRÍCULA PRESENCIAL nos dias 27/02 e 28/02. %REFTIME

A Seção de Alunos fica na sala 12, ao lado do CEC e das mesas azuis.
\begin{itemize}
\item[-] Horário de atendimento: 10:00 às 13:00 e 17:00 às 20:00, de segunda a sexta.
\item[-] E-mail: \tt{saol@ime.usp.br}
\item[-] Telefone: \tt{11 3091-6149} e \tt{11 3091-6279}
\end{itemize}

\item {\bf CAEM:} sigla para o Centro de Aperfeiçoamento do Ensino da
  Matemática. É um órgão de extensão, que oferece cursos, oficinas, palestras e
  presta serviços de assessoria para professores de Matemática. Vocês que fazem
  Licenciatura (futuros professores) também podem usufruir dos serviços do CAEM,
  e até mesmo estagiarem lá. Fica no 1º andar, em frente às escadas.

\item{\bf Xerox:} fica no térreo, saindo da vivência e indo até o final segundo
  corredor à direita (o que não tem os banheiros no final). Para fazer cópias,
  é preciso primeiro comprar fichas na salinha da Atlética (que fica na Vivência).
\end{itemize}

\end{subsecao}

\begin{subsecao}{Bloco C}

Apesar de atualmente abrigar os professores e a secretaria do departamento de
Computação, o misterioso Bloco C é um local a que nós, estudantes, infelizmente não
temos livre acesso. Para garantir tranquilidade e boas condições de trabalho aos
docentes, os alunos devem ser anunciados ou apresentarem uma boa desculpa para
adentrar o local. (Na verdade, os professores do MAC acham que os alunos são monstros
verdes e gosmentos, com $e^{10}$ braços, $\pi$ olhos e que comem criancinhas;
por isso, não querem esse tipo de criatura perambulando pelos seus corredores.)

\end{subsecao}

\begin{subsecao}{Bloco D (vulgo C')}

O bloco D não passa de uma extensão do bloco C. Na verdade, eles são o mesmo bloco,
só que têm entradas independentes. Pertence ao NUMEC, que ninguém sabe ao certo
o que significa. Alguns dizem que ele não existe, e que é apenas fruto da sua
imaginação.

\end{subsecao}

\begin{subsecao}{CCSL}

O anexo do Bloco C abriga o CCSL --- Centro de Competência em \textit{Software}
Livre. O que tem lá? Algumas salas de docentes (afinal, ele é uma extensão do
Bloco C), alguns laboratórios (de sistemas, de inteligência artificial,
de visão computacional, de computação musical...), um pequeno auditório
onde rolam palestras e seminários e, finalmente, o laboratório de extensão,
que é usado para cursos rápidos e palestras com parte prática onde o público
precise de um computador. É esse laboratório que abriga grande parte dos grupos
de extensão do IME que vocês verão na seção ``Atitude, bixes!''.

Se vocês se interessam por \textit{Software} Livre, não deixem de ler a seção
do CCSL desse Guia!

\end{subsecao}

\mapaime{bloco_B_terreo}

\mapaime{bloco_B_superior}

\begin{subsecao}{Biblioteca e SIBi-USP}

A Biblioteca Carlos Benjamin de Lyra do IME-USP é especializada em matemática
aplicada, estatística, ensino da matemática e ciência da computação e possui
um dos mais completos acervos em matemática na América Latina.

É uma das 48 bibliotecas que compõem o Sistema Integrado de Bibliotecas (SIBi)
da Universidade de São Paulo, que juntas abrigam um inestimável acervo de
cerca de 8 milhões de itens que cobrem todas as áreas do conhecimento.

Em atividades diárias, o acervo pode ser consultado diretamente nas
bibliotecas ou pela Internet, por meio do Portal de Busca Integrada e do Banco
de Dados Bibliográficos da USP — Dedalus, que dão acesso público aos registros
bibliográficos de livros, periódicos, todas as teses e dissertações defendidas
na Universidade, além de anais de congresso, catálogos, filmes, iconografias,
jornais, folhetos, produção intelectual dos docentes, entre outros materiais,
conduzindo ao texto completo sempre que possível.

Em sintonia com as novas tecnologias, oferece à comunidade USP acesso remoto a
um acervo digital que compreende mais de 250 mil títulos de livros
eletrônicos, cerca de 20 mil títulos de periódicos, mais de 80 bases de dados,
além dos conteúdos disponíveis em suas bibliotecas digitais.

A Biblioteca Carlos Benjamin de Lyra – IME-USP funciona de 2a. a 6a. feira,
das 08h00 às 21h30, e aos sábados, das 09h00 às 13h00 (período letivo). No
período de férias abre de 2a. a 6a. feira, das 8h00 às 19h30. Ao fazer a
matrícula, os calouros já estarão automaticamente inscritos na biblioteca,
podendo retirar livros dois dias após a matrícula.

Para retirar livros, basta apresentar o cartão provisório junto com um
documento de identificação (RG ou CNH) e cadastrar uma senha. O empréstimo dos 
alunos de graduação será por dez dias, podendo ser renovado até três vezes,
pelo mesmo período, desde que o livro não possua reserva. Poderão ser
retirados no máximo 10 (dez) livros em todo o sistema de bibliotecas, ou seja,
poderão ser emprestados livros de qualquer biblioteca da USP, até que o
limite de dez itens seja atingido. As renovações e reservas podem ser feitas
pela internet, pelo sistema Dedalus.

É importante saber que, embora o empréstimo seja unificado, cada uma das
bibliotecas do Sistema possui regimentos e regulamentos individuais e,
portanto, alguns procedimentos podem variar de uma para outra.

Dúvidas podem atrapalhar seu processo de construção do conhecimento, por isso
a Biblioteca estará sempre à disposição para ajudá-los na realização de
pesquisas nas mais variadas fontes de informação e recursos disponíveis.

Para maiores informações, consultar o site da biblioteca em
\url{http://www.ime.usp.br/bib}

\end{subsecao}
\end{secao}
