\begin{editorial}{Carta Aos Ingressantes}

Agora que vocês entraram na USP, bixes, vocês adquiriram novas
responsabilidades.  Vocês são responsáveis por si mesmos, isto é, ninguém irá se
preocupar com seus problemas acadêmicos (matrículas, notas erradas, dificuldade
com algumas matérias, rixa com professores etc.) se vocês mesmos não se
preocuparem. Existem pessoas que poderão ajudá-los, mas só o farão se vocês
forem procurá-las. Caso contrário, os únicos prejudicados serão vocês.

A faculdade não é o Paraíso (essa estação fica perto da Avenida Paulista), mas pode
melhorar a cada dia. Nós, alunos, também devemos contribuir para essa melhora.
Vocês são o futuro da universidade. Portanto, participem, reclamem, busquem seus
direitos, ajudem e, principalmente, não tenham medo de cara feia, pois isso é o
que não vai faltar.

Lembrem-se que vocês não são mais crianças e já sabem o que querem sem que
outros precisem decidir por você; então procurem o que lhes interessa: iniciação
científica, estágios, monitorias, matérias que não são obrigatórias mas que
vocês gostariam de fazer (mesmo que não tenha nada a ver com seu curso);
participação no CAMat ou na Atlética, esportes no CEPE, artigos no jornal, etc.,
etc., etc.

Não é só porque vocês podem, que vocês devem fazer tudo sozinhos, portanto lembrem que
seus amigos vão ser muito importantes para vocês e para o bom andamento do seu
curso. Procurem combinar atividades fora da faculdade, diferentes do cotidiano,
porque isso ajuda a amenizar o estresse que o dia a dia na faculdade pode
trazer.

Além de tudo, vocês não podem se esquecer de uma parte importante de suas vidas
universitárias, que é estudar. Tentem não deixar para estudar na véspera da
prova porque a probabilidade de vocês não irem bem é bem alta (exceto se vocês
tiverem uma sala do tempo em casa, que nem a do Dragon Ball). A mesma coisa se
aplica aos EPs, ainda mais que, quanto mais próximo da data de entrega, mais
erros vão aparecer e, na maioria dos casos, os EPs não são aceitos depois da
data limite de entrega e, caso sejam aceitos, não valerão a mesma nota.

Além disso, como em todo lugar, existem aquelas pessoas ranzinzas e
pentelhas que acham super bacana acabar com a graça de todo mundo,
criticar o IME e aumentar a sua baixa autoestima dizendo que os cursos
são impossíveis e que vocês não vão se formar nunca, mas não acreditem
nelas, vocês entraram aqui com um propósito. Sigam-no.  

Obs.: não se assustem com palavras e siglas que vocês não entenderam ou 
não entenderem! Continuem lendo, porque tudo será explicado em detalhes, 
mastigado, tim-tim por tim-tim. Saibam que vai faltar um monte de siglas...  
Aprendam-nas seletiva e rapidamente. Destaques para: USP, IME, MAC,
MAT, MAE, MAP, BCC, BM, BMA, LIC, BMAC, BE, CEPE, CEAGESP, CTA, RD, CAMat,
AAAMat, SSG, P1, P2, P3, P4, P5, Pn, PQP, CEC, CNPq (=\$), FAPESP
(=\$\$\$), CG, DP, REC, SUB (esta última, ou talvez as duas ou três últimas,
vocês vão conhecer bem melhor, mais cedo ou mais tarde).

\end{editorial}
