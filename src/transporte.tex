\begin{secao}{Tudo Que Vai Volta (até bixos)}

\begin{subsecao}{Ônibus}

Se vocês não têm como ir nem como voltar, temos algumas dicas:

\begin{enumerate}
  \item Trabalhem muito para comprar um carro,
  trabalhem mais para pagar a gasolina,
  venham para a USP de carro e, obrigatoriamente, dêem carona a um veterano;

  \item Peçam a uma pessoa amiga para trazê-los e buscá-los durante seus
  longos anos de IME;

  \item Conheça alguém que, por sorte, mora perto da sua casa, estuda na USP,
  tenha o mesmo horário que você, seja legal e tenha carro. Traduzindo, s-o-n-h-e;

  \item Estiquem o dedão e esperem, esperem, esperem, esperem... a boa vontade
  de alguém para dar carona;

  \item Mudem-se para uma casa perto da USP;

  \item Peguem uma bicicleta e pedalem;

\end{enumerate}

OK, vocês decidiram serem bixos normais e pegar ônibus! Mas vocês não sabem nem
dar sinal pra ele parar né? Sem problemas, vamos tentar ajudar!

Se vocês vão usar ônibus para ir ou voltar da USP, vocês precisam conhecer
basicamente dois pontos de ônibus: o ponto da FAU e o ponto da
FEA. Ambos ficam na Av. Prof. Luciano Gualberto (que a partir de agora será
chamada de Rua dos Bancos, para todo e todo o sempre).

Para tentar facilitar:
\begin{itemize}
	\item Av. Prof. Luciano Gualberto = Rua dos Bancos;
	\item Av. Prof. Lineu Prestes = Rua do HU;
	\item Av. Prof. Mello Moraes = Rua da Raia.
\end{itemize}

O mais provável é que vocês cheguem na USP por um desses
pontos e vão embora pelo outro.

Esses dois pontos ficam em lados opostos da Rua dos Bancos, um de frente para o
outro, basta atravessar a rua.

Aqui está a lista de ônibus que passam em cada um dos pontos. Para mais detalhes
sobre cada linha, vocês podem usar o site da SPTrans ou o Google Maps.

{\bf Linhas municipais:}

{\bf Ponto da FAU}

(Esses os levam pra fora da USP)
\begin{center}
	\begin{tabular}{|c|c|c|}
      \hline
	  Letreiro & Cor & Interligações (em ordem)\\
	  \hline
	  177H-10 - Metrô Santana & Azul & Metrô: L4, L2, L3 e L1\\
	  7181-10 - Term. Princ. Isabel & Laranja & CPTM: L9\\
	  7411-10 - Praça da Sé & Laranja & Metrô: L4, L2, L3 e L1\\
	  7725-10 - Rio Pequeno & Laranja & - \\
	  809U-10 - Metrô Barra Funda & Laranja & L2 \\
	  8022-10 - Metrô Butantã* & Laranja & CIRCULAR USP\\
      \hline
	\end{tabular}
\end{center}

Esses dois ônibus passam no ponto, mas estão CHEGANDO na USP, portanto não sejam
ridículos dando sinal para eles pararem e muito menos peguem um desses ônibus
nesse ponto.

\begin{center}
	\begin{tabular}{|c|c|}
	  \hline
	  Letreiro & Cor\\
	  \hline
	  701U-10 - Butantã-USP & Azul\\
	  702U-10 - Butantã-USP & Laranja\\
	  \hline
	\end{tabular}
\end{center}

{\bf Ponto da FEA}

(Esses os levam pra fora da USP)
\begin{center}
	\begin{tabular}{|c|c|c|}
      \hline
	  Letreiro & Cor & Interligações (em ordem)\\
	  \hline
	  701U-10 - Metrô Santana & Azul & Metrô: L4, L2, L3 e L1\\
	  702U-10 - Term. Pq. D.Pedro II & Laranja & Metrô: L4, L2 e L3\\
	  7725-10 - Terminal Lapa & Laranja & CPTM: L8\\
	  8012-10 - Metrô Butantã* & Laranja & CIRCULAR USP\\
      \hline
	\end{tabular}
\end{center}

Novamente, esses quatro ônibus passam no ponto, mas estão CHEGANDO na USP!
\begin{center}
	\begin{tabular}{|c|c|}
	  \hline
	  Letreiro & Cor\\
	  \hline
	  177H-10 - Butantã-USP & Azul\\
	  7181-10 - Cidade Universitária & Laranja\\
	  7411-10 - Cidade Universitária & Laranja\\
	  809U-10 - Cidade Universitária & Laranja\\
	  \hline
	\end{tabular}
\end{center}

As linhas marcadas com um * são as linhas circulares da SPTrans. Segue abaixo suas descrições!

{\bf Circulares:}

Também conhecido como ``circulenda'' ou ``secular'' (aos sábados, ``milenar'',
e aos domingos, ``anos-luz''), é o meio de transporte mais barato dentro da USP.
Foi criado para os USPianos se locomoverem dentro do Campus, mas em muitas vezes
é melhor andar do que ficar esperando. Existem 2 itinerários distintos, com
trajetos aproximadamente reversos. Fiquem atentos para não darem uma de bixo
burro e se perderem, hein?

Em 2012 foram implantadas as linhas 8012/10 e 8022/10 - Metro Butantã/Cidade Universitária,
que funcionam como circulares USP, e nós alunos não pagamos, pois elas aceitam
o bilhete USP (BUSP - Retire logo o seu!!).

Para acompanharem as rotas dos circulares e acompanhá-los em tempo real, bem como
ver os portões do campus que estão abertos e seu horário de funcionamento, pode-se
usar o aplicativo “Portões USP", disponível para Android e escrito por ex-alunos
do IME.

Como sabemos que alguns bixos ainda tem dificuldade com a leitura, colocamos o
 mapa das duas linhas nas próximas páginas.

{\bf Linhas intermunicipais:}

Agora, se vocês moram mais longe ainda (outra cidade, outro estado, outro país,
outra dimensão...) e não querem ou não podem se mudar para São Paulo, existem
algumas linhas de ônibus fretados para cidades mais próximas (ou não). Se por
acaso alguma cidade não esteja aí, procurem se informar a respeito, pois não significa
necessariamente que não haja ônibus da USP para lá. Aí estão elas:

\begin{itemize}
  \item {\bf Empresa Urubupungá.}\\
    Tel: 3658-7777
    Site: {\tt www.urubupunga.com.br}\\
    280BI1- São Bernardo do Campo (Centro)\\
    Cor: Cinza\\
    Onde pegar para sair da USP: ponto da FAU\\
    Av Magalhães de Castro, Av Marginal Pinheiros (Shopping Eldorado), Av Dos
    Bandeirantes, Av Eng. Luiz Carlos Berrini, Av Roque Petroni Jr. (Shopping
    Morumbi), Av Prof Vicente Rão, Av Cupecê (Diadema), Av Fábio Eduardo Ramos
    Esquivel (Diadema), Av Piraporinha (Diadema), Av Lucas Nogueira Garcez
    (Diadema), Av Urubupungá.

  \item {\bf Fretados Jundiaí - USP}\\
    Viação MIMO\\
    Tel: 4522-7788\\
    {\tt www.viacaomimo.com.br}\\
    Principais horários:\\
    Ida: 6h20; 7h20; 12h50; 18h00 (na Rodoviaria de Jundiaí)\\
    Volta: 11h50; 17h20; 23h00 (No ponto da FEA)

  \item {\bf São José dos Campos}\\
    Redenção\\
    tel: (12) 3931-3047\\
    {\tt www.redencaoturismo.com.br}\\
    Ida: 06h15 (na gruta em S. José)\\
    Volta:17h40 (no ponto da p2)

  \item {\bf Bragança}\\
    N.S. Fátima\\
    {\tt www.saexbra.com.br}\\
    Tel: 4032-4723 e 7344-2007\\
    Ida: 05h50 (Lgo do Tabão/Habbibs)\\
    Volta:17h00 (Av prof almeida Prado)

  \item {\bf Campinas}\\
    Sta. Cruz\\
    Tel:3868-5995\\
    {\tt www.gruposantacruz.com.br}

  \item {\bf Santos}\\
    Náutica Turismo\\
    Tel: (13) 9112-8860

  \item {\bf Santos / S. Vicente}\\
    Transul\\
    Tel: 6954-4466

  \item {\bf ABC}\\
    Dinâmica ABC-USP\\
    4352-0565 / 4109-0172

  \item {\bf Sorocaba}\\
    Fretado Diurno\\
    (15) 9715-1676 (Márcio)

\end{itemize}

\end{subsecao}


\begin{subsecao}{Veículos no Campus}
Saibam por onde entrar na USP. Lembrem-se de terem sempre a carteirinha USP ou
comprovante de matrícula com RG em mãos.
\begin{itemize}
  \item {\bf Portaria 1 (P1):} R. Afrânio Peixoto. Funciona 24h por dia todos os
    dias, mas a entrada é controlada de segunda à sexta das 20h às 05h, aos sábados
    após as 14h e domingo o dia inteiro. É por onde entram os ônibus municipais.

  \item {\bf Portaria 2 (P2):} Av. Escola Politécnica. Funciona das 5h30 às 20h
    de segunda a sexta. Entrada controlada de seg a sex das 20h às 24h. Fechada
    aos sábados, domingos e feriados. Única entrada para caminhões.

  \item {\bf Portaria 3 (P3):} Av. Corifeu de Azevedo Marques. Tem o mesmo horário
    de funcionamento da P2, porém abre aos sábados até as 14h.

  \item {\bf Portaria 1/2:} R. Eng. Teixeira Soares. Funciona de segunda a sexta das 5h30 às 20h.


%FIXME
\end{itemize}
\mapa{8012-10}
\mapa{8022-10}
\begin{itemize}

  \item {\bf Portarias de pedestre (Mercadinho, São Remo, HU, CPTM e
      Vila Indiana):} Funcionam de 2ª a 6ª, das 05 às 23 hs. Mas dá para passar
      por elas a qualquer dia e horário portando a carteirinha USP.

\end{itemize}

Aos fins de semana, ficam abertas apenas as portarias 1 e 3 (esta última fecha mais cedo). É preciso se identificar.

Os ônibus entram no sábado até as 14h e não entram no domingo, com exceção aos circulares, que entram na Universidade a qualquer hora. Se vocês vêm de carro, saibam que a universidade dispõe de bolsões de estacionamento
gratuito em torno das Unidades.

\end{subsecao}

\begin{subsecao}{Pontos de táxi}
Existem alguns pontos de táxi espalhados pela Cidade Universitária. Eis suas
localidades:

\begin{itemize}
\item Ponto da FEA / ECA (atrás do Banespa)\\
Fone: 3091-4488

\item Ponto da Reitoria\\
Fone: 3091-3556

\item Ponto do Hospital Universitário\\
Fone: 3091-3536
\end{itemize}
\end{subsecao}

\end{secao}
