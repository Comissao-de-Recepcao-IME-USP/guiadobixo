\begin{subsecao}{Estatística}

Se vocês, bixes espertos, acabaram de ingressar no curso de Bacharelado em
Estatística do IME, PARABÉNS! Se forem alunos dedicados, com certeza serão
estatísticos bem-sucedidos, pois emprego é o que não falta! Mas não vão
pensando que vai ser moleza... Aqui vai um resumo do longo caminho que vocês
terão pela frente.

O 1º ano do curso de Bacharelado em Estatística é composto por matérias básicas
dessa e de outras áreas aqui do IME. Assim, vocês vão ter que aprender Cálculo,
Álgebra Linear, Programação etc.

A partir do 2º semestre do 2º ano, o curso vai ficando mais direcionado. É nesse
semestre que será oferecida uma das disciplinas mais importantes (e mais
difíceis) do curso: Inferência Estatística.

O 3º ano é composto, quase que exclusivamente, por matérias da
Estatística. Vocês vão passar o ano todo fazendo listas e mais listas de
exercícios e vão perceber que é preciso que vocês sejam bixes (bixe é eterno e
universal, mesmo que vocês estejam no 3º ano) esforçados para conseguir o tão
sonhado diploma.

Finalmente, no último ano, vocês poderão pôr em prática um pouco de tudo o que
aprenderam, entrando em contato com pesquisadores de outras áreas, elaborando
relatórios, apresentações etc. Se vocês quiserem saber um pouco mais sobre
isso, é só procurarem o CEA (Centro de Estatística Aplicada). Certamente vocês
serão muito bem recebidos.

Não se esqueçam de que nós, veteranes da Estatística, estamos sempre à
disposição para esclarecer qualquer dúvida sobre as disciplinas e,
principalmente, sobre os professores. Além disso, sempre acontecem no IME
palestras de alunos já graduados e que estão no mercado de trabalho. Acompanhá-las
também é uma boa dica se vocês se sentem meio perdidos sobre a graduação e
o seu futuro emprego.

Enfim, aproveitem o curso, façam muitos amigos e não se esqueçam de que há vida lá
fora!

\end{subsecao}
