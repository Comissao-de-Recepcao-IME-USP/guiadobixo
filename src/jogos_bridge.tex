\begin{subsecao}{Bridge}

O Bridge é um jogo pouco conhecido aqui no Brasil mas que é muito jogado em
vários outros países pelo mundo. Usa a mesma dinâmica de vazas dos outros jogos
citados anteriormente e ainda adiciona um contrato* a ser cumprido por uma das
parcerias.

Ultimamente alguns veteranos têm se interessado bastante pelo jogo e não é
difícil encontrar uma ou mesmo duas mesas simultâneas de Bridge na vivência.

O jogo é dividido em duas partes: leilão e carteio. A parte do carteio é bem
parecida com uma Posi no King porém com uma diferença fundamental: as 13 cartas
de um dos jogadores fica à vista, tanto para o seu parceiro quanto para os seus
adversários. Além disso, durante o leilão, várias informações são trocadas entre
as parcerias para tentar se chegar ao melhor número de vazas que podem ser
feitas.

Como já deu pra perceber, o jogo é bastante diferente dos outros e seu
aprendizado é um pouquinho mais complicado, então não vamos explicar nesse Guia
todas as regras, pontuação e convenções utilizadas.

Mas se vocês se interessaram e gostariam de entender o que que esse bando de
gente vê nesse jogo, ou o que são aqueles cartões que as pessoas usam antes de
começar a jogar de verdade, não deixem de aparecer na vivência e pedir pra um
veterano te ensinar como funciona!

\end{subsecao}
