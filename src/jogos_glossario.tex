\begin{subsecao}{Glossário:}

Mico: Carta de um naipe que somente um jogador tem.

Bater mico: Jogar um mico. Pode ser uma jogada boa, mas normalmente é
ruim. Ela requer uma percepção de jogo bastante avançada que bixes,
como você, ainda não tem.

Cortar o baralho: Tirar uma quantidade de cartas de cima do baralho para mudar
o ponto onde começa a distribuição das cartas.

Bater: Acabar com suas cartas, terminando, assim, o jogo.

Vaza: Conjunto de 1 carta de cada jogador, jogadas em sentido horário. Todos
devem jogar o mesmo naipe da primeira carta, ou jogar qualquer outra carta se
não tiverem esse naipe.

Mão: Conjunto de (normalmente) 13 cartas que cada jogador recebe várias vezes
durante o jogo. Pode também ser usado como sinônimo de rodada, como
em ``Ganhei 3 pontos na mão anterior''.

Carteador: Aquele que distribui as cartas. Na verdade é mais relacionado com
quem começa jogando (normalmente começa o jogo aquele à esquerda do Carteador),
já que normalmente as cartas são embaralhadas por qualquer um.

Trunfo: Naipe escolhido para ser mais forte que os outros. Em uma vaza, a carta
mais alta do primeiro naipe aberto ganha, a não ser que uma carta com naipe do
trunfo tenha sido jogada. Nesse caso, ganha o trunfo mais alto.

Responder o naipe: Jogar uma carta do mesmo naipe que abriu a vaza, ou jogar
qualquer carta se não tiver uma carta de tal naipe.

Contrato: Número de vazas que uma parceria diz que vai fazer antes das cartas
serem jogadas.

Bolsa: Continua lendo que já chega nessa parte.

Posi(tiva) ou Neg(ativa): Para determinarmos se a mão é boa para jogar Posi ou
Neg usamos uma regrinha em que: o A vale 4 pontos, o K vale 3, o Q vale 2 e o J
vale 1 ponto. A soma de todos os pontos do jogo é igual a 40 que dividido por 4
dá 10 pontos para jogador em média. Assim, se você tem um pouco mais de 10
pontos na mão, é uma boa idéia pedir Posi, e se tiver poucos pontos, é bom
então pedir Neg.

Touchar: É quando um jogador não tem mais cartas do naipe que foi aberto e
descarta uma carta desfavorável aos seus adversários. Por exemplo, em uma Neg
Homens, ele pode jogar um valete em uma vaza que seus adversários estão fazendo.

Cortar:  É quando um jogador não tem mais cartas do naipe que foi aberto e joga
um trunfo.

Baldar: É quando um jogador não tem mais cartas do naipe que foi aberto e
descarta uma carta.

Destrunfar: É abrir uma mão com uma carta do trunfo e fazer com que todos
respondam o naipe com o objetivo de diminuir o número de trunfos dos
adversários.

Void: É quando o jogador vem sem cartas de um determinado naipe ou elas acabam
no decorrer do jogo. “Vim void em paus”. Quer dizer que quando o jogador
recebeu suas 13 cartas, nenhuma delas era do naipe de paus.

Quinto/Quarto/Terceiro: É a distribuição dos naipes em nossa mão. Se temos 3
cartas de copas, por exemplo, dizemos que estamos terceiro em copas. Se tem 1
carta de espadas, dizemos que estamos primeiro em espadas.

Finesse: É uma aposta estatística no posicionamento das cartas para
fazer sua jogada.

\end{subsecao}
