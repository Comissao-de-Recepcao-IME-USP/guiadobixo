\begin{secao}{CEC - Centro de Ensino de Computação}

\begin{subsecao}{O que é?}

O Centro de Ensino de Computação (CEC) foi criado em 1993, pelo
Departamento de Ciência de Computação do IME, com o objetivo de
oferecer apoio às aulas dos cursos de graduação do Instituto e para os
treinamentos/cursos oferecidos à comunidade USP e não USP, através de
seus laboratórios com recursos de informática.
Está localizado no Bloco B do IME, ao lado da Seção de Alunos. O horário
de funcionamento do CEC é de 2a a 6a das 8h00 às 22h00 e nos recessos
escolares o horário de abertura é alterado para 9h00.

\end{subsecao}

\begin{subsecao}{Estrutura}

O CEC possui 137 computadores, distribuídos em 04 (quatro)
salas/laboratório, sendo 02 (duas) delas com capacidade para até 60
pessoas e com prioridade de uso para as aulas de graduação do IME e
para os cursos/treinamentos.
Nos outros laboratórios, os computadores são dual boot (Linux e Windows) e
possuem os aplicativos solicitados pelos docentes do Instituto. São
utilizados pelos alunos de graduação dos cursos do IME para a realização
dos trabalhos e pesquisas acadêmicos.

\end{subsecao}

\begin{subsecao}{Como funciona?}

Possui rede cabeada e para utilizá-la é necessário ter uma senha de
acesso (conta de usuário). Você precisa ser aluno matriculado no IME para
solicitar a senha de uso, pois o CEC não é uma sala pró-aluno (para ver
como as salas pró-aluno funcionam acesse www.usp.br/proaluno/). 
Para solicitar a senha de acesso, envie e-mail para cec-senha@ime.usp.br,
assunto: CEC senha, informe seu nome completo e número USP. Utilize
seu e-mail institucional (@usp.br - se esqueceu entre em
www.pedidoemail.usp.br).

Informações, como cota de disco/impressão, regras de utilização e faqs,
podem ser visualizadas em http://www.cec.ime.usp.br.

\end{subsecao}

Obs.: a partir de 2011, alguns pedidos de login são recebidos com maior simpatia
e resultam em logins e senhas iguais para Windows e Linux, ou seja, uma senha a
menos para você ter de lembrar.

Obs.2: bixes, ao frequentarem o CEC, fiquem atentos ao ar-condicionado. Se
estiver ligado, deem preferência a usarem calças, blusas, jaquetas e meias de
lã. Cobertores são opcionais. Se não, boa sorte ou \textit{hasta la vista}!

\end{secao}
