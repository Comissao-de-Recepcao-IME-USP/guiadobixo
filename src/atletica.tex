\begin{secao}{A Atlética} %REFTIME - Quase tudo aqui é REFTIME...

\begin{subsecao}{O que é a AAAMat?}

É a Associação Atlética Acadêmica da Matemática. É formada por alunos do IME e
tem como objetivo organizar e divulgar atividades esportivas e eventos (festas)
para a comunidade IMEana, visando seu desenvolvimento físico e mental,
e também a integração ($\int$) entre os alunos de diferentes cursos, de diversos
anos e de outras faculdades.

Dentre essas atividades há a IMEteria, a bateria dos alunos do nosso instituto,
que acompanha as equipes IMEanas em jogos, eventos, etc.

\end{subsecao}
%%%%%%%%%%%%%%%%%%%%%%%%%%%%%%%%%%%%%%%%%%%%%%%%%%%%%%%%%%%%%%%%%%%%%%%%%%%%%%%%

Algumas das atividades da Atlética:

\begin{subsecao}{Atividades esportivas internas:}

Na AAAMat há responsáveis para organizar treinos e campeonatos das seguintes
modalidades: futebol de campo, futsal, basquete, vôlei, handebol, atletismo,
natação, tênis de mesa, tênis de campo, xadrez, judô, beisebol, softbol e
frisbee. Para que os treinos continuem ocorrendo, precisamos sempre da presença
de novos atletas. Então, se você está interessado em aprender um desses esportes
ou já tem familiaridade com algum deles, avise-nos. Venha treinar com nossos
times e competir com eles.

Os dias e horários dos treinos e jogos serão sempre informados através do \textit{site}
e do mural da Atlética, localizado na entrada do bloco B.

Além disso, a Atlética promove anualmente campeonatos internos com o objetivo de
integrar ($\int$) os alunos do IME. Já foram promovidos campeonatos de xadrez,
futsal, truco, Winning Eleven, LoL, Mario Kart, Super Smash Bros, Mario Tenis,
Guitar Hero, sinuca e seletivas de natação, tênis de campo e tênis de mesa.

Ideias e sugestões sobre novas modalidades, campeonatos, inters, etc. são sempre
 bem-vindas! Conversem com a gente!

\end{subsecao}
%%%%%%%%%%%%%%%%%%%%%%%%%%%%%%%%%%%%%%%%%%%%%%%%%%%%%%%%%%%%%%%%%%%%%%%%%%%%%%%%

A AAAMat também representa o IME em diversos campeonatos universitários. São
eles:

\begin{subsecao}{BichUSP}

Como o próprio nome já diz, o BichUSP é um campeonato disputado entre as
faculdades da USP em que apenas os bixos participam. O campeonato acontece logo
nas primeiras semanas de aula, sempre nos finais de semana, contando sempre com
a participação das torcidas incentivando seus times e já mostrando a vocês, bixos,
a rivalidade histórica entre algumas faculdades.

Esse ano ele acontecerá nas seguintes datas:
\begin{itemize}
  \item 27 e 28/2/2016 - Atletismo, Natação, Tênis de Mesa, Xadrez e Tênis de
        Campo.
  \item 5 e 6/3/2016 - Handebol e Futsal.
  \item 12 e 13/3/2016 - Futebol de Campo, Rugby e Softbol.
  \item 19 e 20/3/2016 - Vôlei e Futsal.
\end{itemize}

O IME tem como tradição sempre revelar grandes talentos e grandes equipes.
Abaixo seguem vitórias marcantes nesse campeonato:

\begin{center}
	\begin{tabular}{c|c}
	  \hline
	  Ano & Campeão\\
	  \hline
	  2005 & Basquete Masculino \\
	  2005 & Tênis de Mesa Feminino \\
	  2007 & Atletismo Masculino\\
	  2009 & Atletismo\\
	  2011 & Tênis de Campo Feminino\\
	  2012 & Basquete Feminino\\
	  2014 & Tênis de Mesa Masculino\\
	  2014 & Futsal Masculino\\
	  2015 & Futsal Masculino\\
 	  \hline
	\end{tabular}
\end{center}

%REFTIME outras colocações importantes do último ano

bixos, neste ano cabe a vocês mostrarem a raça IMEana e correrem atrás do
troféu!

Contudo, tenham em mente que o importante no BichUSP não é saber jogar, mas sim
ter vontade de participar.  Portanto, se você acha que não joga muito bem, não
tem problema. Se por acaso você tem alergia a esporte, compareça e torça pelos
nossos times. O BichUSP é uma ótima maneira de começar a conhecer melhor os
VETERANOS e calouros do IME e de outras faculdades.

Fique atento para os treinos especiais do BichUSP, visitando o \textit{site} e
observando o mural da Atlética.

\end{subsecao}
%%%%%%%%%%%%%%%%%%%%%%%%%%%%%%%%%%%%%%%%%%%%%%%%%%%%%%%%%%%%%%%%%%%%%%%%%%%%%%%%
\begin{subsecao}{Copa USP e Jogos da Liga}

São os campeonatos internos da USP. A maioria das faculdades da USP participa
dessas competições. A Copa USP ocorre no 1º semestre, e os Jogos da Liga,
no segundo. Os jogos ocorrem sempre aos finais de semana.

A Copa USP é um dos mais tradicionais campeonatos da USP, havendo uma
rivalidade muito grande entre algumas faculdades. Os times são separados em
duas divisões, de acordo com as colocações nos anos anteriores. O IME, embora
tenha poucos atletas, teve a maioria de seus times disputando a Série Azul,
divisão mais forte, e muitos deles alcançando resultados expressivos.

%TODO Transformar parágrafo abaixo numa tabela
O Handebol Masculino foi vice-campeão dos Jogos da Liga em 2006 e campeão da
Copa USP em 2009. O Basquete Masculino se destacou em 2004 e em 2005, sendo
campeão tanto da Copa USP como dos Jogos da Liga nos dois anos seguidos e ficou
em 3º lugar em 2006, assim como as meninas do Basquete. O Futebol de
Campo e o Xadrez ficaram em 1º lugar na Copa USP em 2006. O Vôlei
Masculino foi 4º colocado da Copa USP Série Azul em 2007 e campeão da Copa
USP em 2010, além de vice nos Jogos da Liga em 2010. O Futebol Masculino chegou
a 3º nos Jogos da Liga em 2010.
Em 2008, tivemos excelentes resultados nos Jogos da Liga. Ficamos entre os
primeiros lugares em praticamente todas as modalidades, colocando-nos entre as
maiores atléticas da USP.

Como dissemos, o IME é famoso por sua inflamada torcida, que muitas vezes já
ajudou nossas equipes nos momentos mais difíceis. Portanto, se você pratica
esporte terá muitas opções e, caso você não jogue nada, junte-se à nossa torcida!

\end{subsecao}
%%%%%%%%%%%%%%%%%%%%%%%%%%%%%%%%%%%%%%%%%%%%%%%%%%%%%%%%%%%%%%%%%%%%%%%%%%%%%%%%
\begin{subsecao}{BIFE}

O BIFE é a competição mais tradicional do IME, onde você verá mais de 150
IMEanos confraternizando, jogando, torcendo e festejando juntos.

Trata-se de um campeonato entre dez faculdades da USP. As competições acontecem em cidades do interior do Estado, onde ficamos 4 dias
alojados para cochilos de poucas horas, com diversos jogos durante o dia e as
melhores festas durante a noite e madrugada.

BIFE são as iniciais das quatro fundadoras: BIO, IME, FAU e ECA.
Além dessas, participam da organização as faculdades: Vet, Geo, Física, FFLCH,
Psico e Química. E de vez em quando temos a ilustre presença de convidadas.

A competição terá sua décima sétima edição em 2016, e cresce cada vez mais. %REFTIME
São quase dois mil alunos da USP \sout{enchendo a cara} confraternizando e
torcendo loucamente pelos seus tIMEs.

%REFTIME
Temos um histórico muito bom neste Inter:
\begin{center}
	\begin{tabular}{c|c|c}
	  Ano & Cidade & Campeão\\
	  \hline
	  1999 & Jacareí & IME\\
	  2000 & Não Houve & - \\
	  2001 & Serra Negra & IME\\
	  2002 & Socorro & ECA\\
	  2003 & São Sebastião & IME\\
	  2004 & Cruzeiro & FFLCH\\
	  2005 & Jacareí & FFLCH\\
	  2006 & Lorena & IME\\
	  2007 & Piedade & IME\\
	  2008 & Itapeva & IME\\
	  2009 & Cruzeiro & IME\\
	  2010 & Barra Bonita & IME\\
	  2011 & Casa Branca & IME\\
	  2012 & Barra Bonita & IME\\
	  2013 & Sumaré & ECA\\
	  2014 & Cidade/Araraquara & FFLCH\\
	  2015 & Taquaritinga & FFLCH\\
	  2016 & Registro & FFLCH\\
	  2017 & ???? & ????
	\end{tabular}
\end{center}

%REFTIME
Sim, somos decacampeões do BIFE.  Infelizmente, não conseguimos o título nos
últimos quatro anos, mas em 2017, cabem a vocês, bixos, se juntarem aos tIMEs e
ajudá-los a recuperarmos o lugar de onde nunca deveríamos ter saído! Também é
papel de vocês vibrar junto com o IME, ir em todas as festas e aproveitar ao máximo o
seu primeiro ano na universidade. O ano de vocês serem bixos a vontade.

\end{subsecao}
%%%%%%%%%%%%%%%%%%%%%%%%%%%%%%%%%%%%%%%%%%%%%%%%%%%%%%%%%%%%%%%%%%%%%%%%%%%%%%%%

Outras atividades:

\begin{subsecao}{Vendas}

Além desse monte de atividades citadas acima, a Atlética também vende
adesivos, calculadoras, canecas, chaveiros, samba-canção, cadernos, estojos,
lápis, borracha, régua, caneta, Mousepad, camisetas e agasalhos do IME,
entre outros. Não deixe de visitá-la e adquirir seus produtos.

\end{subsecao}
%%%%%%%%%%%%%%%%%%%%%%%%%%%%%%%%%%%%%%%%%%%%%%%%%%%%%%%%%%%%%%%%%%%%%%%%%%%%%%%%

\begin{subsecao}{Festas}

A Atlética e o CAMat já promoveram muitas festas e \textit{happy hours}. O sucesso
deles depende em grande parte da participação dos IMEanos. Convide seus
amigos e participe das festas, outra maneira de conhecer e se
integrar ($\int$) com seus colegas bixos e VETERANOS.

Promovemos as festas da IMEteria, ForrIME, RockIME, I Will SurvIME, Rompendo o hIMEn,
Melhores do Ano, Festival de Bandas (junto com o CAMat), entre outras, e
auxiliamos as festas pré-BIFE (Desmame, Engorda e Abate). Todas imperdíveis.

E lembre-se, as festas no seu primeiro ano na universidade são as melhores de
todas!

\end{subsecao}
%%%%%%%%%%%%%%%%%%%%%%%%%%%%%%%%%%%%%%%%%%%%%%%%%%%%%%%%%%%%%%%%%%%%%%%%%%%%%%%%

\begin{subsecao}{Como falar com a Atlética?}

Sempre que tiverem alguma dúvida, reclamação ou sugestão, vocês podem falar
pessoalmente com qualquer membro da Atlética, participar da nossa lista de e-mails \url{aaamat-diretoria@googlegroups.com} ou ir até a Atlética, na sala
B-18, dentro da Vivência.  Perguntem por aí, vejam no mural, liguem para {\tt 3091-6378} ou mandem um \textit{e-mail} para \url{atletica@ime.usp.br}.
Saibam todas as novidades nos seguintes endereços eletrônicos:

\textit{Site} da Atlética: \url{https://www.ime.usp.br/~atletica}

Curta nossa página no Facebook: \url{fb.com/aaamat.ime}

Siga-nos no Twitter: \url{@aaamat}

Siga-nos no Instagram: \url{@atleticaime}

A Atlética inicia 2016 esperando ideias e sugestões que serão muito bem %REFTIME
recebidas. Não deixe de colaborar e participar dos eventos por ela promovidos.

Vocês são SEMPRE bem-vindos em nossa sala, atividades e eventos!

\end{subsecao}
\end{secao}
