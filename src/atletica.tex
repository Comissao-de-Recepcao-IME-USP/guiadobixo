\begin{secao}{A Atlética} %REFTIME - Quase tudo aqui é REFTIME...

\begin{subsecao}{O que é a AAAMat?}

É a Associação Atlética Acadêmica da Matemática - a entidade mais divertida do
mundo!!!! - que tem como objetivo trazer os melhores momentos da sua vida
universitária! A Atlética é formada por um grupo de IMEanos (gestão) que é
responsável por organizar atividades esportivas e eventos (festas, pizzadas,
premiações, etc) para a comunidade IMEana.

\end{subsecao}
%%%%%%%%%%%%%%%%%%%%%%%%%%%%%%%%%%%%%%%%%%%%%%%%%%%%%%%%%%%%%%%%%%%%%%%%%%%%%%%%

Algumas das atividades da Atlética:

\begin{subsecao}{Atividades esportivas internas:}

Na AAAMat existem diretores de modalidade (DMs) que são pessoas responsáveis
pelos treinos e campeonatos dos seguintes esportes: futebol de campo, futsal,
basquete, vôlei, handebol, atletismo, natação, tênis de mesa, tênis de campo,
xadrez, judô, beisebol, softbol, ultimate frisbee, bridge, rugby e e-sports
(\textit{League of Legends}, \textit{Hearthstone} e
\textit{Counter Strike: Global Offense}). Além disso, contamos com a nossa 
amada BatIMEduca, bateria do IME juntamente com a Pedago.

Os dias e horários dos treinos/jogos de cada uma dessas modalidades serão
sempre informados através do site e do mural da Atlética, localizado na entrada
do bloco B (aquele mural verde, colado na parede em frente a lanchonete).

Com essa variedade de modalidades, não tem desculpa pro sedentarismo hein,
bixo? Se você não conhece nenhuma delas, a gente te apresenta e se já conhece,
vem dar aquele “ooooi, sumido!!!” pra aquele esporte que você largou por conta
da fuvest! Ah, não esqueça de torcer pelos nossos atletas! Raça e coração, IME!

Além disso, a Atlética promove anualmente campeonatos internos daqueles jogos
que a gente passa hooooras fritando em casa. Já foram promovidos campeonatos de
Winning Eleven, LoL, Mario Kart, Super Smash Bros, Mario Tenis, Guitar Hero e
sinuca.

Ideias e sugestões sobre novas modalidades, campeonatos, inters, etc. são
sempre muito bem-vindas! Conversem com a gente!

\end{subsecao}
%%%%%%%%%%%%%%%%%%%%%%%%%%%%%%%%%%%%%%%%%%%%%%%%%%%%%%%%%%%%%%%%%%%%%%%%%%%%%%%%

A AAAMat também representa o IME em diversos campeonatos universitários. São
eles:

\begin{subsecao}{BichUSP}

De nome intuitivo e charmoso, o BichUSP é um campeonato disputado entre as
faculdades da USP em que apenas os bixos (VOCÊS!) participam. O campeonato
acontece logo nas primeiras semanas de aula, sempre aos finais de semana. Aqui,
vocês têm a chance de suar a camisa IMEana pela primeira vez e ver a torcida
indo ao delírio em cada jogada - ganhando ou perdendo, seus veteranos estarão
vibrando por vocês!

%REFTIME
Esse ano o BichUSP acontecerá nas seguintes datas:

\begin{itemize}
  \item Tênis todos os dias do BichUSP
  \item 10 e 11/03/2018 - Basquete e Handebol.
  \item 17 e 18/03/2018 - Natação, Atletismo, Tênis de Mesa e Xadrez.
  \item 24 e 25/03/2018 - Futebol de campo e Rugby.
  \item 07 e 08/04/2018 - Vôlei e Futsal.
\end{itemize}

``Mas, Atlética, eu não sei jogar nenhum desses esportes :('' - Não tem
problema, a gente te ensina! Teremos treinos especiais para que vocês conheçam
a modalidade, os DM’s, os técnicos, a gente e os outros bixos que te
acompanharão nesse momento único da graduação!

Se você acha que tem alergia a esportes, dá uma chance da gente te mostrar o
contrário! São várias modalidades com várias dinâmicas diferentes, alguma delas
com certeza vai se encaixar no que você gosta! Se quiser só assistir no começo
e vir torcer com a gente, apareçam nos jogos que nós gritamos "VERMELHO E
BRANCO ATÉ MORRER" todos juntos!

Pra vocês se inspirarem, fizemos essa tabelinha que mostra quantos bixos
brilharam em anos anteriores. Estamos ansiosos pra completar ela com as
conquistas que virão esse ano:

%REFTIME
\begin{center}
  \begin{tabular}{c|c}
    \hline
    Ano & Campeão\\
    \hline
    2005 & Basquete Masculino \\
    2005 & Tênis de Mesa Feminino \\
    2007 & Atletismo Masculino\\
    2009 & Atletismo\\
    2011 & Tênis de Campo Feminino\\
    2012 & Basquete Feminino\\
    2014 & Tênis de Mesa Masculino\\
    2014 & Futsal Masculino\\
    2015 & Futsal Masculino\\
    2016 & Vôlei Masculino\\
    2017 & Xadrez, Rugby Misto e Rugby Feminino (IME+EFEE)\\
    2018 & Xadrez, Futebol de Campo F.\\
    2019 & VEM BIXES!\\
    \hline
  \end{tabular}
\end{center}

\end{subsecao}
%%%%%%%%%%%%%%%%%%%%%%%%%%%%%%%%%%%%%%%%%%%%%%%%%%%%%%%%%%%%%%%%%%%%%%%%%%%%%%%%
\begin{subsecao}{Copa USP}

A Copa USP é o primeiro campeonato após o BichUSP, e existem duas séries (Azul:
1º divisão e Laranja: 2º divisão). São nesses jogos que colocamos em prática
tudo o que fizemos nos treinos semanais para brilharmos nos jogos da fase de
grupos e então seguir arrasando nos jogos mata-matas.

\end{subsecao}
%%%%%%%%%%%%%%%%%%%%%%%%%%%%%%%%%%%%%%%%%%%%%%%%%%%%%%%%%%%%%%%%%%%%%%%%%%%%%%%%
\begin{subsecao}{Jogos da Liga}

Acontece no segundo semestre (UFA, já passou a P1 de cálculo, vem Cálculo 2!!!
\sout{Ou não}) e nessa competição não existe separação por séries. Somente as
faculdades da USP jogam e as disputas são sorteadas para formarem grupos e
então apenas os melhores colocados seguem para a fase final. Essa é a
oportunidade perfeita para gritar um "CHUPA POLI" na arquibancada.

\end{subsecao}
%%%%%%%%%%%%%%%%%%%%%%%%%%%%%%%%%%%%%%%%%%%%%%%%%%%%%%%%%%%%%%%%%%%%%%%%%%%%%%%%
\begin{subsecao}{NDU}

Esse campeonato acontece duas vezes ao ano, e várias faculdades de São Paulo
(tanto das USP quanto algumas não-USP) competem na cidade em busca dos melhores
resultados. Confira com o DM da modalidade se o time está participando da
competição.

\end{subsecao}
%%%%%%%%%%%%%%%%%%%%%%%%%%%%%%%%%%%%%%%%%%%%%%%%%%%%%%%%%%%%%%%%%%%%%%%%%%%%%%%%
\begin{subsecao}{BIFE}

O BIFE É O MELHOR EVENTO ESPORTIVO DO MUNDO! As iniciais das quatro fundadoras
(Bio, IME, Fau e Eca) formam a sigla que dá nome a esses jogos universitários
que a gente tanto ama. Trata-se de um campeonato entre nove faculdades da USP:
a VET, GEO, Fisica, FFLCH, Química e, é claro, as quatro fundadoras já citadas.

Funciona assim: em um determinado feriado, jogadores, torcedores, festeiros e
simpatizantes se deslocam até alguma cidade do interior do Estado. A cidade nos
dá um alojamento (leia-se: local para tomar um banho quentinho e descansar no
aconchego de sua barraca), alguns ginásios e um local para as festas. São
quatro dias muito divertidos e engraçados, onde há rivalidade apenas dentro de
quadra - porque fora é muito amor e integração!

Nosso histórico neste Inter é de parar o trânsito! Olhem só:

%REFTIME
\begin{center}
  \begin{tabular}{c|c|c}
   Ano & Cidade & Campeão\\
   \hline
   1999 & Jacareí & IME\\
   2000 & Não Houve & - \\
   2001 & Serra Negra & IME\\
   2002 & Socorro & ECA\\
   2003 & São Sebastião & IME\\
   2004 & Cruzeiro & FFLCH\\
   2005 & Jacareí & FFLCH\\
   2006 & Lorena & IME\\
   2007 & Piedade & IME\\
   2008 & Itapeva & IME\\
   2009 & Cruzeiro & IME\\
   2010 & Barra Bonita & IME\\
   2011 & Casa Branca & IME\\
   2012 & Barra Bonita & IME\\
   2013 & Sumaré & ECA\\
   2014 & Cidade/Araraquara & FFLCH\\
   2015 & Taquaritinga & FFLCH\\
   2016 & Registro & FFLCH\\
   2017 & Avaré & FFLCH\\
   2018 & Casa Branca & ICBIÓ\\
   2019 & ??? & VAMO IME!!!
  \end{tabular}
\end{center}

%REFTIME
Nesses últimos seis anos não conseguimos o título MAS ESSE ANO VAI! Nossos 
times contam com vocês para que juntos possamos ser hendecampeões! (essa
palavra existe mesmo)

\end{subsecao}
%%%%%%%%%%%%%%%%%%%%%%%%%%%%%%%%%%%%%%%%%%%%%%%%%%%%%%%%%%%%%%%%%%%%%%%%%%%%%%%%
\begin{subsecao}{Títulos}

Fruto de muito treino, empenho, suor, torcida e amor pelo IME-USP, reunimos
abaixo algumas de nossas conquistas:

%REFTIME
\begin{center}
  \begin{tabular}{c|c|c|c}
    Ano & Campeonato & Modalidade & Colocação\\
    \hline
    2006 & Jogos da liga  & Handebol Masc.  & 2º\\
    2009 & Copa USP       & Handebol Masc.  & 1º\\
    2012 & Copa USP       & Handebol Masc.  & 2º\\
    2016 & IMEACHECA      & Handebol Masc.  & 2º\\
    2018 & Jogos da Liga  & Handebol Masc.  & 2º\\
    2017 & Copa USP       & Baseball Fem.   & 2º\\
    2011 & BOBPAI         & Baseball        & 1º\\
    2016 & BOBPAI         & Baseball        & 2º\\
    2016 & Liga Paulista  & Baseball        & 3º\\
  \end{tabular}
\end{center}
\begin{center}
  \begin{tabular}{c|c|c|c}
    Ano & Campeonato & Modalidade & Colocação\\
    \hline
    2016 & Wakaba         & Softball        & 2º\\
    2016 & Softparty      & Softball        & 1º\\
    2011 & Jogos da liga  & Basquete Fem.   & 1º\\
    2012 & Copa Camp      & Basquete Fem.   & 2º\\
    2012 & Copa USP       & Basquete Fem.   & 3º\\
    2014 & Interfarofa    & Futsal Fem.     & 1º\\
    2015 & NDU            & Futsal Fem.     & 2º\\
    2017 & Copa USP       & Futsal Fem.     & 1º\\
    2017 & Jogos da Liga  & Futsal Fem.     & 2º\\
    2017 & IMEACHCA       & Futsal Fem.     & 1º\\
    2015 & Camp. G-4      & Vôlei Masc.     & 2º\\
    2016 & Copa USP       & Vôlei Masc.     & 1º\\
    2017 & Copa USP       & Vôlei Masc.     & 2º\\
    2017 & NDU            & Vôlei Masc.     & 1º\\
    2018 & Copa USP       & Vôlei Masc.     & 1º\\
    2018 & BIFE           & Vôlei Masc.     & 2º\\
    2018 & NDU            & Vôlei Masc.    & 2º\\
    2015 & Camp. G-4      & Handebol Fem.   & 1º\\
    2015 & Integramix     & Handebol Fem.   & 1º\\
    2016 & CUPA           & Handebol Fem.   & 2º\\
    2016 & Interfarofa    & Handebol Fem    & 1º\\
    2018 & Copa USP       & Handebol Fem.   & 1º\\
    2009 & Intercalouros  & Atletismo       & 1º\\
    2011 & LUPAA          & Atletismo       & 1º\\
    2015 & Integramix     & Futebol Campo   & 1º\\
    2017 & Copa USP       & Futebol Campo M & 1º\\
    2011 & Copa USP       & Futsal Masc.    & 1º\\
    2012 & Copa Camp      & Futsal Masc.    & 1º\\
    2013 & NDU            & Futsal Masc.    & 1º\\
    2013 & Jogos da Liga  & Futsal Masc.    & 2º\\
    2015 & Integramix     & Futsal Masc.    & 1º\\
    2015 & NDU            & Futsal Masc.    & 2º\\
    2016 & IMEACHECA      & Vôlei Fem.      & 1º\\
    2016 & Gran Prix USP  & Vôlei Fem.      & 3º\\
    2017 & Gran Prix USP  & Vôlei Fem.      & 3º\\
    2017 & Copa USP       & Jiu-jitsu       & 2º\\
    2017 & Jogos da Liga  & Tênis Campo M   & 2º\\
    2017 & NDU            & Xadrez          & 1º\\
    2017 & Copa USP       & Xadrez          & 3º\\
    2017 & Jogos da Liga  & Xadrez          & 2º\\
    2017 & TUES           & Hearthstone     & 2º\\
    2018 & BIFE           & Basquete Masc.  & 1º\\
    2018 & Jogos da Liga  & Basquete Masc.  & 2º\\
    2018 & BIFE           & Natação Masc.   & 2º\\
    2018 & Jogos da Liga  & Natação Masc.   & 2º\\
    2018 & BIFE           & Rugby Masc.     & 2º\\
    2018 & BIFE           & Futebol Campo F.& 2º
  \end{tabular}
\end{center}

\end{subsecao}
%%%%%%%%%%%%%%%%%%%%%%%%%%%%%%%%%%%%%%%%%%%%%%%%%%%%%%%%%%%%%%%%%%%%%%%%%%%%%%%%
Outras atividades:

\begin{subsecao}{Vendas}

A Atlética também quer te ajudar a vestir o vermelho e branco (que, a essa
altura, já corre em suas veias! :D) e traz pra você diversos produtos
personalizados, tais como: adesivos, tatuagens, canecas, talabartes, chaveiros,
samba-canção, cadernos, estojos, mousepads, chinelos, camisetas e agasalhos do
IME, pra você sair por aí esbanjando seu amor pelo IME-USP $<$3

Chegou na hora da prova e esqueceu seu kit bixo em casa? A gente te salva --
também vendemos lápis, borracha, régua, caneta e calculadora!

\end{subsecao}
%%%%%%%%%%%%%%%%%%%%%%%%%%%%%%%%%%%%%%%%%%%%%%%%%%%%%%%%%%%%%%%%%%%%%%%%%%%%%%%%
\begin{subsecao}{Festas}

%REFTIME
A Atlética e o CAMat já promoveram muitas festas e happy hours. Atualmente,
promovemos a I Will SurvIME, a FofIMEduca (Com a galera da Pedagogia), o 
JuniME, a Melhores do Ano, o sarau e alguns HHs durante o ano e auxiliamos 
as festas pré-BIFE (Desmame, Engorda e Abate). Todas imperdíveis!
Esperamos vocês!

\end{subsecao}
%%%%%%%%%%%%%%%%%%%%%%%%%%%%%%%%%%%%%%%%%%%%%%%%%%%%%%%%%%%%%%%%%%%%%%%%%%%%%%%%
\begin{subsecao}{Como falar com a Atlética?}

A salinha da AAAMat é a B-18. Ela fica dentro da vivência e é pequena, mas
sempre cabe mais um! Sempre que precisarem conversar com a Atlética, vocês
podem ir até lá e falar com qualquer membro da gestão. Além disso, também temos
outros meios de contato, tais como o telefone: (11) 3091-6378 ou o e-mail:
atletica@ime.usp.br

Se quiserem ficar por dentro de tudo que acontece na atlética, vocês também
podem:

\begin{itemize}
  \item Acompanhar o site da Atlética: https://www.ime.usp.br/~atletica
  \item Curtir nossa página no Facebook: fb.com/aaamat.ime
  \item Seguir a  gente no Instagram: @aaamat_imeusp
  \item Participar da nossa lista de e-mails: basta enviar um e-mail em branco
        para aaamat-diretoria+subscribe@googlegroups.com e seguir as instruções
        enviadas pro seu e-mail.
\end{itemize}

%REFTIME
A Atlética inicia 2019 esperando vocês, bixos lindos, para que juntos possamos
trazer muitos títulos e troféus para casa! É importantíssimo que vocês saibam
que estamos abertos para qualquer tipo de crítica, dúvida, ideia ou sugestão.

Vocês são SEMPRE muito bem-vindos em nossa sala, atividades, times e eventos!
\end{subsecao}
\end{secao}
