\begin{secao}{Dicas}

Como todo bixo é perdido, aqui vão algumas dicas pra você não ficar perguntando o tempo todo:

%TODO Checar isso aqui.
{\bf Biblioteca:} Para se inscrever, basta comparecer à biblioteca munido do
cartão USP, definitivo ou provisório. O empréstimo é unificado, sendo possível
retirar livros em qualquer uma das bibliotecas da USP mediante apresentação do
cartão USP. Não deixe de assistir à palestra da Biblioteca na primeira semana
de aula! Consulte a programação.

{\bf Correios:} tem uma agência em frente à prefeitura do campus.

{\bf Bancos e Caixas Eletrônicos:} Agências do Santander, Bradesco,
Nossa Caixa, Banco do Brasil, HSBC e Itaú na Av. Prof. Luciano
Gualberto; Caixas eletrônicos perto do bandejão da Física, em frente à dentro da reitoria, no CEPE, enfim, em lugares aleatórios.

\begin{subsecao}{Cultura na USP}

{\bf Museu de Arqueologia e Etnologia (MAE):} ao lado da Prefeitura do Campus.

% TODO Descobrir se o MAC está reformando ou será demolido?
{\bf Museu de Arte Contemporânea (MAC):} próximo ao CRUSP.

{\bf Paço das Artes:} em frente à Faculdade de Educação.

{\bf Museu do Brinquedo:} fica na Faculdade de Educação, Bloco B.

{\bf Museu do Crime:} na Academia de Polícia perto da P1.

{\bf Museu do Instituto Oceanográfico:} adivinha?

{\bf Museu da Geociências:} Lá mesmo.

{\bf Instituto Butantan:} Próximo à História.

{\bf Museu da Faculdade de Veterinária} Perto do P3. Ponto final do circular.

{\bf Estação Ciência:} fica fora da USP. R. Guaicurus, 1274 - Lapa

{\bf Museu Paulista, vulgo Ipiranga: }também fora da USP, no Parque da
Independência - S/N  - Ipiranga.

{\bf Museu de Zoologia: }também fora da USP, na Av. Nazaré, 481  -
Ipiranga (atrás do Ipiranga).

{\bf CinUSP Paulo Emílio} Dentro do campus, próximo ao bandejão central, existe uma sala de cinema. Durante todo o ano ocorrem várias mostras cinematográficas, nas quais são exibidos inúmeros filmes. As sessões são gratuitas e a programação pode ser conferida no seguinte site: {\tt http://www.usp.br/cinusp/}


\end{subsecao}

\begin{subsecao}{Onde beber?}

Se você é um bixo que curte entornar os canecos de vez em quando, então agora
deve estar pensando, até que enfim vamos falar de algo que presta! Lembre-se de
levar um VETERANO para pagar a ele algumas doses, pois graças a eles que você
está recebendo essas dicas, então sem mais delongas, aqui vão alguns lugares
firmeza para se fazer isso à vontade:

{\bf FAU:} O grêmio da FAU sempre vende cervejas, cada hora uma marca e tamanho diferente. Para chegar, basta ir no primeiro andar à direita até o fim.

{\bf Física:} Embora seja a Física, lá é um lugar gostoso para tomar vários
tipos de cerveja, que só é vendida após as 18:00 horas, e comer alguns
salgados baratos. No próprio C.A. deles se vende cerveja.

{\bf FFLCH:} Vá até o prédio da História e Geografia e procure o Aquário, se estiver alguém lá dentro eles vendem cerveja.

%FIXME Qual é a situação atual do QiB?
{\bf ECA:} Famosa Quinta i Breja, adivinha que dia da semana isso acontece?
Acertou bixo, de quinze em quinze dias nas quintas-feiras, parabéns, como
prêmio você pode pagar uma breja para o seu VETERANO favorito!

{\bf Rei das batidas:} Muito famoso não só por quem estuda na USP, o Rei,
como é carinhosamente chamado, fica fora da USP, saindo pela P1. Vende
diversas batidas e, é claro, cerveja...

{\bf Bar do frango:} Não se sabe qual é o verdadeiro nome desse bar, mas ele é
uma alternativa ao Rei, quando este se encontra muito lotado. Apesar do péssimo atendimento e do aspecto horrível do lugar é bom para beber sem
tumulto. É frequentado principalmente no começo do ano. Se encontra atrás do
Rei.

É claro que também há as festas, que ocorrem em qualquer lugar da USP, e nelas
há ainda outras misturas alcoólicas impossíveis.

É nosso dever informar também que o álcool é uma substância altamente viciante
e, quando bebida em excesso, pode trazer graves consequências à sua saúde, às
vezes à saúde de outra pessoa, à sua família e principalmente ao seu bolso.
Portanto, não se esqueça de abastecer os seus VETERANOS.

\end{subsecao}
\end{secao}
