\begin{subsecao}{USPGameDev: Pesquisa e Desenvolvimentos de Jogos na USP}

\figurapequenainline{uspgamedev}

Valve. Blizzard. Rockstar. Nintendo. USPGameDev. O que esses nomes têm em comum?
São nomes de grupos de desenvolvedores de jogos. E um deles tem sua sede na USP.

Constituído primariamente de alunos da USP de diversas áreas (na prática não) o
USPGameDev (UGD) foi criado em 2009, já tendo lançado pelo menos sete jogos e
até mesmo publicado seu próprio \textit{kit} de desenvolvimento (ganhem pontos
de hipster sendo nossos \textit{early adopters}!). Aliás, não trabalhamos apenas
com jogos digitais (aqueles que você joga no computador, console, portátil ou
celular), mas também com jogos analógicos (cartas, tabuleiro, esporte, etc.)!!!

E não é nada tão complicado assim. Mesmo tendo acabado de entrar na faculdade,
vocês também podem criar seu próprio jogo com a ajuda do USPGameDev. Tanto que
alguns dos nossos últimos jogos lançados foram produtos de grupos de bixos com
tempo livre demais em suas vidas. Nenhum conhecimento é necessário, até porque
um dos principais objetivos do grupo é desenvolvermos conhecimento e experiência
na área!

Justamente por isso, o UGD também oferece cursos e \textit{workshops} para a
comunidade USP sobre diversos assuntos envolvendo desenvolvimento de jogos.
Fiquem de olho! Além disso, participamos de \textbf{game jams} (ou hackatons):
eventos regionais e internacionais onde temos de 24 a 72 horas para fazer um
jogo com base em um tema que só é revelado na hora!

Interessados? Acessem nosso muito bem desenvolvido \textit{site} e dêem uma
conferida na página de downloads! Confiram: \url{http://uspgamedev.org}

\textbf{Para participar, basta entrar em contato conosco e marcar um dia para
conversarmos por onde você pode começar!}

%FIXME: esses vspaces estão aqui para puxar o texto que vem depois para mais
%       perto.
\vspace{-1.5em}

\begin{center}
  \Large
  \url{http://uspgamedev.org/contato}
\end{center}

\vspace{-1.5em}

\end{subsecao}
