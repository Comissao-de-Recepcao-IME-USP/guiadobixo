\begin{subsecao}{USPGameDev: Pesquisa e Desenvolvimentos de Jogos na USP}

\figurapequenainline{uspgamedev}

Valve. Blizzard. Rockstar. Nintendo. USPGameDev. O que esses nomes têm em comum?
São nomes de grupos de desenvolvedores de jogos. E um deles tem sua sede na USP.

Constituído primariamente, mas não exclusivamente, de alunos da USP de diversas
áreas (computação, matemática, design, música, letras, \textit{etc}.) o
USPGameDev (UGD) foi criado em 2009  e já lançou dezenas de jogos (um deles no 
Steam!\footnote{\texttt{https://store.steampowered.com/app/827940/Marvellous\_Inc/}}) 
e até mesmo o seu próprio \textit{kit} de desenvolvimento. Trabalhamos com jogos 
digitais e analógicos (video-games e jogos de tabuleiro, por exemplo). Vale 
ressaltar que adotamos a filosofia de software livre (\textit{``livre'' de 
``liberdade'', não necessariamente grátis}).

Queremos aprender e ensinar desenvolvimento de jogos como a atividade 
multifacetada que ela é. Você, que acabou de ingressar, também pode criar o seu 
próprio jogo dentro do USPGameDev, além de participar dos muitos projetos que já 
estão em andamento. Tanto que alguns dos nossos últimos jogos foram produtos de 
grupos de ingressantes (com muito tempo livre ou em disciplinas que aceitavam 
jogos como trabalhos). 

Não é necessário conhecimento prévio algum! Todos nós começamos a vida sem saber 
programar, desenvolver, projetar, desenhar, \textit{etc}. Isso porque não somos 
uma \textit{empresa} de jogos, mas um grupo de estudos. E, portanto, buscamos 
aprender o que não sabemos e ensinar o que sabemos.
Justamente por isso, o UGD também oferece cursos e \textit{workshops} livremente 
para a comunidade USP sobre diversos assuntos envolvendo desenvolvimento de 
jogos. Fiquem de olho! Além disso, participamos de \textbf{game jams} (ou 
hackathons): eventos regionais e internacionais onde temos de 24 a 72 horas para 
fazer um jogo com base em um tema que só é revelado na hora!

Interessados? Acessem nosso muito bem desenvolvido \textit{site} e deem uma 
conferida na página de downloads! 

\textbf{Para participar, basta entrar em contato (via email ou o que seja) 
conosco e marcar um dia para conversarmos}. Ser um membro não é lá muito formal, 
a gente bate um papo e vê o que seria legal para você fazer. O grupo é horizontal e 
cada um escolhe quanto participa.

\begin{description}
  \item[Site:] \url{http://uspgamedev.org}
  \item[Fórum:] \url{forum.uspgamedev.org}
  \item[E-mail:] contato@uspgamedev.org
  \item[Facebook:] \url{facebook.com/UspGameDev}
\end{description}

\end{subsecao}
