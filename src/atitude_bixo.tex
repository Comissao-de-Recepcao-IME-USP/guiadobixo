\begin{secao}{Atitude, bixos!}

Na USP, os alunos têm a liberdade e apoio de se organizarem
para montar grupos de debates, ciclos de palestras, grupos
de desenvolvimento e até mesmo grupos para jogarem alguma coisa (como
RPG, Magic, Yu-Gi-Oh ou algum esporte).
Portanto, caso você tenha algum projeto em mente não hesite
em se organizar com seus amigos e se informar em como avançar com essa ideia.
Lembre-se de que seus VETERANOS estão aí para te aconselhar e tirar suas
dúvidas (e para você pagar cervejas para eles, é claro).

É possível também você se juntar com alguns amigos e formar grupos de
estudos, seja para alguma matéria com a qual vocês tenham dificuldade, para
discutir aquele EP/lista de exercícios que ninguém está conseguindo
fazer ou simplesmente estudar algum tópico de interesse mútuo.

Esses são alguns dos exemplos dos grupos que foram direta ou indiretamente
criados por alunos do nosso Instituto!

% RDs --------------------------------------------------------------------------
\begin{secao}{O que é RD?}

RD é o Representante Discente. É um forte elo de ligação entre professores e alunos. 
RD é um aluno que representa os nossos interesses frente aos diversos conselhos existentes.
O RD ajuda, junto com os conselhos, a decidir coisas como autorização para festas, 
mudanças no currículo, aumento de vagas na FUVEST, mudança no corpo docente (às 
vezes lutamos para tirar algum professor), enfim, coisas desse tipo e muitas mais. 

Acho que você já percebeu o quanto é importante ter um aluno em cada um desses conselhos. 
Infelizmente, não costumamos preencher todas as vagas que nos é de direito. Isso se 
deve ao desinteresse de alguns ou  falta de tempo da maioria de seus VETERANOS. 

É, bixo, qué você quem tem mais tempo para fazer as coisas funcionarem aqui, já que 
ainda não sabe o que é Rec, DP, Trabalho, Estágio etc. Portanto, se você quer fazer 
alguma coisa pelo lugar onde você estuda, está aí uma dica. Para você ser RD é 
necessário se candidatar. O mandato é de um ano.\footnote{Até 2008 ou por aí, as eleições de RD eram no primeiro semestre do ano. Agora, são geralmente no final, ou seja, como bixo você não poderá atuar como RD, mas candidate-se no final do ano!}. Observação: ser um RD é também uma boa maneira de saber como pensam os seus professores e como as coisas funcionam aqui.

Em 2012, excepcionalmente, as eleições serão feitas no comecinho do ano (13, 14 e 15 de março). De acordo com o edital (nos murais, emails, fique atento!), você, bixo, por mais que não possa se canidatar a nenhum cargo, pode exercer seu direito de voto! Procure conversar com seus VETERANOS para saber melhor como funcionam essas coisas. Por enquanto, vai aí um breve resumo do que mais ou menos acontece em cada um dos órgãos nos quais temos direito a representate(s).
  
No IME, temos 26 cargos de RD, sendo 10 necessariamente de
pós-graduação e 14 necessariamente de graduação (Os dois cargos
restantes são livres). Todos os cargos tem direito a um suplente.
 
Existem diferentes níveis de hierarquia na administração.
 
{\bf As CoCs,
Comissões Coordenadoras de Curso (Lic, Pura, Estatística, Aplicada e
Computação)} são as mais próximas dos alunos. Temos um cargo de aluno em cada comissão. São comissões
pequenas, que tratam
dos problemas internos de cada curso: mudança de currículo,
requerimentos, optativas. Subordinada à CG e ao conselho do relativo
departamento. Analogamente, temos um cargo em cada Comissão
Cordenadora de Programa (de Pós).
 
{\bf Os Conselhos de Departamento (MAT, MAE, MAC e MAP)} tem uma dinâmica
um pouco diferente das CoCs, são mais formais. Cada conselho
se reúne (quase) mensalmente e são formados (em geral) por mais pessoas,
sendo
que existem regras sobre participação dos diferentes níveis
hierárquicos de professores (Titular, Associado, Doutor e Assistente). Nesses
conselhos, além de aprovar algumas das decisões das Comissões
Coordenadoras de Curso e de Programa (pós) e distribuição de
carga didática, são discutidos re-oferecimento de curso, revisão de prova,
supervisão das atividades dos docentes, afastamentos (temporários ou não),
contratação de professores e muitas outras coisas.

Os Conselhos de Departamento são subordinados à Congregação e ao CTA.
 
{\bf A Comissão de Graduação (CG)}, basicamente, avalia requerimentos,
mudança/criação de cursos e jubilamentos.
Analogamente, existe a Comissão de Pós-Graduação (CPG). Ambas são
subordinadas à Congregação.
 
{\bf Comissão de Espaço Físico (COESF)} é um orgão consultivo do CTA, formado
por representantes de diversos "ocupadores de espaço": Biblioteca, Centro
de Software Livre, Matemateca. Também tem representantes de cada
departamento. É presidida pelo vice-diretor. O RD daqui é o mesmo do CTA.
 
{\bf A Comissão de Cultura e Extensão (CCEx)} quase nunca tem reunião. Cuida das
atividades de extensão: Matemateca, CAEM, etc...
 
Os dois conselhos mais importantes são o CTA e a Congregação, ambos
presididos pelo Diretor.
 
{\bf O Conselho Técnico e Administrativo (CTA)} cuida de todas questões não
acadêmicas: Orçamento, reformas, avaliação dos funcionários, xerox,
lanchonete. É formado pelos 4 chefes de departamento, diretor, vice diretor, um
representante dos funcionários e um RD.
 
{\bf A Congregação} é o órgão máximo do Instituto. Com muitos professores, a
maioria titular. São dois RDs de graduação e um de Pós. Basicamente, neste
órgão, são rediscutidas e aprovadas (ou não) muitas das decisões dos órgãos
subordinados. Os membros da Congregação tem voto na eleição para Reitor e
Vice-Reitor. 
 
Bom, bixo, caso você não tenha lido o começo desse texto, não é difícil perceber que é muito importante ter um aluno em cada um desses conselhos. Pergunte, participe, vote. Saiba do que anda acontecendo! 

%\quadrinhos3

\pagebreak
\end{secao}


% Rede Linux -------------------------------------------------------------------
\begin{subsecao}{Rede Linux}

\figurapequenainline{rede_linux}

\begin{subsubsecao}{Introdução}

A Rede Linux é uma rede de computadores, administrada por alunos do IME e
que fornece diversos serviços para os VETERANOS e até mesmo para vocês, bixos.
Ela disponibiliza:

%FIXME
\vspace{-1em}

\begin{itemize}
\item 2 salas de computadores (no bloco A) com todo\footnote{ Se um programa
estiver faltando, mande um email pra admin@linux.ime.usp.br pedindo-o.} tipo de
programa necessário para suas atividades acadêmicas (com pelo menos uma que fica
aberta 24 horas por dia, 7 dias por semana\footnote{ Mas talvez vocês não
consigam entrar no bloco A depois da meia noite, que é quando a portaria
fecha.});
\item Uma página na internet para cada aluno;
\item Um \textit{e-mail} para cada aluno;
\item Espaço para você guardar seus arquivos;
\item Acesso remoto via ssh (linux.ime.usp.br);
\item Impressoras;
\item Admins dispostos e capazes, para o caso de algum usuário ter alguma boa
ideia para adicionar a esta lista;
\end{itemize}
\end{subsubsecao}

\begin{subsubsecao}{O Linux}

A rede utiliza em todos os seus computadores um sistema operacional chamado
Linux. Esse é um sistema desenvolvido de forma colaborativa pelos usuários
e empresas interessados nele (se quiser saber mais a respeito, pesquise
por ``software livre''!).

O Linux não é um sistema mais difícil de usar que o Windows. É apenas
diferente em alguns aspectos. Além de tudo, existem cursos de Linux que são
organizados pelos alunos do IME. Os admins costumam promover esses cursos.
Fiquem atentos aos emails!

Não se deixem intimidar pelo sistema. Se vocês se derem ao trabalho de
aprender a utilizá-lo bem, verão que ele é bastante flexível, e até mesmo
interessante (tanto quanto um sistema operacional pode ser =P).

\end{subsubsecao}

\begin{subsubsecao}{Os admins}

Os admins são alunos do bacharelado em ciência da computação (vulgo BCC) que
são responsáveis por administrar a rede. Entre outras coisas, isso quer dizer
manter os computadores funcionando, ajudar os alunos a usar a rede (com
cursos\footnote{ Fiquem atentos aos emails!!!} e resolvendo dúvidas nos
horários de plantão\footnote{ Na página da rede, estão os horários de todos os
admins.}) e também implementar coisas novas na rede (aceitamos sugestões!)

Os admins são escolhidos por um treinamento que acontece de dois em dois anos,
em todo ano par. Mais informações serão divulgadas quando este estiver próximo
a ocorrer.

\end{subsubsecao}
\begin{subsubsecao}{Como criar uma conta?}

Basta passar na Admin, na sala 125 do bloco A (como vocês são bixos: bloco A é o da
biblioteca, bloco B aquele que tem muitas salas de aula e que vocês vão passar boa
parte da vida de vocês). Contatos:

%FIXME
\vspace{-1em}

\begin{description}
\item [e-mail:] admin@linux.ime.usp.br
\item [Página:] \url{www.linux.ime.usp.br}
\item [Sala:] 125, bloco A
\end{description}

%FIXME
\vspace{-.5em}

\end{subsubsecao}

\end{subsecao}


% IME Júnior -------------------------------------------------------------------
\begin{subsecao}{IMEjr: A Nossa Empresa}

\figurapequenainline{imejr_logo}

Em meados de 1991, surgia a Empresa Junior de Informática, Matemática e
Estatística do IME (IMEjr). Uma Empresa Junior é uma Associação Sem Fins
Lucrativos administrada por estudantes de graduação (que é o que você é agora)
e tem o objetivo de complementar a formação do aluno em termos da integração
entre teoria e prática, além de incentivar o empreendedorismo entre os alunos
do Instituto

Entre as atividades da IMEjr estão o desenvolvimento de projetos em todas as
áreas do IME e a organização de palestras, cursos e workshops. Logo, estamos
abertos tanto a alunos interessados em aprender a administrar uma empresa
quanto a desenvolver atividades e projetos.

Uma diferença entre nós e uma empresa comum é que nossos integrantes têm muito
mais liberdade de trabalhar e participam de projetos que auxiliam a aprofundar
mais sua formação, do que a maioria dos estágios por aí.

Contamos com vocês neste ano, e já garantimos que existem atividades prontas.
Esperamos seu contato! LEMBRE: nem sempre de aulas e livros é feito um
estudante com boa formação. Por isso, anote em sua agenda, celular, bloquinho
de nota, ...:

%FIXME
\vspace{-1em}

\begin{description}
\item [Sala:] 258, bloco A
\item[E-mail:] \url{imejr@ime.usp.br}
\item[Website:] \url{www.ime.usp.br/imejr}
\item[Facebook:] \url{fb.com/IMEJuniorUSP}
\end{description}

%FIXME
\vspace{-.5em}

\end{subsecao}


% USPGameDev: Pesquisa e Desenvolvimentos de Jogos na USP ----------------------
\begin{subsecao}{USPGameDev: Pesquisa e Desenvolvimentos de Jogos na USP}

Valve. Blizzard. Rockstar. Nintendo. USPGameDev. O que esses nomes têm em comum?
São nomes de grupos de desenvolvedores de jogos. E um deles tem sua sede na USP.

Constituído primariamente de alunos da USP de diversas áreas (na prática não) o 
USPGameDev (UGD) foi criado em 2009, já tendo lançado cinco jogos e 
publicado seu próprio \textit{kit} de desenvolvimento para jogos 2D e 3D que 
oferece suporte para diversas plataformas, tais como Windows, Linux, Mac e Android. 

Mas não é nada tão complicado assim. Mesmo tendo acabado de entrar na faculdade,
você também pode criar seu próprio jogo com a ajuda do USPGameDev. Tanto que nossos
últimos jogos lançados foram produtos de grupos de bixos com tempo livre demais em
suas vidas. Simplesmente apareça em uma de nossas reuniões e participe. Nenhum
conhecimento é necessário, até porque um dos principais objetivos do grupo é
aprendermos!

Além disso, o UGD também oferece cursos e \textit{workshops} para a comunidade USP
sobre diversos assuntos envolvendo desenvolvimento de jogos. Fique de olho!

Acesse nosso muito bem desenvolvido \textit{site}: 
\url{http://uspgamedev.org}

Para saber mais sobre os horários e datas das reuniões e como participar, acesse: 
\url{http://uspgamedev.org/contato}

\end{subsecao}



%FIXME GAMBIARRA para não quebrar página num lugar zuado
\pagebreak

% Diversime --------------------------------------------------------------------
\begin{subsecao}{DiversIME}

O grupo de Diversidade do Instituto de Matemática e Estatística tem como objetivo
 apoiar e unir a comunidade LGBT do IME. O grupo, fundado em 2012, possui hoje
 no facebook mais de 100 membros e nosso gráfico é do tipo $y = ax + b$ com $a < 0$,
 ou seja, crescemos constantemente.

A ideia principal é lutar contra o preconceito que ainda existe na sociedade,
 instruir as pessoas do instituto sobre o assunto e fazer amigos!

Esse ano, temos uma proposta de realizarmos reuniões mensais para estudar o
 assunto e nos apoiar mutuamente, seja em auxílio a estudantes com dificuldades
 na aceitação, com famílias que não aceitam ou até mesmo com a falta de
 namoradxs, o importante é participar.

E tudo isso também ocorre no nosso grupo secreto do facebook! É só pedir pra um
 veterano membro que ele te adiciona lá. Como o grupo é secreto, apenas membros
 podem te ver lá, garantindo sua segurança se você não se sentir confortável
 para abrir isso para a sociedade. Vale lembrar, é claro, que o grupo não
 abrange apenas LGBTs, mas também héteros amigos e interessados em apoiar
 a causa, participar de nossas discussões e causações. Tendo o coração e a mente
 abertos, são todos bem-vindos!

Sigam também nossa página no Facebook: {\tt fb.com/diversimeusp} !
Acredite, você não está sozinho, e a partir do DiversIME, você pode encontrar
 pessoas que passaram pelas mesmas dificuldades e maravilhas que você. Se você
 é imeano e compreende a importância da auto-aceitação e do respeito à
 diversidade, então você já é parte do DiversIME!

\end{subsecao}


% Existimos! -------------------------------------------------------------------
\begin{subsecao}{$\exists$xistimos!}

\figurapequenainline{existimos}

$\exists$xistimos! surgiu em  2014 com a proposta de criar um espaço de
confiança entre as alunas do IME, onde cada uma possa ter a liberdade e
segurança para discutir suas vivências e propostas.

O desconforto com a forma como as mulheres são tratadas no IME e nas ciências
exatas em geral fez com que nos juntássemos para discutir como as questões de
gênero se manifestam no instituto e quais são as formas de agir para evitar
situações desconfortáveis ou preconceituosas.

Desde então promovemos uma série de eventos e intervenções para que tal debate
atinja toda a comunidade imeana, além de realizarmos reuniões periódicas apenas
com mulheres para que possamos conversar e nos ajudar, em qualquer situação,
mas em especial naquelas onde possamos ser vítimas ou testemunhas de preconceito e
machismo. Todas vocês estão convidadas a participar das nossas reuniões, 
alunas de outros institutos, professoras, funcionárias e mulheres
em geral são muito bem vindas $<$3. 

Para falar conosco ou participar do grupo basta enviar um email
para 3xistimos@gmail.com ou existimos@google.groups.com ou ainda pelo facebook:
https://www.fb.com/3xistimos/

Para denúncias ou relatos anônimos acesse: http://bit.ly/existimos
\end{subsecao}


% Grupo A5 ---------------------------------------------------------------------
\begin{subsecao}{Grupo A5}

\figurapequenainline{grupo_A5}

Somos um grupo de estudantes de graduação e pós-graduação do IME e, anualmente,
organizamos eventos acadêmicos no instituto.  Tudo começou em 2012, quando dois
alunos da pura perceberam que alguns assuntos muito interessantes sobre
matemática e sobre o meio acadêmico nem sempre eram desenvolvidos em sala de
aula, e resolveram organizar um evento para levar um destes assuntos aos demais
estudantes instituto. Assim, juntamente ao CAMat, organizaram um ciclo de
palestras sobre "Os 7 Problemas do Milênio", e o evento foi um sucesso.  Vários
professores e estudantes começaram a pedir que mais eventos como este fossem
organizados, e foi então que um deles teve a ideia de criar um grupo
independente das demais instituições do IME (sim, o Grupo A5 é independente. Não
é vinculado ao CAMat e nem a nenhum outro grupo, apesar de aceitar parcerias em
alguns eventos), com a finalidade de complementar a formação dos estudantes do
IME e de quem quiser participar, levando palestras e outros eventos sobre temas
relevantes e não explorados no currículo, de forma gratuita e com linguagem de
fácil entendimento.  E assim, no final de 2013, o Grupo A5 oficialmente nasceu,
com nome e logo e com novos integrantes no grupo.  Desde então, não paramos
mais. Em 2014 organizamos o ciclo de palestras "IC ou Não IC? - Eis a Questão",
em 2015 organizamos o evento "História da Matemática", que foi indicado como um
dos destaques de Cultura e Extensão de 2015 pelo IME.

Para mais informações do Grupo A5 e dos eventos já organizados por nós, acessem
nosso site \url{www.ime.usp.br/~a5} (ainda está em construção, mas em breve os
vídeos das palestras e demais informações estarão lá). E não deixem de curtir
a página Grupo A5 no facebook: \url{www.fb.com/pagina.GrupoA5} (é aqui que
vocês terão em primeira mão os detalhes de tudo o que for feito por nós).

Vale ressaltar que o Grupo A5 é formado e mantido por estudantes do IME, então o
sucesso e a continuidade do grupo dependem de todos; começando por vocês,
bixos. Então venham, assistam, dêem sugestões, participem. E se gostarem,
juntem-se ao nosso grupo!

\end{subsecao}



% Olimpíadas de Matemática e Informática ---------------------------------------
% Maratona de Programação ------------------------------------------------------
\begin{subsecao}{Olimpíadas de Conhecimento}

\begin{itemize}

\item{\bf Matemática: }

Bom pessoal, se vocês entraram no IME, muito provavelmente já participaram
de alguma Olimpíada de Matemática no Ensino Fundamental e/ou Médio. A 
boa notícia é que vocês vão poder continuar participando se quiserem,
e quem nunca participou tem a oportunidade de começar agora.

Mas por quê participar? As Olimpíadas Universitárias de Matemática são uma
oportunidade de se divertir resolvendo problemas difíceis de Matemática e agregar
valor ao currículo ao mesmo tempo. Elas são parecidas com as Olimpíadas de
Ensino Médio, mas com conteúdo de Matemática da graduação (essencialmente 
Cálculo, Análise, Álgebra Linear, Álgebra, Combinatória e Teoria dos Números), 
mas com enfoque em problemas que exigem criatividade e técnicas mais inovadoras,
muitas das quais vocês provavelmente não verão durante toda a graduação.

De quais olimpíadas podemos participar? Como alunos de graduação, vocês podem
participar da Olimpíada Iberoamericana de Matemática Universitária (OIMU),
Olimpíada Brasileira de Matemática (OBM) e Olimpíada Internacional de 
Matemática (IMC).

Como fazemos para nos preparar? Os sites institucionais dessas olimpíadas
tem todo o material necessário para vocês que querem estudar e se preparar
para elas.

Como fazemos para participar? Inscrevam-se pelo site ou entrem em contato com
o professor Yoshiharu. Para o IMC aconselha-se ter ganhado medalha na OBM,
já que é necessário apoio financeiro do IME por ser uma olimpíada internacional.

Mas nós, bixos, temos chance? Como foi o desempenho de IMEanos nelas? Nós obtivemos
sucesso nestas olimpíadas. Ganhamos medalhas em todas as três competições e
o resultado mais recente foi uma medalha de bronze na IMC e ouro na OBM.

Links institucionais:

\url{http://www.cimat.mx/Eventos/oimu/}

\url{http://www.imc-math.org/}

\url{http://www.obm.org.br/opencms/}

Se tiverem alguma dúvida, não hesitem em perguntar a algum VETERANO sobre os
Olímpicos!

\item{\bf Informática: }

``Informática? Vocês mexem com Word, Excel e PowerPoint então?''

Responder essa pergunta já virou rotina para competidores da
Olimpíada Brasileira de Informática (OBI). Não, Informática 
não é Word. Oras, então o que é a OBI?

A OBI é uma competição de lógica e matemática. As provas são pequenos
problemas que você deve resolver com programas de computador.

Apesar de ser uma olimpíada que requeira um conhecimento mínimo de 
linguagem de computação, as provas em si geralmente não contém nada 
de teoria computacional mais avançada. 

Isso por que a OBI é voltada para alunos do Ensino Médio e recém
ingressantes. Quer dizer que vocês, bixos recém formados do Médio,
são nossa única esperança de ganhar medalhas e trazer glória ao IME!
Isso também quer dizer que essa é a única chance de vocês de fazer a OBI,
uma competição relativamente tranquila comparada à Maratona de 
Programação.

Para participar, basta falar com o Professor Carlinhos (\url{http://www.ime.usp.br/~cef/}), 
ou com ex-competidores, amáveis VETERANOS e lindas criaturas, que estarão disponíveis
durante a semana de recepção e em todos os treinos da Maratona de Programação também!

Para mais informações, acessem \url{http://olimpiada.ic.unicamp.br/}

\end{itemize}

\begin{subsubsecao}{Maratona de Programação}

À primeira vista, a Maratona de Programação pode soar um tanto
surreal. Nerds correndo pela USP ao mesmo tempo que resolvem
problemas de computação e matemática? Infelizmente esse não
é o caso.

A Maratona de Programação se resume à resolução de problemas.
Se você adora resolver desafios, quebrar a cabeça com novos
e excitantes problemas e acumular toneladas de dinheiro, esse
é o lugar perfeito para você!

A competição consiste de uma série de problemas que englobam
temas como programação dinâmica, grafos e estruturas de dados.
Times de três pessoas devem resolver a maior quantidade de
desafios em cinco horas de programação. E tudo isso com direito
a um lanche gratuito durante a prova.

Mas não temam, bixos. Não é só por que vocês acabaram de entrar que
a probabilidade de ganhar uma medalha seja nula. Inclusive, na primeira
fase da maratona, uma equipe de bixos tem vaga garantida para a
fase brasileira.

Além da fama, constantes pedidos por autógrafos e dinheiro de sobra,
a maratona também vai lhes trazer um conhecimento muito mais
adiantado em relação ao dos seus colegas de classe, e até oportunidades
de emprego em empresas de renome, como Google, Facebook e IBM.

Se vocês se interessaram pela maratona e querem saber os horários dos
treinos, como participar ou saber mais, acessem:

\url{http://www.ime.usp.br/~maratona}

Acessem também o \textit{site} da competição nacional:

\url{http://maratona.ime.usp.br/}

\end{subsubsecao}


\end{subsecao}


\end{secao}
