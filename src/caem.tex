\begin{secao}{CAEM - Centro de Aperfeiçoamento do Ensino da Matemática}

O Centro de Aperfeiçoamento do Ensino da Matemática tem como principal
objetivo prestar assessoria aos professores de Matemática das redes
públicas (escolas municipais e estaduais) e particulares. Dentre
outras atividades, o CAEM oferece cursos, oficinas, palestras e
seminários voltados para professores de matemática dos níveis
Infantil, Fundamental e Médio. IMEanos podem usufruir e participar,
inclusive bixos. Aqui vão algumas informações úteis:

%Daqui para baixo é texto do próprio CAEM, com leves mudanças.

\begin{itemize}

\item \textbf{Como se cadastrar}: Para se cadastrar no CAEM a pessoa
  deve ter em mãos um comprovante de residência e 2 fotos
  3x4. Qualquer pessoa pode se cadastrar, independente se é aluno ou
  não.

\item \textbf{Materiais disponíveis}: O CAEM conta com um acervo
  voltado para o ensino da matemática nos níveis fundamental e médio,
  além de materiais de auxílio para aula como material dourado,
  sólidos, régua e compasso de lousa, DVD's. \textbf{Quais deles podem
    ser emprestados}: Livros e DVD's podem ser emprestados na sua
  maioria por um período de 1 semana (até 4 materiais, podendo
  renovar). Materiais de auxílio em sala podem ser emprestados em
  quantidade limitada e por apenas 2 dias.

\item \textbf{Informações sobre as oficinas}: Para todas as oficinas
  do CAEM são oferecidas apenas 5 vagas gratuitas a alunos do IME. Não
  são aceitos alunos além das 5 vagas, mesmo que esteja disposto a
  pagar, pois as oficinas e cursos do CAEM são voltadas a professores
  atuantes para trocas de experiência.

\item \textbf{Custos para o aluno}: Alunos do IME tem 5 vagas
  gratuitas nas oficinas.

\item \textbf{Horas de ATPAs}: Os certificados das oficinas e palestras
 do CAEM podem ser usados para as ATPA's.

\item \textbf{Horário de funcionamento}: Segundas e Quartas-feiras,
  das 10h00 às 19h00. Terças, Quintas e Sextas-feiras, das 10h00 às
  21h00.
\end{itemize}

Para mais informações, acesse o site \url{www.ime.usp.br/caem} ou veja
a página do CAEM no facebook procurando por ``CAEM IME USP''. Para
contato, envie um e-mail para {\tt caem@ime.usp.br}. Telefone e Fax:
(11) 3091-6160.


\end{secao}
