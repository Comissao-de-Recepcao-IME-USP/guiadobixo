\begin{secao}{Comissão de Trote e \textit{Kit}-bixo}

A Comissão de Recepção aos Calouros, também conhecida como Comissão de Trote, é
responsável por auxiliar os ingressantes em seus primeiros momentos imeanos.
Sabemos que não é um momento fácil. Vocês estão entrando em uma nova fase de
suas vidas, em um lugar estranho, com pessoas estranhas (em todos os sentidos),
e nosso objetivo é fazer com que vocês se sintam bem-vindos e se integrem
($\int$) com seus coleguinhas e seus VETERANOS!

Vocês já devem ter conhecido alguns de nós durante a matrícula, aquelas pessoas
bem legais que estavam devidamente uniformizadas, cuidando para que nenhum
bêbado ou pessoa de má índole machucasse vocês! Parte da Comissão ajuda os
alunos a preencher os formulários e a se matricular direitinho. Enquanto isso,
outra parte impede que a espera na fila de matrícula seja tediosa e confusa:
todos os bixos são devidamente pintados, carimbados e tosados, numa tentativa
de torná-los mais charmosos. Duro esse trabalho, não?

A Comissão de Trote organiza a super Semana de Recepção, cheia de atividades
legais! (Vocês devem ter recebido, junto com aquela papelada na matrícula, a
programação da Semana. Se não receberam, procurem arrumar uma, logo!).
E adivinhem quem organiza o magnífico encontro dos bixos: IMEntrando, que deverá
ocorrer em março? %REFTIME

A Comissão de Trote é formada pelos mais animados e divertidos VETERANOS. Como
dissemos, eles organizam a matrícula, agilizam a papelada, mantêm um clima
alegre na recepção, fazem isso e aquilo na Semana de Recepção...
Vocês devem estar pensando ``Puxa! Como nós, bixos, podemos retribuir
tamanha dedicação?'' É simples, bixos: {\bf\em comprem o \textit{kit}-bixo}!!!

O \textit{kit}-bixo, como vocês devem saber, é um conjunto de coisas
importantíssimas para vocês, ingressantes perdidos! Ele contém dois tipos de
itens:
\begin{itemize}
\item itens úteis;
\item itens essenciais.
\end{itemize} %REFTIME (Isso muda todo ano)
Dentre eles temos uma camiseta, que serve para que nós os identifiquemos como
bixos e para que as pessoas na rua achem que vocês são inteligentes (fiquem
tranquilos, bixos, vocês não são tão melhores assim); materiais personalizados
com o símbolo do IME pra vocês usarem nas aulas e mostrarem a seus amigos; uma
caneca (também personalizada), pra vocês economizarem muitos copos no bandejão e
que ainda serve como convite para o IMEntrando; adesivos pra vocês colarem no
carro que vão ganhar de presente; calculadora para usarem nas suas provas de
Estatística; squeeze pra aguentarem o calor do começo do ano; uma tatuagem
personalizada; um pen-drive de 8GB, para vocês guardarem todos os arquivos e EPs
das matérias; um caderno personalizado feito com muito \sout{suor} carinho pela
Comissão e várias outras coisas, todas contidas em uma mochila sport do IME-USP,
afinal não dá pra carregar tudo isso na mão, né?

Adquiram o maravilhoso \textit{kit}-bixo do IME-USP. Ele estará à venda na
matrícula e na semana de recepção, até durarem os estoques. (Não vão chorar
depois, hein?)

Para terminar, além de tudo isso, a Comissão é que faz esse maravilhoso guia que
vocês estão lendo agora (ou estão só olhando as figuras, vai
saber...). Esperamos que vocês gostem do nosso trabalho! Qualquer coisa, nos
procure! Entrem na nossa página no Facebook: {\tt fb.com/troteimeusp} contando o
que vocês sentiram ao ler o guia, o que fizeram com vocês na matrícula (com o
nome do VETERANO na denúncia), o que vocês acharam da semana de recepção e se
vocês se sentiram bem-vindos ou querem voltar logo para perto da sua
mãe. Estaremos sempre prontos a ajudá-los!
\end{secao}
