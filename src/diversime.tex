\begin{subsecao}{DiversIME}

\figurapequenainline{diversime_2}

O grupo de apoio à diversidade do Instituto de Matemática e Estatística tem como
objetivo unir e ajudar todas as pessoas que queiram expressar sua diversidade,
seja ela de orientação sexual e afetiva, identidade de gênero, racial, social ou
qualquer outra. A ideia principal do grupo é lutar contra o preconceito que
ainda existe na sociedade, instruir as pessoas do instituto sobre o assunto
(aliás, de fora dele também), conhecer gente e ideias.

Nossa página no facebook é \url{fb.com/diversimeusp}, e lá você pode pedir para
a gente te adicionar no grupo secreto (quem está de fora não consegue ver quem
faz parte do grupo), e pelo grupo também acompanhar as nossas evoluções. As duas
plataformas tem diversos moderadores, e o grupo é bem aceito e conhecido. Temos
também uma lista de e-mails se você não for muito-do-tipo de facebuquiar. É só
mandar um e-mail se identificando para
\url{diversime-usp+subscribe@googlegroups.com}.

Só para nos gabarmos, vamos dizer aqui que as MAIORES festas desse Instituto
foram as maravilhosas edições que fizemos da I Will SurvIME, voltadas para a
comunidade diversa dessa Universidade de São Paulo. Recordes de público, nos
surpreendendo e dando trabalho em todas as edições. Para fazê-las, precisamos
do esforço e dedicação das maravilhosas pessoas que compõe o DiversIME, que
as realizam com a ajuda da Atlética.

Acredite, você não está só, e através do DiversIME você pode encontrar pessoas
que passaram pelas mesmas dificuldades e pelas mesmas maravilhas. Se você
compreende a importância da auto-aceitação e do respeito, então já tem tudo para
ser parte do DiversIME!

\end{subsecao}
