\begin{secao}{DiversIME}

O grupo de Diversidade do Instituto de Matemática e Estatística tem como objetivo
 apoiar e unir a comunidade LGBT do IME. O grupo, fundado em 2012, possui hoje
 no facebook mais de 100 membros e nosso gráfico é do tipo $y = ax + b$ com $a > 0$,
 ou seja, crescemos constantemente.

A ideia principal é lutar contra o preconceito que ainda existe na sociedade,
 instruir as pessoas do instituto sobre o assunto e fazer amigos!

Esse ano, temos uma proposta de realizarmos reuniões mensais para estudar o
 assunto e nos apoiar mutuamente, seja em auxílio a estudantes com dificuldades
 na aceitação, com famílias que não aceitam ou até mesmo com a falta de
 namoradxs, o importante é participar.

E tudo isso também ocorre no nosso grupo secreto do facebook! É só pedir pra um
 veterano membro que ele te adiciona lá. Como o grupo é secreto, apenas membros
 podem te ver lá, garantindo sua segurança se você não se sentir confortável
 para abrir isso para a sociedade. Vale lembrar, é claro, que o grupo não
 abrange apenas LGBTs, mas também héteros amigos e interessados em apoiar
 a causa, participar de nossas discussões e causações. Tendo o coração e a mente
 abertos, são todos bem-vindos!

Sigam também nossa página no Facebook: {\tt fb.com/diversimeusp} !
Acredite, você não está sozinho, e a partir do DiversIME, você pode encontrar
 pessoas que passaram pelas mesmas dificuldades e maravilhas que você. Se você
 é imeano e compreende a importância da auto-aceitação e do respeito à
 diversidade, então você já é parte do DiversIME!

\end{secao}
