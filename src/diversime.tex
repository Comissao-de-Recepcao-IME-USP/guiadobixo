\begin{subsecao}{DiversIME}

\figurapequenainline{diversime_2}

O grupo de apoio à diversidade do Instituto de Matemática e Estatística tem como
objetivo unir e ajudar todas as pessoas que queiram expressar sua diversidade,
seja ela de orientação sexual e afetiva, identidade de gênero, racial, social ou
qualquer outra. A ideia principal do grupo é lutar contra o preconceito que
ainda existe na sociedade, instruir as pessoas do instituto sobre o assunto
(aliás, de fora dele também), conhecer gente e ideias.

Nossa página no facebook é \url{fb.com/diversimeusp}, e lá você pode pedir para
a gente te adicionar no grupo secreto (quem está de fora não consegue ver quem
faz parte do grupo), e pelo grupo também acompanhar as nossas evoluções. As duas 
plataformas tem moderadores, e procuramos ser acessíveis à debates e atividades 
que as e os estudantes têm interesse em desenvolver. 

Acredite, você não está só, e através do DiversIME você pode encontrar pessoas
que passaram pelas mesmas dificuldades e pelas mesmas maravilhas. Se você
compreende a importância da auto-aceitação e do respeito, então já tem tudo para
ser parte do DiversIME!

Curta a nossa página no Facebook para saber dos eventos que vamos organizar 
durante o ano e saber de outros eventos da USP e Comunidades. 

\end{subsecao}
