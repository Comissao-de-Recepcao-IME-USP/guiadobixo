\begin{subsecao}{MaratonUSP}

\figurapequenainlineapertada{maratonusp}
O MaratonUSP é o grupo de programação competitiva do IME. O seu principal foco é
o estudo de algoritmos e estruturas de dados, abordando uma variedade de temas
que vão desde filas e pilhas, até Teoria dos Jogos e outros tópicos avançados de
matemática. O grupo é hoje uma referência para o Brasil inteiro com nossas
diversas conquistas em competições e com nosso canal do YouTube, que em apenas
em menos de um ano já acumula milhares de visualizações.

As Maratonas de Programação consistem de provas realizadas em grupos de 3 alunos
em que o objetivo é resolver o maior número de problemas no menor tempo
possível - bem como uma corrida! A grande diferença é que você não espera tanto
quanto provas convencionais para saber se acertou ou não uma questão: as respostas
saem na hora, dando inclusive a oportunidade de tentar novamente um problema. Como 
o foco é o trabalho em equipe, cada time só tem 1 computador durante toda a
competição, tornando a cooperação uma chave para o sucesso.

Na pandemia, para se preparar para esse tipo de prova tão diferente, o nosso
grupo organiza aulas dinâmicas abertas a todos. Elas são voltadas inclusive
para calouros que entraram sem conhecimento de programação. As aulas acontecem
online e posteriormente são postadas no YouTube.

A dinâmica dos encontros, com a formação de trios, proporciona uma interação maior
entre bixes, além de aprimorar as habilidades de trabalho em equipe. Tudo é feito
com muito carinho pelos próprios membros do grupo. É uma oportunidade incrível para
desenvolver habilidades que envolvam não apenas códigos, mas também oratória e
didática. Viajar para dar aulas e ministrar cursos - inclusive fora do país - também
é uma possibilidade.

Todo esse ambiente faz com que nosso grupo tenha resultados incríveis nas Maratonas,
sendo Tetracampeão da fase nacional e tendo participado diversas vezes da etapa
mundial da competição, em países como a China, Tailândia, Rússia, Estados Unidos
e muito mais - tudo de graça! Não da para esquecer que a própria fase nacional
acontece em um local diferente todo ano, uma ótima oportunidade de conhecer mais
do nosso país.

A vida de Maratoneiro não impacta só sua passagem pela universidade, ela também
abre portas para o futuro. É super comum os ex-membros irem trabalhar em empresas
como Google, Facebook, Microsoft, tanto aqui como também em grande parte no
exterior. As habilidades desenvolvidas durante as competições, viajando pelo
mundo dando aulas, preparando vídeos para o YouTube ou simplesmente estudando
com seus colegas durante uma tarde vão ser úteis tanto no mercado de trabalho
como na academia, além de render anos de muita diversão.

\begin{description}
\item [Facebook:] \url{facebook.com/MaratonUSP}
\item[Site:] \url{www.ime.usp.br/~maratona}
\item[Youtube:] \url{www.youtube.com/c/maratonusp}
\item[Telegram:] \url{https://t.me/maratonusp2021}

\end{description}

\end{subsecao}
