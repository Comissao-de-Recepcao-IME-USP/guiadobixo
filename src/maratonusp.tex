\begin{subsecao}{MaratonUSP}

\figurapequenainlineapertada{maratonusp}

O MaratonUSP é um grupo de estudos de programação competitiva. O principal evento é a Maratona de Programação, uma competição anual organizada pela Sociedade Brasileira de Computação (SBC). A competição é restrita a universitários que se organizam em times de três pessoas para resolver os mais variados problemas em forma de código. Apesar da coletividade, há uma dificuldade a mais: cada equipe dispõe de apenas um computador.

A Maratona é dividida em várias fases. Primeiro, há uma seleção interna dos times que representarão a USP na competição. Uma das vagas é reservada para um time de bixos! Depois, há uma fase regional com outras universidades da capital paulista, onde a USP tem um histórico de vitórias absoluto.

As coisas são consideravelmente mais difíceis na fase nacional, conhecida como Final Brasileira. As melhores universidades do Brasil se reúnem em algum canto do país para disputar as vagas brasileiras para a fase mundial. Exemplos recentes são Belo Horizonte, Foz do Iguaçu e Salvador. A hospedagem, comida e transporte são todos cobertos pela organização e pela universidade. Viagem de graça! Os melhores times brasileiros vão para a Final Mundial, sempre num país diferente. Nós estivemos recentemente no Marrocos, na Tailândia e na China. Em hotéis 5 estrelas! De graça!

Grandes empresas têm muito interesse nos alunos que participam da Maratona. É muito comum encontrar ex-maratonistas em empresas como Google, Facebook, etc. O pensamento abstrato e o trabalho em equipe desenvolvidos na Maratona são características muito bem avaliadas pelos recrutadores.

O ano começa com o bixeCamp, uma série de aulas focadas em ensinar o básico de programação e algoritmos. As aulas são no início dos treinos, que ocorrem no CEC (do lado da Seção de Alunos) à partir das 14h às sextas-feiras. O treino acaba às 19h, mas não é necessário ficar até o final. Alunos já experientes em programação são encorajados a comparecer também. Te esperamos lá!

\begin{description}
\item [Facebook:] \url{facebook.com/MaratonUSP}
\item[Site:] \url{www.ime.usp.br/~maratona}
\end{description}

\end{subsecao}
