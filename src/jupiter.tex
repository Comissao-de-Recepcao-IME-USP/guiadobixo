\begin{secao}{JupiterWeb}

O \sout{Luciferweb} Jupiterweb é o sistema que administra (quase) tudo que
vocês, bixos, fazem na sua graduação. Desde as suas matrículas até a emissão de
um documento para pagar meia no cinema, o Jupiter vai ser muito útil em
vários momentos se vocês conhecerem tudo que ele tem a oferecer.

\begin{subsecao}{Calendário}

Muito útil pra saber quando é \sout{feriado} que abre a matrícula ou o período
de requerimento.

\end{subsecao}

\begin{subsecao}{Disciplinas}

Mostra todas as disciplinas da USP (até as que não estão mais disponíveis).
Vocês podem pesquisar por código, por algum trecho do nome, por unidade e por
horário. Além de ser útil para conferir algumas informações sobre aquela
matéria que vocêm precisa cursar, ainda ajuda a procurar matérias diferentes
pela USP (como “Pintura”, “Psicologia da Morte” ou “Pesca Sustentável”).

\end{subsecao}

\begin{subsecao}{Grade Horária}

Lista as disciplinas em que vocês estão inscritos/matriculados/pendentes, com os
horários de cada uma. É onde geralmente vocês conferem se tá tudo certo com a
sua matrícula ou não. Quando as aulas começam, o ideal é que apareça
“Matriculado” (a matéria fica cinza). Se alguma disciplina aparecer como
pendente (fica azul), então corram na seção de alunos para conferir o que está
errado!

\end{subsecao}

\begin{subsecao}{Resumo Escolar}

O resumo escolar é um documento enorme com tudo de bom (e de ruim) que
aconteceu na sua vida acadêmica na USP. Lá vocês podem conferir cada
disciplina cursada (com nota), as disciplinas que vocês trancaram, seus
aproveitamentos de estudo, quantos créditos vocês já fizeram e a tão mística
“Média ponderada”.

\end{subsecao}

\begin{subsecao}{Acompanhamentos}

\begin{itemize}
  \item \textbf{Dados do Programa}

    Compilação da sua vida universitária, desde o Vestibular até a hora que
    vocês se formam.
  \item \textbf{Evolução no Curso}

    É o jeito visual de acompanhar as disciplinas que vocês já fizeram ou que
    estão faltando. Sua principal função é ser printado ao final do curso (com
    tudo verde) para postar no Facebook e fazer sucesso com a família na
    internet.
  \item \textbf{Rendimento Acadêmico}

    Mais um lugar pra vocês conferirem suas notas, quantidade de matérias
    cursadas, e quantidade de créditos cursados, só que aqui tudo aparece em
    forma de gráfico.
  \item \textbf{Perfil de deficiências}

    Questionário da USP sobre deficiências.

  \item \textbf{Meus benefícios e bolsas}

    Onde ficam registradas as suas bolsas e alguns serviços prestados na
    universidade que valem dinheirinhos. Também mostra as datas que a USP lhes
    paga para vocês organizarem sua vida financeira!!

  \item \textbf{Dados pessoais}

    O nome é bem intuitivo. Mantenham tudo atualizado para não ter problemas com
    a SPTrans, SAS ou qualquer coisa que confira seus dados em algum momento.

\end{itemize}

\end{subsecao}

\begin{subsecao}{Emissão de Documentos}

Emite alguns documentos que precisamos no dia a dia, como um comprovante de
matrícula ativa. Dependendo do documento, vocês vão precisar ir até a seção de
alunos, mas os mais requisitados vocês encontram por aqui.

\end{subsecao}

\begin{subsecao}{Matrícula}

Aqui começa uma das partes mais legais da Universidade: A matrícula! Pois é
bixos, vocês não vão cursar mais o que a sua instituição de ensino manda, mas o
que vocês quiserem! Apesar de ter uma grade de matérias obrigatórias, vocês
podem cursá-las na ordem que lhes for mais conveniente e sempre ir intercalando
com optativas eletivas e optativas livres.

Não que isso seja uma boa prática. O seu curso foi pensado e montado com uma
sequência de matérias que faça sentido na sua vida, mas vocês têm independência
para quebrar isso caso seja melhor pra vocês.

\textbf{Vocês não precisam se preocupar com isso agora! As matérias do primeiro
semestre já são inseridas automaticamente, vocês só precisam fazer a própria
matrícula a partir do segundo semestre.}

Quando a hora chegar, vocês vão clicar em “Fazer matrícula” e escolher o tipo de
matéria que vocês querem cursar: Obrigatórias, Optativas eletivas, Optativas
Livres ou Extracurriculares. Quando vocês escolhem um desses blocos, vão
aparecer uma série de matérias disponíveis pra vocês com as turmas (cada turma
tem um horário e um professor específico). Basta vocês irem escolhendo quais
querem cursar, conferir umas 15 vezes e salvar.

Pode até parecer um monstro de 7 cabeças, mas está beeeeem longe disso!
Aproveitem esse momento para consultar aquele veterane que você ama muito e que
vai te mostrar como é tudo bem tranquilo.

Mas quando é mesmo que eu faço isso?

O Jupiter vai abrir algumas “Interações de matrícula”, geralmente duas ou três.
Apenas nesses dias, o botãozinho de “Fazer matrícula” fica disponível. Vocês
recebem e-mails da USP, fica um banner no meio do saguão do bloco B, fazem
evento no facebook e divulgam a data no calendário da USP… Não tem desculpa pra
não saber quando é!

No dia de abertura das interações, o JupiterWeb revela suas verdadeiras garras
(e lentidões) dando absolutamente todos os erros possíveis e tentando
atrapalhar a sua vida. Então, talvez esse não seja o melhor horário pra tentar
se matricular. A ordem de matrícula não importa, então podem ficar tranquilos e
fazer em qualquer momento de qualquer uma das interações (MAS NÃO SE ESQUEÇAM DE
FAZER EM ALGUM MOMENTO!!!).

Depois de cada interação, o Jupiter vai colocar vocês em um ranking que olha seu
semestre, o que vocês deveriam estar cursando, suas notas e o que mais for
conveniente para decidir se vocês ganham ou não uma vaga na disciplina.
Geralmente isso não é um problema com matérias obrigatórias (principalmente se
vocês seguirem sua grade ideal direitinho) ou matérias com poucos inscritos. Mas
se vocês estiver tentando uma optativa muito concorrida (como ``Psicologia da
morte'', por exemplo), se prepare que vocês podem perder uma vaga muito sonhada.
Esse resultado pode ser acessado no ``Resultado das consolidações''.

\end{subsecao}

\begin{subsecao}{Requerimentos}

Por vários motivos, vocês podem ter sua matrícula de uma disciplina negada ou
até mesmo nem conseguir se matricular porque essa disciplina não tem vagas
reservadas para o seu curso. Isso não quer dizer que vocês não vão conseguir
cursar, mas que vocês vão ter que enfrentar mais burocracias para isso. Os
requerimentos, também conhecidos como melhores amigos do aluno da graduação,
são pedidos formais em que vocês inserem a matéria da qual vocês foram
chutados/negados/impedidos de cursar e uma boa justificativa. Esse requerimento
vai para o coordenador do seu curso e possivelmente para o professor de
matéria, pessoa que vai decidir se o seu motivo é bom o suficiente e se o
número de vagas comporta a sua presença.

Vocês podem fazer um requerimento para cursar qualquer matéria, por qualquer
motivo, mas se o pedido for muito esdrúxulo, vocês podem receber um não.

O resultado dos requerimentos demoram 84 anos e um pouco para sair. Por isso,
vocês vão acompanhar as aulas da disciplina requerida mesmo sem saber o
resultado, confiante de que você vai receber um OK depois de um tempo. Existe
um prazo limite para aprovar ou negar um requerimento, então fiquem tranquilos
que até a P1 ou P2, vocês já devem saber se estão matriculados ou não.

Dica: Conversar com os coordenadores e o professor responsável da disciplina
pode ajudar! Se vocês querem muito cursar uma matéria, não tenham vergonha de
lutar por ela :)

\end{subsecao}

\end{secao}
