\begin{secao}{JúpiterWeb}

O \sout{Luciferweb} Jupiterweb é o sistema que administra (quase) tudo que
você, bixo, faz na sua graduação. Desde as suas matrículas até a emissão de
um documento para pagar meia no cinema, o Jupiter vai ser muito útil em
vários momentos se você conhecer tudo que ele tem a oferecer.

\begin{subsecao}{Calendário}

Muito útil pra saber quando é \sout{feriado} que abre a matrícula ou o período
de requerimento.

\end{subsecao}

\begin{subsecao}{Disciplinas}

Mostra todas as disciplinas da USP (até as que não estão mais disponíveis).
Você pode pesquisar por código, por algum trecho do nome, por unidade e por
horário. Além de ser útil para conferir algumas informações sobre aquela
matéria que você precisa cursar, ainda te ajuda a procurar matérias diferentes
pela USP (como “Pintura”, “Psicologia da Morte” ou “Pesca Sustentável”).

\end{subsecao}

\begin{subsecao}{Grade Horária}

Lista as disciplinas que você está inscrito/matriculado/pendente com os
horários de cada uma. É onde geralmente você confere se tá tudo certo com a sua
matrícula ou não. Quando as aulas começam, o ideal é que você esteja
“Matriculado” (a matéria fica cinza). Se alguma disciplina aparecer como
pendente (fica azul), então corra na seção de alunos para conferir o que está
errado!

\end{subsecao}

\begin{subsecao}{Resumo Escolar}

O resumo escolar é um documento enorme com tudo de bom (e de ruim) que
aconteceu na sua vida acadêmica na USP. Lá você pode conferir cada disciplina
cursada (com nota), as disciplinas que você trancou, seus aproveitamentos de
estudo, quantos créditos você já fez e a tão mística “Média ponderada”.

\end{subsecao}

\begin{subsecao}{Acompanhamentos}

\begin{itemize}
  \item \textbf{Dados do Programa}

    Compilação da sua vida universitária, desde o Vestibular até a hora que
    você se forma.
  \item \textbf{Evolução no Curso}

    É o jeito visual de acompanhar as disciplinas que você já fez e o que está
    faltando. Sua principal função é ser printado ao final do curso (com tudo
    verde) para postar no Facebook e fazer sucesso com a família na internet.
  \item \textbf{Rendimento Acadêmico}

    Mais um lugar pra você conferir suas notas, quantidade de matérias
    cursadas, e quantidade de créditos cursados, só que aqui tudo aparece em
    forma de gráfico.
  \item \textbf{Perfil de deficiências}

    Questionário da USP sobre deficiências.

  \item \textbf{Meus benefícios e bolsas}

    Onde ficam registradas as suas bolsas e alguns serviços prestados na
    universidade que valem dinheirinhos. Também mostra as datas que a USP te
    paga para você organizar sua vida financeira!!

  \item \textbf{Dados pessoais}

    O nome é bem intuitivo. Mantenha tudo atualizado para não ter problemas com
    a SPTrans, COSEAS ou qualquer coisa que confira seus dados em algum momento.

\end{itemize}

\end{subsecao}

\begin{subsecao}{Emissão de Documentos}

Emite alguns documentos que precisamos no dia a dia, como um comprovante de
matrícula ativa. Dependendo do documento, você vai precisar ir até a seção de
alunos, mas os mais requisitados você encontra por aqui.

\end{subsecao}

\begin{subsecao}{Matrícula}

Aqui começa uma das partes mais legais da Universidade: A matrícula! Pois é
bixo, você não vai cursar mais o que a sua instituição de ensino manda, mas o
que você quer! Apesar de ter uma grade de matérias obrigatórias, você pode
cursá-las na ordem que lhe for mais conveniente e sempre ir intercalando com
optativas eletivas e optativas livres.

Não que isso seja uma boa prática. O seu curso foi pensado e montado com uma
sequência de matérias que faça sentido na sua vida, mas você tem independência
para quebrar isso caso seja melhor pra você.

\textbf{Você não precisa se preocupar com isso agora! As matérias do primeiro
semestre já são inseridas automaticamente, você só precisa fazer a própria
matrícula a partir do segundo semestre.}

Quando a hora chegar, você vai clicar em “Fazer matrícula” e escolher o tipo de
matéria que você quer cursar: Obrigatórias, Optativas eletivas, Optativas
Livres ou Extracurriculares. Quando você escolhe um desses blocos, vão aparecer
uma série de matérias disponíveis pra você com as turmas (cada turma tem um
horário e um professor específico). Basta você ir escolhendo quais quer cursar,
conferir umas 15 vezes e salvar.

Pode até parecer um monstro de 7 cabeças, mas está beeeeem longe disso!
Aproveitem esse momento para consultar aquele veterane que você ama muito e que
vai te mostrar como é tudo bem tranquilo.

Mas quando é mesmo que eu faço isso?

O Júpiter vai abrir algumas “Interações de matrícula”, geralmente duas ou três.
Apenas nesses dias, o botãozinho de “Fazer matrícula” fica disponível. Você
recebe e-mails da USP, fica um banner no meio do saguão do bloco B, fazem
evento no facebook e divulgam a data no calendário da USP… Não tem desculpa pra
não saber quando é!

No dia de abertura das interações, o JúpiterWeb revela suas verdadeiras garras
(e lentidões) dando absolutamente todos os erros possíveis e tentando
atrapalhar a sua vida. Então, talvez esse não seja o melhor horário pra tentar
se matricular. A ordem de matrícula não importa, então pode ficar tranquilo e
fazer em qualquer momento de qualquer uma das interações (MAS NÃO SE ESQUEÇA DE
FAZER EM ALGUM MOMENTO!!!).

Depois de cada interação, o Jupiter vai colocar você em um ranking que olha seu
semestre, o que você deveria estar cursando, suas notas e o que mais for
conveniente para decidir se você ganha ou não uma vaga na disciplina.
Geralmente isso não é um problema com matérias obrigatórias (principalmente se
você seguir sua grade ideal direitinho) ou matérias com poucos inscritos. Mas
se você estiver tentando uma optativa muito concorrida (como “Psicologia da
morte, por exemplo), se prepare que você pode perder uma vaga muito sonhada.
Esse resultado pode ser acessado no “Resultado das consolidações”.

\end{subsecao}

\begin{subsecao}{Requerimentos}

Por vários motivos, você pode ter sua matrícula de uma disciplina negada ou até
mesmo nem conseguir se matricular porque essa disciplina não tem vagas
reservadas para o seu curso. Isso não quer dizer que você não vai conseguir
cursar, mas que você vai ter que enfrentar mais burocracias para isso. Os
requerimentos, também conhecidos como melhores amigos do aluno da graduação,
são pedidos formais em que você insere a matéria que você foi
chutado/negado/impedido de cursar e uma boa justificativa. Esse requerimento
vai para o coordenador do seu curso e possivelmente para o professor de
matéria, pessoa que vai decidir se o seu motivo é bom o suficiente e se o
número de vagas comporta a sua presença.

Você pode fazer um requerimento para cursar qualquer matéria, por qualquer
motivo, mas se o pedido for muito esdrúxulo, você pode receber um não.

O resultado dos requerimentos demoram 84 anos um pouco para sair. Por isso,
você vai acompanhar as aulas da disciplina requerida mesmo sem saber o
resultado, confiante de que você vai receber um OK depois de um tempo. Existe
um prazo limite para aprovar ou negar um requerimento, então fiquem tranquilos
que até a P1 ou P2, você já deve saber se está matriculado ou não.

Dica: Conversar com os coordenadores e o professor responsável da disciplina
pode te ajudar! Se você quer muito cursar uma matéria, não tenha vergonha de
lutar por ela :)

\end{subsecao}

\end{secao}
