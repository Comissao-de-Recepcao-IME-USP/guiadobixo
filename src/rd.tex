\begin{secao}{O que é RD?}

Antes de mais nada, RD significa Representante Discente. O RD é um aluno que 
representa nossos interesses frente aos diversos conselhos e comissões existentes, 
sendo um forte elo de ligação entre professores e alunos. O RD ajuda a tomar 
decisões que impactam todo o IME, como, autorização para festas, mudanças 
no currículo, aumento de vagas na FUVEST, quantidade de bolsas, reformas, 
mudança no corpo docente (às vezes lutamos para tirar algum professor), 
enfim, coisas desse tipo e muitas mais.

Acho que já deu para perceber o quanto é importante ter um aluno em cada um desses
conselhos. Infelizmente, não costumamos preencher todas as vagas. Isso se 
deve ao desinteresse de alguns ou falta de tempo da maioria de seus VETERANOS. 
E vocês é quem tem mais tempo para fazer as coisas funcionarem aqui, já que 
ainda não sabem o que é rec, DP, trabalho, estágio, etc. Portanto, se quiserem 
fazer alguma coisa pelo lugar onde vocês, bixos, vão estudar, está aí
uma dica.

Como RDs vocês poderão entender melhor o funcionamento do Instituto e ajudar no 
processo de melhorá-lo. Vocês também poderão melhorar o relacionamento entre 
estudantes, professores e servidores e entender como os professores pensam. 
As eleições são organizadas pelo CAMat no final do segundo semestre e o 
mandato é de um ano.

Aqui vai um breve resumo do que mais ou menos acontece em cada um dos colegiados 
nos quais temos direito a representante(s):

No IME, temos 23 cargos de RD com 31 vagas no total, sendo 17 reservadas 
para graduação, 11 para pós e 3 livres. Todos tem direito a um suplente.
Caso o RD não possa ir a alguma reunião ou, por algum motivo da vida, tenha 
que abandonar o cargo, o suplente assume em seu lugar e o cargo não fica 
sem representante.

Existem diferentes níveis de hierarquia na administração.

{\bf As CoCs,
Comissões Coordenadoras de Curso (Lic, Pura, Estatística, Aplicada, BMAC e
Computação)} são as mais próximas dos alunos. Temos um cargo de aluno em cada
comissão. São comissões pequenas, que tratam dos problemas internos de cada
curso: mudança de currículo, requerimentos, optativas, etc. São subordinadas 
à CG e ao conselho do relativo departamento. Analogamente, temos um cargo em cada
Comissão Coordenadora de Programa (de Pós).

{\bf Os Conselhos de Departamento (MAT, MAE, MAC e MAP)} têm uma dinâmica um
pouco diferente das CoCs: são mais formais. Cada conselho se reúne (quase)
mensalmente e são formados (em geral) por mais pessoas, sendo que existem
regras sobre participação dos diferentes níveis hierárquicos de
professores (Titular, Associado, Doutor e Assistente). Nesses conselhos, além
de aprovar algumas das decisões das Comissões Coordenadoras de Curso e de
Programa (pós) e distribuição de carga didática, são discutidos re-oferecimento
de curso, revisão de prova, supervisão das atividades dos docentes,
afastamentos (temporários ou não), contratação de professores e muitas outras
coisas.
Os Conselhos de Departamento são subordinados à Congregação e ao CTA.

{\bf A Comissão de Graduação (CG)}, basicamente, avalia requerimentos,
mudança/criação de cursos e jubilamentos. Analogamente, existe a Comissão de
Pós-Graduação (CPG). Ambas são subordinadas à Congregação.

{\bf A Comissão de Cultura e Extensão (CCEx)} quase nunca tem reunião. Cuida
das atividades de extensão: Matemateca, CAEM, etc.

Também há comissões mais específicas, como a comissão de estágio, a comissão de 
pesquisa (do doutorado) e o Centro de Competência em Software Livre (CCSL), da computação.

Os dois conselhos mais importantes são o CTA e a Congregação, ambos presididos
pelo Diretor.

{\bf O Conselho Técnico e Administrativo (CTA)} cuida de todas as questões não
acadêmicas: Orçamento, reformas, avaliação dos funcionários, Xerox, etc. É 
formado pelos quatro chefes de departamento, diretor, vice-diretor, um
representante dos funcionários e um RD.

{\bf A Congregação} é o órgão máximo do Instituto. Inclui muitos professores, a
maioria titular. São dois RDs de graduação e um de Pós. Basicamente,
nesse órgão, são rediscutidas e aprovadas (ou não) muitas das decisões
dos órgãos subordinados. Os membros da Congregação tem voto na eleição para
Reitor e Vice-Reitor.


Bom, caso não tenha ficado claro desde o começo desse texto, percebam que é 
muito importante ter um aluno em cada um desses conselhos. Se estiverem tendo 
problemas com professores, requerimentos, etc, ou simplesmente quiserem saber 
o que anda aconecendo, procurem o RD certo pra conversar. Perguntem, participem, 
votem e façam o IME um lugar melhor.

\end{secao}
