\begin{subsecao}{Fodinha}

Assim como Truco, Fodinha (Te fode, Se fode aí, entre outras variações de como é chamado) é um jogo de buteco, inclusive Fodinha é o jogo
que você joga quando quer jogar Truco, mas tem uma quantidade ímpar de 
pessoas, pouco baralho para muita gente ou porque não vai dar tempo, entre outros motivos. Mas
diferente do Truco o Fodinha é um jogo individual jogado de 3 até o que o seu bom senso permitir de jogadores

Assim como Truco também utiliza-se um baralho sem as cartas 8, 9 e 10, a sequência
de cartas é a mesma, (da maior) 3, 2, A, K, J, Q, 7, 6, 5, 4 (pra menor)
e existe uma manilha(se você não sabe o que é, mais pra frente a gente explica).

A cada mão é distribuído um número diferente de cartas para cada jogador. Os
jogadores começam o jogo com uma carta cada, e a cada mão aumenta em 1 a
quantidade da cartas recebidas até não ser possível mais distribuir essa quantidade de cartas para os jogadores, a partir daí 
o jogo começa a voltar, ou seja em cada mão uma carta a menos é distribuída até que uma carta apenas seja distribuída para 
cada um, essa é a última rodada do jogo. 

Após a distribuição, uma carta é virada, ela determina qual será a 
manilha (mais uma semelhança com Truco) da rodada. A manilha será sempre a próxima carta mais forte da que foi virada.
Se a carta virada for um J, a maninha será o K. Isso significa que nessa mão, o
K passa a ser a carta mais forte do jogo. Entre as manilhas existe uma
hierarquia de naipe. A carta de paus $\clubsuit$  é a mais forte seguida da de copas $\heartsuit$,
espadas $\spadesuit$ e ouros $\diamondsuit$.

Em seguida rodando no sentido anti-horário, começando pela direita do carteador cada jogador deverá "apostar" 
quantas rodadas ele "faz", após todos apostarem começa a 1ª rodada, onde cada jogador (na mesma ordem que apostaram) 
devem descartar uma carta da sua mão, quando todos tiverem descartado, o jogador que jogou a maior carta faz a rodada 
e uma nova rodada começa, seguindo no mesmo sentido, começando pelo jogador que fez a última rodada, até que acabem
as cartas nas mãos dos jogadores.

Ao final da mão então cada jogador irá comparar o número de rodadas que fez com o que apostou e receberá de pontos (ou fodes) 
o módulo da diferença entre os 2. No final do jogo perde aquele que tem mais pontos, e ganha o que tem menos.

Em cada mão o último jogador a apostar é obrigado a falar um número de forma que não seja possível todo mundo acertar a 
aposta, ou seja, se por exemplo está na 7ª rodada e a soma das apostas dos jogadores até agora é 5, o último jogador não pode 
apostar que faz 2.

Durante uma rodada se um jogador joga uma carta de número igual a um que já saiu naquela rodada as duas se cancelam e saem da 
disputa, mesmo que sejam as maiores cartas da rodada, de forma que uma carta menor faça a rodada. Essa regra não vale para 
manilhas, pois existe uma hierarquia entre elas.

A primeira e a última mão são especiais, ou sejas nas duas mãos com apenas uma 
carta os jogadores colocam a carta que receberem na testa de forma que todos os outros jogadores 
vejam sua carta menos ele próprio, assim ele deve fazer sua aposta, baseado na cartas dos outros e não na sua.

