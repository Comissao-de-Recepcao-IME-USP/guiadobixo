\begin{secao}{CCSL - Centro de Competência em Software Livre}

Se você já se cansou da definição tradicional de software (``hardware é a parte
que você chuta, software é a parte que você xinga'') e prefere pensar que
``software é a parte com que você faz o que quiser'', o Software Livre é para você. 
Baixar de graça? Sim! Copiar para o amigo? Claro! Fuçar, mexer e hackear? Com
 certeza! E aqui no IME ele faz parte do ensino e da pesquisa, literalmente, há
 décadas! Tanto que o IME tem o CCSL --- Centro de Competência em Software Livre
 --- que existe para dar apoio a essas atividades internamente e promover a
 difusão do Software Livre fora da Universidade também.

Você pode participar das reuniões do Centro, que ocorrem a cada 15 dias, e
 colaborar com as atividades que ele promove; ou pode mergulhar em algum dos
 projetos apoiados por ele, melhorando o código. Alguns dos projetos podem vir a
 ser ótimos temas para seu TCC ou para trabalhos de iniciação científica (com
 bolsa! :D). Então você ainda junta o útil ao agradável.\\
Visite \url{http://ccsl.ime.usp.br} e inscreva-se na lista de discussão,
 conheça os projetos e veja quando será a próxima reunião!

Ah, mas você ainda nem sabe direito o que é Software Livre? Não se preocupe.
Você ainda vai ouvir falar muito dele, tanto no IME quanto fora dele. Além
disso vai haver uma apresentação a respeito no começo do semestre, fique ligado! %REFTIME

\end{secao}
