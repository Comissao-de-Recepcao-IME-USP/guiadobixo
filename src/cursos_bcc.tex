\begin{subsecao}{Computação}

Muito bem, bixo, você conseguiu passar em Computação! E, depois de tanto esforço
e dedicação, você vai finalmente poder descansar e relaxar, certo? Errado!

Se você pretende se formar no tempo ideal (4 anos), você precisará se dedicar
em tempo integral ($\int$) ao curso, pelo menos nos dois primeiros anos. Ou
seja, arrume um bom paitrocínio, se possível. Senão, uma boa alternativa é
pedir uma bolsa trabalho do COSEAS, que paga um salário mínimo e só vai
tomar 40h do seu mês e, portanto, não vai atrapalhar tanto seus estudos.
Normalmente você não vai conseguir fazer estágios de verdade antes do 3º
ano, por causa das aulas do período da tarde. Então aproveite seu curso! 
Preocupe-se em trabalhar quando tiver mais tempo ``livre''.

Nos próximos dois anos, você vai estudar todo tipo de matérias, que na maioria
das vezes, irá parecer completamente inútil (e algumas vezes realmente
será).

Tudo começa com a temida trilogia de quatro Cálculos:
\begin{itemize}
\item Cálculo I - O Guia do Computeiro das Galáxias
\item Cálculo II - O Gradiente do Fim do Universo
\item Cálculo III - A Integral de Linha, O Rotacional e Tudo O Mais
\item Cálculo IV - Até mais, e Obrigado pelo 5 bola!
\end{itemize}

Como se não bastasse, ainda temos que passar pelas maratonas
de Álgebra (Álgebra I e II e Álgebra Linear), Física (I e II),
Estatística (Estatística I, II e Processos Estocásticos), e ainda dos MAC's que
na verdade são MAT's (MAC300, Programação Linear). Mas não entre em pânico!
Algumas matérias legais (as da computação de verdade) vão aparecer também em
momentos aleatórios e cada vez mais constantes.

Muitos dizem que toda essa maratona de matemática foi inventada para
torturá-lo. Eles estão certos. Mas na verdade, ela serve para te dar uma
boa ``base'' em matemática, já que toda a teoria da computação envolve
matemática e, como futuro possível pesquisador (isso é Ciência da Computação,
que é diferente de SOS computadores, Microcamp e afins), você precisa estar
preparado para trabalhar com ela. Além disso, dizem que a matemática desenvolve
um raciocínio lógico extremamente necessário para a programação (basta notar
que as pessoas que são boas em programação geralmente são boas em matemática,
ou não).

Em 2015, a grade do BCC passou por algumas alterações, algumas matérias
tiveram sua ementa alterada (como Álgebra I, que agora se chama
Fundamentos de Matemática para a Computação) e entraram matérias novas como Introdução à 
Ciência da Computação. Essa mudança tem como objetivo mudar a estrutura curricular
atual do nosso curso. 

Em 2016, outras mudanças à essa estrutura serão inseridas.
Para ver como será o currículo, acesse: http://bcc.ime.usp.br/curriculo2016/ .

Para quaisquer dúvidas quanto às mudanças e às adaptações da grade, pode-se
contatar o professor Coelho (coelho@ime.usp.br).

A partir de certo momento, que você mesmo determina, é possível seguir uma
ou mais áreas mais específicas da computação, como Computação Gráfica ou
Inteligência Artificial por exemplo, puxando determinadas matérias como
optativas (que na verdade você é obrigado a cursar). Você pode criar uma grade
bem legal, de acordo com seu gosto, pois você escolhe quais matérias vai
cursar (tirando as obrigatórias).

Essa formação teórica prepara você para contornar todo tipo de problema que
você possa vir a encontrar em sua vida profissional. Na verdade, não. Na sua
vida profissional, você pode ter, por exemplo, que programar em $C\#$, Asp.NET,
aprender uma nova linguagem de programação bizarra ou fazer alguma coisa que
aparentemente não tem nada a ver com o que você aprendeu na faculdade. E você
dirá ``Mas eu não tive uma aula de Como Programar na Linguagem Stavromula
Beta!''. O que importa é que você (teoricamente) sabe os princípios da
programação e pode aplicar esse conhecimento para dominar rapidamente ``toda''
e ``qualquer'' linguagem, tecnologia, etc. O BCC não é um curso que ensina N
linguagens (na verdade, N = 2 ou 3, dependendo da boa vontade dos professores) e
como usar M programas e recursos. O BCC é um curso que ensina a técnica e a
teoria que te dará uma base sólida para você aprender qualquer
coisa. E essas N + M coisas que vão te ensinar vão te ajudar bastante a
entender tudo.

Finalmente, esteja sempre atento aos eventos promovidos pela Empresa Júnior,
pelo CAMAT e pelo instituto, que ajudarão a complementar sua formação. Boa
sorte, pois você vai precisar. Use Linux e memorize esta
mensagem: ``SegmentationFault''. Ela será uma assombração que perseguirá você
pelo resto do curso.

\end{subsecao}
