\begin{subsecao}{Computação}

Muito bem, bixos, vocês conseguiram passar em Computação! Depois de tanto esforço
e dedicação, vocês vão finalmente poder descansar e relaxar, certo? Errado!

Se vocês pretendem se formar no tempo ideal (4 anos), vocês precisarão se
dedicar em tempo integral ($\int$) ao curso, pelo menos nos dois primeiros
anos. Ou seja, arrumem um bom paitrocínio, se possível. Senão, caso vocês ainda
precisem daqueles papeis coloridos que deixam as pessoas felizes, uma boa
alternativa é pedir uma bolsa trabalho do COSEAS, que paga um salário mínimo e
só vai tomar 40h do seu mês e, portanto, não vai atrapalhar tanto seus estudos.
Normalmente vocês não vão conseguir fazer estágios de verdade antes do 3º ano,
por causa das aulas do período da tarde. Então aproveitem o curso!  Preocupem-se
em trabalhar quando tiverem mais tempo ``livre''.

%REFTIME
A partir deste ano está sendo introduzida uma nova grade curricular para o
BCC. Isso mesmo bixos, vocês serão nossas cobaias. Mas não se preocupem, a
mudança foi extensivamente discutida durante os últimos anos por professores em
conjuntos de VETERANOS voluntários, além de suportado por dados obtidos por
pesquisas feitas com alunos de todos os anos da história do BCC. Portanto foi
projetada com muito amor de forma a otimizar sofrimento e falta de vida
social. Mais informações sobre a reforma podem ser encontradas no site
\url{http://bcc.ime.usp.br/curriculo2016/}. Informações mais completas podem ser
encontradas no pequeno relatório de 1000 páginas, acessível em
\url{http://www.ime.usp.br/~batista/reformulacao.pdf}.

bixos, vocês podem estar pensando ``Há! Finalmente sai do colégio! Faço
computação e passarei o dia inteiro no computador! Relogios digitais sao uma
grande ideia!''. Adivinhem, a realidade não é nem um pouco próxima disso. Antes
da reforma o curso já começava com a famosa trilogia dos quatro cálculos:
\begin{itemize}
\item Cálculo I - O Guia do Computeiro das Galáxias
\item Cálculo II - O Gradiente do Fim do Universo
\item Cálculo III - A Integral de Linha, O Rotacional e Tudo O Mais
\item Cálculo IV - Até mais, e Obrigado pelo 5 bola!
\end{itemize}
Com a última reforma do currículo a trilogia se tornou de fato só 3 cálculos,
sendo os cálculos III e IV colapsados em um Cálculo III.V especial para o BCC,
mas a piada foi mantida pois seus VETERANOS gostam muito dela. Ainda assim, não
se engane. Isso ainda continua sendo bastante cálculo em suas vidas.

E a matemática não para por ai. Durante os dois primeiros anos vocês ainda
precisam fazer duas matérias de estatística, vetores e geometria, álgebra linear, uma
optativa de ciências, e como não bastasse, existem matérias do MAT disfarçadas
como MAC (como MAC0105, equivalente a Álgebra I de seus VETERANOS). Muitos dizem
que toda essa maratona de matemática foi inventada para torturá-los. Eles estão
certos. Mas alem disso, ela serve para dar uma boa ``base'' em matemática, já
que toda a teoria da computação envolve matemática e, como futuros possíveis
pesquisadores (isso é Ciência da Computação, que é diferente de SOS
computadores, Microcamp e afins), vocês precisam estar preparados para trabalhar
com ela. Além disso, dizem que a matemática desenvolve um raciocínio lógico
extremamente necessário para a programação (basta notar que as pessoas que são
boas em programação geralmente são boas em matemática, ou não). Mas não se
preocupem, vocês terão algumas MAC's durante esses primeiros anos.

Um momento importante na graduação de um BCÇoide é o segundo ano. Nele começamos
a ter que escolher optativas eletivas de acordo com nossos gostos por áreas
específicas da computação. Vai acontecer de optativas que vocês queiram fazer
acabem não sendo oferecidas nos momentos em que vocês podiam fazê-las, bixos,
mas é a vida. Ainda assim, a liberdade no curso quanto a escolha de optativas é
bem grande. Assim vocês podem montar o curso de acordo com o gosto de vocês,
como por exemplo escolher matérias para o lado de inteligência artificial, assim
vocês podem criar um programa chamado Smith e acabar com a Matrix.

Uma opção nova no curso são as trilhas ou ênfases, que são caminhos que vocês
podem seguir no curso. Se vocês completarem, até o final do curso, os requisitos
de alguma trilha, você ganha um diploma com uma nota de ênfase em alguma área da
computação. Como esse ainda é um conceito do novo currículo com o qual seus
VETERANOS nunca tiveram experiência, não podemos ainda dizer muito sobre. Mas
não se preocupem, vocês cursarão as matérias MAC0101 e MAC0102, feitas
justamente para que você descubra como o BCC é um curso divertido, para que
vocês conheçam as possibilidades de áreas e trilhas.

Como vocês puderam perceber, bixos, BCC é uma formação bem teórica. Essa
formação teórica prepara você para contornar todo tipo de problema que você
possa vir a encontrar em sua vida profissional. Na verdade, não. Na sua vida
profissional, você pode ter, por exemplo, que programar em $C\#$, Asp.NET,
aprender uma nova linguagem de programação bizarra ou fazer alguma coisa que
aparentemente não tem nada a ver com o que você aprendeu na faculdade. E você
dirá ``Mas eu não tive uma aula de Como Programar na Linguagem Stavromula
Beta!''. O que importa é que você (teoricamente) sabe os princípios da
programação e pode aplicar esse conhecimento para dominar rapidamente ``toda'' e
``qualquer'' linguagem, tecnologia, etc. O BCC não é um curso que ensina N
linguagens (na verdade, N = 2 ou 3, dependendo da boa vontade dos professores) e
como usar M programas e recursos. O BCC é um curso que ensina a técnica e a
teoria que te dará uma base sólida para você aprender qualquer coisa. E essas N
+ M coisas que vão te ensinar vão te ajudar bastante a entender tudo.

Finalmente, esteja sempre atento aos eventos promovidos pela Empresa Júnior,
pelo CAMAT e pelo instituto, que ajudarão a complementar sua formação. Boa
sorte, pois você vai precisar. Use Linux e memorize esta
mensagem: ``Segmentation Fault''. Ela será uma assombração que perseguirá você
pelo resto do curso.

\end{subsecao}
