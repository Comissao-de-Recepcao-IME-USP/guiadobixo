\begin{subsecao}{Espadas} 

Bixos, se vocês já leram sobre o Cagando, e entenderam meio por cima como é o
jogo de Copas, então Espadas será fácil pra vocês.

Para começar, 13 cartas são distribuídas para cada um dos jogadores, que jogam
em parceria com o jogador à frente. Neste jogo o naipe de espadas será sempre o
trunfo.

Seguindo a partir da esquerda do carteador, cada jogador escolhe o número de
vazas que acha que vai fazer. Como é um jogo de duplas, as pedidas de cada
parceria serão somadas e ambos devem jogar para cumprir esse contrato*. 

Além disso, qualquer jogador pode dizer que não fará nenhuma vaza, um contrato
chamado de NIL, que é especial, pois separa o jogo de seu parceiro, tendo cada
um o seu contrato.

Nesse jogo, um jogador só pode abrir espadas depois que algum outro jogador já
tenha jogado uma carta de espadas em uma vaza de outro naipe.
\begin{description}

\item[Pontuação:]

Para cada vaza de um contrato são atribuídos 10 pontos. Se a parceria falha em
cumprir tal contrato, a dupla perde o valor do contrato. Se a parceria consegue
cumprir tal contrato, ela ganha o valor do contrato, e mais um ponto
na bolsa* da parceria para cada vaza feita a mais que o estipulado.
Se o jogador que fez a vaza tenha pedido Nil, a dupla não ganha mais pontos.
Apenas sobe o valor da bolsa*.

\item[Bolsa:]

Para evitar que os contratos sejam feitos muito baixos, e estimular a precisão
nas escolhas iniciais, cada equipe mantém uma bolsa, que é uma pontuação
separada que vai enchendo conforme vazas a mais que o contrato são feitas. Uma
bolsa estoura quando 10 vazas são adicionadas, tirando 100 pontos da parceria
que fez essas vazas a mais.

\item[O Nil:]
Quando alguém diz que não irá fazer vaza alguma, essa jogada é chamada de Nil.
Tal jogada separa o contrato de seu parceiro, e vale por si só 100 pontos. Um
nil cumprido ganha 100 pontos, enquanto um nil perdido, além de perder tais
pontos, adiciona as vazas feitas na bolsa e não ganha nenhum ponto extra por
vaza.

\item[O Blind Nil:]
Situações dramáticas pedem por atitudes dramáticas, e o Blind Nil é uma delas.
Como o nome já diz, o Blind Nil é pedido sem ver as cartas e por isso, vale o
dobro dos pontos!

\end{description}
Ganha o jogo a equipe que chega em 500 pontos primeiro, ou vocês ainda podem
perder o jogo chegando a -200 pontos. Essa pontuação pode ser alterada pelo
veterano em virtude dos horários de aula ou outros fatores limitantes de
tempo...

\end{subsecao}
