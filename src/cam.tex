\begin{subsecao}{Comissão de Acolhimento da Mulher - CAM}

A Comissão de Acolhimento da Mulher (CAM) foi criada no dia 9 de Novembro de 2016,
pela Portaria IME 1677, após a realização de várias reuniões abertas com amplos
debates durante praticamente um ano com o Coletivo Mulheres do IME, o qual 
apresentou a proposta ao Diretor do Instituto, tendo sido aprovada pela Congregação
da Instituição por unanimidade.

Somos uma Comissão assessora da diretoria do IME cuja a principal atribuição 
é dar acolhimento às vítimas de discriminação de gênero, de assédio moral e 
sexual e de violência contra a mulher, quando essas ocorrências envolverem 
pessoas da comunidade do IME ou tenham ocorrido em suas dependências.

Essas discriminações tornam a convivência mais difícil e provocam nas mulheres
a sensação de desrespeito e de não pertencimento. A criação de uma Comissão de 
Acolhimento da Mulher contribuirá para o combate institucional à violência contra
a mulher, à desigualdade de gênero e aos efeitos da cultura patriarcal na academia.

A Comissão é formada por duas professoras, duas funcionárias e duas alunas, 
eleitas por seus pares. Conheça a gestão atual: 

\textbf{Professoras}: 
\begin{itemize}
  \item Lucia Junqueira (lucia@ime.usp.br)
  \item Viviana Giampaoli (vivig@ime.usp.br)
\end{itemize}

\textbf{Funcionárias}: 
\begin{itemize}
  \item Gianne Uchôa (gianne@ime.usp.br)
  \item Luiza Ribeiro Camilo (luizinha@ime.usp.br)
\end{itemize}

\textbf{Alunas}: 
\begin{itemize}
  \item Ana Luiza Tenório (ana.tenorio@usp.br)
  \item Cinthia Saraiva (cinthia.saraiva.santos@usp.br)
\end{itemize}

As mulheres que procurarem a comissão poderão, se quiserem, indicar com qual ou 
quais de seus membros desejam conversar.

Para mais informações, acesse o site \url{https://www.ime.usp.br/~cam} ou veja
a página da CAM no facebook procurando por "Comissão de Acolhimento da Mulher - CAM" 
ou envie um e-mail para {\tt cam@ime.usp.br}. 


\end{subsecao}
