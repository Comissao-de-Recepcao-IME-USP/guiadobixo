\begin{subsecao}{Rede Linux}

\figurapequenainline{rede_linux}

\begin{subsubsecao}{Introdução}

A Rede Linux é uma rede de computadores, administrada por alunos do IME e
que fornece diversos serviços para os VETERANOS e até mesmo para vocês, bixos.
Ela disponibiliza:

\begin{itemize}
\item 2 salas de computadores (no bloco A) com todo\footnote{ Se um programa
estiver faltando, mande um email pra admin@linux.ime.usp.br pedindo-o.} tipo de
programa necessário para suas atividades acadêmicas (com pelo menos uma que fica
aberta 24 horas por dia, 7 dias por semana\footnote{ Mas talvez vocês não
consigam entrar no bloco A depois da meia noite, que é quando a portaria
fecha.});
\item Uma página na internet para cada aluno;
\item Um \textit{e-mail} para cada aluno;
\item Espaço para você guardar seus arquivos;
\item Acesso remoto via ssh (linux.ime.usp.br);
\item Impressoras;
\item Admins dispostos e capazes, para o caso de algum usuário ter alguma boa
ideia para adicionar a esta lista;
\end{itemize}
\end{subsubsecao}

\begin{subsubsecao}{O Linux}

A rede utiliza em todos os seus computadores um sistema operacional chamado
Linux. Esse é um sistema desenvolvido de forma colaborativa pelos usuários
e empresas interessados nele (se quiser saber mais a respeito, pesquise
por ``software livre''!).

O Linux não é um sistema mais difícil de usar que o Windows. É apenas
diferente em alguns aspectos. Além de tudo, existem cursos de Linux que são
organizados pelos alunos do IME. Os admins costumam promover esses cursos.
Fiquem atentos aos emails!

Não se deixem intimidar pelo sistema. Se vocês se derem ao trabalho de
aprender a utilizá-lo bem, verão que ele é bastante flexível, e até mesmo
interessante (tanto quanto um sistema operacional pode ser =P).

\end{subsubsecao}

\begin{subsubsecao}{Os admins}

Os admins são alunos do bacharelado em ciência da computação (vulgo BCC) que
são responsáveis por administrar a rede. Entre outras coisas, isso quer dizer
manter os computadores funcionando, ajudar os alunos a usar a rede (com
cursos\footnote{ Fiquem atentos aos emails!!!} e resolvendo dúvidas nos
horários de plantão\footnote{ Na página da rede, estão os horários de todos os
admins.}) e também implementar coisas novas na rede (aceitamos sugestões!)

Os admins são escolhidos por um treinamento que acontece de dois em dois anos,
em todo ano par. Mais informações serão divulgadas quando este estiver próximo
a ocorrer.

\end{subsubsecao}
\begin{subsubsecao}{Como criar uma conta?}

Basta passar na Admin, na sala 125 do bloco A (como vocês são bixos: bloco A é o da
biblioteca, bloco B aquele que tem muitas salas de aula e que vocês vão passar boa
parte da vida de vocês). Contatos:
\begin{description}

\item [e-mail:] admin@linux.ime.usp.br
\item [Página:] \url{www.linux.ime.usp.br}
\item [Sala:] 125, bloco A

\end{description}
\end{subsubsecao}

\end{subsecao}
