\begin{subsecao}{Pura}

\quadrinhos{19}

Ufa, vocês chegaram à Pura! Sejam bem-vindos! Nesse curso vocês serão apresentados a
muitas das diversas áreas da matemática pura. Muitas pessoas entram no curso
sem saber direito do que ele se trata e descobrem que afinal era bem diferente
daquilo que esperavam que fosse. Mas estamos aqui para ajudar vocês!
%\begin{enumerate}[label=\roman{*})]

i) Como é o curso da Pura?

\begin{itemize}

\item  Existem matérias extremamente úteis e práticas para o dia-a-dia e é
exatamente por isso que vocês vão acabar as odiando (Ex. Estatística,
Computação, Física...). Mas felizmente elas acabam antes do fim do segundo ano!
E se por acaso acabarem gostando podem pegar mais matérias sobre esses assuntos
como optativas.
\item No segundo ano começam as matérias específicas do curso, como Análise Real,
Anéis e Corpos, EDO... Aí é que a coisa fica interessante e vocês vão sentir
mais o que é a Pura! Talvez vocês não saibam ainda, mas matemática a partir de
agora vai ser algo muito maior do que fazer contas para achar uma resposta e aplicar
algoritmos. A partir de agora tudo tem que ser provado. O que matemáticos
fazem é provar teoremas e estudar estruturas matemáticas como corpos, grupos,
espaços topológicos, espaços vetoriais, espaços de medida... Vocês provavelmente
não sabem o que é nada disso, mas não se preocupem, vocês vão descobrir.
\item O Bacharelado em matemática não tem TCC! Mas não comemorem ainda, no final da
graduação vocês vão precisar fazer uma matéria chamada "Introdução ao trabalho Científico"
em que precisarão fazer um trabalho de Iniciação Científica com algum professor. A
disciplina vai ajudar muito aqueles que pretendem seguir carreira acadêmica, mas não
precisam se preocupar com isso por enquanto, pois é recomendado que só cursem essa
matéria no último ano.
\item Em 2015, a matéria Números Inteiros passou a ser ministrada no primeiro
semestre de curso, e esse vai ser o melhor momento para aprenderem a fazer boas
demonstrações. Em 2017, juntaram as duas geometrias diferenciais em apenas uma
matéria e cálculo VI passou a ser obrigatória. Agora o curso tem os cálculos 1, 2, 3,
5 e 6 (Para onde foi cálculo 4? A demonstração fica a cargo do leitor).
O curso está sempre se atualizando para oferecer o melhor para os estudantes,
então se mobilizem para continuar melhorando a Pura.
\item  Vocês vão ter que estudar muito, mas nem por isso desistam. Vão às monitorias,
peçam ajuda a seus colegas e aos veteranes, se ajudem e lembrem que ninguém é melhor 
que ninguém por saber mais disso ou daquilo!

\end{itemize}

ii) O que fazer depois de se formar??

O objetivo principal do Bacharelado em Matemática é formar bons pesquisadores
em... Matemática! Para quem não sabe, a matemática não está completa (e nunca
vai estar!), isto é, sempre tem alguma coisa nova para descobrir. Caso seja isso
que você queira, Mestrado e Doutorado te aguardam depois desse curso! Se vocês
pensam que quem se forma nesse curso só pode ser professor/pesquisador, vocês
estão muito enganados! O curso forma pessoas que sabem analisar e resolver problemas
metodicamente (vocês vão ver que estarão pensando com mais clareza em breve). Por
mais que esse não seja o objetivo do curso, isso acaba acontecendo e os alunos
que não têm viés acadêmico tiram bom proveito disso. Vocês muito provavelmente
não aprenderão a aplicar matemática em outras ciências, mas terão plena capacidade
de irem atrás disso sozinhos e o mercado gosta disso. Quem se formou na Pura pode
trabalhar em vários locais: universidades, colégios, bancos, empresas...

iii) Como lidar com o curso da Pura?

O essencial é gostar de matemática, ter gosto pela descoberta e pelo raciocínio
em matemática! No começo, você provavelmente vai achar incrível o simples processo
de provar coisas. ``Uau, saí de $A$ e por passos lógicos cheguei em $B$, isso é
magia!''. Infelizmente, essa empolgação inicial passa, porém acabamos desenvolvendo
um gosto pelo estudo de estruturas matemáticas (como espaços vetoriais etc.
citados acima). E gostamos de verdade. E prosseguimos estudando essas coisas.

Um meio para te ajudar é trocar ideias com seus amigos: além de conversar
sobre matemática (você vai fazer isso bastante por aqui), vocês podem formar um
grupo unido que esteja disposto a enfrentar as matérias, EPs (exercícios de
programação para entregar), provas etc., além de estudarem juntos e, claro,
aprenderem juntos. O curso da pura é mais divertido depois que você se enturma. Não se
esqueça também que há muitos veteranes que gostam de ajudar e o farão se você pedir!

Talvez já tenham te contado que esse curso pode ser fácil de entrar, mas
costumam se formar uns seis de nós por ano (e olhe lá!). E foi em 2011 que a Pura
bateu seu recorde de formandos ao mesmo tempo: foram dezesseis! Isso não acontecia
desde pelo menos a época do Jacy Monteiro (que logo você vai descobrir que é o
nome do auditório)! E ainda teve uns três que se formaram em
três anos. Nunca desanime com as pessoas que querem abaixar a sua moral dizendo
que você nunca vai se formar. É mentira! Todos nós amamos matemática, mas sabemos
que matemática é trabalho duro e persistência (mas isso é ruim?). Então nunca
desanime com os comentários dos outros e não desista!

É importante lembrar que a vida não é só estudar. Aprenda a conciliar o seu estudo
com seus amigos, alguns hobbies, namoro etc. Ao contrário do que falam, é
possível ter vida social e estudar matemática. Inclusive tiveram alunos com notas
muito boas e que até se formaram em menos do que quatro anos que sabidamente
sempre tinham algum tempo reservado para as três coisas citadas. É importante
sempre ter um tempo para esfriar a cabeça também. É estranho, mas o nosso cérebro
trabalha enquanto estamos em repouso ou nos entretendo, e quando voltamos a estudar
às vezes algo que parecia difícil fica fácil. Mas não vai entender isso errado,
hein? Tem que estudar, senão não tem jeito!

iv) O que mais preciso saber sobre a Pura???

Primeiramente, apesar de toda a dificuldade, a Pura tem uma carga horária
relativamente menor do que a maioria dos outros cursos... Teoricamente, é
possível se formar em 3 anos e meio, ou até menos. E existem pessoas que o
fazem (ou tentam, pelo menos). Mas tome cuidado: Além de difícil, você
corre o risco de não aprender nada e tirar notas bem mais baixas. Notas altas
são importantes para pedir bolsas, intercâmbio etc., então pode ser que, se você
quer adiantar sua formatura para conseguir alguma dessas coisas, isso seja um
tiro no pé. É bom agir com cautela. Normalmente, o tempo que você pode vir a ter
a menos de aula precisará ser gasto estudando por conta própria. Por isso, tome
cuidado para não se sobrecarregar. Mas não se esqueça que há muitas atividades
que podem ser abandonadas sem se prejudicar caso o curso fique pesado (e isso não é feio, viu?).

Tente tirar proveito da relativa flexibilidade da grade de horários: enquanto
no 1º ano você tem todas as aulas certinhas todo dia, com o passar do
curso você terá menos aulas (que tenderão a ficar mais difíceis), e sua
grade poderá ficar cheia de buracos. Não tenha medo do trancamento parcial,
quando você tiver medo de bombar em alguma matéria, ou quando não se der bem com
um professor: em boa parte dos cursos pode valer a pena deixar determinada
matéria para depois do que fazer com algum professor com quem você não se dê
bem. Mas antes de tomar alguma atitude tão extrema, você tem sempre seus colegas
de turma e seus colegas veteranes. No IME tem muita gente disposta a te ajudar e
com o tempo você vai encontrá-los!

Finalizando, deixamos para vocês os seguintes conselhos:
\begin{enumerate}

\item Informem-se sobre atividades extracurriculares como o programa de
Iniciação Científica e uma série de palestras com professores que, muito possivelmente,
realizar-se-ão durante o ano. Também há programas voltados para o primeiro ano
com o propósito único de te auxiliar. A oportunidade é única, então se informe e aproveite!
\item Tome consciência de que você, na grande maioria das vezes, vai ter que
estudar muito;
\item Não desanime com as pessoas que dizem que você não vai conseguir. Se
você gostar da coisa, você consegue! É normal dar escorregadas e ir mal em algumas
provas durante o seu curso, e pode ser até que você reprove em alguma coisa,
mas isso não quer dizer que você não serve para a coisa! Se você gosta do curso,
você vai conseguir chegar tão longe quanto você quiser. Mas é claro que nada
acontece sem esforço!
\item Se você está com dúvidas, pergunte. Não importa se você vai perguntar
pro professor, pro colega, pro monitor, pro cachorro, pro defunto, pro exú, pro
Goku... Mas dúvidas pequenas hoje geralmente se tornam problemas enormes
no fim do semestre, e esse tipo de coisa tem o potencial de te reprovar em alguma
disciplina, além de, é claro, prejudicar seu aprendizado.
\item Existem muitos veteranes que gostam de ajudar, basta pedir. Com o tempo 
você vai descobrir quem são. Não tenha medo deles!
\item Se você entrou sem saber ao certo como é o curso, seus colegas podem
te ajudar a gostar! Esperamos fortemente que você goste e estamos dispostos a te ajudar com isso!
\item E se você entrou sabendo, não seja arrogante... você pode ajudar os seus
colegas a gostar!
\item E claro, esperamos que o curso seja mesmo aquilo que você
espera e que você também seja feliz com ele.

\end{enumerate}
Para terminar, façam amigos no IME: eles vão entender vocês como ninguém. Qualquer dúvida,
vocês podem nos procurar. Estaremos sempre dispostos a ajudá-los.

\end{subsecao}
