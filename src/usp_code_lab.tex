\begin{subsecao}{USPCodeLab}

\figurapequenainlineapertada{uspcodelab}

O CodeLab é um grupo de extensão que tem por objetivo criar um espaço
colaborativo para o desenvolvimento de tecnologia na USP.

O grupo foi fundado por estudantes do BCC em 2015 com o nome IME Workshop, mas atualmente
conta com integrantes de vários cursos e lugares. O grupo possui quatro núcleos: Cidade Universitária (IME),
Each, São Carlos (ICMC) e, por fim, um núcleo fora da USP na UFABC.

Nosso foco é aprender
na prática ferramentas e técnicas de desenvolvimento de software que permitam
solucionar problemas do mundo real. Durante o semestre, organizamos grupos de estudos e cursos.
Utilizamos os recursos mais avançados, criados por grandes empresas como Google
e Facebook e pela comunidade de software livre, para criar projetos legais
propostos pelos próprios participantes! Alguns exemplos são o nosso \textit{bot}
do bandejão (@uspbandexbot, no Telegram) e o  \href{http://compreqs.surge.sh/}{CompReqs}, um sistema gerenciamento
de pré-requisitos para as disciplinas do BCC.

O CodeLab também organiza Hackathons. Durante os eventos, os participantes se reúnem em times e são
desafiados a fazer, em apenas 24h, um protótipo de software ou hardware
relacionado ao tema da competição. Diversão garantida para quem gosta de comida,
café e muita programação! Grandes exemplos são o SheHacks, um hackathon focado somente para mulheres
cis e trans, e o InterHack, o maior Hackathon universitário do Brasil!

Curtam nossa página do Facebook e entrem no nosso grupo do Telegram para saber
datas e horários das nossas reuniões abertas. Participem do CodeLab!

\begin{description}
\item[Facebook:] \url{uclab.xyz/facebook}
\item[Telegram:] \url{uclab.xyz/telegram}
\item[Youtube:] \url{uclab.xyz/youtube}
\item[Site:] \url{uclab.xyz/site}
\item[Instagram:] @uspcodelab
\end{description}

\end{subsecao}
