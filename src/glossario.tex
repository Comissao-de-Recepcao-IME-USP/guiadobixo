\begin{secao}{Glossário}

Esta parte é, sem dúvida, uma das mais importantes do guia. Aqui vocês
encontrarão todas as explicações para as maiores dúvidas do universo, apesar de
já sabermos que a resposta para a vida, o universo e tudo mais é 42. Com
certeza, após lerem este trecho sua vida vai mudar: vocês saberão, por exemplo,
porque o céu é azul e com quantos paus se faz uma canoa.

\begin{subsecao}{Sobre as matérias}

{\bf Teorema:} Um teorema é uma afirmação que pode ser provada. Provar
teoremas é a principal atividade dos matemáticos. Deles surgem Lemas,
Corolários, Proposições, e tantas outras coisas que você só vai entender
completamente o que significam quando precisar escrever sobre eles, o que vai
acontecer logo, logo!

{\bf Iniciação Científica:} grupo de estudantes, coordenado por um professor, que
estuda um determinado assunto paralelamente ao curso. No caso de estudantes que
queiram bolsas de estudo, é adotado um plano de estudos mais rigoroso.

{\bf P1, P2, P3:} sofrimento de todo aluno de graduação. A quantidade de sofrimento
depende do professor da disciplina.

{\bf SUB:} prova que você faz quando vai mal nas P1, P2 e P3, ou
quando você simplesmente não vai. Sua aplicação e utilidade depende do
professor ministrante.

{\bf REC:} prova que você faz quando vai mal na SUB. Pra fazer a REC, precisa ter
pelo menos média 3 nas outras avaliações.

{\bf DP:} matéria que você faz quando vai mal na REC.
\end{subsecao}

\begin{subsecao}{Sobre programas}

{\bf Computador:} objeto com vontade própria, sensível, que requer muito
carinho e atenção.

{\bf EP:} Exercício-Programa. Algo que você vai ter que fazer muitas vezes, e
vai dar muito trabalho.

{\bf GCC:} compilador mais recomendado para seus EP's, por suas inúmeras
qualidades. Atenção: ele ainda fará você se sentir incompetente.

{\bf Hello World:} Um clássico da programação universal.

{\bf Segmentation Fault:} Efeito computacional aleatório causado pela ``véspera
de entrega de EP''. Desenvolvido por Murphy.

{\bf Stack Overflow:} mensagem que aparece na tela do computador
quando ele se recusa a funcionar.

{\bf Teorema Fundamental do EP:} ''O EP só funcionará no dia da entrega.'' Não
confunda com o Corolário 42 da Lei de Murphy: ''O EP só {\bf não} funcionará no
dia da entrega!''

{\bf Linux:} Sistema operacional criado totalmente em linguagem C, graças a um
esforço mundial de milhares de programadores e experts em informática, composto
por aproximadamente 7 mil arquivos e 5 milhões de linhas, e com o qual você não
tem capacidade para trabalhar.

{\bf Windows:} Vírus. Porém tão bem mascarado que parece até a coisa correta a
se usar.
\end{subsecao}

\begin{subsecao}{Sobre a USP}

{\bf CEPE:} Centro de Práticas Esportivas - lugar onde você poderá praticar
todos os esportes que quiser.

{\bf SAS:} ao lado da praça do relógio. É onde os alunos fazem a carteirinha
de passes de ônibus, EMTU e metrô, além de solicitar os auxílios eteceteras que
estão explicados na seção respectiva.

{\bf CRUSP:} Conjunto residencial da USP. Se você se inscrever, torça para
pegar um apartamento num dos blocos já reformados, ou então torça para não
conseguir nenhum.

% {\bf Colméia:} conjunto de favos.

% {\bf Favos:} um monte de prédios hexagonais encravados no meio do CRUSP.

{\bf CINUSP:} Cinema da USP localizado no favo 4 da Colméia. Toda semana ele
passa um filme de qualidade. Informe-se sobre a programação em {\tt www.usp.br/cinusp}.

{\bf Pelletron:} prédio da Física que na realidade é um acelerador de
partículas.

{\bf P1, P2, P3:} não confundam com as do item das matérias. Aquelas eras AS P1, P2, P3;
estes aqui são OS P1, P2, P3. São os portões da USP. O P1, que é o principal, se localiza
no cruzamento da Av. Afrânio Peixoto com a Rua Alvarenga, além de que é por ele que
entram os circulares vindos do metrô. O P2 fica na Av. Escola Politécnica. Já o P3,
fica na Av. Corifeu de Azevedo Marques. Fique atento quanto aos horários dos portões,
pois eles não ficam abertos o tempo todo.

{\bf H.U. (Hospital Universitário):} para onde são mandados bixes que se machucam na semana de Recepção.
Lá são realizados os tratamentos e experiências com
exposição a radiação, exposição a aspirantes a médicos, teste de
paciência/resistência a dor assim que pega a senha, alto nível de gesso no
estômago, queda espontânea (ou não) de cabelo etc.

{\bf Vet (Veterinária):} para onde são mandados biches que se machucaram na Semana de Recepção.

{\bf Psico:} Para onde são mandados os bixes (e veterenes também) que querem plantões psicológicos porque saúde mental é importante deve ser tratada com seriedade, Porra.

\end{subsecao}
\end{secao}
