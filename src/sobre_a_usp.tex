\begin{secao}{Um Pouco Sobre a USP}

Vocês, que são novos na USP, devem saber desde cedo que aqui há muita
burocracia. É bom que se acostumem com ela, já que vocês terão que enfrentá-la.

O reitor é presidente do maior órgão da USP, o Conselho Universitário (abrevia-se
C.O. para evitar frases do tipo: ``Vou ter uma reunião no CU hoje'', ``O CU não
está funcionando muito bem esse semestre'', ``Os alunos não tem acesso ao CU''
etc.) que determina TODOS os rumos da universidade.

Abaixo dele vêm as coordenadorias e unidades. O SAS (Superintendência de Assistência Social),
por exemplo, é o departamento responsável pelos serviços
que a universidade oferece (não é a melhor coisa do mundo, mas oferece) para a
comunidade universitária: ônibus circulares, bandejões, moradia para estudantes
(CRUSP) etc.

Alguns outros lugares que vocês devem saber que existem são o HU (Hospital
Universitário), o CEPE (Centro de Práticas Esportivas, leia “cepê”), o banheiro
da FEA (Faculdade de Economia, Administração e Contabilidade), que fica na
frente do IME e é uma das faculdades mais bem abastecidas financeiramente
(apelidada “carinhosamente” de Shopping), a Física (certas aulas de laboratório são lá)
 e a querida Faculdade de Educação (para o pessoal da Licenciatura).

\begin{subsecao}{E-mail USP}

Bixes, atenção e cuidado ao e-mail que vocês receberam (ou vão receber)! É um
email@usp.br, do sistema “oficial” de e-mails da USP, o que significa que é
através dele que a Universidade, o Jupiterweb, a diretoria do IME, o CAMat e
alguns desocupados lhes enviarão comunicados oficiais, o que pode envolver desde
oportunidades diversas para intercâmbios, estágios, monitorias e tudo o mais que
vocês, simples bixes imeanos, sejam capazes de se imaginar fazendo. Por este e-mail
vocês receberão informações antes mesmo que as principais (é, nem todas são
colocadas em murais não) sejam também colocadas nos murais.

\end{subsecao}

\begin{subsecao}{Eduroam}

O Eduroam (EDUcation ROAMing, não tem nada a ver com alguém chamado Eduardo) é um
serviço de wi-fi que \sout{às vezes} provê acesso à Internet dentro de diversas
universidades (inclusive a USP). Mesmo que você seja aluno da USP, se viajar para 
uma universidade no exterior que possui o Eduroam, ainda poderá ter acesso à Internet 
usando o mesmo login!

Para usar o Eduroam, basta colocar o seu número USP no campo de login seguido por 
@usp.br. Por exemplo, se o seu número USP for 123456789, então deverá colocar no
campo de login 123456789@usp.br. No campo da senha, deverá colocar a sua senha única 
(a mesma senha que vocês usam para logar no Jupiterweb).

Tem várias outras informações sobre as configurações na hora de fazer login que
\sout{ninguém entende}, mas se algum de vocês entender, estão no site 
https://eduroam.usp.br/como-usar/.

Para mais informações, acessem https://eduroam.usp.br/.

\end{subsecao}

\begin{subsecao}{O SAS}

A Superintendência de Assistência Social, que fica próxima à praça do
relógio, é o órgão da USP responsável pelo bem estar financeiro dos alunos — não
é lá a melhor coisa do mundo, mas ajuda... É onde vocês podem conseguir seus
milhares de benefícios, tais como auxílios financeiros, moradia gratuita (CRUSP)
ou bolsa alimentação (conhecido como vale-bandex) e dentistas gratuitos. Ou
seja, há bolsas de todo modelo, tamanho, designer, estação e preferência
gastronômica nula que preferir. É também responsável pelo Setor de Passes
Escolares.

Se algum de vocês, bixes, estudou a vida inteira em escola pública, tem baixa renda,
sofreu agressões ou foi reprimido na infância, não pense que está sozinho nessa,
muitos também passaram por isso e vocês conseguem encontrar apoio aqui, permanência
estudantil é um direito e não deve ser motivo de vergonha, portanto busquem
auxílio,
pois podemos ajudar. Vão a seguir as diversas alternativas para todos que precisarem:

{\bf Moradia e auxílios financeiros}

Se vocês infelizmente não têm tanto acesso a meios de locomoção, ou dinheiro para
pagar transportes ou mesmo repúblicas, saibam que diversos auxílios podem ser
oferecidos para vocês para amenizar a sua situação:

Com a carteirinha provisória que você recebeu na matrícula, você deverá entrar
na página do SAS ({\tt http://sites.usp.br/sas/}) e solicitar a inscrição
para o processo de moradia e alojamento. Lá você terá um formulário, três
trilhões de coisas para preencher e assim que te chamarem, deve apresentar os
documentos (hum!) necessários para a assistente (você irá “ganhar” uma). Na
improvável hipótese do site estar fora do ar, como todo ano acontece, você terá
que ir ao SAS e pedir alojamento na USP, lá no Bloco G – sem trocadilhos - do
CRUSP.

A prioridade é dada aos residentes de outros estados ou interior de São Paulo,
mas os moradores de São Paulo também podem solicitar, dadas as proporções da
cidade e o tempo de duas horas para que moradores da Zona Norte venham às aulas.

Vale ressaltar, bixes, que vocês tomem o cuidado de não fazer o mesmo que outros
bixes burros de outros anos, que confundiram alojamento com moradia. São dois
requerimentos distintos e vocês deverão solicitar os dois se realmente quiserem
garantir um lugar para tomar banho e dormir.

Os requerimentos conterão perguntas sobre a situação socioeconômica, bem como os
documentos que deverão ser trazidos para comprová-la. Perguntas como renda
salarial, quantas pessoas contribuem com ela, números de bens móveis e imóveis,
situação habitacional, tipo de escola em que estudaram, se trabalham e há quanto
tempo, quanto gastam para vir à USP, o tempo de ida etc. Ainda há um espaço
para descreverem alguma particularidade não exposta nas perguntas que, obviamente,
receberá um parecer técnico.

Na classificação final da MORADIA, se vocês conseguiram uma pontuação grande, vocês
podem escolher entre a moradia no CRUSP ou um auxílio financeiro (bolsa moradia)
de R\$400,00 para vocês poderem alugar quartos, casas, hotéis, ou mesmo transporte
para ida e volta pra sua terra. Há casos em que a classificação final lhes
permite ter benefício apenas ao alojamento OU à bolsa, mas ao apartamento de
três quartos individuais (CRUSP) não.

Se vocês não conseguirem por nenhum desses meios, podem tentar a hospedagem, que é
simplesmente vocês ficarem no apartamento de alguém que more no CRUSP. Mas fiquem
atentos às datas de requerimento depois do resultado da seleção, pois vocês podem
ficar sem essa chance. Se mesmo assim vocês não conseguirem nada (bixes azarados!),
e acharem que os entenderam mal na entrevista ou coisa assim, vocês podem pedir para
entrar com recurso, e terem mais uma chance de esclarecer melhor a sua situação
(a.k.a. “cantar a assistente social”), ou também procurar a
AMORCRUSP (Associação dos Moradores do CRUSP que fica no Bloco F, das 14h às 18h).

{\bf Alimentação}

Vocês podem solicitar também o auxílio alimentação na página do SAS, que
consiste nos vale-bandex da USP e são válidos para almoço e jantar. Vocês deverão
passar por outra seleção que também inclui questionários sócio-econômicos,
comprovantes, e mais papéis.

{\bf Bolsa-trabalho}

Destina-se a alunos de graduação vinculados a projetos de extensão de serviços à
coletividade. Os projetos são selecionados anualmente, de acordo com sua relevância
para as finalidades da universidade pública e os estudantes vinculam-se por
afinidade acadêmica ou científica. Cada bolsa é de 1 (um) salário mínimo por
40 horas de trabalho mensais. Além da seleção socioeconômica feita pela
DPS (Divisão de Promoção Social), há uma seleção técnica feita pelos supervisores
dos projetos.

Mas fiquem espertos! Para tudo tem prazo e o SAS não é obrigado a ficar esperando a
boa vontade de aparecerem por lá de ninguém. Qualquer dúvida, liguem pro SAS.

{\bf Atendimento odontológico gratuito}

Antigamente para vocês agendarem o atendimento odontológico gratuito era necessário fazer uma
carteirinha no HU (Hospital Universitário) e em seguida, comparecer ao Bloco G
do CRUSP com a carteirinha e agendar, porém hoje em dia, graças à tecnologia,
isso não é mais necessário, você só vai precisar mandar uns e-mails, preencher uns
formulários e esperar \sout{uma eternidade}.

Primeiro você precisar mandar um e-mail solicitando triagem para triagemodonto.sas@usp.br,
vão te responder com um formulário que você vai preencher e enviar de volta, após isso eles
te colocarão na fila de espera pra triagem, no final deve demorar muitos meses pra
eles te chamarem e quando fizerem isso você só vai ter que comparecer ao Bloco G do CRUSP
munido do seu cartão USP (que a essa altura do campeonato você já estará com o definitivo).

No final será demorado, porém o serviço é gratuito e realmente ajuda. Eles também atendem casos
de emergência como por exemplo dor de dente, algum dente quebrado ou qualquer outra situação
do gênero. No caso de dúvidas, são poucos os que podem te ajudar nesse assunto, então o melhor
a se fazer é ligar pra lá: (11) 3091-3393.

%FIXME
\pagebreak
{\bf Setor de passe escolar}

Será nesse lugar que você resolverá boa parte dos seus problemas com o SAS e com o
Bilhete Único. Aqui é onde você carrega o vale bandex, entrega as documentações para
as bolsas, assina boa parte dos contratos e recebe orientação sobre as mais diversas
burocracias em que está se metendo.

Para as linhas da EMTU, SPTRANS, METRO, vocês devem entrar no site do SAS e fazer
o pré-cadastro para o respectivo cartão ({\tt http://sites.usp.br/sas/}). Talvez demore um pouco, pois a USP tem que
avisar para a SPTrans que vocês passaram na Fuvest. Fiquem atentos! Qualquer dúvida,
liguem para a sessão de passe escolar do SAS: (11) 3091-3581.

Há também um cartão especial para alunos de universidades públicas provenientes
de outras cidades do interior de São Paulo. Se algum de vocês vêm de algum desses domos
ignotos (como Resende, Caçapava ou Guaíra), pode se dirigir ao guichê da sua empresa de
transporte intermunicipal (Cometa, Danúbio Azul, etc.), apresentar seu Cartão USP,
preencher um formulário e eles aguardarão confirmação do seu instituto. Então
você pega seu cartão na Seção de Alunos e, na compra de passagens entre São Paulo
e sua cidade-natal, paga 50\% do valor normal. Só não deixe isto por último na
sua lista de necessidades porque existe um período do ano em que os guichês
liberam seus formulários; em resumo, espiche suas orelhas e corra para a rodoviária.

\end{subsecao}

\begin{subsecao}{Atendimento Psicológico}

Existem inúmeros serviços na USP que oferecem atendimento psicológico
aos alunos, cada um com um foco e um público-alvo. Dentre eles, tem
o Serviço de Aconselhamento Psicológico (SAP), o Plantão Psicológico
LEFE e o Ateliê Aberto.

Bixes, sempre que precisarem, não tenham receio de procurar apoio
psicológico, por mais banal que vocês possam achar o motivo. Vocês
podem frequentar os plantões psicológicos para conversar sobre 
qualquer coisa que esteja lhes incomodando, sejam professores, 
matérias, EPs ou algo do tipo sem criar qualquer
problema para vocês no IME. Se for necessário, pode haver processos 
de encaminhamento ou retorno para novas entrevistas.

Para maiores informações, acesse o site 
\url{https://www5.usp.br/servicos/atendimento-psicologico/}

\end{subsecao}
%\pagebreak
% Utilize para começar uma nova página do lado esquerdo do Guia!

\begin{subsecao}{Bandejão}
%TODO Fazer as duas páginas ficarem na mesma "visualização" no guia impresso
%Ou seja, usar esquema Odd/Even do LaTeX.
% Isso foi deixado de lado nessa versão de 2016 para que o mapa dos circulares
% ficasse no lugar certo e não tivesse nenhuma página em branco
% (como o cleardoublepage ali em cima)
% 2017: não sei se entendi o que era para ser feito, mas usei o
%       \cleardoublepage para a imagem do calvin não ficar zuada.


Os bandejões, vulgarmente conhecidos como Restaurantes do SAS, são os lugares
em que vocês podem se alimentar razoavelmente a um preço analogamente razoável.
Os tickets custam R\$ 2,00, sendo estes carregados na carteirinha USP. O
lugar para carregar o ``bandejão único'' é no SAS, perto do bandejão
Central. Além disso, é possível verificar o saldo
atual e comprar mais créditos via boleto no site
{\tt https://uspdigital.usp.br/rucard/} ou pelo aplicativo "Cardapio USP",
disponível para Android e iOS (não se esqueça de olhar também os outros 
aplicativos da USP disponíveis, pois podem ser muito úteis).
É uma ótima alternativa para evitar as infinitas filas dos guichês de venda. Os 
créditos demoram até 02 (dois) dias úteis para estarem disponíveis. 
Dica: Confira o saldo da carteirinha no site ou no aplicativo antes de tentar usar!

Existe uma lenda que sobremesas especiais aparecem em certos bandejões perto de
datas comemorativas. Mas não se engane - normalmente, elas não aparecem no
cardápio (talvez para não ficar muito cheio de gente).

\cleardoublepage
\figuragrande{bandex_calvin}

O cardápio semanal do bandejão pode ser visto no site {\tt
http://sites.usp.br/sas/} ou pelo aplicativo ``Cardapio USP'', mas se vocês 
estiverem com preguiça de ver em um desses lugares, é 
bem provável que um veterane já saiba e resolva informar se questionado com muita
educação.

O cardápio é geralmente composto de arroz (com opção normal e integral), feijão,
prato principal (carne/ovos), acompanhamento (legumes ou verduras refogadas,
cremes, molhos), salada (algumas folhas), sobremesa (frutas e, às vezes, algo mais
requintado), pãozinho (normal e integral) e suco, além de temperos genéricos
(jamais perguntem do que eles são feitos). Se vocês são vegetarianos, o
acompanhamento nunca contém carne (e estão tentando fazer com que sejam veganos
também), e todos os bandejões tem uma alternativa vegetarina de PVT para o
prato principal, normalmente vindo na forma de ração, mas existindo as formas
de lasanha, kibe, e outros.


Não se esqueçam de levar as suas canecas do Kit-bixe se forem comer nos
bandejões, tanto para ostentar a posição de bixes do IME, quanto para salvar o
meio ambiente, evitando o uso de copos descartáveis. Além disso, levar a caneca
é extremamente vantajoso, já que a quantidade de suco que cabe no copo
descartável é pequena, fazendo vocês viajarem diversas vezes até a máquina de
suco.

PS: O Efeito Bandex é proporcional à quantidade de salitre utilizado em cada
bandejão.\\
PS 2: Nunca, em hipótese alguma, jamais, visite a cozinha do seu bandejão de
preferência, pois você corre o risco de nunca mais almoçar na vida. Como já
dizia o velho sábio ``A ignorância é uma virtude''.\\
PS 3: Playstation 3.

Consultem a tabela abaixo para decidir em qual dos bandejões vocês vão comer.

\figuragrande{bandex}
O horário das refeições é o seguinte:\\
Café da manhã: 7h às 8h30min (somente no Central, aos sábados fica aberto das
7h às 9h e aos domingos das 8h às 9h30)\\
Almoço: 11h15min às 14h15min (na Química é das 11h às 14h)\\
Jantar: 17h30 às 19h45 (não tem no PCO)\\

Para mais informações, visitem o site do SAS: {\tt http://sites.usp.br/sas/}

\end{subsecao}

\begin{subsecao}{Outros lugares para comer na USP}

Caso vocês, bixes, não queiram comer no bandejão, seja porque estão com medo do
prato do dia ou simplesmente porque querem comer algo diferente (e
possivelmente de sabor melhor), saibam que há alguns outros lugares na USP onde
vocês podem comer.

Alguns dos preços a seguir podem não estar 100\% corretos, mas a ordem de
grandeza sim =).

{\bf Lanchonete da Física}

Fica um pouco antes do bandejão da Física, nela vocês podem almoçar pagando R\$
40,90 o quilo, sempre tem uma razoavel quantidade de saladas e pratos
quentes, além disso, tem churrasco com uma boa variadade de opções.
Além do almoço por quilo, lá vende alguns salgados e lanches como sanduíches e
beirutes. E um pouco antes da lanchonete, há um lugar que vende cookies por R\$
2,50 (3 unidades). Vocês vão sentir o cheiro dos cookies quando forem bandejar.

{\bf Restaurante do IPEN}

Fica na parte de cima da rua entre o IF e o Parque Esporte Para Todos, para
entrar lá vocês precisam da carteirinha USP. Lá o quilo é R\$ 24,50, porém não
tem uma variedade muito grande de comida, tem marmitex também que é R\$ 11,70.
Ele só abre para pessoas de fora às 13h e antes de ir para o restaurante vocês
devem se identificar.

{\bf Lanchonete da FAU}

Localizada dentro da FAU assim que você sobe a rampa. Lá além de lanches é
servido almoço tanto por quilo (custa R\$ 43,90) como prato feito. Tirando o por
quilo, qualquer outra coisa deve ser paga no caixa antes de retirar. Além
disso, no terceiro andar da FAU tem a mesinha de doces que os alunos deixam lá
para vender, a variedade e quantidade de doces varia bastante com o dia e o
preço deles costuma ser no máximo R\$2,50.

{\bf Restaurante da FEA}

Fica atrás da FEA, lá o quilo tem bastante coisa diferente e algumas até que
sofisticadas, porém o preço do quilo lá é R\$61,90, provavelmente é o mais caro
da USP. Geralmente é frequentado por professores e funcionários da USP. Além do
self-service, há algumas opções de prato feito que custam R\$22,00.

{\bf Trailers de lanche da \sout{ECA} Química}

Ficam na calçada com a Avenida Lineu Prestes, em frente ao Instituto de
Química. Nesses carrinhos vende pastel, sanduíches, churros, tapioca e salgados 
em geral. Os preços dos lanches variam entre R\$3,00 e R\$10,00. As opções de
sanduíches e a quantidade de sabores de tapioca é bem grande e todos em geral
são muito bons. Já o pastel você pode escolher de 1 a 5 ingredientes para por
no pastel, portanto o preço varia de acordo com a quantidade de ingredientes que
vocês vão querer.

{\bf Cachorro-quente da Reitoria}

Fica na rotatória da biblioteca Brasiliana (ou a rotatória depois do CEPE, se assim preferir)
lá o cachorro-quente pode ser feito tanto no pão de cachorro-quente como na baguete. Além disso,
há a opção de por catupiry e de por uma salsicha adicional. O preço do cachorro-quente varia
entre R\$9,00 e R\$12,00.

{\bf Poke da Reitoria}

Recentemente abriu um lugar que vende Poke (um prato havaiano que mistura peixe cru em cubos, arroz, shoyu e frutas) ao lado do cachorro-quente. Esta iguaria custa aproximadamente R\$23, e você pode escolher os ingredientes que vão no seu prato, tipo Spoletto. Dá pra fazer um Poke vegano, com cogumelos em vez de peixe. Além de Poke, também vende Temakis, Hot Rolls, e até Açai. A qualidade é boa mas as filas podem ser demoradas.

{\bf Pastel do IEE}

Esse lugar maravilhoso fica a apenas alguns passos do IME. Apenas às terças-feiras,
em um canto escondido do IEE (pergunte a um veterane como chegar!) fica o famoso pastel do IEE.
Há muitas opções de sabores, além de poder comprar um caldo de cana para acompanhar.
Geralmente, o preço do pastel é o mesmo ou mais barato que o pastel de feira.

\end{subsecao}

\end{secao}
