\begin{secao}{Um Pouco Sobre a USP}

Você, que é um novo aluno da USP, deve saber desde cedo que aqui há muita
burocracia. É bom que se acostume com ela, já que você terá que enfrentá-la. 

O reitor é presidente do maior órgão da USP, o Conselho Universitário (abrevia-se
C.O. para evitar frases do tipo: ``Vou ter uma reunião no CU hoje'', ``O CU não
está funcionando muito bem esse semestre'', ``Os alunos não tem acesso ao CU'',
etc.) que determina TODOS os rumos da universidade.

Abaixo dele vêm as coordenadorias e unidades. O COSEAS (Coordenadoria de Saúde
e Assistência Social), por exemplo, é o departamento responsável pelos serviços
que a universidade oferece (não é a melhor coisa do mundo, mas oferece) para a
comunidade universitária: ônibus circulares, bandejões, moradia para estudantes
(CRUSP) etc.

Alguns outros lugares que você deve saber que existem são o HU (Hospital
Universitário), o CEPE (Centro de Práticas Esportivas, leia “cepê”), o banheiro
da FEA (Faculdade de Economia, Administração e Contabilidade), que fica na
frente do IME e é uma das faculdades mais bem abastecidas financeiramente
(apelidada "carinhosamente” de Shopping), a Física (as aulas de laboratório –
que só a Estat faz - são lá) e a querida Faculdade de Educação (para o pessoal
da Licenciatura).

\begin{subsecao}{E-mail USP}

bixo, atenção e cuidado ao e-mail que você recebeu no ato da matrícula! É um
bixo@usp.br, do sistema “oficial” de e-mails da USP, o que significa que é
através dele que a Universidade, o Jupiterweb, a diretoria do IME, o CAMat e
alguns desocupados lhe enviarão comunicados oficiais, o que pode envolver desde
oportunidades diversas para intercâmbios, estágios, monitorias e tudo o mais que
você, um simples bixo imeano, seja capaz de se imaginar fazendo. Por este e-mail
você receberá informações antes mesmo que as principais (é, nem todas são
colocadas em murais não) sejam também colocadas nos murais. Vale aqui uma máxima
de V: “Se não querem concorrência, não farão propaganda.” 

Um modo de fugir do design deprimente da página principal do @usp.br é
configurando-o para que seus e-mails sejam encaminhados para um servidor que
você já use. Fica a dica! 

\end{subsecao}

\begin{subsecao}{O COSEAS}

A Coordenadoria de Saúde e Assistência Social, que fica próxima à praça do relógio,
é o órgão da USP responsável pelo bem estar financeiro do aluno - não é lá a
melhor coisa do mundo, mas ajuda... É onde você pode conseguir seus milhares de
benefícios, tais como auxílios financeiros, moradia gratuita (CRUSP) ou bolsa
alimentação (conhecido como vale-bandex) e dentistas gratuitos. Ou seja, há
bolsas de todo modelo, tamanho, designer, estação e preferência gastronômica
nula que preferir. É também responsável pelo Setor de Passes Escolares. 

Se você, bixo, estudou a vida inteira em escola pública, se seu irmão te batia,
seu pai roubava seu dinheiro, ou você foi reprimido na infância, não pense que
você é o único aqui e por isso merece ser mimado. Vão a seguir as diversas
alternativas para você tentar:

{\bf Moradia e auxílios financeiros}

Se você infelizmente não tem tanto acesso a meios de locomoção, ou dinheiro para
pagar transportes ou mesmo repúblicas, saiba que diversos auxílios podem ser
oferecidos para você para amenizar a sua situação:

Com a carteirinha provisória que você recebeu na matrícula, você deverá entrar
na página do COSEAS ({\tt http://www.usp.br/coseas}) e solicitar a inscrição
para o processo de moradia e alojamento. Lá você terá um formulário, três
trilhões de coisas para preencher e assim que for chamado, apresentar os
documentos (hum!) necessários para sua assistente (você irá "ganhar” uma). Na
improvável hipótese do site estar fora do ar, como todo ano acontece, você terá
que ir o COSEAS e pedir alojamento na USP, lá no Bloco G – sem trocadilhos - do
CRUSP.

A prioridade é dada aos residentes de outros estados ou interior de São Paulo,
mas os moradores de São Paulo também podem solicitar, dadas as proporções da
cidade e o tempo de duas horas para que moradores da Zona Norte venham às aulas. 

Vale ressaltar, bixo, que você tome o cuidado de não fazer o mesmo que outros
bixos burros de outros anos, que confundiram alojamento com moradia. São dois
requerimentos distintos e você deverá solicitar os dois se realmente quiser
garantir um lugar para tomar banho e dormir.

Elas conterão perguntas sobre a sua situação socioeconômica, bem como os
documentos que deverão ser trazidos para comprová-la. Perguntas como renda
salarial, quantas pessoas contribuem com ela, números de bens móveis e imóveis,
situação habitacional, tipo de escola em que estudou, se trabalha e há quanto
tempo, quanto gasta para vir à USP, o tempo de ida e etc. Ainda há um espaço
para descrever alguma particularidade não exposta nas perguntas que, obviamente
receberá um parecer técnico.

Na classificação final da MORADIA, se você conseguiu uma pontuação grande, você
pode escolher entre a moradia no CRUSP ou um auxílio financeiro (bolsa moradia)
de R\$300,00 para você poder alugar um quarto, casa, hotel, ou mesmo transporte
para ida e volta pra sua terra.  Há casos em que a classificação final lhe
permite ter benefício apenas ao alojamento OU à bolsa, mas ao apertamento de
três quartos individuais não (CRUSP). Nesse caso, escolha a bolsa e com o
dinheiro, leve seu VETERANO para beber.

Se você não conseguir por nenhum desses meios, pode tentar a hospedagem, que é
simplesmente você ficar no apertamento de alguém que more no CRUSP. Mas fique
atento às datas de requerimento depois do resultado da seleção, pois você pode
ficar sem essa chance. Se mesmo assim você não conseguir nada (bixo azarado!),
e achar que te entenderam mal na entrevista ou coisa assim, você pode pedir para
entrar com recurso, e ter mais uma chance de esclarecer melhor a sua situação
(a.k.a. “cantar a assistente social”), ou também procurar a
AMORCRUSP (Associação dos Moradores do CRUSP que fica no Bloco F, das 14h às 18h).

{\bf Alimentação}

Você pode solicitar também o auxílio alimentação na página do COSEAS, que
consiste nos vale-bandex da USP e são válidos para almoço e jantar. Você deverá
passar por outra seleção que também inclui questionários sócio-econômicos,
comprovantes, e mais papéis.

{\bf Bolsa-trabalho}

Destina-se a alunos de graduação vinculados a projetos de extensão de serviços à
coletividade. Os projetos são selecionados anualmente, de acordo com sua relevância
para as finalidades da universidade pública e os estudantes vinculam-se por
afinidade acadêmica ou científica. Cada bolsa é de 1 (um) salário mínimo por
40 horas de trabalho mensais. Além da seleção socioeconômica feita pela
DPS (Divisão de Promoção Social), há uma seleção técnica feita pelos supervisores
dos projetos.

Mas fique esperto! Para tudo tem prazo e o COSEAS não pode ficar esperando a sua
boa vontade de aparecer por lá. Qualquer dúvida, ligue pro COSEAS.

{\bf Atendimento odontológico gratuíto}

Para você agendar o atendimento odontológico gratuito, é necessário fazer uma
carteirinha no HU (Hospital Universitário) e em seguida, comparecer ao Bloco G
do CRUSP com a carteirinha e agendar. Para colocação e manutenção de aparelhos,
lá eles lhe indicam para o atendimento na odontologia (pois afinal, é muita
crueldade usar animais como ratos e macacos como cobaias). 

{\bf Setor de passe escolar}

Para as linhas da EMTU, SPTRANS, METRO, você entrar no site do Coseas e fazer
seu pré-cadastro. 

O endereço é o mesmo que todos os recursos do COSEAS: {\tt http://www.usp.br/
coseas}. Talvez demore um pouco, pois a USP tem 
que avisar para a SPTrans que você passou na Fuvest. Fique atento! Qualquer 
dúvida, ligue para a sessão de passe escolar do Coseas: 3091-3581 

Há também um cartão especial para alunos de universidades públicas provenientes
de outras cidades do interior de São Paulo. Caso você, bixo, venha de algum desses domos ignotos
(como Resende, Caçapava ou Guaíra), pode se dirigir ao guichê da sua empresa de
transporte intermunicipal (Cometa, Danúbio Azul, etc.), apresentar seu Cartão USP,
preencher um formulário e eles aguardarão confirmação do seu instituto. Então
você pega seu cartão na Seção de Alunos e, na compra de passagens entre São Paulo
e sua cidade-natal, paga 50\% do valor normal. Só não deixe isto por último na
sua lista de necessidades porque existe um período do ano em que os guichês
liberam seus formulários; em resumo, espiche suas orelhas e corra para a rodoviária.

\end{subsecao}

\begin{subsecao}{Serviço de Atendimento Psicológico (SAP)}

Às quartas-feiras das 9 horas às 11 horas (ou até as vagas acabarem),
o Instituto de Psicologia oferece um plantão psicológico para todos a
comunidade USP, ou seja, você bixo pode frequentar esses plantões sempre
que achar necessário. Nesses plantões, que podem ser individuais ou em grupo
, você pode conversar com psicólogos sobre qualquer coisa que esteja te
incomodando, seja professores, matérias, EPs ou algo do tipo sem criar qualquer
problema para você no IME. Se for necessário, pode haver processo de
encaminhamento ou retornos para novas entrevistas. Os retornos são marcados de
segunda-feira a sexta-feira das 7 horas às 19 horas ou aos sábados das
8 horas às 13 horas.

Para maiores informações, a secretaria funciona das 9 horas às 12 horas e das
13:30 horas às 16:30 horas de segunda a sexta-feira. 
 
\end{subsecao}

\pagebreak

\figuragrande{bandex_calvin}


\begin{subsecao}{Bandejão}
%TODO Fazer as duas páginas ficarem na mesma "visualização" no guia impresso
%Ou seja, usar esquema Odd/Even do LaTeX.


Os bandejões, vulgarmente conhecidos como Restaurantes do COSEAS, são os lugares
em que você pode se alimentar razoavelmente a um preço analogamente razoável.
Os tickets custam R\$ 1,90, sendo estes carregados em sua carteirinha USP. O 
lugar para carregar seu ``bandejão único'' é no COSEAS, perto do bandejão 
Central. No caso vocês receberam a carteirinha provisória (sim, aquela 
branca que você não sabia para que servia), que será substituida pela carteirinha
permanente no futuro. Além disso, é possível verificar o seu saldo atual no
site {\tt https://uspdigital.usp.br/rucard/}.

Existe uma lenda que anuncia que guichês de venda serão abertos próximos aos
institutos – um pouquinho de bom-senso sempre bem-vindo -, como o IME e
a Veterinária, mas por enquanto estamos apenas na espera.

O cardápio semanal do bandejão pode ser visto no site {\tt http://www.usp.br/coseas}
ou pelo aplicativo ``USP'' disponível para Android e iOS mas se você estiver
com preguiça de ver em um desses lugares, é bem provável que um VETERANO já saiba
e resolva te informar se questionado com muita educação.

O cardápio é geralmente composto de arroz (com opção normal e integral), feijão,
prato principal (carne/ovos), acompanhamento (legumes ou verduras refogadas, 
cremes, molhos), salada, sobremesa, pãozinho e suco, além de temperos genéricos
(jamais pergunte do que eles são feitos). Se você é vegetariano, o 
acompanhamento nunca (ou quase nunca) contém carne, e todos os bandejões tem uma
alternativa vegetarina de PVT para o prato principal, normalmente vindo na forma
de ração, mas existindo as formas de lasanha, kibe, e outros.


Não se esqueça de levar a sua caneca do Kit-bixo se for comer nos bandejões,
tanto para ostentar sua posição como bixo do IME, quanto para salvar as florestas, 
evitando o uso de copos descartaveis. Além disso, levar a caneca é extremamente
vantajoso, já que a quantidade de suco que cabe no copo descartável é pequena,
fazendo você viajar diversas vezes até a maquina de suco.


PS: O Efeito Bandex é proporcional à quantidade de salitre utilizado em cada bandejão.\\
PS 2: Nunca, em hipótese alguma, jamais, visite a cozinha do seu bandejão de preferência,
pois você corre o risco de nunca mais almoçar na vida. Como já dizia o velho sábio ``A
ignorância é uma virtude''.\\
PS 3: Playstation 3.

Consulte a tabela abaixo para decidir em qual dos bandejões você vai comer.

\figuragrande{bandex}
O horário das refeições é o seguinte:\\
Café da manhã: 7h às 8h30min (O horário é diferente aos fins de semana)\\
Almoço: 11h15min às 14h15min (Na Química é das 11h às 14h)\\
Jantar: 17h30 às 19h45\\

Para mais informações, visite o site do COSEAS: \href{http://www.usp.br/coseas/COSEASHP/COSEAS2010_restaurantes.html}{http://www.usp.br/coseas/}
\pagebreak

\end{subsecao}

\begin{subsecao}{Outros lugares para comer na USP}

Caso você bixo não queira comer no bandejão, seja porque está com medo do prato do dia
ou simplesmente porque quer comer algo diferente (e possivelmente de sabor melhor), saiba
que há alguns outros lugares na USP onde você pode comer.

{\bf Lanchonete da Física}

Fica um pouco antes do bandejão da Física, nela você pode almoçar pagando R\$ 29 reais no
quilo, sempre tem uma razoavel quantidade de saladas e pratos quentes, além disso, tem
churrasco com uma boa variadade de opçoes. 
Além do almoço por quilo, lá vende alguns salgados e lanches como sanduíches e beirutes.

{\bf Restaurante do IPEN}

Fica na parte de cima da rua entre o IF e o Parque Esporte Para Todos, para entrar lá você
precisa da carteirinha USP. Lá o quilo é R\$ 16,50, porém não tem uma variedade muito grande
de comida. Ele só abre para pessoas de fora às 13h e antes de ir para o restaurante você deve
se identificar.

{\bf Lanchonete da FAU}

Localizada dentro da FAU assim que você sobe a rampa. Lá além de lanches é servido almoço
tanto por quilo (custa R\$24,90) como prato feito. Tirando o por quilo, qualquer outra coisa
deve ser paga no caixa antes de retirar. Além disso, no terceiro andar da FAU tem a mesinha
de doces que os alunos deixam lá para vender, a variedade e quantidade de doces varia bastante
com o dia e o preço deles costuma ser no máximo R\$2,50.

{\bf Restaurante da FEA}

Fica atrás da FEA, lá o quilo tem bastante coisa diferente e algumas até que sofisticadas,
porém o preço do quilo lá é R\$47,90, provavelmente é o mais caro da USP. Geralmente
é frequentado por professores e funcionários da USP. Além do self-service,
há algumas opções de prato feito que custam R\$14,00.

{\bf Trailers de lanche da ECA}

Fica atrás do prédio principal da ECA, nesses carrinhos vende pastel, sanduíches, tapioca e
salgados em geral. Os preços dos lanches variam entre R\$3,00 e R\$10,00. As opções de
sanduíches e a quantidade de sabores de tapioca é bem grande e todos em geral são muito bons.
Já o pastel você pode escolher de 1 a 5 ingredientes para por no pastel, portanto o preço
varia de acordo com a quantidade de ingredientes que você vai querer.

{\bf Cachorro-quente da Reitoria}

Fica na rotatória da biblioteca Brasiliana (ou a rotatória depois do CEPE, se assim preferir)
lá o cachorro-quente pode ser feito tanto no pão de cachorro-quente como na baguete. Além disso,
há a opção de por catupiry e de por uma salsicha adicional. O preço do cachorro-quente varia
entre R\$6,00 e R\$9,00.

\end{subsecao}

\end{secao}
