\begin{subsecao}{Bach. em Matemática Aplicada e Computacional}

Estava em dúvida entre Matemática e Computação? Gosta de outras áreas também?
Então, bixo, BMAC foi a escolha certa pra você!

BMAC é um curso dentro do IME que relaciona a ``Matemática Teórica'' com
ferramentas Estatísticas e computacionais a fim de resolver problemas práticos
de diversas áreas, não necessariamente ligadas a exatas. Assim você terá uma boa
formação de Cálculo, Álgebra, Computação e Estatística, além de se especializar
em alguma habilitação que pode ser na área de Biológicas, Econômica, Elétrica,
Mecânica e outras.

Hoje em dia o mercado de trabalho está bem atrativo para os formandos no curso.
Empresas grandes e bancos procuram esse perfil dinâmico para postos de análise
financeira, crédito ou ainda em áreas de previsão Estatística como a
previdenciária.

No ramo acadêmico, os avanços com a Bioinformática e o aumento do uso de
ferramentas Estatísticas e computacionais nas pesquisas avançadas requisita
profissionais com conhecimentos avançados em exatas e que saibam adaptar tais
conhecimentos à área em questão. Além disso, os avanços em pesquisas ligadas à
própria matemática, também com aplicações em outras áreas, como Sistemas
Dinâmicos, estão em alta, e o IME é um dos grandes responsáveis pela produção
científica nacional nessa área.

Esses são apenas alguns exemplos de onde você está entrando! Com o tempo, você vai
descobrir que as possibilidades são maiores ainda! Lembre que o
curso é Noturno, o que possibilita que você trabalhe durante o dia, apesar de talvez
ficar um pouco pesado para levar algumas matérias. Você também pode fazer como
a maioria e ficar varzeando na vivência o dia todo.

O curso é o mais novo no IME, assim como essa área de atuação, o que deixa o
curso bem flexível, e os alunos costumam manter um bom diálogo com os
coordenadores do curso a fim de melhorá-lo. Também não se intimide em falar com
os VETERANOS que fazem esse curso, pois às vezes a falta de uma boa conversa
causa uma catástrofe, como uma possível transferência para a POLI (Argh!).

\end{subsecao}
