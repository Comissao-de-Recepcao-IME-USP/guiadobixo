\begin{subsecao}{Bach. em Matemática Aplicada e Computacional}

Estavam em dúvida entre Matemática e Computação? Gostam de outras áreas também?
Então, bixos, BMAC foi a escolha certa pra vocês!

Se você ingressou no BMAC provavelmente você é um pouco mais velho que o pessoal
do IME, já está trabalhando, ou está em busca de uma segunda graduação. Caso não
esteja nesse perfil prepare-se pra encontrar alunos assim na sua turma, mas não se
preocupe sempre encontramos pessoas parecidas conosco pra formar amizades.

BMAC é um curso dentro do IME que relaciona a ``Matemática Teórica'' com
ferramentas estatísticas e computacionais a fim de resolver problemas práticos
de diversas áreas, não necessariamente ligadas a exatas. Assim vocês terão uma boa
formação de Cálculo, Álgebra, Computação e Estatística, além de ao final do terceiro
semestre precisarem escolher uma habilitação pra se especializar, que são elas:

\begin {center}
  \begin {tabular}{|c|c|}
    Habilitação & Unidade \\
    \hline
    Ciências Biológicas & Bio\\
    Sistemas e Controle & Poli\\
    Mecatrônica e Sistemas Mecânicos & Poli\\
    Métodos Matemáticos & IME\\
    Saúde Animal & VET? \\
    Estatística Econômica & FEA \\
    Comunicação Científica & ECA \\
    Saúde Pública & MED? \\
    Fisiologia e Biofísica & MED? \\
  \end {tabular}
\end {center}

De maneira geral, as quatro primeiras habilitações são bem parecidas com o curso da Matemática
Aplicada, inclusive os primeiros semestres dos dois cursos são praticamente iguais,
então se você fizer amizade com umas pessoas da Aplicada vai ser legal pra trocar
\sout{a resposta das provas que os professores aplicaram de manhã} informações úteis.
E infelizmente, para o azar de quem trabalha, nem todas as habilitações são integralmente
no noturno, algumas habilitações tem aulas somente no período da tarde. As habilitações
que são oferecidas integralmente no noturno são: Ciências Biológicas, Estatística Econômica,
..... , portanto se você planejava fazer uma graduação toda a noite em uma habilitação
diferente das acimas listadas trate de rever seus planos.

Nesse ano em questão temos uma habilitação que será extinta, a ....,
e para 2017 teremos uma nova habilitação que será "Ciências Aturiais/Atuárias",
essa habilitação também vai ser na FEA e vai ser integralmente no noturno,
ela já estava sendo planejada faz um tempo e esse ano já devia entrar em vigor,
porém devido à algumas burocracias ela só entrará em 2017.

Muita gente do IME já falou que BMAC não existe, ou que um aluno do BMAC é uma
espécie em extinção, isso acontece pois o curso é noturno, então boa parte
dos alunos começam a trabalhar conforme o tempo passa, isso se já não trabalham.
E hoje em dia o mercado de trabalho está bem atrativo para alunos do BMAC.
Empresas grandes e bancos procuram esse perfil dinâmico para postos de análise
financeira, crédito ou ainda em áreas de previsão Estatística como a
previdenciária.

No ramo acadêmico, os avanços com a Bioinformática e o aumento do uso de
ferramentas Estatísticas e computacionais nas pesquisas avançadas requisita
profissionais com conhecimentos avançados em exatas e que saibam adaptar tais
conhecimentos à área em questão. Além disso, os avanços em pesquisas ligadas à
própria matemática, também com aplicações em outras áreas, como Sistemas
Dinâmicos, estão em alta, e o IME é um dos grandes responsáveis pela produção
científica nacional nessa área.

Esses são apenas alguns exemplos de onde vocês estão entrando, bixos do BMAC! Com o tempo,
vocês vão descobrir que as possibilidades são maiores ainda! Lembrem-se que o
curso é Noturno, o que possibilita que vocês trabalhem durante o dia, apesar de talvez
ficar um pouco pesado para levar algumas matérias. Vocês também podem fazer como
a maioria e ficar varzeando na vivência o dia todo.

O curso é o mais novo no IME, assim como essa área de atuação, o que
deixa o curso bem flexível, e os alunos costumam manter um bom diálogo
com os coordenadores do curso (Sônia e Mané, decorem esses nomes) a fim de melhorá-lo.
Também não se intimidem em falarem com os VETERANOS que fazem esse curso, pois às
vezes a falta de uma boa conversa causa uma catástrofe, como uma possível
transferência para a POLI (Argh!).

\end{subsecao}
