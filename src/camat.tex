\begin{secao}{O CAMat}

O CAMat - cuja sigla não combina com o nome - é o Centro Acadêmico da Matemática, Estatística e Computação.  Um centro acadêmico é uma entidade estudantil que representa es estudantes de determinado curso. No nosso caso, todes aqueles matriculades na graduação e pós-graduação do IME-USP.

A sala da entidade fica no interior da Vivência, na sala 18 do bloco B, sendo habitada por uma vendinha de guloseimas e bebidas para matar a larica e sede nossa de cada dia, e, como a Vivência, tem seu acesso livre e aberto a todes! Além dos comes e bebes, também emprestamos - mediante a apresentação do cartão USP - calculadoras científicas, baralhos e outros jogos!

Importante reforçar que a atuação do CAMat não se basta (ou não deveria bastar) à salinha ou Vivência. É dever do CAMat defender nossos interesses e lutar pelos nossos direitos enquanto estudantes, organizando eventos de teores acadêmico, cultural, festivo, político e outros, além de nos representar perante as demais instâncias e entidades da Universidade. E, embora seja tocado por uma gestão eleita anualmente em pleito democrático, todes têm o poder de opinar, questionar e participar do rumo que este centro acadêmico toma. Lembre-se: o CAMat é feito por e para es estudantes. Assim, todas as atividades promovidas pelo CAMat são discutidas e decididas em reuniões abertas, nas quais, pelo menos desde a gestão 2010, tode IMEane têm direito a voz e voto. As reuniões são semanais, alternando o horário sendo alternado entre almoço e janta, sediadas na Vivência; os dias são divulgados no início do semestre.

Agora, para salvar suas costas dos milhares de livros de defesa contra as artes das Trevas (Cálculo, Álgebra, Análise...) que você precisará carregar, o CAMat também aluga armários mediante uma taxa relativamente insignificante. E, como a distribuição ocorre através de sorteio - podendo escolher entre armário alto ou baixo -, é necessário ficar atento às datas do período de inscrição e chamada do sorteio! O período de renovação de armários é destinado àqueles que já têm armário alugado, os armários que sobram após este período são destinados ao sorteio. Caso ainda sobrem armários após o primeiro sorteio, tem o segundo sorteio (mas não conte com ele!)

Período de Renovação: 17/02 a 06/03

Período de Inscrição 1º sorteio: 17/02 a 13/03

Chamada do 1º Sorteio: a partir de 16/03

O CAMat, juntamente com es estudantes, organiza um Banco de Provas (grita no inbox ou e-mail que a gente mostra o link), com VÁAAARIAS provas dos anos anteriores feitas pelos maravilhosos veteranes. Não se esqueça de contribuir com sua prova ao final do semestre, não deixe o Banco de Provas morrer!
  
Nem só de serviços é feito um centro acadêmico. Então fica ligado nesses eventos do balaco-baco:

\begin{subsecao}{Eventos}

\textbf{Campeonato de Sinuca}

Se você gosta de jogar sinuca, cola na Vivência que a mesa está sempre disponível e basta pedir para jogar para ter sua vez nela. Se tem interesse em aprender a jogar, cola na Vivência e não se sinta intimidade pelas pessoas que já estão na mesa, elas já passaram pelo que você está passando e estão dispostas a te ajudar a aprender.

Fora as partidinhas entre aulas, temos também um campeonato de sinuca planejado, se liga nas datas: 
Período de Inscrição: 02/03 a 16/03
Início do Campeonato: 17/03

A inscrição custa R\$10,00 e o prêmio do campeonato é de 30\% do valor arrecadado com as inscrições. Então se você quiser um dindin bora chamar todo mundo pra participar. Mais informações em breve.


\textbf{An(IME)$^2$}

E logo antes da Semana Santa, nós organizamos o An(IME)$^2$, 
o evento de animes e mangás do IME. Contando 
com PokéBingo, Just Dance, Karaokê e uma sessão especial do CinIME. É uma sexta-feira para deixar seu lado otaku correr livre na corrida Naruto, elevar seu cosmo e se preparar para uma semana de descanso (e bastante estudo).


O CAMat também tem várias formas de você entrar em contato sempre que precisar:

\begin{itemize}
\item e-mail: \textit{camat.usp@gmail.com}
\item Facebook: \url{fb.com/CAMatUSP}
\item Twitter: \url{twitter.com/CAMatRadical}
\item Telefone: 3091-6293
\end{itemize}
LEMBRE: não deixem de participar (ao menos para conhecer) das reuniões do
CAMat!

LEMBRE$^2$: OCUPE a salinha do CAMat e a Vivência, esses são espaços seus também!

\end{subsecao}

\end{secao}
