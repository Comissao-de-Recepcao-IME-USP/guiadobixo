\begin{secao}{O CAMat}

O Centro Acadêmico da Matemática, Estatística e Computação é o que chamamos de
CAMat. É um órgão reconhecido pelo IME, feito pelos estudantes e para os
estudantes.

É dever do CAMat lutar pelos nossos direitos, organizando debates, palestras,
além, é claro, de eventos culturais, encontros, feiras de livros, festas...

O CAMat representa os alunos junto ao IME e todas as demais entidades do nosso
Instituto. O CAMat não é simplesmente uma sala, um espaço. O CAMat somos nós,
todos os alunos do IME, anualmente representados por um grupo de alunos que
vence um pleito democrático, do qual você também participará, seja votando ou,
logo, logo, sendo votado!

Todas as atividades promovidas pelo CAMat são discutidas e decididas em
reuniões abertas, nas quais, pelo menos desde a gestão 2010, todo aluno do IME tem
direito a voz e voto. Elas serão a melhor oportunidade para você apresentar suas
ideias, questionar, ou simplesmente ficar a par do que o CAMat
anda fazendo.

A sala do CAMat  fica dentro da Vivência, na sala 18 do bloco B. Você tem acesso
livre lá dentro para conversar, ver TV, ler jornais e revistas, jogar sinuca,
pebolim, xadrez, videogame, BARALHO (aqui você vai aprender um número inimaginável de
jogos de baralho...)\footnote{Veja a seção desse guia dos jogos da
Vivência} e até estudar.

Para salvar suas costas dos milhares de livros de defesa contra as artes das
Trevas (Cálculo, Álgebra, Análise...) que você precisará carregar, o CAMat
também aluga armários aos alunos a uma taxa insignificante. Fiquem atentos, pois no
início das aulas do 1º semestre ocorrerá a renovação dos armários com
aqueles alunos VETERANOS que já os têm e certamente sobrará uma boa quantidade
de armários para outros, incluindo vocês, bixos.

Além desses, o CAMat oferece alguns outros serviços:

\begin{itemize}
  \item Organizamos um Banco de Provas, que contém várias provas dos anos
    anteriores feitas pelos maravilhosos VETERANOS. Não se esqueça de contribuir
    com sua prova também.
  \item Emprestamos baralhos, violões e fichas de pôquer mediante apresentação
    de carteirinha USP.
  \item Em seus momentos de fome e sede fique tranquilo, pois no CAMat você
    também encontra alguns "comes e bebes", prontos para matar aquela fome e
    sede que teimam em aparecer praticamente a toda hora... Ah! Os preços são
    praticamente de atacado!
  \item Temos também uma máquina de café disponível a todos que queiram lutar
    contra o sono... É só chegar lá e preparar, de graça!
  \item Plantões de dúvidas para dar aquela força nas disciplinas! (veja no
    final desta seção)
\end{itemize}

Bom, mas o CAMat não é e nem deve ser feito só de serviços. Há atualmente
muitas mudanças no nosso instituto, e precisamos garantir que os alunos não
saiam prejudicados. Em 2010, trouxemos, junto com o
DCE, uma exposição sobre a ditadura militar para o saguão do IME. Para abrir a
exposição e discutir o assunto com a gente, recebemos o então Ministro dos
Direitos Humanos, Paulo Vannuchi. Achamos fundamental discutir assuntos não só
relativos ao IME, mas também à USP, ao Brasil e ao mundo.

O CAMat também tem várias formas de você entrar em contato sempre que precisar
(quando roubarem seu lanche, puxarem seu cabelo ou te chamarem de bobo):

\begin{description}
\item [Página:] \url{http://www.ime.usp.br/~camat}
\item [e-mail:] camat.usp@gmail.com
\item [grupo de e-mails:] camat-aberto@googlegroups.com.br (faça parte!)
\item [Facebook:] \url{fb.com/CAMatUSP}
\item [Telefone:] 3091-6293
\end{description}

LEMBRE: não deixem de participar (ao menos para conhecer) das reuniões do
CAMat. Local e horários serão propriamente divulgados.

LEMBRE$^2$: vocês são SEMPRE bem-vindos na salinha do CAMat.

\begin{subsecao}{Plantão de dúvidas}

Pra quem tá perdido no curso, sentindo falta de alguma coisa que não aprendeu
direito no colégio, não está conseguindo acompanhar as disciplinas e
eventualmente quer um conselho de alguém que está há mais tempo no IME, temos o
Plantão de Dúvidas! 

Ao longo do semestre estudantes do próprio IME (e às vezes professores) se
voluntariam pra dar aulas com assuntos de ensino médio com que vocês podem não
estar confortáveis e são importantes: funções, trigonometria, logaritmos etc.

O próprio instituto, pela Comissão de Graduação, nos ajuda todos os anos a
encontrar salas e voluntários pra essas aulas, mas os detalhes de dias e
horários são decididos apenas quando começa o semestre, de forma que não temos
como já informar onde serão esses plantões. 

Contudo, fiquem de olhos nos seus e-mail institucionais, murais e na seguinte
página: \url{https://www.facebook.com/plantaoduvidas/}

Não hesite em nos procurar!

\end{subsecao}

\end{secao}
