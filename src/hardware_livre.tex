\begin{subsecao}{Hardware Livre}

\figurapequenainline{hardwarelivre}

Nós, do Hardware Livre USP, desde 2013 nos encontramos semanalmente para
discutir sobre projetos de hardware. Estudamos e brincamos com diversos tipos de
hardware e software, desde Arduino até centrífugas de laboratório e impressoras
3D.

\textit{``Não sabemos programar e temos medo de tomar choque, podemos participar
do grupo?''}

Claro que podem! Provavelmente ao seu lado agora tem um microprocessador que
está pegando suas informações e processando de alguma maneira. Entender como
esses dispositivos funcionam é algo essencial para qualquer pessoa que queira
transformar o mundo em que vive construindo novos dispositivos para atender
suas necessidades.

\textit{``Ficamos sabendo que o grupo foi criado por alunos do BCC, não vamos
nos misturar com eles!''}

Não falem isso, um dos aspectos mais legais da universidade é a interação entre
diversas áreas do conhecimento e pessoas! No grupo já passaram pessoas de
diversos lugares, como da ECA, POLI, física, biologia… Até um pessoal do ITA
veio tirar umas dúvidas conosco. Então, se vocês não são do BCC, têm um motivo a
mais para participar.

Ao longo de todo o primeiro semestre, realizaremos uma série de workshops para
introduzir conceitos de programação e de hardware a quem nunca mexeu. Fiquem
ligados na nossa página no Facebook e no nosso site para saber mais! Mas vocês
também não precisam esperar até lá para nos encontrar — vejam em nosso site o
local semanal dos nossos encontros ou converse conosco no nosso grupo do
Telegram.

\begin{center}
  \Large
  \url{hardwarelivreusp.org}

  \url{facebook.com/Hardwarelivreusp}

  \url{Telegram: tiny.cc/hardwarelivre}
\end{center}

\end{subsecao}
