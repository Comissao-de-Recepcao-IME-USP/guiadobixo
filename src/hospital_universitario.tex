\begin{secao}{Hospital Universitário}
   \begin{quote}\emph{O HU USP é o hospital de ensino de excelência utilizado
pelos Cursos de Atenção à Saúde da USP.  O hospital privilegia as pesquisas
relacionadas aos problemas de saúde  mais comuns da população brasileira. O
atendimento é regionalizado para o bairro do Butantã, sempre com enfoque no
ensino e pesquisa''}- trecho obtido da página do HU (\url{http://www.hu.usp.br/})
\textit{alguns} anos atrás
   \end{quote}

O que isso quer dizer? Os estudantes de Medicina, Ciências Farmacêuticas,
Odontologia, Saúde Pública, da Escola de Enfermagem e do Instituto de
Psicologia, mantendo contato direto também com os Institutos de Ciências
Biomédicas, de Biologia, de Química, a Faculdade de Arquitetura e Urbanismo,
a Escola Politécnica e a Escola de Comunicações e Artes precisam de cobaias para
suas atividades/experiências. O HU é um santo lugar que recebe alguns fracos
de espírito que bebem demais e ficam incapacitados de fazer qualquer atividade
fisiológica. Para fazer o cartão do hospital, vocês devem levar:

\begin{itemize}
   \item a carteirinha USP;
   \item um documento oficial com foto (RG, CNH...);
   \item o CPF; e
   \item o CNS (Cartão Nacional de Saúde, do SUS).
\end{itemize}

Assim vocês possuirão alguns privilégios no atendimento, em situações de
emergência vocês são atendidos rapidamente (algo entre 600 minutos, como
diz na senha de espera) e não precisam soletrar o nome da mãe enquanto
estiverem desmaiados.

\end{secao}
