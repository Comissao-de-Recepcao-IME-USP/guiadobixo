\begin{subsecao}{Pokeralho}

Um dos jogos mais jogados da vivência em seu passado, retornando a ser jogado ultimamente, o
Pokeralho é uma mistura de Presidente (também conhecido como milionário) com
Poker. No pokeralho, cada jogador recebe 13 cartas. Quem embaralha e distribui é
selecionado de forma randômica, sendo bixos uma das prioridades quando este sabe
como fazer isso. A ordem das cartas é: 2 A K Q J T 9 8 7 6 5 4 3, do mais forte
para o mais fraco, exceto nos jogos de Straight, que será explicado mais a
frente. Os naipes também possuem uma ordem, que é: Espadas, Copas, Paus e Ouros,
do mais forte para o mais fraco.

As mãos utilizadas são, em ordem de força e separadas pelo número de cartas:
\begin{itemize}

\item \textbf {1 carta:}
\begin{itemize}
\item Conhecido como \textbf{single}, é uma carta qualquer.
\end{itemize}
\item \textbf {2 cartas:}
\begin{itemize}

\item \textbf{Par:} Quaisquer duas cartas de mesmo valor.
\end{itemize}
\item \textbf {3 cartas:}

\begin{itemize}
\item \textbf{Trinca:} Quaisquer três cartas de mesmo valor.
\end{itemize}
\item \textbf {5 cartas:}

\begin{itemize}
\item \textbf{Straight [Seqüência]:} Cinco cartas seguidas, de qualquer naipe,
aqui há uma regra especial, a carta Ás só pode começar ou terminar uma
sequência.
\item \textbf{Flush:} Cinco cartas de um mesmo naipe.
\item \textbf{Full House:} Uma trinca e um par.
\item \textbf{Quadra:} Quatro cartas de mesmo valor, com direito a um descarte
para completar 5 cartas.
\item \textbf{Straight Flush:} Cinco cartas seguidas do mesmo naipe. O Ás só
pode começar ou terminar uma sequência. 
\end{itemize}

\end{itemize}

O jogo se inicia com aquele que tem o $\diamondsuit$3, a carta mais fraca. Ele
então joga uma das mãos acima (não é necessário que ele utilize
o $\diamondsuit$3 nessa jogada, só que ele a tenha), e em ordem, os jogadores
jogam uma mão de mesmo número de cartas e maior força que a anterior ou passam
a vez (ou seja, se alguém abriu uma dupla, as pessoas só podem responder com
uma dupla, se alguém abrir com um jogo de $5$ cartas, então só podem ser
jogadas mãos de $5$ cartas dentre as descritas acima). 

Para os jogos de 1 e 2 cartas, a força é dada primeiro pelo valor da carta e
depois pelo naipe da mesma, assim um $\clubsuit$3  pode ser jogado sobre
um $\diamondsuit$3, mas não sobre $\heartsuit$ 3 ou uma carta de valor 4 ou
maior de qualquer naipe. Para jogos de 3 cartas, a força é dada só pelo valor
da carta. Para jogo de 5 cartas, primeiro vem a força do tipo de
jogada (Straight $<$ Flush $<$ FullHouse $<$ Quadra $<$ Straight Flush). Para duas
jogadas iguais, temos os seguintes critérios:
\begin{itemize}
	\item Straight : a maior carta da sequência é que determina a força.
	\item Flush: o naipe é o primeiro desempate, seguido pela carta de maior
valor.  \footnote{ $\clubsuit$5 $\clubsuit$8 $\clubsuit$9 $\clubsuit$10 $\clubsuit$K é
maior que $\diamondsuit$2 $\diamondsuit$J $\diamondsuit$Q $\diamondsuit$K
$\diamondsuit$A e menor que $\clubsuit$3 $\clubsuit$4 $\clubsuit$6 $\clubsuit$7
$\clubsuit$2 ou qualquer FLUSH de $\heartsuit$  ou $\spadesuit$ }
	\item Full House: é visto pelas cartas da trinca.	
	\item Quadra: valor da quadra. Ignore a carta de descarte.
	\item Straight Flush, quando aparecer um alguém lhe ensina direito.
\end{itemize}

Quando 3 jogadores passarem a vez, o último a jogar torna, podendo escolher
qualquer mão para jogar, inclusive com mais ou menos cartas que a anterior, e o
jogo prossegue assim até que alguém acabe com todas as cartas da sua mão. 

Quando um jogador bate*, as cartas nas mãos dos outros jogadores são contadas e
cada jogador recebe pontos de acordo com o numero de cartas que sobrou na mão,
esse numero é dobrado se a pessoa tiver entre 7 a 10 cartas, e triplicado se
forem 11 ou mais. Acaba o jogo quando alguém alcançar 51 pontos ou mais, nesse
momento quem tiver menos pontos ganha.

Há uma vertente do pokeralho que é o pokeralho em dupla, onde cada jogador faz
dupla com a pessoa a sua frente, o jogo é procedido normalmente, com algumas
diferenças: 
\begin{itemize}
\item Depois que cada jogador recebe as 13 cartas e as arruma, ele então
escolhe 3 cartas para passar para a dupla, e a dupla escolhe 3 cartas para
passar para o outro jogador (essa escolha deve ser feita sem troca de mensagem
entre os parceiros). 
\item Quando alguém bate, primeiro cada jogador faz a conta do total de pontos
da própria mão (dobrando / triplicando da mesma forma que no pokeralho padrão)
e depois cada dupla soma o total de pontos. 
\item O jogo termina quando uma das duplas faz 102 ou mais, essa dupla perdeu o
jogo. 

\end{itemize}
\end{subsecao}
