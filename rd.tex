\begin{secao}{O que é RD?}

RD é o Representante Discente. É um forte elo de ligação entre professores e alunos. 
RD é um aluno que representa os nossos interesses frente aos diversos conselhos existentes.
O RD ajuda, junto com os conselhos, a decidir coisas como autorização para festas, 
mudanças no currículo, aumento de vagas na FUVEST, mudança no corpo docente (às 
vezes lutamos para tirar algum professor), enfim, coisas desse tipo e muitas mais. 

Acho que você já percebeu o quanto é importante ter um aluno em cada um desses conselhos. 
Infelizmente, não costumamos preencher todas as vagas que nos é de direito. Isso se 
deve ao desinteresse de alguns ou  falta de tempo da maioria de seus VETERANOS. 

É, bixo, qué você quem tem mais tempo para fazer as coisas funcionarem aqui, já que 
ainda não sabe o que é Rec, DP, Trabalho, Estágio etc. Portanto, se você quer fazer 
alguma coisa pelo lugar onde você estuda, está aí uma dica. Para você ser RD é 
necessário se candidatar. O mandato é de um ano.\footnote{Até 2008 ou por aí, as eleições de RD eram no primeiro semestre do ano. Agora, são geralmente no final, ou seja, como bixo você não poderá atuar como RD, mas candidate-se no final do ano!}. Observação: ser um RD é também uma boa maneira de saber como pensam os seus professores e como as coisas funcionam aqui.

Em 2012, excepcionalmente, as eleições serão feitas no comecinho do ano (13, 14 e 15 de março). De acordo com o edital (nos murais, emails, fique atento!), você, bixo, por mais que não possa se canidatar a nenhum cargo, pode exercer seu direito de voto! Procure conversar com seus VETERANOS para saber melhor como funcionam essas coisas. Por enquanto, vai aí um breve resumo do que mais ou menos acontece em cada um dos órgãos nos quais temos direito a representate(s).
  
No IME, temos 26 cargos de RD, sendo 10 necessariamente de
pós-graduação e 14 necessariamente de graduação (Os dois cargos
restantes são livres). Todos os cargos tem direito a um suplente.
 
Existem diferentes níveis de hierarquia na administração.
 
{\bf As CoCs,
Comissões Coordenadoras de Curso (Lic, Pura, Estatística, Aplicada e
Computação)} são as mais próximas dos alunos. Temos um cargo de aluno em cada comissão. São comissões
pequenas, que tratam
dos problemas internos de cada curso: mudança de currículo,
requerimentos, optativas. Subordinada à CG e ao conselho do relativo
departamento. Analogamente, temos um cargo em cada Comissão
Cordenadora de Programa (de Pós).
 
{\bf Os Conselhos de Departamento (MAT, MAE, MAC e MAP)} tem uma dinâmica
um pouco diferente das CoCs, são mais formais. Cada conselho
se reúne (quase) mensalmente e são formados (em geral) por mais pessoas,
sendo
que existem regras sobre participação dos diferentes níveis
hierárquicos de professores (Titular, Associado, Doutor e Assistente). Nesses
conselhos, além de aprovar algumas das decisões das Comissões
Coordenadoras de Curso e de Programa (pós) e distribuição de
carga didática, são discutidos re-oferecimento de curso, revisão de prova,
supervisão das atividades dos docentes, afastamentos (temporários ou não),
contratação de professores e muitas outras coisas.

Os Conselhos de Departamento são subordinados à Congregação e ao CTA.
 
{\bf A Comissão de Graduação (CG)}, basicamente, avalia requerimentos,
mudança/criação de cursos e jubilamentos.
Analogamente, existe a Comissão de Pós-Graduação (CPG). Ambas são
subordinadas à Congregação.
 
{\bf Comissão de Espaço Físico (COESF)} é um orgão consultivo do CTA, formado
por representantes de diversos "ocupadores de espaço": Biblioteca, Centro
de Software Livre, Matemateca. Também tem representantes de cada
departamento. É presidida pelo vice-diretor. O RD daqui é o mesmo do CTA.
 
{\bf A Comissão de Cultura e Extensão (CCEx)} quase nunca tem reunião. Cuida das
atividades de extensão: Matemateca, CAEM, etc...
 
Os dois conselhos mais importantes são o CTA e a Congregação, ambos
presididos pelo Diretor.
 
{\bf O Conselho Técnico e Administrativo (CTA)} cuida de todas questões não
acadêmicas: Orçamento, reformas, avaliação dos funcionários, xerox,
lanchonete. É formado pelos 4 chefes de departamento, diretor, vice diretor, um
representante dos funcionários e um RD.
 
{\bf A Congregação} é o órgão máximo do Instituto. Com muitos professores, a
maioria titular. São dois RDs de graduação e um de Pós. Basicamente, neste
órgão, são rediscutidas e aprovadas (ou não) muitas das decisões dos órgãos
subordinados. Os membros da Congregação tem voto na eleição para Reitor e
Vice-Reitor. 
 
Bom, bixo, caso você não tenha lido o começo desse texto, não é difícil perceber que é muito importante ter um aluno em cada um desses conselhos. Pergunte, participe, vote. Saiba do que anda acontecendo! 

\quadrinhos3

\pagebreak
\end{secao}
