\begin{subsecao}{Estatística}

Se você, bixo esperto, acabou de ingressar no curso de Bacharelado em
Estatística do IME, PARABÉNS! Se for um aluno dedicado, com certeza será
um estatístico bem-sucedido, pois emprego é o que não falta!!! Mas não vá
pensando que vai ser moleza... Aqui vai um resumo do longo caminho que você
terá pela frente.

O 1º ano do curso de Bacharelado em Estatística é composto por matérias
básicas dessa e de outras áreas aqui do IME. Assim, você vai ter que aprender
Cálculo, Álgebra Linear, Programação, etc.

A partir do 2º semestre do 2º ano, o curso vai ficando mais
direcionado. É nesse semestre que será oferecida uma das disciplinas mais
importantes (e mais difíceis) do curso: Inferência Estatística.

O 3º ano é composto, quase que exclusivamente, por matérias da
Estatística. Você vai passar o ano todo fazendo listas e mais listas de
exercícios e vai perceber que é preciso ser um bixo (bixo é eterno e universal,
mesmo que você esteja no 3º ano) esforçado para conseguir o tão sonhado
diploma.

Finalmente, no último ano, você poderá pôr em prática um pouco de tudo o que
aprendeu, entrando em contato com pesquisadores de outras áreas, elaborando
relatórios, apresentações, etc. Se você quiser saber um pouco mais sobre isso, é
só procurar o CEA (Centro de Estatística Aplicada). Certamente você será muito
bem recebido.

Não se esqueça de que nós, VETERANOS da Estatística, estamos sempre à
disposição para esclarecer qualquer dúvida sobre as disciplinas e,
principalmente, sobre os professores.

Aproveitem o curso, façam muitos amigos e não se esqueçam de que há vida lá fora!

\end{subsecao}
