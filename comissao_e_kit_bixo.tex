\begin{secao}{Comissão de Trote e \textit{Kit}-bixo}

A Comissão de Recepção aos Calouros, também conhecida como Comissão de Trote,
é responsável por auxiliar os ingressantes em seus primeiros momentos imeanos.
Sabemos que não é um momento fácil. Você está entrando em uma nova fase da
sua vida, em um lugar estranho, com pessoas estranhas (em todos os sentidos), e
nosso objetivo é fazer com que você se sinta bem-vindo e se integre ($\int$)
com seus coleguinhas e seus VETERANOS!

Você já deve ter conhecido alguns de nós durante a matrícula, aquelas pessoas
bem legais que estavam devidamente uniformizadas, cuidando para que nenhum
bêbado ou pessoa de má índole te machucasse! Parte da Comissão ajuda os
alunos a preencher os formulários e a se matricular direitinho. Enquanto isso,
outra parte impede que a espera na fila de matrícula seja tediosa e confusa:
todos os bixos são devidamente pintados, carimbados e tosados, numa tentativa
de torná-los mais charmosos. Duro esse trabalho, não?

A Comissão de Trote organiza a super Semana de Recepção, cheia de atividades
legais! (Você deve ter recebido, junto com aquela papelada na matrícula, a
programação da Semana. Se não recebeu, procure arrumar uma, logo!).
E adivinhe quem organiza o magnífico encontro dos bixos: IMEntrando, que deverá
ocorrer em abril? %REFTIME

A Comissão de Trote é formada pelos mais animados e divertidos VETERANOS. Como
dissemos, eles organizam a matrícula, agilizam a papelada, mantêm um clima
alegre na recepção, fazem isso e aquilo na Semana de Recepção...
Você deve estar pensando ``Puxa! Como eu, bixo, posso retribuir
tamanha dedicação?'' É simples, bixo: {\bf\em compre o \textit{kit}-bixo}!!!

O \textit{kit}-bixo, como você deve saber, é um conjunto de coisas importantíssimas
para você, ingressante perdido! Ele contém dois tipos de itens:
\begin{itemize}
\item itens úteis;
\item itens essenciais. 
\end{itemize}
Dentre eles temos uma camiseta, que serve para que
nós o identifiquemos como bixo e para as pessoas na rua acharem que você é
inteligente (fique tranquilo, bixo, você não é tão melhor assim);
 materiais personalizados com o símbolo do IME pra você usar nas aulas e mostrar a seus amigos;
 uma caneca (também personalizada), pra você economizar muitos copos no bandejão
e que ainda serve como convite para o IMEntrando;
 Adesivos pra você colar no carro que vai ganhar de presente;
 Calculadora para usar nas suas provas de Estatística;
 Squeeze pra aguentar o calor do começo do ano;
 Um caderno personalizado feito com muito \sout{suor} carinho pela Comissão e
 várias outras coisas, todas contidas em uma mochila sport do IME-USP, afinal
 não dá pra carregar tudo isso na mão, né?

Adquira o maravilhoso \textit{kit}-bixo do IME-USP. Ele estará à venda na matrícula e na
semana de recepção, até durarem os estoques. (Não vai chorar depois, hein?)

Para terminar, além de tudo isso, a Comissão é que faz esse maravilhoso guia que
você está lendo agora (ou está só olhando as figuras, vai saber...). Esperamos
que você goste do nosso trabalho! Qualquer coisa, nos procure! Entre na nossa página
no Facebook: {\tt fb.com/troteimeusp} contando o que você 
sentiu ao ler o guia, o que fizeram com você na matrícula (com o nome do 
VETERANO na denúncia), o que
você achou da semana de recepção e se você se sentiu bem-vindo ou quer voltar logo
para perto da sua mãe. Estaremos sempre prontos a ajudá-lo! (ou a reprimi-lo, depende
da situação XD)

\end{secao}

