\documentclass[11pt]{report}
\usepackage[portuguese]{babel}
\usepackage[utf8]{inputenc}
%\usepackage[dvips]{graphicx}
\usepackage{graphicx}
\usepackage{enumitem}
\usepackage[normalem]{ulem}
\usepackage{latexsym}
\usepackage{amsmath}
%\usepackage{hyperref}

\newcounter{qcounter}

\setlength{\parindent}{0pt}
\setlength{\parskip}{3ex plus 0.5ex minus 0.5ex}

\addtolength{\voffset}{-0.8cm}
\addtolength{\textheight}{1.6cm}
\addtolength{\hoffset}{-1.0cm}
\addtolength{\textwidth}{2.0cm}

\pagestyle{plain}

% =============== Seção de definições de macros ========================

% Delimita uma seção:
\newenvironment{secao}[1] {
    \framebox[\textwidth] {
        \rule[-1.2ex]{5ex}{5.5ex}
        {\Large\sf #1}
        \hspace{\stretch{1}}
    } \addcontentsline{toc}{chapter}{#1}
    %\begin{sf}
    \nopagebreak[4]
} { 
    %\end{sf}
}


% Delimita uma subseção 
\newenvironment{subsecao}[1] {
    \rule[0ex]{2.5ex}{2.5ex}
    {\Large\sf #1}
	 \addcontentsline{toc}{section}{#1}
    \nopagebreak[4]
}{ }

% Coloca uma figura (sem ser quadrinhos)
\newcommand{\figura}[1] {
    \begin{figure}[!htbp]
      \begin{center}
        \includegraphics[width=\textwidth]{imagens/#1.pdf}
      \end{center}
    \end{figure}
}

% Coloca quadrinhos
\newcommand{\quadrinhos}[1] {
    \figura{quad#1}
}


% ============================ Documento ===============================
\begin{document}

%Renomeando o índice----------------------------------------------------
%\renewcommand{\contentsname}{\center Não se perca, bixo...}
\renewcommand{\contentsname}{\center Esse guia contém...}

% Capa -----------------------------------------------------------------
\figura{capa}
\clearpage
\pagebreak

%Coloca o índice---------------------------------------
\tableofcontents
\newpage

% Editorial ------------------------------------------------------------
\begin{secao}{Editorial}

Bixo (opa, bixo é com letra minúscula), foi difícil chegar até aqui. Você está meio ou completamente perdido. Temos apenas uma sugestão: aproveite esta etapa. Faça da sua estadia na USP o melhor tempo da sua vida. Você verá que a USP tem muitas e muitas coisas a oferecer. Não se preocupe apenas em estudar e passar de ano, como você fez durante sua vida inteira; aproveite TUDO (você ainda vai descobrir a definição de TUDO). Você pode não acreditar nisto agora, mas saiba que viverá momentos inesquecíveis aqui no IME, alguns fantásticos e outros deploráveis. 

Este guia foi feito para você, bixo que já sabe ler (se não souber, apenas olhe as figuras), possa aprender um pouquinho do que é a USP, o IME e a vida de universitário que se inicia agora. Ele foi escrito numa forma descontraída e fácil para que você consiga entender; mesmo assim, se pintar alguma dúvida, você pode se dirigir a qualquer VETERANO, e sua dúvida será sanada (e, quem sabe, talvez você também comece uma nova e forte amizade). Outra coisa: LEIA E DECORE COMPLETAMENTE ESTE GUIA PARA NÃO PAGAR MICO. Pensando bem, você vai pagar mico de qualquer jeito; ainda assim, seja um mínimo precavido e leia.

Lembre-se: este é seu último ano como bixo. Aproveite!

\vspace{\stretch{1}}
\rule{\textwidth}{0.5ex}\rule{2ex}{0.5ex}

\begin{small}
\begin{tabular}{|p{\textwidth}|}
\hline
\\[0.2pt]
{\large\bf Guia do bixo 2012} \\
Uma publicação da Comissão de Trote \\
\\
\makebox[4cm][l]{{\bf Editores}} Luiz e David.\\
%
%\makebox[4cm][l]{{\bf Capa}} David e Wil-Kazuo.\\
%
\makebox[4cm][l]{{\bf Textos}} André Verri (Deco),Antonieta, Fábio da Yumi, Gizela Fonseca,\\
\makebox[4cm][l]{{\bf       }} Lucas Cavalcanti, Marina Trindade, Mauricio Camilo,\\
\makebox[4cm][l]{{\bf       }} Paula Corradi, Pedrosa, Pedrão, Renata Aguemi,\\ 
\makebox[4cm][l]{{\bf       }} Ricardo Yasuda,  Yumi, David, Wil-Kazuo e autores dos textos\\ 
\makebox[4cm][l]{{\bf       }} dos guias anteriores não supracitados.\\
%
\makebox[4cm][l]{{\bf Layout}} btco (Guia 2007)                          \\
\makebox[4cm][l]{{\bf Revisão geral}} Dado a falta de tempo, bixo, a revisão é com você!\\
\makebox[4cm][l]{{\bf Agradecimentos:}} \\
Ao Donald Knuth (por inventar o \TeX\makebox{} e salvar-nos do Word na preparação
desse guia), ao btco pela iniciativa de começar este guia em \LaTeX\makebox{},
ao gimp, por nos ajudar a editar as figuras, aos bixos, por lerem o guia todo
e decorarem as músicas e os 
dez mandamentos. Ao museu de filatelia de São Jorge, por nos auxiliar na capa do guia e aos organizadores dos guias anteriores, afinal nada se cria,
tudo se copia. Por último, à gráfica e às pessoas da Comissão que
perderam as férias para que este guia ficasse pronto. \\
\makebox[4cm][l]{{\bf       }}                                            \\
\hline
\end{tabular}
\end{small}

\pagebreak

\begin{subsecao}{Os Dez Mandamentos}
  \begin{enumerate}
  \item O VETERANO tem sempre razão;
  \item Na improvável hipótese de o bixo ter razão, entra imediatamente
        em vigor o primeiro mandamento;
  \item Em qualquer evento social, as despesas correm sempre por conta
        do bixo;
  \item O bixo tem o direito de permanecer calado (exceto quando interpelado
        por um VETERANO). Tudo o que disser pode e será usado contra ele;
  \item O bixo deve se apresentar imediatamente em caso de convocação por
        um VETERANO. Os desertores serão severamente punidos;
  \item Não são válidos no IME os direitos constitucionais do bixo à vida,
        liberdade e igualdade;
  \item O bixo deve estar pronto para assumir as seguintes funções para um
        VETERANO: cadeira, cinzeiro, moleque de recados, etc, quando as
        circunstâncias assim o exigirem; e também quando não o exigirem.
  \item O bixo deve amar respeitar e os seus VETERANOS acima de qualquer
        coisa;
  \item Para os casos não abrangidos por estas regras, a decisão final
        correrá por conta dos VETERANOS.
  \item Todo bixo é BURRO.
  \end{enumerate}

  
Como bixo, você tem todo o direito de reclamar sobre os mandamentos! Qualquer reclamação deverá ser protocolada em três vias datadas, assinadas e autenticadas, com firma reconhecida em cartório, e assim encaminhada à Comissão de Trote 2012 via mala direta. As reclamações serão incineradas e os reclamantes severamente punidos. Obs.: alguns VETERANOS sugeriram que incinerássemos os reclamantes também. A medida está em estudo, devido ao custo da operação e ao lixo tóxico produzido.


\end{subsecao}
\end{secao}

\pagebreak

% Carta Aos Ingressantes ----------------------------------------------------
\begin{secao}{Carta Aos Ingressantes}

Agora que você entrou na USP, bixo, você adquiriu novas responsabilidades. Você é responsável por si mesmo, isto é, ninguém irá se preocupar com seus problemas acadêmicos (matrículas, notas erradas, dificuldade com algumas matérias, rixa com professores, etc.) se você mesmo não se preocupar. Existem pessoas que poderão te ajudar, mas só o farão se você for procurá-las. Caso contrário, o único prejudicado será você.

A faculdade não é o Paraíso (essa estação fica no final da Paulista), mas pode melhorar a cada dia. Nós, alunos, também devemos contribuir com essa melhora. Você é o futuro da universidade. Portanto, participe, reclame, procure seus direitos, ajude e, principalmente, não tenha medo de cara feia, pois isso é o que não vai faltar. Já ouviu falar em “like a Rodas”? Então se prepare. 

Lembre-se que você não é mais criança, já sabe o que quer sem que outros fiquem decidindo por você; então procure o que lhe interessa: iniciação científica, estágios, monitorias, matérias que não são obrigatórias mas que você gostaria de fazer (mesmo que não tenha nada a ver com o seu curso); participação no CAMat ou Atlética, esportes no CEPE, artigos no jornal, etc, etc, etc.

Você não é mais criança, mas também não é o bambambã, então se lembre de que seus amigos vão ser muito importantes para você e para o bom andamento do seu curso. Procure combinar atividades fora da faculdade, diferentes do cotidiano, porque isso ajuda a amenizar o stress que o dia-a-dia na faculdade pode trazer.

Além disso, como em todo lugar, existem aquelas pessoas ranzinzas e pentelhas que acham super bacana acabar com a graça de todo mundo, criticar o IME e fazer a sua baixo-estima ficar muito alta dizendo que os cursos são impossíveis e que você não vai se formar nunca, mas não acredite neles, você entrou aqui com um propósito, então: siga-o. Por mais que você passe 15 anos aqui, vai se formar (ou não).

Obs: Esse guia é auto-explicativo, portanto não se assuste com palavras e siglas
que você não entendeu, continue lendo, porque tudo será explicado com detalhes,
mastigado, tim-tim por tim-tim. Saiba que não vai faltar um monte de siglas.
Aprenda-as seletiva e rapidamente. Destaques para: USP, IME, MAC, MAT, MAE, MAP,
BCC, BM, BMA, LIC, BMAC, BE, CEPE, CEAGESP, CTA, RD, CAMat, AAAMat, SSG, PUTUSP,
P1, P2, P3, P4, P5, Pn, PQP, CEC, CNPq (=\$), FAPESP (=\$\$\$), CG, DP, REC, SUB
(esta última, ou talvez as duas ou três últimas, você vai conhecer bem melhor
mais cedo ou mais tarde).

\end{secao}
\pagebreak
% Comissão de Trote e o Kit-bixo -------------------------------------------

\quadrinhos1
\begin{secao}{Comissão de Trote e o Kit-bixo}

A Comissão de Recepção aos Calouros, também conhecida como Comissão de Trote,
é responsável por auxiliar os ingressantes em seus primeiros momentos imeanos.
Nós sabemos que não é um momento fácil. Você está entrando em uma nova fase da
sua vida, em um lugar estranho, com pessoas estranhas (em todos os sentidos) e
o nosso objetivo é fazer com que você se sinta bem vindo e se integre ($\int$) com
seus coleguinhas e seus VETERANOS!

Você já deve ter conhecido alguns de nós durante a matrícula, aquelas pessoas
'bem' legais que estavam devidamente uniformizadas cuidando para que nenhum bêbado 
ou pessoa de má índole te machuque! Uma parte da Comissão ajuda os alunos a 
preencher os formulários e a se matricular direitinho. Enquanto isso,
outra parte impede que a espera na fila de matrícula seja tediosa e confusa: todos os bixos
são devidamente pintados, carimbados e tosados, numa tentativa de torná-los
mais charmosos. Além disso, na fila, os bixos são estimulados a cantar as canções
folclóricas do IME. Duro esse trabalho, não?

A Comissão de Trote organiza a super Semana de Recepção, cheia de atividades
legais! (Você deve ter recebido, junto com aquela papelada na matrícula, a
programação da Semana. Se não recebeu, procure arrumar uma logo!).                      
%No tradicional evento de Doação de Sangue, a Comissão também vai estar lá para
%segurar a sua mãozinha (vai?). 
E adivinhem quem organiza a magnífica FESTA
DOS BIXOS, logo depois do início das aulas??

A Comissão de Trote é formada pelos mais animados e divertidos VETERANOS. Como
dissemos, eles organizam a matrícula, agilizam a papelada, mantêm um clima
alegre na recepção, fazem isso e aquilo na semana de Recepção, organizam dúzias
de festas...Você deve estar pensando ``Puxa, como eu, bixo, posso retribuir
tamanha dedicação?'' É simples, bixo: {\bf\em compre o kit-bixo}!!!

O kit-bixo, como você deve saber, é um conjunto de coisas importantíssimas
para você, ingressante perdido! O kit-bixo contém dois tipos de itens:
\begin{enumerate}

\item itens úteis; 
\item itens essenciais. Exemplos: a sua camiseta de bixo, que
serve para que nós o identifiquemos como bixo, e para as pessoas na rua acharem
que você é inteligente (fique tranquilo, bixo, você não é tão melhor assim);
materiais essenciais e numa linha personalizada especial
 para a sua sobrevivência e bom desempenho no IME; o convite para a festa dos bixos;
 alguns brindes (tem um CD ótimo!),
etc. 

\end{enumerate}
A Comissão usa o dinheiro para financiar suas atividades e reparte o excedente
entre o CAMat e a Atlética. Na verdade, esse dinheiro é uma importante fonte de
renda para esses órgãos. Assim, ao comprar o kit, você não só estará
satisfazendo seus mais íntimos desejos consumistas, como também estará ajudando
diretamente o Centro Acadêmico e a Atlética!!! Portanto deixe de ser muquirana e
adquira o maravilhoso kit-bixo do IME-USP. Ele estará à venda na matrícula e na
semana de recepção, até durarem os estoques. (não vai chorar depois, hein?)

%\figura{intbixos}

Para terminar, além de tudo isso, a Comissão é que faz esse maravilhoso guia que
você está lendo agora (ou está só olhando as figuras, vai saber...). Esperamos que
goste do nosso trabalho! Qualquer coisa, nos procure! Mande um email para comissaodetrote@gmail.com contando o que você sentiu 
ao ler o guia, o que fizeram com você na matrícula (com o nome do VETERANO na denúncia),
o que você achou da semana de recepção e se você se sentiu bem-vindo ou quer voltar logo
para a sua mãe. Estamos sempre prontos para ajudá-lo! (ou reprimí-lo, depende da situação XD) 

\end{secao}

% O CAMat ------------------------------------------------------------------
\begin{secao}{O CAMat}

O Centro Acadêmico da Matemática, Estatística e Computação é o que chamamos de CAMat. Esse é um órgão reconhecido pelo IME, feito pelos estudantes e para os estudantes.
 
É dever do CAMat lutar pelos nossos direitos, organizando debates, palestras, além, é claro, eventos culturais, encontros, feiras de livros, festas...
 
O CAMat representa os alunos junto ao IME e todas as demais entidades do nosso Instituto. O CAMat não é simplesmente uma sala, um espaço, o CAMat somos nós, todos os alunos do IME, anualmente representados por um grupo de alunos que vence um pleito democrático, do qual você participará seja votando ou logo, logo, sendo votado!

Todas as atividades promovidas pelo CAMat são discutidas e decididas em reuniões, nas quais, pelo menos desde a gestão 2010, todo aluno do IME tem direito a voz e voto. Elas serão a melhor oportunidade para você dar suas idéias, questionar qualquer coisa,
ou simplesmente ficar a par do que o CAMat anda fazendo. 
 
A sala do CAMat  fica dentro da Vivência, sala 18 do bloco B. Você tem acesso livre lá
dentro para conversar, ver TV, ler jornais e revistas, jogar sinuca, xadrez, BARALHO
(aqui você vai aprender um número inimaginável de jogos de baralho)\footnote{Vide a mais nova seção desse guia dos jogos na Vivência}... e até estudar.
 

Para salvar suas costas dos milhares de livros de defesa contra as artes das Trevas
(Cálculo, Álgebra, Análise...) que você precisará carregar, o CAMat também aluga
armários aos alunos a uma taxa insignicante. Fique atento, no início das aulas do primeiro semestre ocorrerá a renovação dos armários com aqueles alunos VETERANOS que já os tem e certamente sobrará uma boa quantidade de armários para outros inclusive vocês, bixos.
 
Além desses, o CAMat oferece alguns outros serviços:

\begin{itemize}
\item Organizamos um banco de provas com provas dos anos anteriores feitas pelos
maravilhosos VETERANOS e que você pode consultar. Não se esqueça de contribuir com sua prova também.
\item Emprestamos baralhos, violões e fichas de poker mediante apresentação de carteirinha USP.
\item Vendemos tíquetes do bandejão na nossa salinha. Temos o acordo com a COSEAS de vender ao preço de custo (R\$ 1,90), mas nada o impede de doar os 10 centavos do troco para o seu CA...
\item Para seus momentos de fome e sede fique tranquilo, pois no CAMat você também encontra alguns "Comis e Bebis", prontos para matar aquela fome e sede que teimam em aparecer praticamente toda hora... Ah os preços são praticamente de atacado!
\item Temos também uma máquina de café disponível a todos quantos queiram lutar contra o sono... É só chegar lá e preparar, é de graça!
\end{itemize}

Bom, mas o CAMat não é e nem deve ser feito só de serviços. Há atualmente muitas mudanças no nosso instituto e precisamos garantir que os alunos não saiam prejudicados. De oito anos pra cá, o CAMaEm 2010, trouxemos, junto com o DCE, uma exposição sobre a ditadura militar para o saguão do IME. Para abrir a exposição e discutir o assunto com a gente, recebemos o então Ministro dos Direitos Humanos, Paulo Vannuchi. Achamos fundamental discutir assuntos não só relativos ao IME, mas também à USP, ao Brasil e ao mundo. 
 

O CAMat também tem uma página na WEB {\tt http://camatimeusp.org}, um e-mail {\tt camat@ime.usp.br}, um grupo de e-mails {\tt camat-aberto@googlegroups.com.br} (faça parte!) e um telefone {\tt 3091-6293} para você entrar em contato sempre que precisar (quando roubarem seu lanche, puxarem seu cabelo ou te
chamarem de bobo).

LEMBRE-SE: não deixe de participar (ao menos para conhecer) das reuniões do CAMat. Local e horários serão propriamente divulgados

LEMBRE-SE$^2$: você é sempre bem vindo na salinha do CAMat (mesmo sendo bixo).

\end{secao}

% A Atlética -------------------------------------------------------------
\quadrinhos2
\begin{secao}{A Atlética}

  \begin{subsecao}{O que é a AAAMat?}

    É a Associação Atlética Acadêmica da Matemática. É formada por alunos do IME e tem como objetivo organizar e divulgar atividades esportivas e eventos (festas) para a comunidade imeana, visando o seu desenvolvimento físico e mental, e também a integração [$\int$] entre os alunos de diferentes cursos, de diversos anos e de outras faculdades. 

Dentre essas atividades há a IMEteria, a bateria dos alunos do nosso instituto, que acompanha as equipes imeanas em jogos, eventos, etc. 

  
  \end{subsecao}  
  
  Algumas das atividades da Atlética:

\pagebreak

  \begin{subsecao}{Atividades esportivas internas:}

A Atlética promove anualmente campeonatos internos com o objetivo de integrar ($\int$) os alunos do IME. Já foram promovidos campeonatos de xadrez, futsal, pebolim, sinuca e seletivas de natação e de truco.

Novos desafios estão vindo por aí! Idéias e sugestões são sempre bem vindas!


    \end{subsecao}
   
  A Atlética também representa o IME em diversos campeonatos universitários. São eles:

  \begin{subsecao}{Bixusp}


Como o próprio nome já diz, o Bixusp é um campeonato disputado entre a maioria das faculdades da USP em que apenas os bixos participam. O campeonato é disputado nos finais de semana do mês de março e conta com grande participação das torcidas incentivando seus times à vitória e já mostrando a você, bixo, a rivalidade histórica entre algumas faculdades.

O IME tem como tradição sempre revelar grandes talentos e grandes equipes: em 2005 o basquete masculino ganhou a primeira medalha de ouro em uma modalidade coletiva na história da participação imeana. Neste mesmo ano faturamos o Tênis de Mesa Feminino e ficamos em segundo lugar no Masculino. Em 2007, o Xadrez imeano, o Basquete Masculino, o Futsal Masculino e o Futebol de Campo conquistaram o terceiro lugar e fomos campões no Atletismo Masculino, garantindo, assim, o segundo lugar no Atletismo geral e quarto lugar na classificação geral. Em 2008, ficamos em segundo lugar no Futsal Masculino. Em 2009, o Atletismo foi campeão, o Tênis de Mesa Masculino foi vice. Em 2010, o Tênis de Mesa Masculino e o Atletismo Feminino foram vice.

O importante no Bixusp não é saber jogar, mas ter vontade de participar. Portanto, se você acha que não joga muito bem, não tem problema. Se por acaso você tem alergia a esporte, compareça e torça pelos nossos times. O Bixusp é uma ótima maneira de começar a conhecer melhor os VETERANOS e bixos (e bixetes) do IME e de outras faculdades. Fique atento aos treinos especiais para o Bixusp. Os dias e horários dos treinos e jogos serão sempre informados através do site e do mural da Atlética, localizado na entrada do bloco B.

  \end{subsecao}

  \begin{subsecao}{Copa USP e Jogos da Liga}

São os campeonatos internos da USP. A maioria das faculdades da USP participa destas competições. A Copa USP ocorre no primeiro semestre e os Jogos da Liga, no segundo semestre. Os jogos ocorrem sempre aos finais de semana.

A Copa USP é um dos mais tradicionais campeonatos da USP, mantendo uma rivalidade muito grande entre algumas faculdades. Os times são separados em duas divisões de acordo com as colocações nos anos anteriores. O IME, embora tenha poucos atletas, teve a maioria de seus times disputando a Série Azul, divisão mais forte, e muitos deles alcançando resultados expressivos.

O Handebol Masculino foi vice-campeão dos Jogos da Liga em 2006 e campeão da Copa USP em 2009. O Basquete Masculino se destacou em 2004 e em 2005, sendo campeão tanto da Copa USP como dos Jogos da Liga nos dois anos seguidos e ficou em terceiro lugar em 2006, assim como as meninas do Basquete. O Futebol de Campo e o Xadrez ficaram em primeiro lugar na Copa USP em 2006. O Vôlei Masculino foi quarto colocado da Copa USP Série Azul em 2007 e campeão da Copa USP em 2010, além de vice nos Jogos da Liga em 2010. O Futebol Masculino chegou a terceiro nos Jogos da Liga em 2010.

Em 2008, tivemos excelentes resultados nos Jogos da Liga. Ficamos entre os primeiros lugares em praticamente todas as modalidades, nos colocando entre as maiores atléticas da USP.

Como dissemos, o IME é famoso por sua inflamada torcida, que muitas vezes já ajudou nossas equipes nos momentos mais difíceis. Portanto, se você pratica esporte terá muitas opções e caso você não jogue nada, junte-se à nossa torcida!

  \end{subsecao}

%Tirado na edição do Guia 2009 pra 2012---------------------------------------
%  \begin{subsecao}{Intercomp}

%O Intercomp fez parte do nosso passado glorioso. Em 1997, 1998, 2001, 2002 e 2003 fomos campeões e em 1999, 2000, 2004 e 2005 fomos vice. Em 2006 o IME não participou por problemas técnicos. Em 2007 tivemos uma participação um tanto quanto frustrante, cultivando muitos vices-campeonatos como o Futebol de Campo, Futsal Masculino, Basquete Masculino, Vôlei Masculino, Natação Masculina, Handebol Feminino e no Xadrez.

%Ao lado da Unicamp, fomos uma das faculdades que mais venceu o Intercomp, faturando 8 torneios no total de 18 disputados. Em 2008, ficamos com o caneco.

%\end{subsecao}

\begin{subsecao}{BIFE}

O BIFE é uma competição contendo apenas faculdades da USP. Essas competições esportivas acontecem em cidades do interior ou litoral, onde ficamos 4 dias alojados pra cochilos de poucas horas, com jogos durante o dia e festas durante a noite e a madrugada.

Em 2011 as faculdades presentes foram: BIO, IME, FAU, ECA, FFLCH, Veterinária, Geologia, Física, Psicologia, Química, FOFITO e RI. As letras são as iniciais dos quatro fundadores: BIO, IME, FAU e ECA.

É uma competição que terá sua décima terceira edição em 2012 e que vem crescendo a cada ano, com um grande número de alunos comparecendo.

Temos um histórico muito bom no BIFE. Em 2010, ganhamos o Atletismo, ficamos em terceiro lugar em 2005, com o vice-campeonato em 2002 e 2004 e fomos campeões em 2000, 2001, 2003, 2006, 2007, 2008, 2009, 2010 e 2011! Sim, somos os atuais campeões do Bife e você, bixo, terá o compromisso de ajudar o IME a vencer em 2012 também! 


\end{subsecao}

Outras atividades:

\begin{subsecao}{Vendas}

Além desse monte de atividades citadas acima, a Atlética também faz e vende adesivos, canecas, chaveiros, camisetas e agasalhos do IME. Não deixe de visitar a Atlética e conhecer os nossos produtos.

ATENÇÃO: fique esperto e compre logo o seu agasalho para não ficar sem!

  \end{subsecao}

\begin{subsecao}{Festas}

A Atlética e o CAMAT já promoveram muitas festas e Happy Hours. O sucesso dessas festas depende em grande parte da participação dos imeanos. Convide seus amigos e participe das festas, uma outra maneira de conhecer e se integrar ($\int$) com seus colegas bixos e VETERANOS.

Uma das nossas festas é a Melhores do Ano, onde os melhores atletas do ano anterior são premiados. E o melhor: é open bar! A festa de 2011 foi muito boa, e com certeza esse ano será melhor ainda! Contamos com sua presença para prestigiar os VETERANOS e se divertir na festa.

\end{subsecao}

\begin{subsecao}{Como falar com a Atlética?}


Sempre que tiver alguma dúvida, reclamação ou sugestão você pode falar pessoalmente com qualquer membro da Atlética ou ir até a sala da Atlética, sala B-18, dentro da Vivência. Pergunte por aí, veja no mural, ligue no telefone: 3091-6378 ou mande um e-mail para atletica@ime.usp.br.

Visite também nosso site na Internet: www.ime.usp.br/~atletica. 

A Atlética inicia 2012 esperando idéias e sugestões que serão muito bem recebidas. Não deixe de colaborar e participar dos eventos por ela promovidos. A Atlética agradece!!!

\end{subsecao}
\end{secao}

% Um Pouco Sobre o IME ----------------------------------------------------

\begin{secao}{Um Pouco Sobre o IME}

O IME, Instituto de Matemática e Estatística (não, bixo, não está errado.
Computação não faz parte da sigla mesmo! Aliás, faz sim, mas só os inteligentes
podem ver!),  tem três blocos: o bloco A, bloco B, bloco C e o novíssimo bloco D
(sim, a conta está certa).

\begin{subsecao}{Bloco A}
No bloco A, as coisas mais importantes são: as salas dos professores, a parte
administrativa (diretoria, secretarias de departamento), algumas salas de aula
(para a pós-graduação), as máquinas de café, a IMEjr, a biblioteca, as ET's e
as Salas Pró Aluno (Rede Linux).

\begin{itemize}
   \item {\bf IMEjr:} é a empresa júnior do IME, que é administrada pelos próprios
     alunos da graduação. Fica na sala 258A.

   \item {\bf Biblioteca:} fica logo na entrada, ao lado direito. Há algumas mesas 		   	individuais (que são muito confortáveis para tirar um cochilo) e algumas salas com 			lousa para estudo em grupo. Você pode pegar livros emprestados para consulta na 			hora, mas para levar para casa precisará fazer uma carteirinha. Atualmente é 				permitida a entrada de mortais comuns (alunos da graduação) no acervo, mas saiba que 	nem sempre foi assim.

   \item {\bf ET's (Estações de Trabalho):} são salas com alguns terminais 
     que têm acesso à Internet e muito mais. Não, bixo, não se anime,
     pois ela não é para você! Essa sala só pode ser usada pelos alunos
     da pós-graduação e por alguns alunos que fazem iniciação científica.

   \item {\bf Salas Pró-Aluno:} mais conhecidas como salas da Rede Linux,
     podem ser usadas por alunos da graduação (ieiii!). Utilizam o Linux,
     que, ao contrário do Windows, é um Sistema Operacional.
\end{itemize}
\end{subsecao}

\begin{subsecao}{Bloco B}

  No bloco B, você deve conhecer as salas de aula (da graduação), a
  Vivência, a máquina de snacks, a de refrigerantes e a de café,
  as mesinhas azuis, o CEC, a seção de alunos,
  o GRECIME, o CAEM e a gráfica (onde você vai tirar suas cópias\footnote{Na realidade, esse é um ano de mudanças em relação à gráfica: tudo indica que não será mais utilizadda para as cópias simples do dia-a-dia e teremos agora uma nova "xerox" para copiar 
\sout {o caderno do colega} o capítulo do livro para estudar para a prova}).

\begin{itemize}
\item {\bf Vivência:} sala 18, onde as pessoas podem dormir, assistir TV,
  jogar um pebolim, sinuca, cartas, xadrez, fliperama e até mesmo estudar.
  Guarde bem esse nome, você vai passar a maior parte do seu tempo lá.
  Lá também estão os armários e as salas do CAMat e da Atlética. 

%\item {\bf Lanchonete:} Na frente da guarita... Sem mais palavras. [tirado na edição de 2012]


\item {\bf Lanchonete:} Até 2011 havia uma no IME, mas nos parece que o inquilino teve suas síndromes de Seu Madruga, atrasou\footnote{A verdade é que o cara não somente atrasou o aluguel, mas anunciou com cada palavra que não o pagaria! E, apesar desta clareza cartesiana, só foi feito o despejo em 2011 (é, você não teve sorte e ficou sem lanchonete, há! Não que tenha perdido muita coisa...), após anos sem aluguel pago. Há projetos e projetos sobre o que será feito no local (há quem diga que a Vivência será transferida para lá).} o aluguel por anos, e teve a lanchonete fechada. 


\item {\bf GRECIME:} Grêmio dos funcionários. Fica na sala 21, e \sout {é} era uma excelente  alternativa à lanchonete.

\item {\bf Máquinas de Snacks e de Refrigerantes:} não existem mais\footnote{O dono das máquinas pagava só 5 \%  da receita das vendas para o CAMat, então a gestão de 2010 decidiu rescindir o contrato. }. Hoje, vc pode ir diretamente à salinha do CAMat comprar seu biscoito, refrigente, chocolate e etc a (quase) preço de custo! Há também boatos de que tem café de graça por lá... 

\item {\bf CEC (Centro de Ensino de Computação):} é um centro munido de
  dezenas de computadores rodando Windows, e alguns rodando Linux. Também
  é possível fazer EPs em situações de desespero (você vai descobrir o que é isso logo logo). Algumas
  aulas de computação são ministradas lá (como o desafios de programação do BCC)
 
\item {\bf Seção de Alunos:} você já a conhece da matrícula. Caso não lembre
  exatamente o local, fica na sala 15, em frente ao CEC.

\item {\bf CAEM:} sigla para o Centro de Aperfeiçoamento do Ensino da Matemática.
  É um órgão de extensão, que oferece cursos, oficinas, palestras e presta serviços
  de assessoria para professores de Matemática. Você que faz Licenciatura (portanto
  futuro professor) também pode usufruir dos serviços do CAEM, e até mesmo ser um
  estagiário de lá. Fica no primeiro andar, em frente às escadas.

\end{itemize}
\end{subsecao}

\begin{subsecao}{Bloco C}

  Apesar de atualmente abrigar os professores e a secretaria do
  departamento de Computação, o misterioso Bloco C é um local que nós,
  estudantes, infelizmente não temos livre acesso. Para garantir
  tranquilidade e boas condições de trabalho aos docentes, o aluno deve
  ser anunciado ou apresentar uma boa desculpa para adentrar o local.
  (Na verdade, os professores do MAC acham que os alunos
  são monstros verdes e gosmentos, com $\pi$ braços, $e^{10}$ olhos e que
  comem criancinhas, por isso não querem esse tipo de criatura perambulando
  pelos seus corredores.)

\end{subsecao}

\begin{subsecao}{Bloco D (vulgo C')}

  O bloco D não passa de uma extensão do bloco C. Na verdade eles são o
  mesmo bloco, só que têm entradas independentes. Pertence ao NUMEC, que
  ninguém sabe ao certo o que significa. Alguns dizem que ele não existe,
  e que é apenas fruto da sua imaginação.

\end{subsecao}


\begin{subsecao}{Nova Construção}

Estão construindo o que seria o Bloco D, mas com essa confusão de blocos talvez esse seja o bloco E, ou quem sabe o bloco neperiano ou F. Só o futuro dirá. Esperemos não passar 2012 apenas observando a “rapidez” da construção, mas se por algum milagre burocrático estivermos errados poderemos finalmente escrever esse tópico do Guia do bixo corretamente\footnote{Agora falando sério, esse prédio abrigará o CCSL - Centro de Competência em Software Livre. Mesmo se você, bixo, não for do BCC, vale a pena dar uma olhada no site: {\tt http://ccsl.ime.usp.br/}}. 


\end{subsecao}
\end{secao}



% Guia de jogos da Vivência -------------------------------------------------

\begin{secao}{Guia de jogos da Vivência }

bixo, como já foi dito muitas vezes, na USP, você não deve apenas estudar, mas também aproveitar TUDO que é oferecido. Se você é um bixo que gosta de jogar baralho, no IME existem muitos VETERANOS que ficarão felizes em lhe chamar pra jogar se estiverem precisando de mais um jogador, e, depois de olhar em todos os lugares, terem achado apenas você para completar a mesa. 

A maior concentração desses VETERANOS acontece na Vivência, e lá eles jogam principalmente os seguintes jogos: Truco, Pokeralho, Cagando, Copas, Espadas e King. Como eles sabem que você provavelmente nunca ouviu falar desses jogos, tiveram a bondade de ensiná-los antes mesmo de você aparecer por lá! Aí está um pequeno manual de jogos de baralho da vivência. Não seja um batedor de mico* e leia-o com atenção. 

Os termos marcados com um * são explicados no Glossário, no fim do guia de jogos. Você bixo, provavelmente não vai entender tudo que está escrito aqui. Nesse caso, é só ir até a Vivência e pedir pra qualquer VETERANO que estiver sentado jogando baralho que te ensine o jogo X.

Vamos então aos jogos!

\begin{subsecao}{Truco}

O Truco é um jogo de boteco, e você já deve ter jogado ou visto alguém jogar em algum momento da sua vida. (bixo, não era só você que passava o intervalo do cursinho, e até algumas aulas jogando truco...) Ele é jogado por quatro jogadores formando duas duplas ou 6 jogadores formando 2 trios, que se sentam alternados à mesa.

Utiliza-se um baralho sem as cartas 8, 9 e 10. No truco a carta mais alta é o 3 seguido pelo 2, A, K, J, Q, 7, 6, 5 e 4. 

O carteador (também chamado de 'pé') embaralha o maço e dá ao jogador da esquerda para que esse corte* o baralho. Daí distribui 3 cartas para cada jogador e vira uma carta sobre a mesa. Essa carta determina qual será a manilha do jogo. A manilha será sempre a carta seguinte em ordem de tamanho da virada. Se a carta virada for um J, a maninha será o K. Isso significa que nessa mão, o K passa a ser a carta mais forte do jogo. Entre as manilhas existe uma hierarquia de naipe. A carta de paus é a mais forte seguida da de copas, espadas e ouros.

A pessoa a direita do carteador (também chamada de 'mão') será a primeira a jogar uma carta. O jogo roda em sentido anti-horário. Todos os participantes deverão jogar uma carta na mesa seguindo a ordem de jogadores. Aquele que jogar a carta mais forte ganha a rodada e torna a jogar na próxima rodada. Ganha a mão a parceria que fizer duas das três rodadas.

\textit{O truco:}

Na sua vez de jogar, um jogador pode pedir "TRUCO!!", aumentando o valor do jogo para 3 pontos. A parceria adversária pode fugir (e perder apenas um ponto), jogar valendo 3 pontos, ou pedir "SEIS MARRECO!", aumentando mais ainda o valor do jogo. Ainda pode ser pedido "Nove" ou "Doze", sempre oferecendo a oportunidade para a equipe adversária fugir, perdendo o valor atual da jogada (Por exemplo, perdendo seis pontos ao fugir de um pedido de "Nove"). 

A rodada melada: Quando a rodada empata, por exemplo, com dois Ases jogados por duplas diferentes, a rodada é dita 'melada' e a segunda rodada decide o jogo. Se as duas cartas que empataram a rodada forem manilhas, neste caso em específico, existe um desempate, que se dá pela força dos naipes.  Caso a segunda rodada também mele, a terceira rodada decide o jogo. Em caso de outro empate, nenhuma das equipes ganha ponto. 

A mão de onze: Quando uma das equipes está com 11 pontos, cada jogador dessa equipe pode checar as cartas do seu parceiro antes de decidir se joga ou não. No caso de aceitarem o jogo, a rodada vale imediatamente 3 pontos (e não pode ser trucada, sob pena de perder o jogo). No caso de não aceitarem, a equipe adversária ganha apenas um ponto. 

O jogo continua assim até que uma das equipes atinja os 12 pontos (ou tentos) e ganhe a partida. 

\end{subsecao}
\begin{subsecao}{Pokeralho}

Um dos jogos mais jogados da vivência em seu passado, agora não tão comumente jogado, a não ser por alguns VETERANOS pouco mais VETERANOS que o comum, o Pokeralho é uma mistura de Presidente (também conhecido como milionário) com Poker. 
No pokeralho, cada jogador recebe 13 cartas. Quem embaralha e distribui é selecionado de forma randômica, sendo bixo uma das prioridades quando este sabe como fazer isso. 
A ordem das cartas é: 2 A K Q J T 9 8 7 6 5 4 3, do mais forte para o mais fraco, exceto nos jogos de Straight, que será explicado mais a frente. 
Os naipes também tem possuem uma ordem, que é: Espadas, Copas, Paus e Ouros, do mais forte para o mais fraco. 

As mãos utilizadas são, em ordem de força e separadas pelo número de cartas:
\begin{itemize}


\item \textbf {1 carta:}
\begin{itemize}
\item Conhecido como \textbf{single}, é uma carta qualquer.
\end{itemize}
\item \textbf {2 cartas:}
\begin{itemize}

\item \textbf{Par:} Quaisquer duas cartas de mesmo valor.
\end{itemize}
\item \textbf {3 cartas:}

\begin{itemize}
\item \textbf{Trinca:} Quaisquer três cartas de mesmo valor.
\end{itemize}
\item \textbf {5 cartas:}

\begin{itemize}
\item \textbf{Straight [Seqüência]:} Cinco cartas seguidas, de qualquer naipe, aqui há uma regra especial, a carta Ás só pode começar ou terminar uma sequência.
\item \textbf{Flush [colors]:} Cinco cartas de um mesmo naipe.
\item \textbf{Full House:} Uma trinca e um par.
\item \textbf{Quadra:} Quatro cartas de mesmo valor, com direito a um descarte para completar 5 cartas.
\item \textbf{Straight Flush:} Cinco cartas seguidas do mesmo naipe. O Ás só pode começar ou terminar uma sequência. 
\end{itemize}

\end{itemize}


O jogo se inicia com aquele 
que tem o $\diamondsuit$3, a carta mais fraca. Ele então joga uma das mãos acima (não é necessário que ele utilize o $\diamondsuit$3 nessa jogada, só que ele a tenha), e em ordem, os jogadores jogam uma mão de mesmo número de cartas e maior força que a anterior ou passam a vez (ou seja, se alguém abriu uma dupla, as pessoas só podem responder com uma dupla, se alguém abrir com um jogo de $5$ cartas, então só podem ser jogadas mãos de $5$ cartas dentre as descritas acima). 

Para os jogos de 1 e 2 cartas, a força é dada primeiro pelo valor da carta e depois pelo naipe da mesma, assim um $\clubsuit$3  pode ser jogado sobre um $\diamondsuit$3, mas não sobre $\heartsuit$ 3 ou uma carta de valor 4 ou maior de qualquer naipe. 
Para jogos de 3 cartas, a força é dada só pelo valor da carta. 
Para jogo de 5 cartas, primeiro vem a força do tipo de jogada (Straight < Flush < FullHouse < Quadra < StraightFlush). 
Para duas jogadas iguais, temos os seguintes critérios:
\begin{itemize}
	\item Straight : a última carta da sequência é que determina a força.
	\item Flush: o naipe é o primeiro desempate, seguido pela carta de maior valor.  ( $\clubsuit$5 $\clubsuit$8 $\clubsuit$9 $\clubsuit$10 $\clubsuit$K é maior que $\diamondsuit$2 $\diamondsuit$J $\diamondsuit$Q $\diamondsuit$K $\diamondsuit$A e menor que $\clubsuit$3 $\clubsuit$4 $\clubsuit$6 $\clubsuit$7 $\clubsuit$2 ou qualquer FLUSH de $\heartsuit$  ou $\spadesuit$  )
	\item Full House: é visto pelas cartas da trinca.	
	\item Quadra: valor da quadra. Ignore a carta de descarte.
	\item Straight Flush, quando aparecer um alguém lhe ensina direito.
\end{itemize}

Quando 3 jogadores passarem a vez, o último a jogar torna, podendo escolher qualquer mão para jogar, inclusive com mais ou menos cartas que a anterior, e o jogo prossegue assim até que alguém acabe com todas as cartas da sua mão. 

Quando um jogador bate*, as cartas nas mãos dos outros jogadores são contadas e cada jogador recebe pontos de acordo com o numero de cartas que sobrou na mão, esse numero é dobrado se a pessoa tiver entre 7 a 10 cartas, e triplicado se forem 11 ou mais. Acaba o jogo quando alguém alcançar 51 pontos ou mais, nesse momento quem tiver menos pontos ganha.

Há uma vertente do pokeralho que é o pokeralho em dupla, onde cada jogador faz dupla com a pessoa a sua frente, o jogo é procedido normalmente, com algumas diferenças: 
\begin{itemize}
\item Depois que cada jogador recebe as 13 cartas e as arruma, ele então escolhe 3 cartas para passar para a dupla, e a dupla escolhe 3 cartas para passar para o outro jogador (essa escolha deve ser feita sem troca de mensagem entre os parceiros). 
\item Quando alguém bate, primeiro cada jogador faz a conta do total de pontos da própria mão (dobrando / triplicando da mesma forma que no pokeralho padrão) e depois cada dupla soma o total de pontos. 
\item O jogo termina quando uma das duplas faz 102 ou mais, essa dupla perdeu o jogo. 

\end{itemize}


\end{subsecao}
\begin{subsecao}{Cagando}

O "Cagando", ou "Cagando no Bequinho" é um jogo rápido e dinâmico que provavelmente vai ser muito jogado nos intervalos das suas aulas. Como a maioria dos jogos da Vivência, é um jogo de vazas*, onde todos os jogadores começam com o mesmo número de cartas, e jogam uma por vez no sentido horário. 

Além de ser classificado como um jogo de vazas, o Cagando também testa sua noção de quão forte está sua mão* e, principalmente, faz você ferrar e rir da cara do seu novo amiguinho bixo. 

A cada rodada é distribuído um número diferente de cartas para cada jogador. Os jogadores começam o jogo com uma carta cada, e a cada rodada aumenta em 1 a quantidade da cartas recebidas. Na última rodada cada jogador terá treze cartas. 

Depois da distribuição, uma carta é virada e o naipe dessa carta será o trunfo* da rodada. Rodando para a esquerda a partir do carteador*, cada jogador chuta o número de vazas que vai ganhar naquela mão (de 0 ao número de cartas distribuídas, bixo). 

Para que seja impossível que todos ganhem pontos, o último jogador nunca pode pedir um número de vazas que faça somar o número de cartas totais. Assim, se 7 cartas foram distribuídas para cada jogador, e as pedidas anteriores foram 3, 0 e 2, o último jogador não pode pedir 2 vazas (completando 7 vazas totais). O jogo continua, sendo que em cada rodada o primeiro que falou na rodada anterior será o último a escolher um número de vazas. 

Ganha uma vaza a maior carta do naipe da primeira carta, a não ser que um trunfo seja jogado. O jogador que ganhou a vaza, torna a abrir a próxima vaza. 

No fim da mão, contam-se quantas vazas foram feitas por cada jogador. Os jogadores que fizeram o número exato de vazas que haviam dito que iriam fazer, ganham esse número como pontuação. Os jogadores que erraram perdem o módulo da diferença (é bixo, até na vivência tem matemática, se você não sabe o que é isso, possivelmente um VETERANO irá te explicar, ou não) entre o número de vazas pedidas e feitas. 

Duas rodadas são especiais: a primeira e a última. 

Na primeira rodada, ficaria muito fácil escolher se você vai fazer ou não sua vaza vendo sua carta, então ninguém pode ver sua própria carta. Em compensação, você pode ver as cartas das outras 3 pessoas, que, assim como você, devem colocar a carta na testa, com a face para os adversários. 

Na última rodada, não sobra nenhuma carta para ser virada como trunfo (todas as 52 cartas foram distribuídas), então a mão é jogada sem trunfo. Além disso, o carteador dessa rodada é sempre aquele que está em último na pontuação.


\end{subsecao}

\begin{subsecao}{$\heartsuit$Copas$\heartsuit$} 

Sim, bixo, é aquele mesmo que você joga no seu computador e sempre acha que ganha com mais de 100 pontos!! Copas é um dos jogos menos jogados na vivência, mas vale a pena conhecer. 

Copas também é um jogo de vazas, mas aqui todas as mãos são compostas por 13 cartas para cada jogador. 

14 das 52 cartas são especiais e valem pontos. Cada carta de copas vale 1 ponto, e a dama de espadas (Moça, Mulher, Procurada, Vadia, Pudim...) vale 13 pontos. Portanto, em cada mão são distribuídos 26 pontos. 

O jogo termina quando um jogador alcança 100 pontos e o vencedor é aquele que tem menos pontos. 

No começo de cada mão, todos os jogadores devem escolher 3 das suas 13 cartas recebidas, para passar para um adversário previamente determinado. A ordem de passada é jogador da esquerda, direita, frente e não passa.

Depois da passagem simultânea de todos os jogadores, o jogador com o $\clubsuit$2, abre o jogo com essa carta.

Em sentido horário, cada jogador, respondendo o naipe*, joga uma carta e o vencedor da vaza recebe todos os pontos que estiverem na mesma.

Na primeira vaza do jogo, é proibido que os jogadores, se não tiverem nenhuma carta de paus, joguem uma das 14 cartas de valor do jogo. A partir da segunda vaza, jogar uma das cartas de valor já é permitido.

 Adicionando uma tensão extra ao jogo, um jogador só pode abrir copas depois que algum outro jogador já tenha jogado uma carta de copas em uma vaza anterior, de outro naipe.

O jogo prossegue até todas as cartas serem jogadas, contando-se os pontos de cada um e anotando no placar. 

Acertando a lua: Se você conseguir, em uma mesma mão, pegar todos os pontos em jogo, 26 pontos são adicionados para seus adversários, enquanto você não ganha nenhum! Se isto levar ao fim do jogo (estourar um jogador com mais de 100 pontos) e você NÃO FOR GANHAR A PARTIDA, então todos os jogadores permanecem com seus pontos e você perde 26! 

\end{subsecao}


\begin{subsecao}{$\spadesuit$Espadas$\spadesuit$} 


bixo, se você já leu sobre o Cagando, e entendeu meio por cima como é o jogo de Copas, então Espadas será fácil pra você. 

Para começar, 13 cartas são distribuídas para cada um dos jogadores, que jogam em parceria com o jogador em frente. Neste jogo o naipe espadas será sempre o trunfo.

Seguindo a partir da esquerda do carteador, cada jogador escolhe o número de vazas que acha que vai fazer. Como é um jogo de duplas, as pedidas de cada parceria serão somadas e ambos devem jogar para cumprir esse contrato*. 

Além disso, qualquer jogador pode dizer que não fará nenhuma vaza, um contrato chamado de NIL, que é especial, pois separa o jogo de seu parceiro, tendo cada um o seu contrato.

Nesse jogo, um jogador só pode abrir espadas depois que algum outro jogador já tenha jogado uma carta de espadas em uma vaza de outro naipe.
\begin{description}

\item[Pontuação:]

Para cada vaza de um contrato são atribuídos 10 pontos. Se a parceria falha em cumprir tal contrato, a dupla perde o valor do contrato. Se a parceria consegue cumprir tal contrato, ela ganha o valor do contrato, e mais um ponto e mais 'um' na bolsa* da parceria para cada vaza feita a mais que o estipulado. Se o jogador que fez a vaza tenha pedido Nil, a dupla não ganha mais pontos. Apenas sobe o valor da bolsa*.

\item[Bolsa:]

Para evitar que os contratos sejam feitos muito baixos, e estimular a precisão nas escolhas iniciais, cada equipe mantém uma bolsa, que é uma pontuação separada que vai enchendo conforme vazas a mais que o contrato são feitas. Uma bolsa estoura quando 10 vazas são adicionadas, tirando 100 pontos da parceria que fez essas vazas a mais.

\item[O Nil:]
Quando alguém diz que não irá fazer vaza alguma, essa jogada é chamada de Nil. Tal jogada separa o contrato de seu parceiro, e vale por si só 100 pontos. Um nil cumprido ganha 100 pontos, enquanto um nil perdido, além de perder tais pontos, adiciona as vazas feitas na bolsa e não ganha nenhum ponto extra por vaza.

\item[O Blind Nil:]
Situações dramáticas pedem por atitudes dramáticas, e o Blind Nil é uma delas. Como o nome já diz, o Blind Nil é pedido sem ver as cartas e por isso, vale o dobro dos pontos!

\end{description}
Ganha o jogo a equipe que chega em 500 pontos primeiro, ou você ainda pode perder o jogo chegando a -200 pontos. Essa pontuação pode ser alterada pelo VETERANO em virtude dos horários de aula ou outros fatores limitantes de tempo...


\end{subsecao}
\begin{subsecao}{King}

O King é o jogo mais jogado por nós, IMEanos e também o mais difícil. Originalmente ele é um jogo individual, mas no IME todos nós jogamos em dupla. É um jogo de vazas e jogado com um baralho de 52 cartas por quatro pessoas.

O jogo é composto por 10 mãos, 4 positivas e 6 negativas. Em cada uma delas, cada participante recebe 13 cartas. Cada dupla tem direito a 2 posis e 3 negs.

A cada rodada, um jogador embaralha e distribui as cartas. A pessoas a esquerda do carteador pedirá posi ou neg* e a pessoa a direita naipe ou tipo. A dupla do carteador é quem dará a saída do jogo.

Nas mãos positivas do King, o objetivo é fazer o maior número de vazas, e nas mãos negativas queremos não fazer alguma coisa específica da vaza.

Quando um jogador pede Posi, seu parceiro vai escolher, baseado na própria mão, um naipe para ser o trunfo. Além dos 4 naipes conhecidos,  os jogadores podem pedir NT, que é a mão sem trunfo. Geralmente pedimos um naipe em que temos 5 cartas ou mais. Costuma-se pedir NT se o jogador não tiver nenhum naipe longo.

Após a escolha do naipe é jogada essa mão. A dupla que fizer mais vazas ganhará pontos.

Nas mão negativas do King não existe trunfo e em cada uma delas queremos negar alguma coisa em específico. As 6 mãos negativas são: Vazas, Copas, Homens , Mulheres, Duas Últimas (2U) e King.

\begin{list}{\textbf{ (\arabic{qcounter}$^{o}$ mão:)}}{\usecounter{qcounter}}

\item \textbf{Vazas -} O objetivo é fazer o menor número de vazas.

\item \textbf{Copas -}  Nessa neg deve-se evitar fazer vazas em que tenham cartas de copas. Nessa mão os jogadores só podem abrir copas quando só tiverem cartas de copas na mão.

\item \textbf{Homem -} Nessa mão, deve-se evitar fazer as vazas que tenham Reis ou Valetes.

\item \textbf{Mulheres -} Nessa mão, deve-se evitar fazer as vazas que tenham Damas.

\item \textbf{2U -} Nessa neg, deve-se evitar fazer apenas as últimas duas vazas. Fazer ou não as 11 primeiras não interfere na pontuação.

\item \textbf{King -} Nessa mão, deve-se evitar fazer a vaza que contenha o rei de copas. Aqui também só é permitido abrir copas quando o jogador só tiver cartas de copas na mão.

\end{list}

O sistema de pontuação é um pouco complicado. Essa parte pode ser pulada, mas estará aqui como referência:
\begin{itemize}

\item Vazas:	  20 pontos por vaza
\item Copas:	  20 pontos por carta de copas
\item Homens:	  30 pontos por homem
\item Mulheres: 50 pontos por mulher
\item 2U:	  90 pontos por cada uma das 2 últimas vazas
\item King:    160 pontos pelo $\heartsuit$ K
\item Posi:	  25 pontos por vaza

\end{itemize}
Como jogamos muito mesmo esse jogo, até uma sociedade para jogarmos King foi criado por alunos daqui do IME. Ela se chama Sociedade Brasileira de King (SBK), e tem até membros IMEanos já formados. A SBK organiza torneios e variantes do jogo pra que você possa se divertir muito com o seu jogo favorito, bixo.

Então, não deixe de aparecer na Vivência e botar em prática todos esses jogos que você acabou de aprender!

\end{subsecao}
\begin{subsecao}{Glossário:}

Mico: Carta de um naipe que somente um jogador tem.

Bater mico: Jogar um mico. Pode ser uma jogada boa, mas normalmente é ruim. Ela requer uma percepção de jogo bastante avançada que você, bixo, ainda não tem.

Cortar o baralho: Tirar uma quantidade de cartas de cima do baralho para mudar o ponto onde começa a distribuição das cartas.

Bater: Acabar com suas cartas, terminando, assim, o jogo.

Vaza: Conjunto de 1 carta de cada jogador, jogadas em sentido horário. Todos devem jogar o mesmo naipe da primeira carta, ou jogar qualquer outra carta se não tiverem esse naipe.

Mão: Conjunto de (normalmente) 13 cartas que cada jogador recebe várias vezes durante o jogo. Pode também ser usado como sinônimo de rodada, como em "Ganhei 3 pontos na mão anterior".

Carteador: Aquele que distribui as cartas. Na verdade é mais relacionado com quem começa jogando (normalmente começa o jogo aquele à esquerda do Carteador), já que normalmente as cartas são embaralhadas por qualquer um.

Trunfo: Naipe escolhido para ser mais forte que os outros. Em uma vaza, a carta mais alta do primeiro naipe aberto ganha, a não ser que uma carta com naipe do trunfo tenha sido jogada. Nesse caso, ganha o trunfo mais alto.

Responder o naipe: Jogar uma carta do mesmo naipe que abriu a vaza, ou jogar qualquer carta se não tiver uma carta de tal naipe.

Contrato: Número de vazas que uma parceria diz que vai fazer antes das cartas serem jogadas.

Bolsa: Continua lendo que já chega nessa parte.

Posi(tiva) ou Neg(ativa): Para determinarmos se a mão é boa para jogar Posi ou Neg usamos uma regrinha em que: o A vale 4 pontos, o K vale 3, o Q vale 2 e o J vale 1 ponto. A soma de todos os pontos do jogo é igual a 40 que dividido por 4 dá 10 pontos para jogador em média. Assim, se você tem um pouco mais de 10 pontos na mão, é uma boa idéia pedir Posi, e se tiver poucos pontos, é bom então pedir Neg.

Vocabulário extra:

Touchar: É quando um jogador não tem mais cartas do naipe que foi aberto e descarta uma carta desfavorável aos seus adversários. Por exemplo, em uma Neg Homens, ele pode jogar um valete em uma vaza que seus adversários estão fazendo.

Cortar:  É quando um jogador não tem mais cartas do naipe que foi aberto e joga um trunfo.

Baldar: É quando um jogador não tem mais cartas do naipe que foi aberto e descarta uma carta.

Destrunfar: É abrir uma mão com uma carta do trunfo e fazer com que todos respondam o naipe com o objetivo de diminuir o número de trunfos dos adversários.

Void: É quando o jogador vem sem cartas de um determinado naipe ou elas acabam no decorrer do jogo. “Vim void em paus”. Quer dizer que quando o jogador recebeu suas 13 cartas, nenhuma delas era do naipe de paus.

Quinto/Quarto/Terceiro: É a distribuição dos naipes em nossa mão. Se temos 3 cartas de copas, por exemplo, dizemos que estamos terceiro em copas. Se tem 1 carta de espadas, dizemos que estamos primeiro em espadas.

Finesse: No King, é uma aposta estatística no posicionamento das cartas para fazer sua jogada.


\end{subsecao}

\end{secao}

\pagebreak

% O que é RD? ----------------------------------------------------
\begin{secao}{O que é RD?}

RD é o Representante Discente. É um forte elo de ligação entre professores e alunos. 
RD é um aluno que representa os nossos interesses frente aos diversos conselhos existentes.
O RD ajuda, junto com os conselhos, a decidir coisas como autorização para festas, 
mudanças no currículo, aumento de vagas na FUVEST, mudança no corpo docente (às 
vezes lutamos para tirar algum professor), enfim, coisas desse tipo e muitas mais. 

Acho que você já percebeu o quanto é importante ter um aluno em cada um desses conselhos. 
Infelizmente, não costumamos preencher todas as vagas que nos é de direito. Isso se 
deve ao desinteresse de alguns ou  falta de tempo da maioria de seus VETERANOS. 

É, bixo, qué você quem tem mais tempo para fazer as coisas funcionarem aqui, já que 
ainda não sabe o que é Rec, DP, Trabalho, Estágio etc. Portanto, se você quer fazer 
alguma coisa pelo lugar onde você estuda, está aí uma dica. Para você ser RD é 
necessário se candidatar. O mandato é de um ano.\footnote{Até 2008 ou por aí, as eleições de RD eram no primeiro semestre do ano. Agora, são geralmente no final, ou seja, como bixo você não poderá atuar como RD, mas candidate-se no final do ano!}. Observação: ser um RD é também uma boa maneira de saber como pensam os seus professores e como as coisas funcionam aqui.

Em 2012, excepcionalmente, as eleições serão feitas no comecinho do ano (13, 14 e 15 de março). De acordo com o edital (nos murais, emails, fique atento!), você, bixo, por mais que não possa se canidatar a nenhum cargo, pode exercer seu direito de voto! Procure conversar com seus VETERANOS para saber melhor como funcionam essas coisas. Por enquanto, vai aí um breve resumo do que mais ou menos acontece em cada um dos órgãos nos quais temos direito a representate(s).
  
No IME, temos 26 cargos de RD, sendo 10 necessariamente de
pós-graduação e 14 necessariamente de graduação (Os dois cargos
restantes são livres). Todos os cargos tem direito a um suplente.
 
Existem diferentes níveis de hierarquia na administração.
 
{\bf As CoCs,
Comissões Coordenadoras de Curso (Lic, Pura, Estatística, Aplicada e
Computação)} são as mais próximas dos alunos. Temos um cargo de aluno em cada comissão. São comissões
pequenas, que tratam
dos problemas internos de cada curso: mudança de currículo,
requerimentos, optativas. Subordinada à CG e ao conselho do relativo
departamento. Analogamente, temos um cargo em cada Comissão
Cordenadora de Programa (de Pós).
 
{\bf Os Conselhos de Departamento (MAT, MAE, MAC e MAP)} tem uma dinâmica
um pouco diferente das CoCs, são mais formais. Cada conselho
se reúne (quase) mensalmente e são formados (em geral) por mais pessoas,
sendo
que existem regras sobre participação dos diferentes níveis
hierárquicos de professores (Titular, Associado, Doutor e Assistente). Nesses
conselhos, além de aprovar algumas das decisões das Comissões
Coordenadoras de Curso e de Programa (pós) e distribuição de
carga didática, são discutidos re-oferecimento de curso, revisão de prova,
supervisão das atividades dos docentes, afastamentos (temporários ou não),
contratação de professores e muitas outras coisas.

Os Conselhos de Departamento são subordinados à Congregação e ao CTA.
 
{\bf A Comissão de Graduação (CG)}, basicamente, avalia requerimentos,
mudança/criação de cursos e jubilamentos.
Analogamente, existe a Comissão de Pós-Graduação (CPG). Ambas são
subordinadas à Congregação.
 
{\bf Comissão de Espaço Físico (COESF)} é um orgão consultivo do CTA, formado
por representantes de diversos "ocupadores de espaço": Biblioteca, Centro
de Software Livre, Matemateca. Também tem representantes de cada
departamento. É presidida pelo vice-diretor. O RD daqui é o mesmo do CTA.
 
{\bf A Comissão de Cultura e Extensão (CCEx)} quase nunca tem reunião. Cuida das
atividades de extensão: Matemateca, CAEM, etc...
 
Os dois conselhos mais importantes são o CTA e a Congregação, ambos
presididos pelo Diretor.
 
{\bf O Conselho Técnico e Administrativo (CTA)} cuida de todas questões não
acadêmicas: Orçamento, reformas, avaliação dos funcionários, xerox,
lanchonete. É formado pelos 4 chefes de departamento, diretor, vice diretor, um
representante dos funcionários e um RD.
 
{\bf A Congregação} é o órgão máximo do Instituto. Com muitos professores, a
maioria titular. São dois RDs de graduação e um de Pós. Basicamente, neste
órgão, são rediscutidas e aprovadas (ou não) muitas das decisões dos órgãos
subordinados. Os membros da Congregação tem voto na eleição para Reitor e
Vice-Reitor. 
 
Bom, bixo, caso você não tenha lido o começo desse texto, não é difícil perceber que é muito importante ter um aluno em cada um desses conselhos. Pergunte, participe, vote. Saiba do que anda acontecendo! 

\quadrinhos3

\pagebreak
\end{secao}
% IME Júnior -------------------------------------------------------------

\begin{secao}{IMEjr: A Nossa Empresa}

Em meados de 1995, surgia a Empresa Júnior de Informática, Matemática
e Estatística do IME (IMEjr). A Empresa Júnior é, como o próprio nome
diz, uma micro-empresa administrada por estudantes de graduação, que é
o que você é agora. Ela tem o objetivo de complementar a formação do
aluno em termos da integração Universidade - Mercado de Trabalho
(Empresas).


Entre as atividades da IMEjr estão o desenvolvimento de
projetos em todas as áreas do IME e a organização de palestras, cursos
e workshops. Com isso, é fácil verificar que a IMEjr está aberta tanto
a alunos interessados em aprender a administrar uma empresa quanto a
desenvolver atividades e projetos.

A diferença entre nós e uma
empresa comum é que nossos integrantes têm muito mais liberdade de
trabalhar e os estagiários participam de projetos mais interessantes,
do ponto de vista de aprofundar sua formação, do que a maioria dos
estágios por aí.

Entre os projetos que já passaram pela IMEjr estão
os contatos com grandes empresas do mercado, análise e desenvolvimento
de sistemas de informática,  em conjunto com o CAMAT. Até um projeto
de sutura cirúrgica já apareceu nas nossas mesas.

Portanto, contamos
com vocês para esse ano, e já garantimos que existem atividades
prontas. Esperamos vocês para nos ajudarem na implementação!

LEMBRE-SE: nem sempre de aulas e livros é feito um estudante com boa
formação. Por isso, anote em sua agenda: IMEjr, sala 258-A, bloco A,
email: {\tt imejr@ime.usp.br}, site na WEB
{\tt www.ime.usp.br/$\sim$imejr/}

\end{secao}

% USPGameDev: Pesquisa e Desenvolvimentos de Jogos na USP---------------

\begin{secao}{USPGameDev: Pesquisa e Desenvolvimentos de Jogos na USP}
{\em Renan (aka Miojo)}

Valve. Blizzard. Rockstar. Nintendo. USPGameDev. O que esses nomes têm em comum? São nomes de grupos de desenvolvedores de jogos. E um deles tem sua sede na USP.

Este grupo, que você já deve ter identificado, consiste primariamente de alunos da USP, teoricamente de diversas áreas (na prática não). Criado a aproximadamente 357(mod 5) anos atrás, esse grupo já lançou um jogo (conhecido popularmente como Horus Eye, que está sofrendo sérias revisões), está desenvolvendo outro (a passo de tartaruga manca) e planeja publicar seu próprio kit de desenvolvimento para jogos 2D (utilizado no Horus Eye, e que você poderá sofrer para criar seus próprios jogos!). Além disso, o USPGameDev ocasionalmente oferece cursos sobre tecnologias utilizadas, que podem ser úteis para diversas outras aplicações. Procure ficar sabendo, nunca se sabe quando isso pode salvar seu EP. 

\end{secao}

% Contas na Rede Linux ----------------------------------------------------

\begin{secao}{Contas na Rede GNU/Linux}
\\
\\
  \begin{subsecao}{Introdução}


A rede GNU/Linux é uma rede de computadores, administrada por alunos do IME e que fornece diversos serviços para os VETERANOS e até mesmo para vocês, bixos. Ela disponibiliza:

\begin{itemize}
\item 3 salas de computadores (no bloco A) com todo \footnote{se um programa estiver faltando, mande um email pra admin@linux.ime.usp.br pedindo-o} tipo de programa necessário para suas atividades acadêmicas (a 125A, que é um corredor do lado da admin, fica aberta 24 horas por dia, 7 dias por semana\footnote{mas talvez você não consiga entrar no bloco A depois da meia noite, que é quando a portaria fecha});
\item Acesso wireless (internet sem fio);
\item Uma página na internet para cada aluno;
\item Um e-mail para cada aluno;
\item Espaço para você guardar seus arquivos;
\item Um email para cada aluno;
\item Espaço para você guardar seus arquivos;
\item Acesso remoto via ssh (shell.linux.ime.usp.br);
\item Impressões;
\item Uma wiki com dicas sobre GNU/Linux e sobre a rede;
\item Um serviço de IRC (irc.linux.ime.usp.br);
\item Listas de discussão (em parte pra você e seus colegas bixos discutirem coisas das matérias e sobreviverem ao IME)\footnote{ users-<curso>-ano@linux.ime.usp.br, onde curso é bcc, bma, bm, be, bmap, lic ou licn. Para mais informações sobre as listas, acesse postino.linux.ime.usp.br};
\item Admins dispostos e capazes, para o caso de algum usuário ter alguma boa idéia para adicionar a essa lista;
\end{itemize}
\end{subsecao}


\begin{subsecao}{O GNU/linux }

A rede utiliza em todos os seus computadores um sistema operacional chamado GNU/Linux. Esse é um sistema desenvolvido de forma colaborativa pelos usuários e empresas interessados nele. (se quiser saber mais a respeito, pesquise por "software livre” ou passe na admin e pergunte!).

O GNU/Linux não é um sistema mais difícil de usar que o Windows. Ele é apenas diferente em alguns aspectos. Além de tudo, existem cursos de GNU/Linux que são organizados pelos alunos do IME. Os admins costumam promover esses cursos, então fique atento!

Então, não se deixe intimidar pelo sistema. Se você se der ao trabalho de aprender a utilizá-lo bem, verá que ele é bastante flexível, e até mesmo interessante (tanto quanto um sistema operacional pode ser =P).


\end{subsecao}

\begin{subsecao}{Os admins}

Os admins são alunos do bacharelado em ciência da computação (vulgo BCC) que são responsáveis por administrar a rede. Entre outras coisas, isso quer dizer manter os micros funcionando, ajudar os alunos a usar a rede (com cursos \footnote{ veja na página da rede (www.linux.ime.usp.br) para saber quando. Talvez os admins passem na sua sala avisando também} e resolvendo dúvidas nos horários de plantão \footnote{na página da rede, estão os horários de todos os admins (especificamente, em {\tt www.linux.ime.usp.br/wiki/A\_Administração)}}
) e também implementar coisas novas na 
rede (aceitamos sugestões !)

Os admins são escolhidos por um treinamento que acontece de dois em dois anos. Você poderá se tornar um porque entrou no ano certo, apesar de ser 2012. Isto porque a seção é realizada entre alunos do segundo ano de BCC nos anos ímpares; sobrevivam e alistem-se já – na verdade, só em 2013. 

\end{subsecao}
\begin{subsecao}{Como criar uma conta?}

Basta passar na admin, na sala 125 do bloco A (como você é bixo: bloco A é o da biblioteca, bloco B aquele que tem muitas salas de aula e a lanchonete fantasma). Contatos:
\begin{description}

\item [Telefone:] 3091-6482
\item [e-mail:] admin@linux.ime.usp.br
\item [Página:] www.linux.ime.usp.br
\item [Sala:] 125, bloco A

\end{description}
\end{subsecao}
\end{secao}
%\figura{quad8}

% CEC -----------------------------------------------------------------------

\begin{secao}{CEC}

O Centro de Ensino de Computação é um dos laboratórios de micros do IME. Está aberto somente aos alunos do IME (graduação, pós-graduação e alunos especiais) e oferece cursos de extensão à comunidade USP e à comunidade não-USP durante todo o ano. Há computadores que usam o Windows e outros que usam o Linux como Sistema Operacional.

Para usar o CEC, você precisa de um login e senha. Para consegui-los, basta levar seu comprovante de matrícula e pedir na secretaria do CEC. Cerca de uma semana depois, você estará cadastrado na rede do CEC e poderá usar os computadores. Lembrando que você precisa de um login para cada sistema operacional: um para usar os PCs com Windows, outro para usar os PCs com Linux. E esse login do Linux Não é O MESMO da Rede Linux do Bloco A, então não vá confundir!

Obs.: em 2011, alguns pedidos de login eram recebidos com maior simpatia e resultavam em logins e senhas iguais para Windows e Linux, ou seja, uma senha a menos para você ter de lembrar. 

E um último aviso: ao frequentar o CEC, fique atento ao ar-condicionado. Se estiver ligado, bixo, dê preferência a usar calças, blusas, jaquetas e meias de lã. Cobertores são opcionais. Se não, boa sorte ou \textit{hasta la vista}! 

\end{secao}

% Dissecando os Cursos ----------------------------------------------------
\begin{secao}{Dissecando os Cursos}


Vamos dissecar os cursos agora. (Argh.. Que horrível!)

Como você já sabe (ou deveria saber) o IME fornece seis cursos: Bacharelado em
Ciência da Computação (BCC), Licenciatura em Matemática (Lic), Bacharelado em
Estatística (Estat), Bacharelado em Matemática (Pura), Bacharelado em Matemática
Aplicada (Aplicada... Duh!)  e Bacharelado em Matemática Aplicada
Computacional (BMAC). Abaixo vão algumas dicas, sugestões e explicações sobre todos
esses cursos:

\begin{subsecao}{Computação}
{\em Lucas Cavalcanti, Fábio da Yumi, Wil-Kazuo}

Muito bem, bixo, você conseguiu passar em Computação! E depois de tanto esforço e dedicação você finalmente vai poder descansar e relaxar, certo? Errado!

Se você pretende se formar no tempo ideal (4 anos), você precisará se dedicar em tempo integral ($\int$) ao curso, pelo menos nos dois primeiros anos. Ou seja, arrume um bom paitrocínio se for possível. Senão uma boa alternativa é pedir uma bolsa trabalho da COSEAS, que paga um salário mínimo e só vai tomar 40h do seu mês e, portanto, não vai atrapalhar tanto os seus estudos. Normalmente você não vai conseguir fazer estágios de verdade antes do $3^{o}$ ano, por causa das aulas do período da tarde. Então aproveite o seu curso! Se preocupe em trabalhar quando tiver mais tempo ''vago''.

Nos próximos dois anos você estudará toda a sorte de matérias, que na maioria das vezes, irão parecer completamente inúteis (e algumas vezes elas realmente serão).

Tudo começa com a temida trilogia de quatro Cálculos:
\begin{itemize}
\item Cálculo I - O Guia do Computeiro das Galáxias
\item Cálculo II - O Gradiente do Fim do Universo
\item Cálculo III - A Integral de Linha, O Rotacional e Tudo O Mais
\item Cálculo IV - Até Mais, Obrigado pelo 5 bola.
\end{itemize}

Como se não bastasse, ainda temos que passar pelas maratonas de Álgebra (Álgebra I e II e Álgebra Linear), Física (I e II), Estatística (Estatística I, II e Processos Estocásticos), e ainda dos MAC's que na verdade são MAT's (MAC300, Programação Linear). Mas não entre em pânico! Algumas matérias legais (as da computação de verdade) vão aparecer também em momentos aleatórios e cada vez mais constantes.

Muitos dizem que toda essa maratona de matemática foi inventada para torturá-lo. Eles estão certos. Mas na verdade, ela serve para te dar uma boa ''base'' em matemática, já que toda a teoria da computação envolve matemática e como futuro possível pesquisador (isso é Ciência da Computação, que é diferente de SOS computadores, Microcamp e afins) você precisa estar preparado para trabalhar com ela. Além disso, dizem que a matemática desenvolve um raciocínio lógico extremamente necessário para a programação (basta notar que as pessoas que são boas em programação geralmente são boas em matemática, ou não).

A partir de um certo momento, que você mesmo determina, é possível seguir uma ou mais áreas mais específicas da computação, como Computação Gráfica ou Inteligência Artificial por exemplo, puxando determinadas matérias como optativas (que na verdade você é obrigado a cursar). Você pode criar uma grade bem legal, de acordo com seu gosto, pois você escolhe quais matérias vai cursar (tirando as obrigatórias).

Essa formação teórica prepara você para contornar todo tipo de problema que você possa vir a encontrar em sua vida profissional. Na verdade não. Na sua vida profissional você pode ter que, por exemplo, programar em $C\#$, Asp.NET, aprender uma nova linguagem de programação bizarra ou fazer alguma coisa que aparentemente não tem nada a ver com o que você aprendeu na faculdade. E você dirá "Mas eu não tive uma aula de Como Programar na Linguagem Stavromula Beta!”. Mas o que importa é que você (teoricamente) sabe os princípios da programação e pode aplicar esse conhecimento para rapidamente dominar "toda” e "qualquer” linguagem, tecnologia, etc. O BCC não é um curso que ensinará N linguagens (na verdade, N = 2 ou 3 dependendo da boa vontade dos professores) e como usar M programas e recursos. O BCC é um curso que ensina a técnica e a teoria que te dará uma base sólida para estar pronto para aprender qualquer coisa. E essas N + M coisas que vão te ensinar vão te ajudar bastante a entender tudo.

Finalmente, estejam sempre atentos aos eventos promovidos pela Empresa Júnior, pelo CAMAT e pelo instituto, que ajudarão a complementar a sua formação. Boa sorte, pois vocês irão precisar. Use Linux e memorize essa mensagem: "SegmentationFault”. Ela será uma assombração que perseguirá você pelo resto do curso.



\end{subsecao}

\begin{subsecao}{Estatística}
{\em Renata Aguemi}

Se você, bixo esperto, acabou de ingressar no curso de Bacharelado em Estatística do IME, PARABÉNS! Se você for um aluno dedicado, com certeza será um estatístico bem sucedido pois emprego é o que não falta!!! Mas não vá pensando que vai ser moleza... Aqui vai um resumo do longo caminho que você terá pela frente.

O primeiro ano do curso de Bacharelado em Estatística é composto por matérias básicas desta e das outras áreas aqui do IME. Assim, você terá que aprender Cálculo, Álgebra Linear, Programação etc.

A partir do segundo semestre do segundo ano, o curso vai ficando mais direcionado. É neste semestre que será oferecida uma das disciplinas mais importantes (e mais difíceis) do curso: Inferência Estatística.

O terceiro ano é composto, quase que exclusivamente, por matérias da Estatística. Você vai passar o ano todo fazendo listas e mais listas de exercícios e vai perceber que é preciso ser um bixo (bixo é eterno e universal, mesmo que você esteja no terceiro ano) esforçado para conseguir o tão sonhado diploma.

Finalmente, no último ano, você poderá pôr em prática um pouco de tudo o que aprendeu, entrando em contato com pesquisadores de outras áreas, elaborando relatórios, apresentações, etc. Se você quiser saber um pouco mais sobre isso é só procurar o CEA (Centro de Estatística Aplicada). Certamente você será muito bem recebido.

Não se esqueça de que nós, VETERANOS da Estatística, estamos sempre à disposição para esclarecer qualquer dúvida sobre as disciplinas e, principalmente, professores.

Aproveitem o curso, façam muitos amigos e não se esqueçam de que há vida lá fora!

\end{subsecao}
\quadrinhos4


\begin{subsecao}{Pura}
{\em Paula Corradi, Marina Trindade e Mauricio ``=o)'' Camilo, Andre “Shinji” Rodrigues, David}


Ufa, você chegou à Pura! Seja bem vindo! Mas, um aviso: se você entrou nesse curso porque se dava bem com a matemática no colégio você vai descobrir que aqui não é bem daquele jeito. Nem por isso desista (vamos até separar nosso texto em itens para ficar mais fácil para você).
%\begin{enumerate}[label=\roman{*})]

i) Como é o curso da Pura?

Agora você deve estar pensando como ele é...
\begin{itemize}

\item Ele é Super-Duper-Mega-Mor-Ever-DeTodos DIFÍCIL... e nem por isso desista.
\item  Você vai ter que estudar muito$^5$, mas nem por isso desista.
\item  Existem matérias extremamente úteis e práticas para o dia-a-dia e é exatamente por isso que você vai acabar odiando elas (Ex. Estatística, Física, Computação...até português, se for forte), mas nem por isso desista. 
\item  Você vai ter que fazer umas duas optativas fora do IME, por isso aproveite para relaxar e abrir sua cabeça. Já pensou em aprender alguma outra língua? Logo você vai perceber que já sabe o alfabeto grego inteiro (maiúsculas e minúsculas), nada mais justo que saber associá-las.Há quem faça mímica na ECA, microeconomia na FEA, métodos anticoncepcionais na enfermagem e até a lenda sobre o ex-aluno que fez "Fauna e Flora” na Biologia!!!

\end{itemize}
ii) O que fazer depois de se formar??

Agora você deve estar pensando: "O que eu faço depois de formado?”... (se você não estava pensando aposto que agora está) 

Sim, as pessoas se formam nesse curso, acredite. O objetivo principal do Bacharelado em Matemática é formar (!?) bons (?!) pesquisadores. Para quem não sabe a matemática não está completa, isto é, sempre tem alguma coisa nova para descobrir. Se você pensa que quem se forma nesse curso só pode ser professor/pesquisador, você está muito enganado! O curso forma pessoas que sabem analisar e resolver problemas metodicamente (você vai ver que está pensando com mais clareza em breve). Com seu potente raciocínio lógico, um bacharel em Matemática pode fazer Pós-Graduação em Engenharia (argh!), Computação, Estatística (argh$^2$!), Física (argh$^3$!), Economia (argh$^5$!). Ele pode trabalhar em vários locais: universidades, colégios, bancos, empresas... Enfim, a vida se torna muito mais fácil se você é matemático. (E se nem tudo der certo você pode vender pipoca na frente de algum teatro de São Paulo.)


iii) Como sobreviver ao curso da Pura???

Bom, como nós ainda estamos cursando, não podemos dizer se vamos sobreviver ou não, mas, de qualquer jeito, podemos dar umas dicas. Um meio para ser bem sucedido é se apoiar em seus amigos: formando um grupo unido que esteja disposto a enfrentar as matérias, línguas estrangeiras (de eventuais professores), EPs (sim, você também faz EPs, se é que você sabe o que é isso), provas, subs, recs, as mesmas matérias de novo todos juntos, o curso da pura nem chega a ser tão doloroso e, na verdade, é até bem divertido. Claro que formar esse grupo não é a coisa mais fácil, já que, quando você começar a prestar atenção nos seus colegas de turma, vai achar eles bem estranhos, mas, depois de um certo tempo, você percebe que eles são bem parecidos com você.

Talvez ja tenham te contado, mas esse curso pode ser fácil de entrar, mas costumam formar-se uns 3 de nós por ano (e olhe la!). E foi no ano passado, 2011, que a Pura bateu o seu recorde de formandos ao mesmo tempo, foram 16! Isso não acontecia desde pelo menos a época do Jacy Monteiro (que você ainda vai saber quem é)! E ainda tiveram uns três malucos que formaram em três anos, mas isso, bixo, isso você não vai contar pra ninguem, nem pros seus pais, que quando você tiver na metade do seu sétimo ano vão te perguntar pela n$^16$-ésima vez por que você não se formou no mesmo tempo daqueles seus amigos.  

v) O que precisa saber sobre a Pura??? 

Primeiramente, apesar de toda a dificuldade, a Pura tem uma carga horária relativamente menor do que a maioria dos outros cursos... Teoricamente, é possível se formar em 3 anos e meio, ou até menos. E existem pessoas que o fazem (ou tentam pelo menos). Mas tome cuidado: Além de extremamente difícil (o curso já é normalmente difícil, não queira torná-lo mais difícil ainda), você corre o risco de não aprender nada e tirar notas bem mais baixas. Normalmente, o tempo que você pode vir a ter a menos de aula precisará ser gasto estudando por conta própria. Por isso, tome cuidado para não se sobrecarregar.

Tente tirar proveito da relativa flexibilidade da grade de horários: enquanto que o primeiro ano você tem todas as aulas certinhas todo dia, com o passar do curso você terá menos aulas (as quais tenderão a ficar mais difíceis), e sua grade poderá ficar cheia de buracos. Não tenha medo do trancamento parcial, quando você tiver medo de bombar alguma matéria, ou quando não se der bem com um professor: em boa parte dos cursos vale mais a pena deixar determinada matérias para depois do que fazer com algum professor com quem você não se dê bem.

Acredite: a Pura só começa realmente no segundo ano. No primeiro, você terá todas as matérias junto com outros cursos, como a estat, a aplicada e o BCC. Aproveite para fazer contatos com as pessoas dos outros cursos, pois depois disso a tendência é se distanciar deles. Só que por esse mesmo motivo, você verá bastante coisa que provavelmente não usará no resto da Pura, além de que no primeiro ano você não vai ter ainda uma boa noção do que será a pura. Você terá uma idéia melhor do que é Matemática de verdade a partir de cursos como Álgebra I e Análise Real.

Muito cuidado com o $5^{o}$ semestre, e o trio parada dura: Álgebra III, topologia e Funções Analíticas.

Recentemente, houve alterações no currículo da Pura. Parabéns bixos! Vocês não precisam mais fazer matérias chatas e que não tem nada a ver com a Pura, como Laboratório de Física e Português. Em compensação, vocês terão que fazer mais duas matérias que não eram obrigatórias: Geometria Diferencial II e Análise Matemática II\footnote{Que mudou de nome para ''Análise Funcional'' em 2011, mas os seus veteranos ainda insitirão em chamá-la de Análise Matemática II por um bom tempo...}, além do que terão que fazer mais créditos de optativas livres fora do IME.

Ah, além disso temos os sacrossantos conselhos que são passados há várias gerações:
\begin{enumerate}
\item	Lembre-se sempre que você gosta de Matemática;
\item	Não tome um curso ruim como parâmetro de como é um determinado assunto;
\item	Lembre-se sempre que você gosta de Matemática;
\item	Persista e lute;
\item	Lembre-se sempre que você gosta de Matemática;
\item	Tome consciência de que você, na grande maioria das vezes, vai ter que estudar muito;
\item	Lembre-se sempre que você gosta de Matemática;
\item	Informe-se sobre atividades extracurriculares como o programa de Iniciação Científica (que é muito bom para formação, talvez até essencial) e uma série de palestras com professores que, muito possivelmente, realizar-se-ão durante o ano;
\item	Lembre-se sempre que você gosta de Matemática;
\item	Não desanime;
\item	Lembre-se sempre que você gosta de Matemática.

\end{enumerate}
Para terminar, faça amigos na Pura, só eles vão te entender. Qualquer dúvida, você pode nos procurar. Estaremos sempre dispostos a ajudá-lo para, assim, preservarmos a nossa espécie !!!

\end{subsecao}

\begin{subsecao}{Licenciatura}
{\em Pedrosa, Cartola e outros}

Olá, bixo! Se você chegou até aqui então parabéns!

Não só porque passou na FUVEST mas também porque entrou em Licenciatura em Matemática, mesmo sendo chamado, pelos seus colegas, de doido, louco entre outros simpáticos adjetivos.

Se você ainda não sabe exatamente o que você fará com o seu curso, tentaremos te explicar, mas espero mesmo que você tenha em mente uma coisa: Você será Professor (a), aquele que tem o dom de sanar as dúvidas dos outros, então aprenda o suficiente para isso. E como fazer isso? Temos algumas sugestões:

Primeiramente, não caia na conversa de seus VETERANOS e colegas bacharelandos que insistem em dizer que o curso de licenciatura é mais fácil que o deles. São cursos diferentes:

Um bacharel é um pesquisador. Portanto, usa a Matemática explorando seus problemas em aberto na esperança de solucionar algum deles, e consequentemente criar outros mais.

Já um licenciado é um professor. Apto a lecionar na Escola Básica e com competências para fazer o aluno compreender esse universo tão mágico que é a Matemática. Se você chegou até aqui com a vontade de ser um professor (a) então provavelmente teve bons professores de matemática. Inspire-se neles, supere-os, aqui você tem a condição ideal para tanto. Somente através de você o mundo poderá ver que matemática também é legal. Ainda mais aquela aprendida na escola, pois a parte difícil fica para ser aprofundada na faculdade, e é o que você estará fazendo nesses n anos que se seguirão.

Você terá uma base de vários ramos da matemática: Geometrias, Cálculos (importante, não bombe neles ou seu curso vai demorar mais para ser concluído!), Estatísticas, Álgebra, computação entre outros. Com o decorrer do curso, você descobrirá qual área acadêmica você prefere fazer as disciplinas de aprofundamento, onde você deverá escolher que matérias você quer se especializar. Tanto pode ser na área de física (para você se tornar um professor de física também!), quanto educação, estatística, álgebra, computação, matemática aplicada em saúde animal e o que mais a sua imaginação (e o Júpiter) permitir. Como pode ver, esse curso é um “coringa” se comparado aos outros.

Além disso, sua formação também abrangerá questões como: o contexto social do aluno, preparação para sala de aula, psicologia da educação e diversas metodologias de ensino. Para isso, você fará disciplinas na Faculdade de Educação a qual lhe preparará melhor nesse contexto (ou pelo menos deveria, é, vá se acostumando...).

Com a nova reforma do MEC para as licenciaturas, implantada na USP em 2006, você também fará mais atividades acadêmicas científicas e culturais, que são: projetos de iniciação científica, oficinas e cursos de aperfeiçoamento, participação em eventos e outras ações que enriqueçam a sua formação profissional e pessoal. Fique esperto: você terá que correr atrás de tudo isso sozinho. Esteja atento com os prazos de entrega dos relatórios de cada semestre. São 200 horas para cumprir! Mas veja pelo lado bom, várias dessas atividades são prazerosas!

Como pode ver, o curso lhe dá um leque bem amplo de escolhas que podem transformá-lo em um excelente professor. Basta você querer. Portanto, bixo, aja!

Agora umas dicas tiradas da cartola: 

Você pode fazer diversas coisas acadêmicas e muitas outras não acadêmicas e consequentemente mais divertidas, porém tudo tem um preço. 
\begin{enumerate}
\item	Podemos passar o ano todo só participando de festas e levar o curso nas coxas, o que será bem divertido e estenderá o tempo que você ficará na faculdade, mas cuidado, tudo tem um limite, e jubilar, apesar dessa palavra vir de júbilo, nesse caso não é uma boa coisa! 
\item	Podemos passar o ano todo na Biblioteca estudando até rachar, ser o nerd da turma (ei, vê se passa cola viu!) e com isso diminuindo o tempo de faculdade. Você será um bom candidato a RD, já pensou nisso? Isso gera coisas boas com relação a bolsas e empregos, então também vale a pena, mas não vá se esquecer de fazer amizades, pois é a única coisa realmente importante. 
\item	O tão difícil meio-termo. É um ideal difícil de ser conquistado, afinal quem já viu um nerd em todas as baladas, ou o baladeiro de plantão que só tira 10? Aliás, vá se acostumando, pois o 10 aqui no IME é virtual... você vai entender isso mais cedo ou mais tarde! Bom, se tudo der certo você vai tirar boas notas (leia-se algo entre 5 até 7), ser mais conhecido/chegado dos professores por se formar de um a três anos a mais que o normal e ainda vai participar das melhores baladas!! Se isso não é bom então vou voltar a fazer minhas listas de Calculo...
\item	Passe em Cálculo, se tenho algo que presta para te dizer é isso, passe em cálculo, bombar aqui vai te atrapalhar muito! Claro que tem outras matérias muito importante para passar também, mas essa é pré esquisito para muitas coisas. Faça uma lista das coisas que tem pré-esquisito para cursar e dê prioridade e elas.
\item	Faça amigos, são eles que vão te ajudar a prosseguir. Muitas vezes pensamos em desistir, e os amigos são aqueles que em último caso nos arrastam, literalmente, para o caminho certo!

\end{enumerate}
%---quadrinhos7---- calvin, adoro lista de calculo

\end{subsecao}

%\quadrinhos7

» Homenagem aos politrecos



%\figura {lumpy3}

\clearpage

\begin{subsecao}{Aplicada}
{\em André Verri (Deco) e Antonieta}

Bem-vindos a um seleto grupo de imeanos. Com o menor número de vagas e o maior índice de desistência, você fazer parte deste curso o torna um indivíduo raro! Calma, calma, você logo vai descobrir que isso acontecia pois este curso era a principal segunda opção para os bixos que queriam virar politrecos. Por isso, achar um veterano deste curso é como achar aquela figurinha premiada, são poucos, mas existem! Sinta-se um privilegiado, pois você entrou no melhor (e mais flexível) curso da USP! 

O Curso de Bacharelado em Matemática Aplicada possui o menor número de créditos (carga horária) entre os cursos do IME (!). Isso significa mais tempo para aprender a jogar Pebolim (Jerônimo, Alberto e Cartola são nomes que vocês vão ouvir bastante ao se arriscarem nessa modalidade!) e King (Trate de aprender), e você logo verá como a vivência está sempre cheia dos seus colegas. Aproveite para se gabar dos outros cursos por você não ter Física e Lab. de Física. Mas não vá se empolgando muito: dificilmente você verá por aí seus VETERANOS, afinal esse também é o curso mais difícil, possui a maior carga de Estatística e computação (perdendo apenas para BE e BCC respectivamente, lógico) e fica cada vez pior a medida que você vai progredindo (bombando). Por isso aproveite bem esse seu primeiro ano, bixo, e tente não encher sua grade horária só porque você tem algum tempo livre, afinal é bom você estar disponível quando for requisitado por um VETERANO.

A partir do 3º semestre você terá que escolher entre umas das habilitações oferecidas podendo, assim, particularizar o seu currículo. As habilitações variam entre áreas tecnológicas e até biológicas:
\begin{description}

\item [Métodos Matemáticos (mais conhecido como Matemática Pura com Requinte):] o curso torna-se bastante teórico, com o currículo muito próximo da Matemática Pura. Aprofunda os conhecimentos na matemática mais abstrata, sendo bastante voltados àqueles interessados em pesquisar. Uma boa opção para aqueles que querem conhecer mais áreas da matemática do que visto pelas outras habilitações. 
\item [Controle e Automação:] estuda a aplicação da matemática em alguns aspectos da engenharia. As disciplinas da habilitação serão dadas na Poli.
\item  [Sistemas e Control:] aplica a matemática a sistemas. Assim como a anterior, as disciplinas da habilitação serão ministradas na Poli.
\item  [Ciências Biológicas:] o enfoque deste curso é na biologia, porém quem decide qual área da biologia se concentrar é o próprio aluno. As disciplinas da habilitação deverão ser escolhidas entre uma lista de eletivas, seguindo o critério de créditos a serem cumpridos. Ao contrário das habilitações politécnicas não serão exigidas disciplinas em outros institutos. 
\end{description}

A grade do curso é praticamente a mesma do noturno, o Bacharelado em Matemática Aplicada e Computacional, sendo as diferenças maiores na parte Estatística do curso e suas habilitações. Algumas habilitações oferecidas ao noturno ainda não são oferecidas ao diurno, no entanto nossos coordenadores estão tomando providências para que estas habilitações sejam oferecidas para ambos os cursos. 


\end{subsecao}

\begin{subsecao}{Bach. em Matemática Aplicada e Computacional}
{\em Pedro Peixoto (Pedrão)}

Estava em dúvida entre Matemática e Computação? Gosta de outras áreas também? Então, bixo, BMAC foi a escolha certa pra você! 

BMAC é um curso dentro do IME que relaciona a "Matemática Teórica” com ferramentas Estatísticas e computacionais a fim de resolver problemas práticos de diversas áreas não necessariamente ligadas a exatas. Assim você terá uma boa formação de Cálculo, Álgebra, Computação e Estatística além de especializar-se em alguma habilitação que pode ser na área de Biológicas, Econômica, Elétrica, Mecânica e outras.  

Hoje em dia o mercado de trabalho está bem atrativo para os formandos do curso. Empresas grandes e bancos procuram esse perfil dinâmico para postos de análise financeira, crédito ou ainda em áreas de previsão Estatística como a previdenciária. 

No ramo acadêmico os avanços com a Bioinformática e o aumento do uso de ferramentas Estatísticas e computacionais nas pesquisas avançadas requisita profissionais com conhecimentos avançados em exatas e que saibam adaptar tais conhecimentos na área em questão. Além disso os avanços em pesquisas ligadas à própria matemática, também com aplicações em outras áreas, como Sistemas Dinâmicos, estão em alta e o IME é um dos grandes responsáveis pela produção científica nacional nessa área.

Esses são apenas alguns exemplos de onde você está entrando! Com o tempo vai descobrir que as possibilidades são maiores ainda! Lembre-se de que como o curso é Noturno, ele possibilita que trabalhe durante o dia, apesar de talvez ficar um pouco pesado para levar algumas matérias. Você também pode fazer como a maioria, e ficar varzeando na vivência o dia todo.

O curso é o mais novo no IME, assim como essa área de atuação, o que deixa o curso bem flexível e os alunos costumam manter um bom diálogo com os coordenadores do curso a fim de melhorá-lo. Também não se intimide em falar com os VETERANOS que fazem este curso, pois às vezes a falta de uma boa conversa causa uma catástrofe, como uma possível transferência para a POLI (Argh!).

\quadrinhos6  

\end{subsecao}
\end{secao}

\pagebreak

%\quadrinhos5

% Um Pouco Sobre a USP ------------------------------------------------------

\begin{secao}{Um Pouco Sobre a USP}

Você, que é um novo aluno da USP, deve saber desde cedo que aqui há muita burocracia. É bom que se acostume com ela, já que você terá que enfrentá-la. 

O atual reitor - Rodas - exerce essa função graças a uma eleição que vai contra tudo o que a democracia possa oferecer. Um colégio eleitoral (do qual os alunos não fazem parte) envia uma lista com três indicações para o governador, para que ele possa escolher o seu preferido. Além disso, o reitor é presidente do maior órgão da USP, o Conselho Universitário (abrevia-se C.O. para evitar frases do tipo: "Vou ter uma reunião no CU hoje”, "O CU não está funcionando muito bem esse semestre”, "Os alunos não tem acesso ao CU”, etc.) que determina TODOS os rumos da universidade.

Abaixo dele vêm as coordenadorias e unidades. A COSEAS (Coordenadoria de Saúde e Assistência Social), por exemplo, é o departamento responsável pelos serviços que a universidade oferece (não é a melhor coisa do mundo, mas oferece) para a comunidade universitária: ônibus circulares, bandejões, moradia para estudantes (CRUSP) etc.

Alguns outros lugares que você deve saber que existem são o HU (Hospital Universitário), o CEPE (Centro de Práticas Esportivas, leia “cepê”), o banheiro da FEA (Faculdade de Economia, Administração e Contabilidade), que fica na frente do IME e é uma das faculdades mais bem abastecidas financeiramente (apelidada "carinhosamente” de Shopping), a Física (as aulas de laboratório – que só a Estat faz - são lá) e a querida Faculdade de Educação (para o pessoal da Licenciatura).

\begin{subsecao}{E-mail USP}

bixo, atenção e cuidado ao e-mail que você recebeu no ato da matrícula! É um bixo@usp.br, do sistema “oficial” de e-mails da USP, o que significa que é através dele que a Universidade, o Jupiterweb, a diretoria do IME, o CAMat e alguns desocupados lhe enviarão comunicados oficiais, o que pode envolver desde oportunidades diversas para intercâmbios, estágios, monitorias e tudo o mais que você, um simples bixo imeano, seja capaz de se imaginar fazendo. Por este e-mail você receberá informações antes mesmo que as principais (é, nem todas são colocadas em murais não) sejam também colocadas nos murais. Vale aqui uma máxima de V: “Se não querem concorrência, não farão propaganda.” 

Um modo de fugir do design deprimente da página principal do @usp.br é configurando-o para que seus e-mails sejam encaminhados para um servidor que você já use. Fica a dica! 



\end{subsecao}



\begin{subsecao}{A COSEAS}
{\em Yumi, Seno e Luiz}

A Coordenadoria de Saúde e Assistência Social, que fica próxima à praça do relógio, é o órgão da USP responsável pelo bem estar financeiro do aluno - não é lá a melhor coisa do mundo, mas ajuda... É onde você pode conseguir seus milhares de benefícios, tais como auxílios financeiros, moradia gratuita (CRUSP) ou bolsa alimentação (conhecido como vale-bandex) e dentistas gratuitos. Ou seja, há bolsas de todo modelo, tamanho, designer, estação e preferência gastronômica nula que preferir. É também responsável pelo Setor de Passes Escolares. 

Se você, bixo, estudou a vida inteira em escola pública, se seu irmão te batia, seu pai roubava seu dinheiro, ou você foi reprimido na infância, não pense que você é o único aqui e por isso merece ser mimado. Vão a seguir as diversas alternativas para você tentar:


{\bf Moradia e auxílios financeiros}


Se você infelizmente não tem tanto acesso a meios de locomoção, ou dinheiro para pagar transportes ou mesmo repúblicas, saiba que diversos auxílios podem ser oferecidos para você para amenizar a sua situação:

Com a carteirinha provisória que você recebeu na matrícula, você deverá entrar na página da COSEAS ({\tt http://www.usp.br/coseas}) e solicitar a inscrição para o processo de moradia e alojamento. Lá você terá um formulário, três trilhões de coisas para preencher e assim que for chamado, apresentar os documentos (hum!) necessários para sua assistente (você irá "ganhar” uma). Na improvável hipótese do site estar fora do ar, como todo ano acontece, você terá que ir a COSEAS e pedir alojamento na USP, lá no Bloco G – sem trocadilhos - do CRUSP.

A prioridade é dada aos residentes de outros estados ou interior de São Paulo, mas os moradores de São Paulo também podem solicitar, dadas as proporções da cidade e o tempo de duas horas para que moradores da Zona Norte venham às aulas. 

Vale ressaltar, bixo, que você tome o cuidado de não fazer o mesmo que outros bixos burros de outros anos, que confundiram alojamento com moradia. São dois requerimentos distintos e você deverá solicitar os dois se realmente quiser garantir um lugar para tomar banho e dormir.

Elas conterão perguntas sobre a sua situação socioeconômica, bem como os documentos que deverão ser trazidos para comprová-la. Perguntas como renda salarial, quantas pessoas contribuem com ela, números de bens móveis e imóveis, situação habitacional, tipo de escola em que estudou, se trabalha e há quanto tempo, quanto gasta para vir à USP, o tempo de ida e etc. Ainda há um espaço para descrever alguma particularidade não exposta nas perguntas que, obviamente receberá um parecer técnico.

Na classificação final da MORADIA, se você conseguiu uma pontuação grande, você pode escolher entre a moradia no CRUSP ou um auxílio financeiro (bolsa moradia) de R\$300,00 para você poder alugar um quarto, casa, hotel, ou mesmo transporte para ida e volta pra sua terra.  Há casos em que a classificação final lhe permite ter benefício apenas ao alojamento OU à bolsa, mas ao apertamento de três quartos individuais não (CRUSP). Nesse caso, escolha a bolsa e com o dinheiro, leve seu VETERANO para beber.

Se você não conseguir por nenhum desses meios, pode tentar a hospedagem, que é simplesmente você ficar no apertamento de alguém que more no CRUSP. Mas fique atento às datas de requerimento depois do resultado da seleção, pois você pode ficar sem essa chance. Se mesmo assim você não conseguir nada (bixo azarado!), e achar que te entenderam mal na entrevista ou coisa assim, você pode pedir para entrar com recurso, e ter mais uma chance de esclarecer melhor a sua situação (a.k.a. “cantar a assistente social”), ou também procurar a AMORCRUSP (Associação dos Moradores do CRUSP que fica no Bloco F, das 14h às 18h).

{\bf Alimentação}

Você pode solicitar também o auxílio alimentação na página da COSEAS, que consiste nos vale-bandex da USP e são válidos para almoço e jantar. Você deverá passar por outra seleção que também inclui questionários sócio-econômicos, comprovantes, e mais papéis.

{\bf Bolsa-trabalho}


Destina-se a alunos de graduação vinculados a projetos de extensão de serviços à coletividade. Os projetos são selecionados anualmente, de acordo com sua relevância para as finalidades da universidade pública e os estudantes vinculam-se por afinidade acadêmica ou científica. Cada bolsa é de 1 (um) salário mínimo por 40 horas de trabalho mensais. Além da seleção socioeconômica feita pela DPS (Divisão de Promoção Social), há uma seleção técnica feita pelos supervisores dos projetos.

Mas fique esperto! Para tudo tem prazo e a COSEAS não pode ficar esperando a sua boa vontade de aparecer por lá. Qualquer dúvida, procure a Yumi, ou ligue pra COSEAS.

{\bf Atendimento odontológico gratuíto}

Para você agendar o atendimento odontológico gratuito, é necessário fazer uma carteirinha no HU (Hospital Universitário) e em seguida, comparecer ao Bloco G do CRUSP com a carteirinha e agendar. Para colocação e manutenção de aparelhos, lá eles lhe indicam para o atendimento na odontologia (pois afinal, é muita crueldade usar animais como ratos e macacos como cobaias). 

{\bf Setor de passe escolar}

Para as linhas da EMTU, SPTRANS, METRO, você entrar no site da Coseas e fazer seu pré-cadastro. O endereço é o mesmo que todos os recursos da COSEAS: (http://www.usp.br/coseas).Talvez demore um pouco, pois a USP tem que avisar para a SPTrans que você passou na Fuvest. Fique atento! Qualquer dúvida, ligue para a sessão de passe escolar da Coseas: 3091-3581 

Há também um cartão especial para alunos de universidades públicas provenientes de outras cidades. Caso você, bixo, venha de algum desses domos ignotos (como Resende, Caçapava ou Guaíra), pode se dirigir ao guichê da sua empresa de transporte intermunicipal (Cometa, Danúbio Azul, etc.), apresentar seu Cartão USP, preencher um formulário e eles aguardarão confirmação do seu instituto. Então você pega seu cartão na Seção de Alunos e, na compra de passagens entre São Paulo e sua cidade-natal, paga 50\% do valor normal. Só não deixe isto por último na sua lista de necessidades porque existe um período do ano em que os guichês liberam seus formulários; em resumo, espiche suas orelhas e corra para a rodoviária.

\end{subsecao}

%\figura{mapausp}
\pagebreak
\figura{bandex}

%\figura{bandex} calvin

\begin{subsecao}{Bandejão}

Os bandejões, vulgarmente conhecidos como Restaurantes da COSEAS, são os lugares em que você pode se alimentar razoavelmente a um preço analogamente razoável. Os tickets custam R\$ 1,90 lá na PQP, quer dizer, na COSEAS, que fica perto do Bandejão Central, sabe, aquele que você tem preguiça de ir andando, então ele custa R\$ 2,00 com alguém na vivência da física (normalmente um japonês de camisa social), no CAMat ou com um tiozinho que vende trufas na química.

Temos de dizer, você foi azarado de novo, bixo – não reclame, THIS... IS... DOISMILEDOOOZE! A partir da metade de 2011, o sistema de tickets foi “atualizado”: agora os tickets estão eletronicamente dentro do seu Cartão USP. São comprados exclusivamente na COSEAS – estamos todos amaldiçoados – e colocados no cartão. 

Nos últimos meses, os seres ocultos que vendem carteirinhas na COSEAS nos perguntavam se queríamos os tickets em papel ou no cartão, contudo, ainda assim, há uma lenda de que o papel teve sua morte agendada e tudo será via eletrônica. Outra lenda anuncia que guichês de venda serão abertos próximos aos institutos distantes da COSEAS – um pouquinho de bom-senso sempre bem-vindo -, como o IME e a Veterinária. Até lá, vamos todos para a COSEAS, exceto se seu nome for Flynn e você retire os tickets diretamente do sistema USP. 

O cardápio semanal do bandejão pode ser visto em um dos murais do CAMat, nos próprios bandejões, no site {\tt http://www.usp.br/coseas}, ou pelo comando {\tt \$ bandex -b nome\_do\_bandejao} na Rede Linux. Se você não é do BCC ou BMAC então substitua essa última informação por “perguntando para alguém o que tem hoje”

O cardápio é geralmente composto de arroz, feijão, prato principal (carne (?)/ovos), acompanhamento (legumes ou verduras refogadas, cremes, molhos), salada, sobremesa, pãozinho e suco, além de temperos genéricos (não pergunte do que eles são feitos, jamais). Se você é vegetariano (meus pêsames), o acompanhamento nunca (ou quase nunca) contém animais mortos.

Não se esqueça de levar a sua caneca do Kit-bixo se for comer na Física – que foi reaberto em 2011, finalmente um sinal de sorte, hein? -, no Central ou no PCO, para economizar copos descartáveis e derrubar menos árvores na Amazônia. Na verdade você pode usá-la nos outros Bandejões, mas não é garantido que ela fique cheia. 

PS: O Efeito Bandex é proporcional à quantidade de salitre utilizado em cada bandejão.\\
PS 2: Playstation 2.\\
PS 3: Há evidências de que o suco de colorido servido na química é o resultado de algumas
experiências que não deram certo.\\
PS 4: Nunca, em hipótese alguma, jamais, visite a cozinha do seu bandejão de preferência,
pois você corre o risco de nunca mais almoçar na vida. Como já dizia o velho sábio ``A
ignorância é uma virtude''.\\
PS 5: O suco do PCO é amarelo ao 12h00 e vai virando ocre a medida que o tempo passa,
seguindo alguma função caótica aleatória desconhecida.\\
PS 6: o Restaurante da Física, como comentado, esteve fechado por meses, tal que os bixos 2011 – seus atuais VETERANOS – só o conhecerão em 2012, já que este restaurante foi aberto em dezembro de 2011; sinta-se um mínimo feliz por isto, bixo, mas ainda torça para que ele esteja maior agora do que fora antes do fechamento. Caso contrário, você definirá o Caos. \\
Consulte a tabela abaixo para decidir em qual dos bandejões você vai comer.

% Quadro bandejao
\figura{bandejao}

\end{subsecao}

\end{secao}

\pagebreak

% Um Pouco Sobre o DCE -------------------------------------------------------
\begin{secao}{Um Pouco Sobre o DCE}


O Diretório Central dos Estudantes (DCE) é a entidade geral de representação
dos estudantes da USP. É o ``pai'' de todos os centros acadêmicos.

Diretório não tem nada a ver com computador. Quer dizer, ter tem, mas o
diretório do DCE não tem. Ai... está muito complicado. Bom, saiba que o DCE
nasceu há quase 30 anos como resultado de uma greve dos estudantes em
razão da morte do professor da USP e jornalista da TV Cultura Wladimir Herzog
(Wlado), considerado um perigoso subversivo (sinônimo de comunista), que fora
levado a prestar depoimento em 1975, na sede do DOI-Codi, órgão responsável
pelo controle de ordem interna (que freqüentemente extravasava seu próprio
controle), e de lá saiu morto.

A nossa sociedade permanecia em silêncio, assustada e constrangida desde a
decretação do AI-5 (Ato Institucional número 5, pelo qual a ditadura criava
formas de repressão social...isso não caiu na FUVEST não?), até o assassinato
de Wlado.

Como dizíamos, essa greve começou na ECA e logo se expandiu por todo o  campus,
paralisando a Física e a FFLCH (Faculdade de Filosofia, Letras e Ciências
Humanas) e fazendo o Movimento Estudantil renascer após a desestruturação que
sofrera durante os, assim chamados, anos de chumbo.

Em 76, todas as entidades estudantis eram consideradas ilegais, menos aquelas
criadas pela lei Suplicy de Lacerda, que permitia a existência de DA's
(Diretórios Acadêmicos) e DCE's atrelados às diretorias das escolas.
Significava que os estudantes podiam ter suas representações estudantis desde
que estas passassem pelo crivo das autoridades de plantão: Reitores e
Diretores nas escolas; Ministros e Secretários de Educação que, por sua vez,
escolhiam os Reitores e Diretores no poder executivo.

O nosso DCE teve uma história diferente: ele nasceu livre; isso significa dizer
que nenhuma de suas diretorias, nesses 20 e tantos anos, teve o aval de outros
que não fossem os estudantes da USP. O que não significa que as diretorias do DCE
tenham sido todas maravilhosas e ativas na vida dos estudantes. 
Mas voltando, o fato do diretório ser livre teve tão grande relevância na época
que a palavra LIVRE foi incorporada ao nome do DCE - Livre da USP (como hoje é
conhecido). Além de livre, o DCE recebeu o nome de um jovem: ``Alexandre Vannucchi Leme'',
que foi um estudante da USP da Geologia. Ele foi morto em 1973, nas mesmas
circunstâncias de Wlado. Por todas suas lutas e resistência, parabéns,
DCE - Livre da USP!

\end{secao}
\pagebreak

\begin{secao}{Hospital Universitário}
   \begin{quote}\emph{O HU USP é o hospital de ensino de excelência utilizado pelos
Cursos de Atenção à Saúde da USP.  O hospital privilegia as pesquisas relacionadas
aos problemas de saúde  mais comuns da população brasileira.O atendimento é regionalizado 
para o bairro do Butantã, sempre com enfoque no ensino e pesquisa''}- Página do HU 
   \end{quote}

 O que isso quer dizer? Os estudantes de Medicina, Ciências Farmacêuticas, Odontologia, Saúde Pública,da Escola de Enfermagem e do Instituto de Psicologia, mantendo contato direto também com o Instituto de Ciências Biomédicas, de Biologia, de Química, Faculdade de Arquitetura e Urbanismo, Escola Politécnica, Escola de Comunicações e Artes, precisam de cobaias para suas atividades/experiências. O HU é um santo lugar onde recebe alguns fracos de espírito que bebem demais e ficam incapacitados de fazer qualquer atividade fisiológica. Você pode levar um comprovante de endereço de qualquer lugar e sua carteirinha USP para fazer o cartão do hospital. Assim você possui alguns privilégios no atendimento, em situações de emergência você é atendido rapidamente (algo entre 600 minutos, como diz na senha de espera) e não precisa  soletrar o nome da sua mãe enquanto estiver desmaiado. 
                      
\end{secao}


\quadrinhos5

% Tudo Que Vai Volta --------------------------------------------------------
\begin{secao}{Tudo Que Vai Volta (até bixo)}

\begin{subsecao}{Ônibus}

Se você é um bixo que não tem como ir nem como voltar, temos algumas dicas:

\begin{enumerate}
  \item Trabalhe muito para comprar um carro,
  trabalhe mais para pagar a gasolina e
  venha para a USP de carro e, obrigatoriamente, dê carona a um VETERANO;

  \item Peça a uma pessoa amiga para trazê-lo e buscá-lo durante seus
  longos anos de IME;

  \item Conheça alguém que, por sorte, mora perto da sua casa, estuda na USP,
  tenha o mesmo horário que você, seja legal e tenha carro. Traduzindo, s-o-n-h-e;

  \item Estique o dedão e espere, espere, espere, espere... a boa vontade
  de alguém para te dar carona;

  \item Mude-se para uma casa perto da USP;

  \item Desista do curso e diga: ``Eu não queria mesmo!!'';

  \item Se nenhuma das alternativas anteriores foi satisfeita, então
  transforme-se numa pessoa normal e pegue ônibus. Abaixo vão as
  principais linhas que passam pela USP.
\end{enumerate}

Eis as principais linhas que entram na USP. Se você não achou um local perto procure outras linhas em algum guia municipal. As fontes usadas foram: o site da USP, o da SPTrans e "o boca-a-boca”, já que como algumas coisas não constavam nos sites ou estavam desatualizadas tivemos que perguntar mesmo. 

{\bf Linhas municipais:}

Essas linhas tem como seu ponto final (com algumas exceções) a Portaria 2,
também conhecida como portaria da POLI (blerg..). Os itinerários não estão completos,
apenas colocamos as ruas/avenidas mais importantes. Para mais detalhes entre no site
da sptrans {\tt www.sptrans.com.br}.
Ei-las:


\begin{itemize}
  \item {\bf 107T - Cidade Universiária / Metrô Tucuruvi}\\
    Cor: Azul-escuro.\\
    Onde pegar para sair da USP: ponto da FAU.\\
    Itinerário: Av Tucuruvi, Av Nova Cantareira, Pr. Orlando Silva, Av Cruzeiro do Sul,
	Av Tiradentes, R. Augusta, Av Cidade Jardim, Av Valdemar Ferreira, Rua dos Bancos.
  
  \item {\bf 701U - Butantã-USP / Jaçanã}\\
    Cor: Azul-escuro.\\
    Onde pegar para sair da USP: ponto da FEA.\\
    Itinerário: Av Tucuruvi, Av Nova Cantareira, Av Cruzeiro do Sul, R. Voluntários da
    Pátria, Av Tiradentes, Av Ipiranga, R. Consolação, Av Dr. Arnaldo, R.
    Cardeal Arcoverde*, Av Eusébio Matoso, Av Valdemar Ferreira**, Rua do HU.

  \item {\bf 177H  - Butantã-USP / Casa Verde (Cuidado!!! Não confundir com a próxima)}\\
    Cor: Azul-escuro.\\
    Onde pegar para sair da USP: ponto da FAU.\\
    Itinerário: Rua Jaguarete, Av Brás Leme, Viad Pacaembu, Av Angélica, Av Dr. Arnaldo,
	R. Cardeal Arcoverde*, Av Valdemar Ferreira**, Rua dos Bancos.

  \item {\bf 177P - Butantã-USP / Casa Verde}\\
    Cor: Azul-escuro.\\
    Onde pegar para sair da USP: ponto da FAU.\\
    Itinerário: Rua Jaguaret, Av Brás Leme, Av Pacaembu, R. Cardoso de Almeida, Av
    Dr. Arnaldo, R. Cardeal Arcoverde*, Av Valdemar Ferreira**, Rua dos Bancos
    
  \item {\bf 637G - Butantã / Grajaú (até a Av. Afrânio Peixoto)}\\
    Cor: Azul-claro.\\
    Onde pegar para sair da USP: na Av. Afrânio Peixoto.\\
    Itinerário: Av Sen. Teotônio Vilela, Av Interlagos, Av Robert Kennedy, Lgo. Socorro, Av
    Adolfo Pinheiro, Av Sto. Amaro, R. Joaquim Floriano, Av Faria Lima, R.
    Sumidouro, Av Valdemar Ferreira, Pr. Vicente Rodrigues, Av Afrânio Peixoto

  \item {\bf 702P - Butantã (até a Portaria 1) / Term. Pq. D. Pedro II (não confundir com
      o próximo)}\\
    Cor: Amarela.\\
    Onde pegar para sair da USP: no ponto da portaria P1.\\
    Itinerário: R. Augusta, R. Colômbia, Av Europa, Av Cidade Jardim, Av Lineu de Paula
    Machado Av Valdemar Ferreira, Pr. Vicente Rodrigues (P1).

  \item {\bf 702U - Butantã-USP / Term. Pq. D.Pedro II}\\
    Cor: Laranja.\\
    Onde pegar para sair da USP: ponto da FEA.\\
    Itinerário: Viad Vinte e cinco de Março, Av Ipiranga, Pr. República, R. Consolação, Av
    Rebouças, Av Eusébio Matoso, Av Valdemar Ferreira**, Rua do HU.

  \item {\bf 7181 - Cidade Universitária / Term. PRINC. Isabel}\\
    Cor: Laranja.\\
    Onde pegar para sair da USP: ponto da FAU.\\
    Itinerário: Av Rio Branco, R. Aurora, Pça República, Av S. Luís, R. Augusta, Av Cidade
    Jardim, Av Lineu de Paula Machado, Av Valdemar Ferreira, Rua dos Bancos

  \item {\bf 7411 - Cidade Universitária / Pça. Da Sé}\\
    Cor: Laranja.\\
    Onde pegar para sair da USP: ponto da FAU.\\
    Itinerário: Pr. João Mendes, Lgo. S. Bento (volta Lgo. S. Francisco) , R. Líbero Badaró,
    Viad do chá, Pr. Ramos de Azevedo, Lgo. Paissandu, Pr. República, Av
    Ipiranga, R. Consolação, Av Rebouças, Av Vital Brasil, Rua dos Bancos.

  \item {\bf 7725 - Metrô V. Madalena / Rio Pequeno (não confundir com o próximo)}\\
    Cor: Laranja.\\
    Onde pegar para sair da USP: Se for sentido R. Pequeno, então ponto da FAU,
    senão ponto da FEA.\\
    Itinerário: Av Gustavo Berthier, Av Rio Pequeno, Av Corifeu de Azevedo Marques, Av
    Escola Politécnica, R. Mello Moraes, Av Lúcio Martins Rodrigues, Rua dos Bancos,
    Rua do HU, Pte. Cidade Universitária, Pr. Arcipreste
    Anselmo de Oliveira, Av Manoel José Chaves, Pr. Panamericana, R. Heitor
    Penteado

  \item {\bf 7725 - Metrô V. Madalena / Expresso-USP (não confundir com o próximo)}\\
    Cor: Laranja.\\
    Onde pegar para sair da USP: Ponto da FEA.\\
    Itinerário:  Rua dos Bancos, Rua do HU, Pte. Cidade Universitária, Pr. Arcipreste Anselmo de Oliveira, Av Manoel José Chaves, Pr. Panamericana, R. Heitor Penteado

  \item {\bf  724A - Aclimação / Cidade Universitária}\\
    Cor: Laranja\\
    Onde pegar pra sair da USP: Ponto da FEA.\\
    Itinerário:  Lgo. Nsa. Sra. da Conceição, R. Cons. Furtado, Av. da Aclimação, 
    R. Topázio, R. Dr. Nicolau de Souza Queiroz, Av. Bernardino de Campos,
    Av. Paulista, R. Card. Arcoverde*, R. Sumidouro, R. Eugênio de Medeiros,
    Av. Vital Brasil, Av. Afrânio Peixoto, P1, Rua dos Bancos.

  \item {\bf 7702 - Terminal Lapa / USP}
    Cor: Laranja\\
    Onde pegar pra sair da USP: Rua do Matão\\
    Itinerário:  R. do Matão, R. dos HU, Av. Afrânio Peixoto, Av Valentim Gentil,	 
    R. Magalhães de Castro, Pte. da Cidade Universitária,Av. Prof. Manuel José Chaves,
    Pr. Panamericana, Av. S. Gualter, R. Bairi, R. Pio XI, R. Tito, R. Francisco Alves,
	R. Jeroaquara, R. Scipiao, Term. Lapa,	 

\end{itemize}
* volta R. Teodoro de Sampaio\\
** volta Av Vital Brasil
 

Dica: Aqui na USP a maioria dos ônibus podem ser dividos em dois tipos: os que
passam antes pela Rua do HU (Av. Prof Lineu Prestes) e os
que passam antes na Rua dos Bancos (Av. Prof. Luciano Gualberto).


{\bf Linhas intermunicipais:}

Agora, se você mora mais longe ainda (outra cidade, outro estado, outro país...) e não quer ou não pode se mudar para São Paulo, existem algumas linhas de ônibus fretados para cidades mais próximas (ou não). Se por acaso a sua cidade não está aí, procure se informar a respeito, pois não significa necessariamente que não haja ônibus da USP para lá. Aí estão elas:

\begin{itemize}
  \item {\bf Empresa Urubupungá.}\\
    Tel: 3658-7777
    Site: {\tt www.urubupunga.com.br}\\
    280BI1- São Bernardo do Campo (Centro)\\
    Cor: Cinza\\
    Onde pegar para sair da USP: ponto da FAU\\
    Av Magalhães de Castro, Av Marginal Pinheiros (Shopping Eldorado), Av Dos
    Bandeirantes, Av Eng. Luiz Carlos Berrini, Av Roque Petroni Jr. (Shopping
    Morumbi), Av Prof Vicente Rão, Av Cupecê (Diadema), Av Fábio Eduardo Ramos
    Esquivel (Diadema), Av Piraporinha (Diadema), Av Lucas Nogueira Garcez
    (Diadema), Av Urubupungá.

  \item {\bf Fretados Jundiaí - USP}\\
    Viação MIMO\\
    Tel: 4522-7788\\
    {\tt www.viacaomimo.com.br}\\
    Principais horários:\\
    Ida: 6h20; 7h20; 12h50; 18h00 (na Rodoviaria de Jundiaí)\\
    Volta: 11h50; 17h20; 23h00 (No ponto da FEA)

  \item {\bf São José dos Campos}\\
    Redenção\\
    tel: (12) 3931-3047\\
    {\tt www.redencaoturismo.com.br}\\
    Ida: 06h15 (na gruta em S. José)\\
    Volta:17h40 (no ponto da p2)

  \item {\bf Bragança}\\
    N.S. Fátima\\
    {\tt www.saexbra.com.br}\\
    Tel: 4032-4723 e 7344-2007\\
    Ida: 05h50 (Lgo do Tabão/Habbibs)\\
    Volta:17h00 (Av prof almeida Prado)

  \item {\bf Campinas}\\
    Sta. Cruz\\
    Tel:3868-5995\\
    {\tt www.gruposantacruz.com.br}

  \item {\bf Santos}\\
    Náutica Turismo\\
    Tel: (13) 9112-8860

  \item {\bf Santos / S. Vicente}\\
    Transul\\
    Tel: 6954-4466

  \item {\bf ABC}\\
    Dinâmica ABC-USP\\
    4352-0565 / 4109-0172

  \item {\bf Sorocaba}\\
    Fretado Diurno\\
    (15) 9715-1676 (Márcio)

  \item {\bf Van (diurno e noturno)}\\
    RR transportes\\
    6919-3345 / 7144-3934\\
    4474-1222 \\

\end{itemize}

\end{subsecao}

\begin{subsecao}{Circular}

Também conhecido como “circulenda” ou “secular” (aos sábados, “milenar”, e aos domingos, “anos-luz”), é o meio de transporte mais barato dentro da USP. Foi criado para os USPianos se locomoverem dentro do Campus, mas em muitas vezes é melhor andar do que ficar esperando. Existem 2 itinerários distintos, com trajetos aproximadamente reversos. Fique atento para não dar uma de bixo burro (duh!) e se perder, hein? 

Obs.: há controvérsias incontáveis, mas há circulares aos fins-de-semana, bixo dedicado. Surge uma terceira linha chamada “Museus”, e fica a cargo do leitor adivinhar por onde ela passa. De qualquer forma, é um itinerário alterado com notável frequência... No último modelo conhecido, os pontos mais próximos para vir para o IME, seriam: para quem vem do P3, o Acesso Vila Indiana, e então se deve descer TODA a Rua do Estupro por vezes chamada de Rua do Matão, e para quem vem do P1, o ponto da FEA. Vale comentar que, aos sábados, esses ônibus passam a cada uma hora e meia, e, aos domingos, existe uma média de 2 circulares passando num ponto, com margem de erro igual a 3. 


\begin{itemize}
  \item {\bf Circular 1}\\
    P3/ Acesso pedestre Vila Indiana/ Biociências/ Filosofia/
    FAU(próximo ao IME)/ IAG(Bandejão da Física)/ Clube dos Funcionários/
    Poli-Civil/ Poli-Metalurgia (Term. De Ônibus e P2)/ Poli-Mecânica/ Pça do
    Relógio/ CRUSP/ acesso de pedestre FEPASA(para pegar o trêm)/ Educação
    Física (P1)/ Educação/ Reitoria (Bandejão Central e Coseas)/ Pça dos
    Reitores/ Antiga Reitoria (Bancos)/ FEA(próximo ao IME)/ Poli-Biênio/
    Prefeitura(Bandejão)/ MAE(acesso de pedestres Rio Pequeno)/ HU/ Biomédicas
    III/ Odontologia/ P3.

  \item {\bf Circular 2}\\
    P3/ Ipen/ Copesp(HU)/ acesso pedestres Rio Pequeno/
    Prefeitura(Bandejão)/ Física(Bandejão)/ Oceanográfico/ Biociências/ CEPAM
    (Bandejão da Química)/ Butantã/ Cultura Japonesa/ Paço das artes/
    P1/ Educação Física/ Acesso pedestres FEPASA(para pegar o trêm)/ Raia
    Olímpica/ Poratria 2/ Terminal de ônibus (mais próximo da P2)/ IPT/
    Eletrotécnica/ FAU (próximo ao IME)/ Geociências(Bancos)/ Letras(mais
    próximo do Bandejão Central)/ História e Geografia/ Fármácia e Química
    (Bandejão da Química)/ Rua do Lago/ Biomédicas/ P3.
\end{itemize}
\end{subsecao}

\begin{subsecao}{Veículos no Campus}
Saiba por onde entrar na USP. Lembre-se de ter sempre a sua carteirinha USP ou
seu comprovante de matrícula com RG em mãos. 
\begin{itemize}
  \item {\bf Portaria 1 (P1):} R. Afrânio Peixoto. Funciona 24h por dia todos os
    dias, mas a entrada é controlada de segunda à sexta das 20h às 05h, aos sábados
    após as 14h e domingo o dia inteiro. É por onde entram os ônibus municipais. 
    
  \item {\bf Portaria 2 (P2):} Av. Escola Politécnica. Funciona das 5h30 às 20h
    de segunda a sexta. Entrada controlada de seg a sex das 20h às 24h. Fechada
    aos sábados, domingos e feriados. Única entrada para caminhões. 
    
  \item {\bf Portaria 3 (P3):} Av. Corifeu de Azevedo Marques. Tem o mesmo horário
    de funcionamento da P2.

  \item {\bf Portaria 1/2:} R. Eng. Teixeira Soares. Funciona de segunda a sexta das 5h30 às 20h. 
    
  \item {\bf Portarias de pedestre (Mercadinho, São Remo, HU, FEPASA e
      Vila Indiana):} Funcionam de 2ª a 6ª, das 05 às 23 hs.

\end{itemize}

Aos fins de semana, ficam abertas apenas as portarias 1 e 3 (esta, fecha mais cedo). É preciso se identificar.

Os ônibus entram no sábado até as 14h e não entram no domingo. Se você vem de carro, saiba que a universidade dispões de bolsões de estacionamento
gratuito em torno das Unidades.

\end{subsecao}

\begin{subsecao}{Pontos de taxi}
Existem alguns pontos de taxi espalhados pela Cidade Universitária. Eis suas
localidades:

\begin{itemize}
\item Ponto da FEA / ECA (atrás do Banespa)\\
Fone: 3091-4488

\item Ponto da Reitoria\\
Fone: 3091-3556

\item Ponto do Hospital Universitário\\
Fone: 3091-3536
\end{itemize}
\end{subsecao}


\end{secao}


% Dicas -----------------------------------------------
\begin{secao}{Dicas}

Como todo bixo é perdido, aqui vão algumas dicas pra você não ficar perguntando
o tempo todo:

{\bf Biblioteca:} Para fazer a inscrição na Biblioteca, você precisa apresentar
2 fotos 3x4, comprovante de residência e o cartão de matrícula. Não deixe de assistir à palestra da Biblioteca
na primeira semana de aula! Consulte a programação.

{\bf Correios:} tem uma agência na Antiga Reitoria (que agora virou nova), perto do Banco do Brasil.

{\bf Bancos e Caixas Eletrônicos:} Agências do Santander Banespa, Bradesco, Nossa Caixa, 
Banco do Brasil, HSBC, Itau e Banco Real na Av. Prof. Luciano Gualberto; Caixas eletrônicos
perto do bandejão da Física, no bandejão da Química, em frente à reitoria, no CEPE, enfim,
em lugares aleatórios.

{\bf Farmácia:} Ao lado do bandejão Central, a farmácia funciona de segunda a sexta, das 9h às 17h e no HU.

{\bf Posto de Gasolina:} na Avenida da Universidade, em frente à Faculdade de Educação.

\begin{subsecao}{Museus pertencentes a USP}

{\bf Museu de Arqueologia e Etnologia (MAE):} ao lado da Prefeitura do Campus.

{\bf Museu de Arte Contemporânea (MAC):} próximo ao CRUSP.

{\bf Paço das Artes:} em frente à Faculdade de Educação.

{\bf Museu do Brinquedo:} fica na Faculdade de Educação, Bloco B.

{\bf Museu do Crime:} na Academia de Polícia perto da P1.

{\bf Museu do Instituto Oceanográfico:} adivinha?

{\bf Museu da Geociências:} Lá mesmo.

{\bf Instituto Butantan:} Próximo à História.

{\bf Museu da Faculdade de Veterinária} Perto do P3. Ponto final do circular.

{\bf Estação Ciência:} fica fora da USP. R. Guaicurus, 1274 - Lapa

{\bf Museu Paulista, vulgo Ipiranga: }também fora da USP, no Parque da Independência - S/N  - Ipiranga.

{\bf Museu de Zoologia: }também fora da USP, na Av. Nazaré, 481  - Ipiranga (atrás do Ipiranga).

\end{subsecao}

\begin{subsecao}{Onde beber?}

Se você é um bixo que curte entornar os canecos de vez em quando, então agora deve estar pensando, até que enfim vamos falar de algo que presta! Lembre-se de levar um Veterano para pagar a ele algumas doses, pois graças a eles que você está recebendo essas dicas, então sem mais delongas, aqui vão alguns lugares firmeza para se fazer isso à vontade: 

{\bf Vet:} Na verdade não fica na veterinária que todos conhecem, é o C.A. da Vet, um lugar paradisíaco, com breja razoavelmente barata, que ficou mais famoso por volta de 2009, em vista das restrições alcoólicas que houveram na USP. Seu local não é revelado por escrito, só no boca-a-boca, por isso vá com um VETERANO experiente, principalmente para pagar cerveja para ele... É, essa parte do guia foi preparado para isso, afinal não passamos meses das nossas vidas fazendo esse guia para não receber nada de volta né???

{\bf Física:} Embora seja a Física, lá é um lugar gostoso para tomar vários tipos de cerveja, que só é vendida após as 18:00 horas, e comer alguns salgados baratos. No próprio C.A. deles se vende cerveja. 

{\bf ECA:} Famosa Quinta i Breja, adivinha que dia da semana isso acontece? Acertou bixo, toda Quinta-feira, parabéns, como prêmio você pode pagar uma breja para o seu veterano favorito!

{\bf Rei das batidas:} Muito famoso não só por quem estuda na USP, o Rei, como
é carinhosamente chamado, fica fora da USP, saindo pela P1. Vende diversas
batidas e, é claro, cerveja...

{\bf Bar do frango:} Não se sabe qual é o verdadeiro nome desse bar, mas ele é uma alternativa ao Rei, quando este se encontra muito lotado. Apesar do péssimo atendimento e do aspecto horrível do lugar é bom para beber sem tumulto. É frequentado principalmente no começo do ano. Se encontra atrás do Rei.

{\bf Morro da coruja:} Um morro perto da física, onde geralmente as pessoas se reúnem para tomar as mais diversas bebidas e misturas alcoólicas possíveis. Um ótimo lugar quando não está chovendo.

É claro que também há as festas, que ocorrem em qualquer lugar da USP, e nelas há ainda outras misturas alcoólicas impossíveis. 

É nosso dever informar também que o álcool é uma substância altamente viciante e, quando bebida em excesso, pode trazer graves consequências à sua saúde, às vezes à saúde de outra pessoa, à sua família e principalmente ao seu bolso. Portanto, não se esqueça de abastecer os seus VETERANOS. 

\end{subsecao}

\end{secao}
% Melodias para a bixarada -------------------------------------------------
\begin{secao}{Músicas para a Bixarada}

%\quadrinhos6


\begin{subsecao}{Grito de Guerra da Matemática}

Arakam Baram Bakam / Tumberê tumberá / Macambê mecambecá \\
Rico reco rico rá / Rá rá rá / Matemá matemá Matemá-ti-ca!
\end{subsecao}

\begin{subsecao}{Musiquinha da Poli}

{\em cantar como ``Ele é um bom companheiro''}

Escola de Viadinho \\
Escola de bunda-mole \\
Se é verdade que o mundo tem cu \\
O cu do mundo é a poli  (3x)
\end{subsecao}

\begin{subsecao}{Caboclo da MAT}

{\em cantar como ``Faroeste Caboclo'' da Legião Urbana}
{\em Essa música é em memória a FUVEST, quando Matemática Aplicada e POLI petenciam a mesma carreira}
\begin{verse}
Cheio de medo em setembro Joãozinho viu que seus dedos tremiam pra fazer a
inscrição

Deixou pra trás a namorada, a motoca, o futebol e as festinhas pra rachar na
revisão

Quando criança só pensava em ser engenheiro ainda mais com o dinheiro que
sonhava em ter na mão

Era o CD lá do colégio onde estudava e todo mundo admirava o boletim desse
cuzão

Ia pra igreja só pra rezar pro seu santo pra pedir a sua ajuda pra prestar
vestibular

Sabia mesmo que ia ser barra pesada porque tinha muito japa pra tomar o seu
lugar

O ano todo se propôs a estudar, passava o dia sem ligar a televisão

Nos feriados não ia viajar, ficava em casa treinando redação

Fazia todos os exercícios da apostila e no fim de cada aula ia falar com o
professor

Às quinze horas ia pro laboratório ver as mitocôndrias da aula anterior

Não entendia como o militarismo dominou nosso país por vinte anos de terror

Ficou cansado de tentar achar resposta e desceu pra lanchonete pra afogar a sua
dor

E lá chegando foi tomar um cafezinho e encontrou um concorrente com quem foi
falar

E o concorrente aumentou seu desespero pois manjava muita coisa que ele tinha
que estudar

Dizia ele, eu vou prestar o ITA... Nesse país prova pior não há

E se não der eu vou pegar engenharia, lá na POLI eu vou tomar o seu lugar

E João não gostou dessa proposta, ele disse ``ai que bosta, eu tô passando mal''

Ele ficou bestificado com a idéia de pegar lista de espera só depois do
carnaval

Meu Deus, é pior ainda, no ano novo eu posso estar lá na Mauá

É brincadeira querer ser engenheiro e só descolar emprego em Taguatinga

Na sexta-feira ele morria de vontade de correr pro banheiro se borrando de
pavor

E conhecia muita gente arrogante que passava do seu lado se dizendo um terror

Ele estudava o relevo da Bolívia, função quadrática e modular

E nos domingos então ele fazia tarefa mínima e complementar

E Joãozinho até a morte se esforçava e o tempo mal sobrava pr'ele se alimentar

E via às duas horas o Vestibulando que passava todas as dicas sobre o
vestibular

Mas ele não queria mais conversa e decidiu que em novembro era hora de rachar

Ele pirou que precisava estudar tanto, virou um bitolado e começou a delirar

E logo, logo os malucos da sua idade viram a calamidade, tem babaca novo aí

E o nosso Joãozinho ficou louco e bateu em todos os japoneses dali

Seus amigos preocupados com a sua sorte deram uma fita de rock pr'ele relaxar

Mas de repente sob uma má influência dos boyzinhos lá do fundo começou a zoar

Já na primeira fase ele penou e só passou porque o corte foi sessenta e três

A demência tomou a sua mente : ``Vocês vão ver, eu vou pegar vocês!!!''

Agora Joãzinho era fodido e estava decidido que não ia se dar mal

Sacava toda a trigonometria e manjava de limites, derivada e integral

Foi quando conheceu uma menina e de toda aquela zona ele se arrependeu

Maria Lúcia era uma bitola linda e o coração dele pra ela o Joãzinho prometeu

Ele dizia que devia estudar, pois engenheiro ele queria ser

Maria Lúcia, pra sempre vou te amar, Engenharia com você quero fazer

O tempo passa e um dia chega a hora de fazer segunda fase coitadinho do João

E ele faz uma prova perigosa diz que espera uma resposta, pode ser um sim ou
não

Não vou correndo pra banca de jornal nem pra pátio do cursinho isso eu não faço
não

Pois eu prefiro ficar na minha casa esperando o resultado com o cu na mão

Maria Lúcia vai comprar o tal jornal e logo após achar seu nome ela procura o
de João

Mas ela volta com tristeza no olhar, olha pra ele e diz ``você pegou a quarta
opção''

Você passou na sua quarta opção, você passou na sua quarta opção

Bacharelado em Matemática é um tesão, eu vou sofrer as conseqüências como um
cão

Não é que Joãozinho estava certo, seu futuro era incerto mas foi se matricular

Matriculou-se e no meio da zoeira descobriu que tinha muitos como ele no lugar

Fez inscrição pro remanejamento e talvez no fim do ano transferência ia tentar

E Joãozinho mantinha a esperança de um dia ir pra Poli estudar química

Mas acontece que um tal Professorzzini terrorista de renome apareceu por lá

Ficou sabendo dos planos de Joãozinho e decidiu que com suas notas ele ia se
ferrar

E ele teve que largar cálculo dois mesmo sabendo derivar e integrar

E decidiu deixar estat pra depois que o Moretin voltasse a lecionar

Professorzzini, professor mais sem vergonha com sua prova enfadonha fez todo mundo
dançar

Desvirginava bixetes inocentes e o nabo era tão quente que nem dava pra sentar

E Joãozinho há muito não via sua amada, e a saudade começou a apertar

Eu vou pra Poli eu vou ver Maria Lúcia, já está em tempo de a gente se
encontrar

Chegando à Poli então ele chorou quando viu Maria Lúcia namorando um japonês

Oh, Maria Lúcia, quanto que você mudou, que estrago que a Poli te fez

Joãozinho era só ódio por dentro e então o japonês para um duelo ele chamou

Amanhã às duas horas no biênio, ou na praça do relógio, seja lá onde for

E você pode escolher as suas armas: derivadas ou matrizes de qualquer versor

Que eu provo que o sub-espaço nulo é o coração dessa piranha a quem jurei o meu
amor

E Joãozinho não sabia o que fazer quando escutou um papo lá no bandejão

Onde falavam dum duelo que iam ver dizendo a hora, o local e a razão

No sábado então às duas horas toda a Poli sem demora foi lá só pra assistir

Um japa que botava pelas costas, encoxou Maria Lúcia e começou a sorrir

Sentindo um ódio na garganta João olhou pros cabacinhos e pros trouxas a
aplaudir

E olhou pros pipoqueiros e as bancas de cachorro-quente que passavam por ali

E se lembrou de quando era uma criança e de tudo que vivera até ali

E decidiu entrar de vez naquela dança, se a Poli é um circo, e daí

E nisso o céu abriu seus olhos e então Maria Lúcia ele reconheceu

Ela queria fazer Álgebra dois pra provar que a Poli não a emburreceu

Politécnico, eu sou homem coisa que você não é, e não me contento em por nas
costas não

Some daqui filha da puta sem vergonha vai pra casa tocar bronha o seu destino é
ser bundão

E Joãozinho deu as costas para os dois, foi pra pura onde encontrou o seu valor

Maria Lúcia se arrependeu depois prestou Fuvest mas no IME não entrou

E a todos declarava que o nosso Joãozinho era gênio que escapou de se foder

Que na alta burguesia lá da Poli todo mundo é bunda mole ninguém sabe o que
fazer

E foi dar monitoria no cursinho pra avisar aos molequinhos pra não esquecer

Ele queria era avisar toda essa gente engenharia é pra demente que só quer
sofrer.
\end{verse}
\end{subsecao}

\begin{subsecao}{Musiquinha da Poli 2}

{\em ``A Casa'' de Vinícius de Moraes}

Eu sou viadinho / Eu sou da POLI \\
Meu cu é de ferro / Meu pinto é mole
\end{subsecao}

\begin{subsecao}{Explode Coração}

{\em cantar como ``Explode Coração'' do Salgueiro}

Explode coração / Na maior felicidade / No IME há tanto tempo \\
Sou campeão mas não termino a faculdade
\end{subsecao}

\begin{subsecao}{USP Maravilhosa}

{\em cantar como ``Cidade Maravilhosa''}

Ó USP Maravilhosa / Cheia de encantos mil \\
Ó USP Maravilhosa / Melhor escola do Brasil

Essa é escola que todos querem / Mas poucos conseguem entrar \\
Você que tentou / E não conseguiu \\
Que vá pra puta que pariu!

Ó USP Maravilhosa / Cheia de encantos mil \\
Ó USP Maravilhosa / Melhor escola do Brasil
\end{subsecao}

\begin{subsecao}{Como É Bom Estar no IME}

{\em cantar como ``Que Bonita sua Roupa''}

Como é bom estar no IME \\
Muito bom mas não se anime \\
Já tivemos aula de cálculo \\
Derivada até que é fácil \\
O difícil é integral

Nós somos imeanos \\
Adoramos os queridos VETERANOS \\
Do IME nós gostamos \\
Por aqui nós ficaremos muitos anos
 
Como é bom estar no IME....

Que porcaria é a poli \\
Viadinhos que não gostam de mulher \\
Bando de bundas-moles \\
A menina mais bonita tem bigode

Como é bom estar no IME....
\\
\\
\end{subsecao}
\end{secao}

%\figura{xkcd}

% Utilidades -------------------------------------------------------------
\begin{secao}{(in)Utilidades}

\begin{subsecao}{Na WEB}

{\tt www.usp.br} - Página da USP. Aqui você encontrará notícias e eventos da
universidade, bem como informações gerais.

{\tt www.ime.usp.br} - Página do IME.
Nesse link, você poderá ver detalhes sobre osl
cursos e obter informações sobre a faculdade.

{\tt www.linux.ime.usp.br} - Página da Rede Linux do IME.
A partir dessa página, você
poderá aprender a explorar alguns dos recursos existentes na rede Linux,
através do FAQ, além de poder acessar seus e-mails da rede Linux pelo Webmail.

{\tt camat.ime.usp.br} - Página do CAMAT.
O site foi remodelado no começo de 2012, não deixem de dar uma olhada. Lá você ficará por dentro não só de eventos, debates e festas, mas também terá acesso a um banco de provas de anos anteriores, que os seus VETERANOS gentil e coletivamente construíram.

{\tt www.ime.usp.br/$\sim$atletica} - Página da Atlética.
Quer acompanhar os jogos do IME
nos campeonatos? Então visite o nosso site e fique ligado em tudo o que está
rolando (literalmente), além de se informar sobre as modalidades que existem no
IME.

{\tt www.sistemas.usp.br/jupiterweb} - Sistema JúpiterWeb.
Aqui você vai encontrar a
sua grade horária e, mais tarde, você poderá fazer as matrículas nas matérias
que irá cursar no semestre, além de ter acesso às suas notas.

{\tt paca.ime.usp.br} - É bixo... acha que vai ser essa moleza pra sempre? Se você acha, está muito enganado! Daqui a pouco você vai receber uma senha para poder enviar seus EPs (vide glossário) nesse endereço... (e não adianta fazer chantagem emocional que o Paca só vai aceitar até 23h55... não entendeu? você vai entender..). É uma evolução do Panda no level 42 e pode ser usado como Orkut de vez em quando. 

{\tt www.parperfeito.com.br} - Tá precisando, hein, bixo!?

{\tt www.prosangue.com.br} - Veja como é o processo de doação de sangue. Haverá
uma campanha para doação de sangue nas primeiras semanas, fique ligado!

{\tt www.google.com.br} - Tudo o que você precisa tá no google. Eles ainda vão
dominar o mundo!

{\tt www.wikipedia.org} - uma enciclopédia livre, onde qualquer pessoa pode
copiar coisas, e acrescentar também.

{\tt www.xkcd.com} - webcomic sobre matemática e computação. Origem de muitas piadas que você ouvirá por aí.

\end{subsecao}


\begin{subsecao}{Telefones}
Colocamos essa seção aqui apenas para que os bixos não nos encham com perguntas
sobre passes e plantão:

{\bf COSEAS:} Seção de Passe Escolar: {\tt 3091-3581}

{\bf Seção de Alunos:} {\tt 3091-6149}

{\bf Telefonista da USP:} {\tt 3091-4313}

{\bf Plantão de Cálculo e Álgebra Linear (serviço gratuito):} {\tt 3037-1773}

\end{subsecao}
\end{secao}

% Glossário ---------------------------------------------------------------

\begin{secao}{Glossário}

Esta parte é, sem dúvida, uma das mais importantes do guia. Aqui você
encontrará todas as explicações para as maiores dúvidas do universo. Com
certeza, após ler este trecho sua vida vai mudar: você saberá, por exemplo,
porque o céu é azul e com quantos paus se faz uma canoa.


\begin{subsecao}{Sobre o IME}
{\bf Armários:} Ficam
na sala 18 do bloco B (sala da Vivência). Além de servirem para
aliviar imeanos do excesso de peso dos livros e cadernos, ajudam o CAMAT a
sobreviver financeiramente.

{\bf Banheiro:} Temos vários e de três tipos: para Homens, para Mulheres e os matinhos para os bixos. Recentemente foram reformados os três tipos, então você bixo, poderá usufruir de um matinho totalmente novo! 

{\bf Bichinho esquisito:} Aquele negócio que está nas camisetas, agasalhos e bonés do IME. Alguns dizem que é o mascote da Atlética; alguns o chamam de Fluffy; ninguém sabe dizer o que ele realmente é, mas apareceu depois que surgiram aquelas bolinhas, estilo porco-espinho, feitas com elásticos coloridos (você provavelmente deve lembrar; foi mais ou menos na época do seu primário). Primo do Cariboo.

{\bf CEC:} É a sigla do Centro de Ensino de Computação. É um dos lugares onde o bixo pode fazer (ou pelo menos tentar) seus EP's e onde são ministrados cursos de
``Computação Instrumental'' para alunos de toda USP (e até de fora).
 Ideal para os que não sabem nem dar login na Linux.

{\bf GRECIME:} É o grêmio dos funcionários. Portanto quando ouvir falar em grêmio não pense CAMat. Não tem nada a ver. Fica em frente às máquinas de salgadinhos (se elas voltarem; se não, fica ao lado da Gráfica). De vez em quando, até 16h ou 20h (varia numa função quadraticamente proporcional ao bom-humor do vendedor e ao próprio vendedor, que não é único), tem bolos maravilhosos de dar água na boca. Outros dias são biscoitos excelentes para se mastigar enquanto se estuda, ou salgados para almoçar quando você tem EP para entregar. Claro que se você que é do noturno e trabalha de dia pode pular esse tópico. 

{\bf IMEJr:} Empresa administrada pelos alunos do IME.

{\bf RD:} Representante Discente. É aquele aluno que representará você nas comissões do IME, Comissões de Curso, Comissão de Graduação, entre outras. Nestas comissões serão tomadas decisões que irão influenciar a sua vida acadêmica. Então a presença de um aluno nelas é imprescindível, pois nada melhor que um aluno para saber das necessidades dos alunos.

{\bf Sala das ET's:} É uma sala no bloco A onde estão as Work Stations (Estações de
Trabalho), usadas pelos professores, alunos de pós-graduação e de iniciação
científica.

{\bf Sala de Vivência:} É a sala 18 do bloco B, a sala mais importante de todas do IME e onde os alunos aproveitam para jogar cartas, sinuca (Formiga está sempre disposto a perder uma lá), pebolim (Cartola, Jerônimo, Adalberto - vocês ainda vão perder para eles), xadrez (Pedrosa), fliperama, Pokémon, Magic, ouvir música, assistir TV, dormir no sofá, e o que mais der na telha de quem quiser relaxar. Você encontrará maus elementos que irão lhe levar ao submundo das drogas, tentando lhe ensinar a jogar king ou bridge... seja careta, caia fora logo e assuma que você gosta de TRUCO. 

{\bf Seção de Alunos:} Aquele lugar onde você faz matrícula (aliás, você deve ir
confirmar a matrícula logo, bixo!), declaração, requerimentos, etc. (assuntos
burocráticos). Vá até o saguão à esquerda do Bloco B (lá onde fica o CEC) entre na fila.
O horário de funcionamento é meio restrito, então não deixe as coisas pra última hora.

{\bf Xerox:} Não ficou pronta até a edição final desse guia. Provavelmente a xerox será entre a Vivência e o Grecime, entrado no Bloco B à direita (aproveite essa informação, bixo, é de primeira mão).
 
\end{subsecao}

\begin{subsecao}{Sobre as matérias}

{\bf Teorema:} Um teorema é uma afirmação que pode ser provada. Provar teoremas é a principal atividade dos matemáticos. Deles surgem Lemas, Corolários, Proposições, e tantas outras coisas que você só vai entender completamente o que significam quando precisar escrever sobre eles, o que vai acontecer logo logo!

{\bf Iniciação Científica:} grupo de alunos, coordenado por um professor, que estuda
um determinado assunto, paralelamente ao curso. No caso de alunos que queiram bolsas
de estudo, é adotado um plano de estudos mais rigoroso.

{\bf SUB:} prova que você faz quando vai mal em alguma outra avaliação, ou quando você simplesmente não vai. Sua aplicação e utilidade depende do professor ministrante

{\bf REC:} prova que você faz quando vai mal na SUB.

{\bf DP:} matéria que você faz quando vai mal na REC.
\end{subsecao}

\begin{subsecao}{Sobre programas}

{\bf Computador:} objeto com vontade própria, sensível, que requer muito carinho e atenção. Normalmente comparado às mulheres, com duas pequenas diferenças: ele faz direito o que você pede e neles podemos fazer upgrades quando quisermos.

{\bf EP:} Exercício-Programa. Algo que você vai ter que fazer muitas vezes, e vai dar
muito trabalho.

{\bf GCC:} compilador mais recomendado para seus EP's, por suas inúmeras
qualidades. Atenção: ele ainda fará você se sentir burro.

{\bf Hello World:} Um clássico da programação universal.

{\bf Segmentation Fault:} Efeito computacional aleatório causado pela ``véspera de
entrega de EP''. Desenvolvido por Murphy.

{\bf Stack Overflow:} mensagem que aparece na tela do computador (Windows)
quando ele se recusa a funcionar. Isso ocorre quando ele está magoado, cansado, ou
simplesmente está ``naqueles dias''.

{\bf Teorema Fundamental do EP:} ``O EP só funcionará no dia da entrega.'' Não
confunda com o Corolário 42 da Lei de Murphy: ``O EP só {\bf não}
funcionará no dia da entrega!''

{\bf Linux:} Sistema operacional criado totalmente em linguagem C, graças a um
esforço mundial de milhares de programadores e experts em informática, composto
por aproximadamente 7 mil arquivos e 5 milhões de linhas, e com o qual você não
tem capacidade para trabalhar.

{\bf Windows:} Vírus. Porém tão bem mascarado que parece até a coisa correta a se usar.
\end{subsecao}

\begin{subsecao}{Sobre a fauna da USP}

{\bf politécnico:} Podemos
defini-los através de seu próprio hino: ``Mulher, mulher
pra quê / eu quero a HP; Mulher, mulher que nada / eu quero a derivada; Mulher,
mulher faz mal / eu quero a integral...''

{\bf VETERANO:} O ser supremo. O senhor de sua vida imeana. Nunca responda,
desrespeite, agrida, afronte ou encare um VETERANO. Se um VETERANO lhe dirigir
a palavra agradeça ao seu Deus, pois você foi abençoado. Com todas as letras
maiúsculas e sempre em fontes TrueType.

{\bf Monitor:} VETERANO
que é mal pago pelo instituto para tirar dúvidas e corrigir
listas de exercícios e/ou EP's de uma determinada matéria.

{\bf Formado:} VETERANO + diploma

{\bf Mestre:} VETERANO + diploma + dissertação

{\bf Doutor:} VETERANO + diploma + dissertação + experiência + tese

{\bf Professor:} Aquele que sabe muito, só não conta pra você. A maior parte não fala português; alguns apenas emitem ruídos estranhos.

(N.do E.): Como vocês podem perceber, o VETERANO está em todas. Portanto, você
deve se esforçar ao máximo para se tornar um de nós (esperamos realmente que você
consiga, pois não vai ser fácil)

{\bf bixo:} O ser
mais inferior da face da Terra. Para encontrar um basta olhar no
espelho. Sempre em minúsculas.
\end{subsecao}


\begin{subsecao}{Sobre a USP}

{\bf Bandejão:} Local de torturas diárias. Ali você irá receber a sua dose
de substâncias estranhas. Se você criar o hábito de comer lá todos os dias, quando
houver guerra biológica ou radioativa você estará imune.

{\bf CEPE:} Centro de Práticas Esportivas - lugar onde você poderá praticar todos os
esportes que quiser.

{\bf Circular:} ônibus interno da USP. Pegue um (se passar) e conheça a universidade toda. Dizemos TODA, porque esse ônibus dá voltas incríveis. Por outro lado, é incrivelmente mais barato que os ônibus da prefeitura, diriamos até que é o preço que vale andar com tal opção (ou falta dela).

{\bf COSEAS:} ao lado da praça do relógio. É onde os alunos fazem a carteirinha de
passes de ônibus, EMTU e metrô, além de solicitar os auxílios eteceteras que estão
explicados na seção respectiva.

{\bf CRUSP:} Conjunto residencial da USP. Se você se inscrever, torça para pegar um
apartamento num dos blocos já reformados, ou então torça para não conseguir
nenhum.

{\bf Colméia:} conjunto de favos.

{\bf Favos:} um monte de prédios hexagonais encravados no meio do CRUSP.

{\bf CINUSP:} Cinema da USP localizado no favo 4 da Colméia. Toda semana ele
passa um filme de qualidade. Informe-se sobre a programação em
{\tt www.usp.br/cinusp} 

{\bf Pelletron:} prédio da Física que na realidade é um acelerador de partículas. Nada de ter idéias mirabolantes, Joselito... 

{\bf PUTUSP:} Se você, bixo, quiser fazer Iniciação (não científica), vá até a
Avenida Valdemar Ferreira (saída principal da USP). Lá estarão os(as)
instrutores(as) dispostos a iniciá-lo 24 horas por dia.

{\bf Vet (Veterinária):} Para
onde são mandados os bixos que se machucaram no trote.

{\bf H.U. (Hospital Universitário):} para onde são mandados os bixos que não foram aceitos na Vet. Lá são realizados os tratamentos e experiências com exposição a radiação, exposição a aspirantes a médicos, teste de paciência/resistência a dor assim que pega a senha, alto nível de gesso no estômago, queda espontânea (ou não) de cabelo etc.

{\bf Psico:} Para onde mandamos os bixos que pensam que são gente.

\end{subsecao}
\end{secao}

\begin{secao}{Considerações Finais}

Este guia chegou ao fim. Você deve estar pensando ``E agora? O que eu, bixo burro, vou fazer,
sozinho nessa faculdade?''.


Se precisar de alguma ajuda, basta procurar algum VETERANO que ele te ajudará (ou rirá 
da sua pergunta de bixo, mas te ajudará mesmo assim! Ou não, depende, mas não custa nada
 tentar!).


Não esqueça de comprar seu kit-bixo para ajudar a recepção e tornar a recepção do ano que vem mais legal que a sua.
Participe da semana de recepção, carinhosamente preparada pelo Instituto e pelos seus VETERANOS devidamente identificados.
Último conselho: Guarde este guia para o resto de sua graduação. Por mais ``bixo'' que seja, você precisará dele.


Qualquer sugestão, elogio, presentes, bajulações entre outras coisas LEGAIS para a Comissão, você receberá atenção no email: comissaodetrote@gmail.com .


\end{secao}
\end{document}


