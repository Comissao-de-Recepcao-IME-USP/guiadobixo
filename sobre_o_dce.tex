\begin{secao}{Um Pouco Sobre o DCE}

O Diretório Central dos Estudantes (DCE) é a entidade geral de representação
dos estudantes da USP. É o ``pai'' de todos os centros acadêmicos.

Diretório não tem nada a ver com computador. Quer dizer, ter tem, mas o
diretório do DCE não tem. Ai... está muito complicado. Bom, saiba que o DCE
nasceu há quase 30 anos como resultado de uma greve dos estudantes em
razão da morte do professor da USP e jornalista da TV Cultura Wladimir Herzog
(Wlado), considerado um perigoso subversivo (sinônimo de comunista), que fora
levado a prestar depoimento em 1975, na sede do DOI-Codi, órgão responsável
pelo controle de ordem interna (que freqüentemente extravasava seu próprio
controle), e de lá saiu morto.

A nossa sociedade permanecia em silêncio, assustada e constrangida desde a
decretação do AI-5 (Ato Institucional número 5, pelo qual a ditadura criava
formas de repressão social...isso não caiu na FUVEST não?), até o assassinato
de Wlado.

Como dizíamos, essa greve começou na ECA e logo se expandiu por todo o  campus,
paralisando a Física e a FFLCH (Faculdade de Filosofia, Letras e Ciências
Humanas) e fazendo o Movimento Estudantil renascer após a desestruturação que
sofrera durante os, assim chamados, anos de chumbo.

Em 76, todas as entidades estudantis eram consideradas ilegais, menos aquelas
criadas pela lei Suplicy de Lacerda, que permitia a existência de DA's
(Diretórios Acadêmicos) e DCE's atrelados às diretorias das escolas.
Significava que os estudantes podiam ter suas representações estudantis desde
que estas passassem pelo crivo das autoridades de plantão: Reitores e
Diretores nas escolas; Ministros e Secretários de Educação que, por sua vez,
escolhiam os Reitores e Diretores no poder executivo.

O nosso DCE teve uma história diferente: ele nasceu livre; isso significa dizer
que nenhuma de suas diretorias, nesses 20 e tantos anos, teve o aval de outros
que não fossem os estudantes da USP. O que não significa que as diretorias do
DCE tenham sido todas maravilhosas e ativas na vida dos estudantes. Mas 
voltando, o fato do diretório ser livre teve tão grande relevância na época
que a palavra LIVRE foi incorporada ao nome do DCE - Livre da USP (como hoje é
conhecido). Além de livre, o DCE recebeu o nome de um jovem:
``Alexandre Vannucchi Leme'', que foi um estudante da USP da Geologia. Ele foi
morto em 1973, nas mesmas circunstâncias de Wlado. Por todas suas lutas e
resistência, parabéns, DCE - Livre da USP!

\end{secao}
