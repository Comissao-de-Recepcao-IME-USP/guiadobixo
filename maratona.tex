\begin{secao}{Maratona de Programação}

A primeira vista a Maratona de Programação pode soar um tanto
surreal. Nerds correndo pela USP ao mesmo tempo que resolvem
problemas de computação e matemática? Infelizmente esse não
é o caso.

A Maratona de Programação se resume à resolução de problemas.
Se você adora resolver desafios, quebrar a cabeça com novos
e excitantes problemas e acumular toneladas de dinheiro, esse
é o lugar perfeito para você!

A competição consiste de uma série de problemas que englobam
temas como programação dinâmica, grafos e estruturas de dados.
Times de três pessoas devem resolver a maior quantidade de
desafios em um determinado tempo. E tudo isso com direito
a um lanche gratuito durante a prova.

Mas não tema, bixo. Não é só por que você é ingressante que
a probabilidade de se ganhar uma medalha seja nula. Na primeira 
fase da maratona, uma equipe de bixos tem vaga garantida para a 
fase brasileira.

Além da fama, constantes pedidos por autógrafos e dinheiro de sobra,
a maratona também lhe traz um conhecimento muito mais 
adiantado aos dos seus colegas de classe, e até oportunidades
de emprego em empresas de renome, como Google, Facebook e IBM.

Se você se interessou pela maratona e quer saber os horários dos
treinos, como participar ou saber mais sobre como funciona, acesse:
http://www.ime.usp.br/~maratona

Se você quer saber sobre coisas mais oficiais sobre a maratona, acesse:
http://maratona.ime.usp.br/

\end{secao}
